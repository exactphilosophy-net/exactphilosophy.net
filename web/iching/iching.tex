
% need * here
\section*{\en{i ching}\de{i ging}\hspace{0.3mm} \moebius{0.017}}

\en{All cultures seem to know some kinds of elements, but let
me consider the 8 trigrams of the Chinese Book of Changes,
the \textsl{I Ching} or \textsl{Yijing}, which may be quite fundamental.}%
\de{Alle Kulturen scheinen eine Art von Elementen zu kennen,
aber lasst mich die 8 Trigramme des chinesischen Buches
der Wandlungen, des \textsl{I Ging} oder \textsl{Yijing}, betrachten, welche
vielleicht ganz grundlegend wären.}

\small
\begin{center}
\begin{tabular}{|l|l|}\hline
\en{\triheaven & heaven, strong, creative, father \\ \hline
\triearth & earth, devoted/yielding, receptive, mother \\ \hline
\trithunder & thunder, inciting movement, arousing, 1st son\\ \hline
\triwater & water, dangerous, abysmal, 2nd son \\ \hline
\trimountain & mountain, resting, keeping still, 3rd son \\ \hline
\triwind & wind/wood, penetrating, gentle, 1st daughter \\ \hline
\trifire & fire, light-giving, clinging, 2nd daughter \\ \hline
\trilake & lake, joyful, joyous, 3rd daughter \\ \hline}%
\de{\triheaven & himmel, stark, schöpferisch, vater \\ \hline
\triearth & erde, hingebend, empfangend, mutter \\ \hline
\trithunder & donner, bewegend, erregend, 1.\ sohn\\ \hline
\triwater & wasser, gefährlich, abgründig, 2.\ sohn \\ \hline
\trimountain & berg, ruhend, stillhalten, 3.\ sohn \\ \hline
\triwind & wind/holz, eindringend, sanft, 1.\ tochter \\ \hline
\trifire & feuer, leuchtend, haftend, 2.\ tochter \\ \hline
\trilake & see, fröhlich, heiter, 3.\ tochter \\ \hline}
\end{tabular}
\end{center}

\normalsize
\en{They seem to resemble Greek elements in pairs,
namely heaven-wind (air), earth-mountain, fire-thunder and
water-lake. Let me rearrange them into another table:}%
\de{Sie scheinen den griechischen Elementen paarweise zu
ähneln, nämlich Himmel-Wind (Luft), Erde-Berg, Feuer-Donner
und Wasser-See. Lasst mich sie in einer weiteren
Tabelle entsprechend anordnen:}

\small
\begin{center}
\begin{tabular}{|l|l|l|l|l|}\hline
\en{\triheaven & heaven & \textbf{air} & rests & male \\ \hline
\triwind & wind/wood & \textbf{air} & moves & female \\ \hline
\trimountain & mountain & \textbf{earth} & rests & male \\ \hline
\triearth & earth & \textbf{earth} & moves & female \\ \hline
\trifire & fire & \textbf{fire} & rests & female \\ \hline
\trithunder & thunder & \textbf{fire} & moves & male \\ \hline
\trilake & lake & \textbf{water} & rests & female \\ \hline
\triwater & water & \textbf{water} & moves & male \\ \hline}%
\de{\triheaven & himmel & \textbf{luft} & ruht & männlich \\ \hline
\triwind & wind/holz & \textbf{luft} & bewegt & weiblich \\ \hline
\trimountain & berg & \textbf{erde} & ruht & männlich \\ \hline
\triearth & erde & \textbf{erde} & bewegt & weiblich \\ \hline
\trifire & feuer & \textbf{feuer} & ruht & weiblich \\ \hline
\trithunder & donner & \textbf{feuer} & bewegt & männlich \\ \hline
\trilake & see & \textbf{wasser} & ruht & weiblich \\ \hline
\triwater & wasser & \textbf{wasser} & bewegt & männlich \\ \hline}
\end{tabular}
\end{center}

\normalsize
\en{Interestingly, the trigrams that correspond to the Greek elements,
i.e.\ resting air and earth, moving fire and water,
are exactly the male trigrams.}%
\de{Interessanterweise sind die Trigramme, die den griechischen
Elementen entsprechen, d.h.\ ruhende Luft und Erde,
bewegtes Feuer und Wasser, genau die männlichen Trigramme.}

% page
\en{Let me map each trigram to the result of a transition
between two elements in Aristotle’s circle of the elements,
ending with the corresponding element and starting with a
male element (fire or air) for the male trigrams (father and
sons) and with a female element (water or earth) for the
female trigrams (mother and daughters):}%
\de{Lasst mich jedes Trigramm dem Resultat eines Über\-gangs
zwischen zwei Elementen in Aristoteles’ Kreis der Elemente
zuordnen, endend mit dem entsprechenden Element
und beginnend mit einem männlichen Element
(Feuer oder Luft) für die männlichen Trigramme (Vater und
Söhne) und mit einem weiblichen Element (Wasser oder
Erde) für die weiblichen Trigramme (Mutter und Töchter):}

\vspace{2.2mm}
\noindent
\en{\includegraphics[scale=0.0605]{i-circle-trigrams.jpg}}%
\de{\includegraphics[scale=0.0602]{i-circle-trigrams-de.jpg}}
\vspace{2.2mm}

\en{The trigrams seem to fit closely:
Thunder as fire that has suddenly come down as lightning from the sky (air),
in contrast to fire steadily clinging to the matter (earth) it burns;
wind as air that gently evaporated from water,
in contrast to gases from a fire risen to heaven;
a lake as water sprung from sources (earth),
in contrast to water fallen down as rain from the sky (air);
a mountain as earth solidified from lava (fire),
in contrast to softly yielding earth from sediments deposited by water.}%
\de{Die Trigramme scheinen gut zu passen:
Donner als Feuer, das plötzlich als Blitz vom Himmel (Luft) heruntergefahren ist,
im Gegensatz zu Feuer, das beständig an der Materie (Erde) haftet, die es verbrennt;
Wind als Luft, die sanft aus Wasser verdunstet ist,
im Gegensatz zu den Gasen eines zum Himmel aufgestiegenen Feuers;
ein See als Wasser, das aus Quellen (Erde) entsprungen ist,
im Gegensatz zu Wasser, das als Regen vom Himmel (Luft) herabgefallen ist;
ein Berg als Erde, die aus Lava (Feuer) erstarrt ist,
im Gegensatz zu sanft nachgebender Erde aus von Wasser abgelagerten Sedimenten.}

\small
\begin{center}
\begin{tabular}{|l|l|rl|l|l|}\hline
\en{\triheaven & heaven & \textbf{air} & $\!\!\!\!\!\!\leftarrow$\,fire & rests & male \\ \hline
\triwind & wind & \textbf{air} & $\!\!\!\!\!\!\leftarrow$\,water & moves & female \\ \hline
\trimountain & mountain & \textbf{earth} & $\!\!\!\!\!\!\leftarrow$\,fire & rests & male \\ \hline
\triearth & earth & \textbf{earth} & $\!\!\!\!\!\!\leftarrow$\,water & moves & female \\ \hline
\trifire & fire & \textbf{fire} & $\!\!\!\!\!\!\leftarrow$\,earth & rests & female \\ \hline
\trithunder & thunder & \textbf{fire} & $\!\!\!\!\!\!\leftarrow$\,air & moves & male \\ \hline
\trilake & lake & \textbf{water} & $\!\!\!\!\!\!\leftarrow$\,earth & rests & female \\ \hline
\triwater & water & \textbf{water} & $\!\!\!\!\!\!\leftarrow$\,air & moves & male \\ \hline}%
\de{\triheaven & himmel & \textbf{luft} & $\!\!\!\!\!\!\leftarrow$\,feuer & ruht & männlich \\ \hline
\triwind & wind & \textbf{luft} & $\!\!\!\!\!\!\leftarrow$\,wasser & bewegt & weiblich \\ \hline
\trimountain & berg & \textbf{erde} & $\!\!\!\!\!\!\leftarrow$\,feuer & ruht & männlich \\ \hline
\triearth & erde & \textbf{erde} & $\!\!\!\!\!\!\leftarrow$\,wasser & bewegt & weiblich \\ \hline
\trifire & feuer & \textbf{feuer} & $\!\!\!\!\!\!\leftarrow$\,erde & ruht & weiblich \\ \hline
\trithunder & donne & \textbf{feuer} & $\!\!\!\!\!\!\leftarrow$\,luft & bewegt & männlich \\ \hline
\trilake & see & \textbf{wasser} & $\!\!\!\!\!\!\leftarrow$\,erde & ruht & weiblich \\ \hline
\triwater & wasser & \textbf{wasser} & $\!\!\!\!\!\!\leftarrow$\,luft & bewegt & männlich \\ \hline}
\end{tabular}
\end{center}

\normalsize
\en{This arrangement is none of the two traditionally known
ones, more similar to Earlier Heaven than Later Heaven:}%
\de{Diese Anordnung ist keine der beiden traditionell
bekannten; sie ähnelt eher der des frühen Himmel als der des
späten Himmels:}

\vspace{2mm}
\en{\includegraphics[scale=0.11]{i-heavens.jpg}}%
\de{\includegraphics[scale=0.11]{i-heavens-de.jpg}}

\vspace{1mm}
% page
\en{More symmetries, some similar to Earlier Heaven:}%
\de{Mehr Symmetrien, einige ähnlich wie beim frühen Himmel:}

\begin{list}{$\bullet$}{\setlength{\leftmargin}{10pt}}

\vspace{-1.5mm}
\item
\en{Daughters and sons are arranged from father to first
to second to third children, and finally to mother.}%
\de{Töchter und Söhne sind vom Vater zum ersten, zweiten
und dritten Kind und schliesslich zur Mutter angeordnet.}

\vspace{-1.5mm}
\item
\en{Opposite trigrams in the circle mirror each other
if you mirror each trigram at the middle line
(i.e.\ swap first and third line) and invert all lines
(yin\lrarr yang).}%
\de{Entgegengesetzte Trigramme im Kreis spiegeln sich
gegenseitig wenn man jedes Trigramm an der mittleren
Linie spiegelt (d.h.\ erste und dritte Linie vertauscht) und
alle Linien umkehrt (Yin\lrarr Yang).}

\vspace{-1.5mm}
\item
\en{Trigrams that transform to or from outer elements have
a broken (yin) line in the middle, which would fit with outer
elements being harder and more brittle, breaking more easily.}%
\de{Trigramme, die sich zu oder aus äusseren Elementen
verwandeln, haben in der Mitte eine unterbrochene (Yin)
Linie, was dazu passen würde, dass äussere Elemente
härter und spröder sind und leichter brechen.}

\vspace{-1.5mm}
\item
\en{Excluding the middle line, between adjacent trigrams in the
circle exactly one line is inverted (yin\lrarr yang).}%
\de{Mit Ausnahme der mittleren Linie ist zwischen benachbarten
Trigrammen im Kreis genau eine Linie invertiert
(Yin\lrarr Yang).}

\end{list}

\en{Let me arrange the circle of elements and trigrams onto
a \mbox{Möbius Strip \moebius{0.013}} as follows (click for larger image):}%
\de{Lasst mich den Kreis der Elemente und Trigramme wie
folgt auf einem \mbox{Möbiusband \moebius{0.013}} anordnen (klicken für ein
grösseres Bild):}

\begin{center}
\en{\includegraphics[scale=0.082]{i-flat.jpg}}%
\de{\includegraphics[scale=0.082]{i-flat-de.jpg}}
\end{center}

\vspace{-4.5mm}
\begin{center}
\includegraphics[scale=0.04]{i-photo.jpg}
\end{center}
\vspace{-2mm}

\en{Inner elements are placed on the inside of the strip,
outer elements on the outside. That way, the strip reminds
of the supposed permeable membrane between in and out,
but with different elements touching: The symbols for the
moving elements fire and water touch on opposite sides of
the strip, coinciding perfectly, and the same is true for the
resting elements earth and air. All lines of the trigrams
on one side of the strip are mirrored by their inverted lines
(yin\lrarr yang) on the other side, so that yin and yang \textsl{are}
different sides of the same on the strip.}%
\de{Die inneren Elemente befinden sich auf der Innenseite
des Streifens, die äusseren Elemente auf der Aussenseite.
Auf diese Weise erinnert der Streifen an die angenomme
durchlässige Membran zwischen innen und aussen, wobei
sich jedoch andere Elemente berühren: Die Symbole für
die sich bewegenden Elemente Feuer und Wasser berühren
sich auf gegenüberliegenden Seiten des Streifens, wobei sie
perfekt übereinstimmen, und dasselbe gilt für die ruhenden
Elemente Erde und Luft. Alle Linien der Trigramme auf
einer Seite des Streifens werden durch ihre invertierten Linien
(Yin\lrarr Yang) auf der anderen Seite gespiegelt, so dass Yin
und Yang auf dem Streifen \textsl{tatsächlich} verschiedene Seiten
desselben sind.}

\en{So, even though fire and water would touch, and maybe
mirror each other between in and out, they could not transform
directly into each other, only indirectly by going along
the single surface of the strip via air or earth.}%
\de{Obwohl sich also Feuer und Wasser berühren und sich
vielleicht zwischen innen und aussen spiegeln würden, könn\-ten
sie sich nicht direkt ineinander verwandeln, sondern nur
indirekt, indem sie über die Luft oder die Erde entlang der
einzigen Oberfläche des Bandes wandeln würden.}

\vspace{0.5mm}
% page
\hspace{-5mm}
\en{\includegraphics[scale=0.13]{i-infinity.jpg}}%
\de{\includegraphics[scale=0.13]{i-infinity-de.jpg}}

\vspace{-0.2mm} % keep
\subsection{\en{leads}\de{fährten}}

\small
\begin{list}{$\bullet$}{\setlength{\leftmargin}{10pt}}

\item
\en{The I Ching is a divination system. By tossing coins or
drawing yarrow sticks, one determines hexagrams (two trigrams)
that are given meanings in the text of the I Ching.
More precisely, the oracle results in two hexagrams, describing
the evolution of the current situation to a new situation.}%
\de{Das I Ging ist ein Verfahren zur Weissagung. Durch Werfen
von Münzen oder Ziehen von Schafgarbenstäbchen werden
Hexagramme (zwei Trigramme) bestimmt, die im Text des
I Ging mit Bedeutungen versehen sind. Genauer gesagt
ergibt das Orakel zwei Hexagramme, die die Entwicklung der
aktuellen Situation zu einer neuen Situation beschreiben.}

\item
\en{This new arrangement of the 8 trigrams and 4 elements in a
circle was inspired by a passage in the introduction of Richard
Wilhelm’s translation of the \textsl{I Ching or Book of Changes}
(translated from German to English by Cary F. Baynes):}%
\de{Diese neue Anordnung der 8 Trigramme und 4 Elemente in
einem Kreis wurde inspiriert durch eine Passage in der
Einleitung von Richard Wilhelms Übersetzung des I Ging:}

\en{“The eight trigrams are symbols standing for changing
transitional states; they are images that are constantly undergoing
change. Attention centers not on things in their state of
being—as is chiefly the case in the Occident—but upon their
movements in change. The eight trigrams therefore are not
representations of things as such but of their tendencies in
movement.”}%
\de{“Die acht Zeichen sind Zeichen wechselnder Übergangs\-zu\-stän\-de,
Bilder, die sich dauernd verwandeln. Worauf das
Augenmerk gerichtet war, waren nicht die Dinge in ihrem Sein—
wie das im Westen hauptsächlich der Fall war—, sondern die
Bewegungen der Dinge in ihrem Wechsel. So sind die acht
Zeichen nicht Abbildungen der Dinge, sondern Abbildungen
ihrer Bewegungstendenzen.”}

\en{So the 8 Chinese trigrams would express essentially the same
elements and changes in a circle as the 4+1 Greek elements,
i.e.\ the fifth element would be contained in the trigrams.}%
\de{Die 8 chinesischen Trigramme würden also im Wesentlichen
die gleichen Elemente und Veränderungen in einem Kreis
ausdrücken wie die 4+1 griechischen Elemente, d.h.\ das fünfte
Element wäre in den Trigrammen enthalten.}

\item
\en{Also in terms of bind/release, the trigrams seem to fit closely:
Fire, heaven, lake and mountain hold their element in place;
thunder, wind, water and earth let it go.}%
\de{Auch in Bezug auf binden/lösen scheinen die Trigramme gut
zu passen: Feuer, Himmel, See und Berg halten ihr Element
fest; Donner, Wind, Wasser und Erde lassen es los.}

\item
\en{No common historical roots are known, nor any roots of the above
arrangement of trigrams in Chinese history, so did both cultures 
mirror nature independently, even unknowingly\,?}%
\de{Es sind keine gemeinsamen historischen Wurzeln bekannt,
auch keine Wurzeln der obigen Anordnung der Trigramme in
der chinesischen Geschichte, haben also beide Kulturen die
Natur unabhängig voneinander gespiegelt, gar unbewusst\,?}

\en{Interpreting earth-water-air as the states of matter solid-fluid-gas
and fire as a chemical reaction or physical phenomenon
that produces light and maybe heat, the elements could be
considered what is most commonly encountered in nature.}%
\de{Interpretiert man Erde-Wasser-Luft als die Aggregatzustände
fest-flüssig-gasförmig und Feuer als eine chemische Reaktion
oder physikalisches Phänomen, das Licht und vielleicht
Wärme erzeugt, könnte man die Elemente als das erachten,
was in der Natur am häufigsten angetroffen wird.}

\en{Elements are elemental necessities of life with air to breathe,
water to drink, food to eat, plus energy/warmth, and they are
elemental and at times traumatic forces of nature with fires
and volcano eruptions, inondations, storms and landslides.}%
\de{Elemente sind elementare Lebensnotwendigkeiten wie Luft
zum Atmen, Wasser zum Trinken, Nahrung zum Essen und
Energie/Wärme, und sie sind elementare und zuweilen
traumatische Naturkräfte wie Brände und Vulkanausbrüche,
Über\-schwemmungen, Stürme und Erdrutsche.}

\en{Conversely, the very nature of oracles is that things are
connected, hence maybe also globally to some degree\,?}%
\de{Umgekehrt liegt es in der Natur eines Orakels, dass die Dinge
miteinander verbunden wären, daher vielleicht zu gewissem
Grad auch weltweit\,?}

% page
\item
\en{Each trigram is part of 15 hexagrams. In the images of the
hexagrams, the wind/wood trigram appears 10 times as wind,
5 times as wood or tree(s); fire 11 times as fire, two times as
lightning, one time as light, one time as sun; water 11 times
as water, two times as clouds, one time as rain, one time as
a spring. The other trigrams appear as themselves.}%
\de{Jedes Trigramm ist Teil von 15 Hexagrammen. In den Bildern
der Hexagramme erscheint das Wind/Holz-Trigramm 10-mal
als Wind, 5-mal als Holz oder Baum/Bäume; Feuer 11-mal als
Feuer, zweimal als Blitz, einmal als Licht, einmal als Sonne;
Wasser 11-mal als Wasser, zweimal als Wolken, einmal als
Regen, einmal als Quelle. Die anderen Trigramme erscheinen
als sie selbst.}

\item
\en{In the yarrow stalk method of consulting the I Ching,
one starts with 50 yarrow stalks and initially puts one away.
This seems to be a reference to the cycles of moon and sun,
because 50+49 lunar months are only about 1.5 days short of 8
solar years, which is also why the Olympics in ancient Greece
were held alternatively every 50 and 49 lunar months. Hence
the moon advances about 3/8 of the circle every solar year,
drawing an eight-pointed star over eight years, as well as
appearing in eight different lunar phases.}%
\de{Bei der Schafgarbenstäbchen-Methode zur Konsultation des
I Ging, beginnt man mit 50 Schafgarbenstäbchen und legt
zunächst eines weg. Dies scheint ein Hinweis auf die Zyklen
von Mond und Sonne zu sein, denn 50+49 Mondmonate sind
nur etwa 1.5 Tage weniger als 8 Sonnenjahre, weshalb auch
die Olympischen Spiele im alten Griechenland abwechselnd
alle 50 und 49 Mondmonate stattfanden. Daher schreitet der
Mond in jedem Sonnenjahr etwa 3/8 des Kreises voran, zeichnet
einen achtzackigen Stern über acht Jahre, und erscheint
dabei in acht verschiedenen Mondphasen.}

\hspace{21mm}\includegraphics[scale=0.06]{i-moonphases.jpg}

\en{Venus never separates more than about 1/8 of the circle from
the sun and appears to stand still 5$\times$2 times in 8 years, drawing
a pentagram that shifts only slightly between cycles. The
Mesopotamian goddess of love Ishtar was associated with
Venus, usually depicted as an eight-pointed star and sometimes
shown together with sun and moon.}%
\de{Die Venus entfernt sich nie weiter als etwa 1/8 des Kreises
von der Sonne und scheint 5$\times$2 Mal in 8 Jahren stillzustehen,
wobei sie ein Pentagramm zeichnet, das sich zwischen
den Zyklen nur leicht verschiebt. Die mesopotamische
Liebesgöttin Ištar wurde mit der Venus in Verbindung gebracht,
die gewöhnlich als achtzackiger Stern und manchmal zusammen
mit Sonne und Mond dargestellt wurde.}

\en{The yin-yang symbol \yinyang\ reminds of moon phases.}%
\de{Das Yin-Yang-Symbol \yinyang\ erinnert an Mondphasen.}

\en{“In its primary meaning yin is ‘the cloudy’, ‘the overcast’
and yang means actually ‘banners waving in the sun’, that
is, something ‘shone upon’, or bright. By transference the
two concepts were applied to the light and dark sides of a
mountain or of a river. In the case of a mountain the southern
is the bright side and the northern the dark side, while in the
case of a river seen from above, it is the northern side that is
bright (yang), because it reflects the light, and the southern
side that is in shadow (yin).” (Wilhelm/Baynes, introduction)}%
\de{“Yin ist in seiner Urbedeutung das Wolkige, Trübe; Yang
bedeutet eigentlich: in der Sonne wehende Banner, also
etwas Beleuchtetes, Helles. Übertragen wurden die beiden
Begriffe auf die erleuchtete und die dunkle (d.h.\ südliche und
nördliche) Seite eines Berges oder Flusses (wo aber die
Süd\-sei\-te im Blick auf den Fluss dunkel, d.h.\ Yin, und die das
Licht reflektierende Nordseite hell, d.h.\ Yang, ist).” (Wilhelm,
Einführung)}

\item
\en{The five Chinese Wu Xing, water, metal, fire, wood and earth,
which are often called “elements” in the West, but literally
mean “moving”, stand most immediately for the five planets
visible to the naked eye, Mercury, Venus, Mars, Jupiter and
Saturn, while the “Four Symbols”, black turtle (plus snake),
white tiger, vermillion bird (phoenix) and azure dragon stand
for the four directions and for constellations in the sky (each
for a group of 7 of the 28 mansions). Together with the I
Ching maybe standing for sun and moon, this would complete
the sky and what it was believed to reflect down on earth.}%
\de{Die fünf chinesischen Wu Xing, Wasser, Metall, Feuer, Holz
und Erde, die im Westen oft als “Elemente” bezeichnet werden,
aber wörtlich “bewegt” bedeuten, stehen am ehesten für
die fünf mit blossem Auge sichtbaren Planeten Merkur,
Venus, Mars, Jupiter und Saturn, während die “vier Symbole”,
die schwarze Schildkröte (plus Schlange), der weisse Tiger,
der zinnoberrote Vogel (Phönix) und der azurblaue Drache
für die vier Himmelsrichtungen und für die Sternbilder stehen
(jeweils für eine Gruppe von 7 der 28 Landsitze). Zusammen
mit dem I Ging, das vielleicht für Sonne und Mond stünde,
würde dies den Himmel vervollständigen und das, was
geglaubt wurde, das er auf der Erde widerspiegeln würde.}

\item
\en{In the five Wu Xing, earth often has a somewhat central role, 
surrounded by things that emerge from it and go back to it:
water from springs, fire from volcanoes, wood growing from
earth and metal mined from it; four very useful ingredients
for humans to shape their worlds, like using fire to smelt ore
into metal tools, which can then be used to cut wood into
houses, furniture, bows, plows, water wheels, etc.}%
\de{In den fünf Wu Xing hat die Erde oft eine zentrale Rolle,
umgeben von Dingen, die aus ihr hervorgehen und zu ihr
zurückkehren: Wasser aus Quellen, Feuer aus Vulkanen, Holz,
das aus der Erde wächst, und Metall, das aus ihr gewonnen
wird; vier sehr nützliche Zutaten für die Menschen, um ihre
Welt zu gestalten, wie z.B.\ die Verwendung von Feuer, um
Erz zu Metallwerkzeugen zu schmelzen, die dann verwendet
werden können, um Holz zu Häusern, Möbeln, Pfeilbögen,
Pflügen, Wasserrädern usw.\ zu verarbeiten.}

% page
\item
\en{In the Chinese zodiac, four star signs are assigned to earth,
arranged in a cross, and in the four sectors in between the
two star signs there are assigned to water, metal, fire and
wood, respectively. This reminds a lot of Aristotle’s circle
with trigrams above, so maybe the Wu Xing earth would
correspond to the static Greek elements and the other four
Wu Xing to the trigrams of the I Ching for the corresponding
transformations\,? Can this be identified in the attributes of
the star signs of the Chinese zodiac\,?}%
\de{Im chinesischen Tierkreis sind vier Sternzeichen der Erde
zugeordnet, die in einem Kreuz angeordnet sind, und die
jeweils zwei Sternzeichen in den vier Sektoren dazwischen sind
je Wasser, Metall, Feuer und Holz zugeordnet. Dies erinnert
sehr an den obigen Kreis von Aristoteles mit Trigrammen, so
dass vielleicht das Wu Xing Erde den statischen griechischen
Elementen entsprechen würde und die anderen vier Wu Xing
den Trigrammen des I Ging für die entsprechenden Transformationen\,?
Kann man dies in den Attributen der Sternzeichen
des chinesischen Tierkreises gespiegelt finden\,?}

\item
\en{Is the association of trigrams with elements and their changes
also closely mirrored in the hexagrams and their changes\,?}%
\de{Widerspiegelt sich die Assoziation der Trigramme mit den
Elementen und ihren Wandlungen auch nahezu exakt in den
Hexagrammen und ihren Wandlungen\,?}

\item
\en{When consulting the I Ching as an oracle, the different lines
are assigned the numbers 6 to 9:}%
\de{Wenn man das I Ging als Orakel konsultiert, werden den
verschiedenen Linien die Zahlen 6 bis 9 zugeordnet:}

\begin{center}
\begin{tabular}{|l|l|l|l|}\hline
\en{6 & old (changing) yin & - - to — & -x-\\ \hline
7 & new (unchanging) yang & — to — & —\\ \hline
8 & new (unchanging) yin & - - to - - & - -\\ \hline
9 & old (changing) yang & — to - - & -o-\\ \hline}%
\de{6 & alt (sich verändernd) yin & - - zu — & -x-\\ \hline
7 & neu (nicht verändernd) yang & — zu — & —\\ \hline
8 & neu (nicht verändernd) yin & - - zu - - & - -\\ \hline
9 & alt (sich verändernd) yang & — zu - - & -o-\\ \hline}
\end{tabular}
\end{center}

\en{These numbers are also associated with the Wu Xing. They
are derived from 5 (earth) plus 1 to 4 (water, fire, wood,
metal), see the Yellow River Map, e.g.\ in Wilhelm/Baynes.}%
\de{Diese Zahlen werden auch mit den Wu Xing assoziiert. Sie
werden abgeleitet von 5 (Erde) plus 1 bis 4 (Wasser, Feuer,
Holz, Metall), siehe die Karte des Gelben Flusses, z.B.\ bei
Wilhelm.}

\en{As a different approach, let me number the elements in Aristotle’s
circle as 1-2-3-4, starting a priori with any element and
going in either direction of the circle. Now, map transformations
of elements to the sum of the three elements involved,
1+2+3 = 6, 2+3+4 = 9, 3+4+1 = 8 and 4+1+2 = 7, where
the element in the middle is the one that is transformed.}%
\de{Als alternativen Zugang nummeriere ich die Elemente in
Aristoteles’ Kreis mit 1-2-3-4, wobei ich a priori mit einem
beliebigen Element beginne und in eine beliebige Richtung den
Kreis rumgehe. Nun ordne ich die Wandlungen der Elemente
der Summe der drei beteiligten Elemente zu: 1+2+3 = 6,
2+3+4 = 9, 3+4+1 = 8 und 4+1+2 = 7, wobei das Element
in der Mitte dasjenige ist, das verwandelt wird.}

\en{This gives also the numbers from 6 to 9 and note that new
yin and yang are obtained for the sequences that cross from
4 to 1, i.e.\ into a \textsl{new} cycle.}%
\de{Daraus ergeben sich ebenfalls die Zahlen von 6 bis 9, wobei
interessant ist, dass sich für Sequenzen mit einem Übergang
von 4 zu 1 neues Yin und Yang ergeben, d.h.\ ein \textsl{neuer} Zyklus.}

\en{Let me number the elements 1-fire, 2-air, 3-water, 4-earth,
starting with the lightest element according to Aristotle:}%
\de{Ich nummeriere nun die Elemente 1-Feuer, 2-Luft, 3-Wasser,
4-Erde, beginnend mit dem leichtesten gemäss Aristoteles:}

\begin{center}
\begin{tabular}{|l|l|l|}\hline
\en{6 & transformation of air & 36 = 6$\times$6 Stratagems\\ \hline
7 & transformation of fire & 49 = 7$\times$7 Qixi (Ch’i\,?)\\ \hline
8 & transformation of earth & 64 = 8$\times$8 I Ching\\ \hline
9 & transformation of water & 81 = 9$\times$9 Tao Te Ching\\ \hline}%
\de{6 & wandlung der luft & 36 = 6$\times$6 Stratageme\\ \hline
7 & wandlung des feuers & 49 = 7$\times$7 Qixi (Chi\,?)\\ \hline
8 & wandlung der erde & 64 = 8$\times$8 I Ging\\ \hline
9 & wandlung des wassers & 81 = 9$\times$9 Tao Te King\\ \hline}
\end{tabular}
\end{center}

\en{This fits astonishingly well with contemporary Western
astrological views of the elements. The 36 Stratagems provide
stratagems to use in politics and war, which fits well with air
as conscious planning mind. The I Ching yields a priori images
of changes in the outer, material world, the element earth,
which are then interpreted in a more detached way. The Tao
Te Ching, which comes in 81 sections, often has something
that flows like water. Besides the 50/49 yarrow stalks, there
is the Qixi Festival on the 7th day of the 7th month of the
year when magpies mythologically build a bridge across the
milky way to briefly reunite two lovers, and ch’i (qì) stands
for life energy and breath (which reminds of pneuma), and is
pronounced almost like the word for 7 (qī) in Chinese.}%
\de{Das passt erstaunlich gut zu zeitgenössischen westlichen
astrologischen Ansichten über die Elemente. Die 36 Strategeme
beschreiben Strategeme für Politik und Krieg, was gut zu Luft
als bewusstem, planendem Geist passt. Das I Ging liefert a
priori Bilder von Veränderungen in der äusseren, materiellen
Welt, dem Element Erde, die dann auf eine etwas abgelöstere
Weise interpretiert werden. Das Tao Te King, gegliedert in 81
Abschnitte, hat oft etwas, das wie Wasser fliesst. Neben den
50/49 Schafgarbenstäbchen gibt es das Qixi-Fest am siebten
Tag des siebten Monats des Jahres, wo Elstern mythologisch
eine Brücke über die Milchstrasse bauen, um zwei Liebende
kurzzeitig wieder zu vereinen, und Chi (qì) steht für Lebensenergie
und Atem (was an pneuma erinnert), und wird auf
chinesisch fast wie das Wort für 7 (qī) ausgesprochen.}

% page
\en{In ancient China, fields in agriculture used to be divided into
squares of 9 = 3 $\times$ 3 fields, with 8 fields (earth) owned by
individual families around a central 9th field that belonged to
all families and contained the well (water).}%
\de{Im antiken China waren Felder in der Landwirtschaft in Quadrate
von 9 = 3 $\times$ 3 Feldern aufgeteilt, wobei 8 Felder (Erde)
einzelnen Familien gehörten und um ein zentrales 9.\ Feld
herum angeordnet waren, welches allen Familien gehörte und
den Brunnen (Wasser) enthielt.}

\vspace{1mm}
\hspace{31mm}
\includegraphics[scale=0.11]{i-fields.jpg}

\item
\en{The most ancient Chinese oracles used bones (typically shoulder
bones of oxen) or turtle plastrons (the belly part of the
turtle shell). Holes were drilled and heated with a heat source
from the back of the plastron to produce cracks on the front.
There seems to be no direct evidence for influence on the I
Ching, so far, while it seems generally admitted that the turtle
would have represented heaven with the upper dome of
its shell and earth with its plastron. On the northern
hemisphere, stars appear to rotate around the north pole in the
sky, the direction assigned to the turtle of the four symbols.}%
\de{Die ältesten chinesischen Orakel verwendeten Knochen (in
der Regel Schulterknochen von Ochsen) oder Wasserschildkröten-Plastrons
(den Bauchteil des Panzers). Darin wurden
Löcher hineingebohrt und mit einer Wärmequelle von der
Rückseite des Plastrons aus erhitzt, um Risse auf der Vorderseite
zu erzeugen. Es scheint bisher keinen direkten Beweis für
einen Einfluss auf das I Ging zu geben, während es allgemein
anerkannt scheint, dass die Wasserschildkröte mit der oberen
Kuppel ihres Panzers den Himmel und mit ihrem Plastron die
Erde repräsentiert hätte. Auf der Nordhalbkugel scheinen sich
die Sterne um den Himmelsnordpol zu drehen, die Richtung,
die der Schildkröte bei den vier Symbolen zugeordnet ist.}

\en{What I have never seen mentioned so far, however, is
something that seems quite obvious, namely that the patterns on
what you can see on the shell of the turtle would mimic the
hexagrams for heaven and earth quite nicely, given also that
there is a hinge around the middle of the plastron:}%
\de{Was ich bisher jedoch noch nie erwähnt gesehen habe, ist
etwas, das ziemlich offensichtlich zu sein scheint, nämlich dass
die Muster, die man auf dem Panzer der Schildkröte sehen
kann, die Hexagramme für Himmel und Erde ziemlich gut
nachahmen würden, wenn man bedenkt, dass es auch ein
Gelenk etwa in der Mitte des Plastrons hat:}

\vspace{0.8mm}
\hspace{3mm}
\includegraphics[scale=0.05]{i-turtleverse.jpg}

\en{A turtle shell has essentially two layers, an outer (softer)
keratinous layer above an inner bone layer. The bone layer of
the plastron has 9 scutes (shields), which were interpreted in
various ways for the bone oracles, while the outer keratinous
layer (shown above on the right) has 6 pairs of scutes, anal,
femoral, abdominal, pectoral, humeral and gular.}%
\de{Der Panzer einer WasserSchildkröte besteht im Wesentlichen
aus zwei Schichten, einer äusseren (weicheren) Keratinschicht
über einer inneren Knochenschicht. Die Knochenschicht des
Plastrons hat 9 Schuppen (Schilde), die auf verschiedene
Weisen für Knochenorakel gedeutet wurden, während die äussere
Hornschicht (oben rechts abgebildet) 6 Schuppenpaare hat:
anal, femoral, abdominal, pectoral, humerus und gular.}

\item
\en{Applying heat to a plastron can cause it to crack, to become
broken. Are yin and yang lines as broken (weak) resp.\ unbroken
(strong) lines in the I Ching thus related to more ancient
oracles involving heat\,? Heat dries up, makes brittle, so would
a yang line correspond to no crack emerging, because it was
wet to start with, hence strong by resisting heat\,?}%
\de{Hitze auf ein Plastron anwenden kann dazu führen, dass es
bricht und Risse bekommt. Stehen Yin- und Yang-Linien als
gebrochene (schwache) bzw.\ ungebrochene (starke) Linien im
I Ging also in Beziehung zu älteren Orakeln, die mit Hitze zu
tun hatten\,? Hitze trocknet aus, macht spröde, würde also
eine Yang-Linie einem nicht entstehenden Riss entsprechen,
weil die Yang-Linie zu Beginn feucht war, also stark indem
sie der Hitze widersteht\,?}

\item
\en{See Billy Culver’s \textsl{Energy Language} website, which inspired
me once to reconsider old attempts to arrange elements and
trigrams on a Möbius Strip \moebius{0.013} or an infinity symbol $\infty$ and
whose style influenced the graphics above, but in my feeling
his images carry more potential than that.}%
\de{Siehe Billy Culvers \textsl{Energy Language} (‘Energiesprache’) Website,
die mich einmal dazu inspiriert hatte, alte Versuche,
Elemente und Trigramme auf einem Möbiusband \moebius{0.013} oder einem
Unendlichkeitssymbol $\infty$ anzuordnen, wieder in Betracht zu
ziehen, und deren Stil die obigen Grafiken beeinflusst hat,
aber gefühlt bergen seine Bilder mehr Potenzial als das.}

% page
\item
\en{Is the female fire trigram a form of inner fire, emo mapped to
some form of eri, that is clinging to a dream, an idea, a wish
despite all outer hardness\,? Is the female earth trigram a form
of inner earth, ero mapped to some form of emi, something
that can yield devotely to outer hardness\,? Is the female lake
trigram a form of outer water, emi mapped to some form of
ero, which brings calm to the outside world without hardness\,?
Is the female wind trigram a form of outer air, eri mapped to
some form of emo, free flowing mind and communication\,?}%
\de{Ist das weibliche Feuer-Trigramm eine Form des inneren Feuers,
emo abgebildet auf eine Form von eri, welche sich trotz
aller äusseren Härte an einen Traum, eine Idee, einen Wunsch
klammert\,? Ist das weibliche Erd-Trigramm eine Form von
innerer Erde, ero abgebildet auf eine Form von emi, auf etwas,
das der äusseren Härte hingebungsvoll nachgeben kann\,? Ist
das weibliche See-Trigramm eine Form von äusserem Wasser,
emi abgebildet auf eine Form von ero, welche Ruhe in
die Aussenwelt bringt, ohne Härte\,? Ist das weibliche Wind-Trigramm
eine Form von äusserer Luft, eri abgebildet auf eine
Form von emo, frei fliessender Geist und Kommunikation\,?}

\item
\en{Is the Chinese approach thus more balanced\,? Conversely, is
the Greek approach more likely to start new things, exactly
because it is maybe initially more imbalanced\,? Are both
needed for ‘full’ balance\,? Is there more\,?}%
\de{Ist der chinesische Ansatz also ausgewogener\,? Führt umgekehrt
der griechische Ansatz eher dazu, neue Dinge zu beginnen,
gerade weil er vielleicht anfangs unausgewogener ist\,?
Sind beide für ein ‘vollständiges’ Gleichgewicht erforderlich\,?
Gibt es noch mehr\,?}

\item
\en{In \textsl{Psychologische Typen} (1921), C.\,G.\ Jung combines extra-
and introversion with implicitly the four elements, which he
terms thinking (‘air’), feeling (‘water’), sensation (‘earth’)
and intuition (‘fire’), into 8 psychological types, maybe
already also inspired by the 8 trigrams of the I Ching:}%
\de{In \textsl{Psychologische Typen} (1921) kombiniert C.\,G.\ Jung Extro-
und Introvertiertheit implizit mit den vier Elementen, die er
als Denken (‘Luft’), Fühlen (‘Wasser’), Empfindung (‘Erde’) und
Intuition (‘Feuer’) bezeichnet, zu 8 psychologischen Typen,
möglicherweise auch schon inspiriert von den 8 Trigrammen
des I Ging:}

\en{“I first met Richard Wilhelm […$\!$] in the early twenties. In
1923 we invited him to Zürich […$\!$]. Even before meeting
him I had been interested in Oriental philosophy, and around
[“etwa”] 1920 had begun experimenting with the I Ching.”
(\textsl{Memories, Dreams, Reflections of C.\,G.\,Jung}, Appendix IV,
recorded and edited by Aniela Jaffé, translated from German
by R.\ and C.\ Winston, 1961, German edition 1971)}%
\de{“Richard Wilhelm lernte ich […$\!$] anfangs der zwanziger Jahre
[kennen]. Im Jahre 1923 luden wir ihn nach Zürich ein
[…$\!$]. Schon bevor ich ihn kennenlernte, hatte ich mich mit
östlicher Philosophie beschäftigt und hatte etwa 1920
angefangen, mit dem I Ging zu experimentieren.” (\textsl{Erinnerungen,
Träume, Gedanken von C.\,G.\,Jung}, Appendix, aufgezeichnet
und herausgegeben von Aniela Jaffé, 1971, englisch 1961)}

\en{Jung uses extraverted in the sense of consciously basing one’s
life mostly on what is outside, and introverted as basing it
consciously mostly on the subjective impression of what is
outside, in both cases with unconscious counterreactions to
some degree. He categorizes thinking and feeling as rational
or judging, because they would judge what they perceive in
their field of attention, while sensing and imagination would
rather just accept the impressions as given. Also, ‘subjective’
is given a bit more weight by Jung than others usually do,
since archetypes inside are considered shared by him.}%
\de{Jung verwendet extrovertiert im Sinn von sein Leben bewusst
hauptsächlich auf das zu stützen, was draussen ist, und
introvertiert es bewusst hauptsächlich auf den subjektiven
Eindruck dessen, was draussen ist, zu stützen, in beiden Fällen
mit bis zu einem gewissen Grad unbewussten Gegenreaktionen.
Er kategorisiert Denken und Fühlen als rational oder
urteilend, weil sie beurteilen würden, was sie in ihrem
Aufmerksamkeitsfeld wahrnehmen, während Empfindung und
Vorstellung Eindrücke eher einfach als gegeben hinnehmen würden.
Auch räumt Jung dem ‘Subjektiven’  etwas mehr Gewicht ein
als andere es gewöhnlich tun, da er Archetypen im Inneren
als untereinander geteilt/verbunden erachtet.}

\item
\en{Love and happiness are felt inside, so maybe ideally not too
much focus outside\,? Nor inside\,? But still sometimes\,?
Or simply be with someone with a different perspective\,?}%
\de{Liebe und Glück werden innen gefühlt, also vielleicht
idealerweise nicht zu viel Fokus nach aussen\,? Und auch nicht nach
innen\,? Aber doch manchmal\,? Oder einfach mit jemandem
zusammen sein, der eine andere Perspektive hat\,?}

\end{list}
