\section{\en{origins}\de{ursprünge}}

\en{It is my impression that a notion of first three “elements”
and later ‘1+3’, as opposed to four equitable elements plus
a fifth element, ‘4+1’, as defined by Aristotle, may have
been subliminally prominent throughout the ages long
before recorded history and Greek philosophers.}%
\de{Mein Eindruck ist, dass eine Vorstellung von erst drei
“Elementen” und später ‘1+3’, im Kontrast zu vier gleichwertigen
Elementen plus einem fünften Element, ‘4+1’, wie von
Aristoteles definiert, unterschwellig schon lange vor der
Geschichtsschreibung und den griechischen Philosophen durch
die Zeitalter hindurch präsent gewesen sein könnte.}

\en{In a nutshell, ‘1’ would have become what makes things
move, energy, fire, and ‘3’ would have evolved to air-water-earth,
the states of matter gas-liquid-solid. Colors would
have been implicitly the light \green{green} of catkins and explicitly
\white{white}\hspace{0.45mm}–\hspace{0.15mm}\red{red}\hspace{0.15mm}–\hspace{0.15mm}\black{black}
in the order of a ripening mulberry, with
roots in prehistoric cults around fire, and likely the moon,
both as single creatress and in trinity, world-wide.}%
\de{Kurz gesagt wäre ‘1’ zu dem geworden, was die Dinge
in Bewegung setzt, Energie, Feuer, und ‘3’ hätte sich zu
Luft-Wasser-Erde entwickelt, den Aggregatzuständen
gas-flüssig-fest. Farben wären implizit das helle \green{Grün}
von Kätz\-chen und explizit
\white{Weiss}\hspace{0.45mm}–\hspace{0.15mm}\red{Rot}\hspace{0.15mm}–\hspace{0.15mm}\black{Schwarz}
in der Reihenfolge
einer reifenden Maulbeere gewesen, mit Wurzeln in prä\-his\-to\-ri\-schen
Kulten um das Feuer und wahrscheinlich um den
Mond herum, sowohl als die einzige Schöpfungsgöttin als
auch in Dreieinigkeit, weltweit.}

\vspace{-1.7mm}
\begin{center}
\includegraphics[scale=0.02]{i-tetrahedron.jpg}
\end{center}
\vspace{-2.3mm}

\en{First named colors in virtually all languages were
white\hspace{0.15mm}-\hspace{0.15mm}red\hspace{0.15mm}-\hspace{0.15mm}black
as the colors of fire (light): \black{black} as dark, \white{white} as
bright, and \red{red} as the colors of fire from flame to embers,
yellow to red. Fire, humanity’s first major discovery, would
have initially been preserved in a raised mound of ashes
(white) around a core of glowing coal (red around black).
There would also have been cults around this, most likely
a universal “white” moon/fire creatress/goddess. In ancient
Greece sacrifices were given into fire and the first sacrifice
always given to Hestia, the goddess of the hearth.}%
\de{Die ersten benannten Farben in praktisch allen Sprachen waren
weiss\hspace{0.15mm}-\hspace{0.15mm}rot\hspace{0.15mm}-\hspace{0.15mm}schwarz,
als die Farben des Feuers (Licht):
\black{schwarz} als dunkel, \white{weiss} als hell, und \red{rot} als die
Farben des Feuers von Flamme bis Glut, gelb bis rot.
Das Feuer, die erste grosse Entdeckung der Menschheit,
wäre anfangs in einem aufgehäuften Aschehügel (weiss) um
einen Kern aus glühender Kohle (rot um schwarz)
aufbewahrt worden. Es hätte wohl auch Kulte um das herum
gegeben, am wahrscheinlichsten um eine universelle “weisse”
Mond-/Feuer-Schöpferin/-Göttin. Im alten Griechenland
wurden Opfer ins Feuer gegeben, und das erste Opfer
wurde immer Hestia, der Göttin des Herdes, dargebracht.}

\vspace{-0.8mm}
\begin{center}
\includegraphics[scale=0.015]{i-hestia.jpg}
\end{center}
\vspace{-1.3mm}

% page
\en{One of the earliest Indian Upanishads, the Chandogya
Upanishad, which dates back to at least around 700 BCE,
relates these three colors to “elements”:
\red{red}\hspace{0.15mm}-\hspace{0.15mm}fire,
\white{white}\hspace{0.45mm}-\hspace{0.15mm}water
and \black{black}\hspace{0.15mm}-\hspace{0.15mm}earth,
probably also since water is more
transparent (and hence “brighter”) and ashes more “fluid”
than earth resp.\ coal. It appears that at some point red
became associated with air instead and the goddess came
to represent fire and moon as the ruler of a trinity of
air-water-earth or sky, sea and underworld. Colors assigned to
elements by Antiochus of Athens around the second century
CE were accordingly
\yellow{yellow}\hspace{0.15mm}-\hspace{0.15mm}fire,
\white{white}\hspace{0.45mm}-\hspace{0.15mm}water,
\red{red}\hspace{0.15mm}-\hspace{0.15mm}air and
\black{black}\hspace{0.15mm}-\hspace{0.15mm}earth,
and at least today’s symbols are triangles.}%
\de{Eine der frühesten indischen Upanishaden, die Chandogya
Upanishad, die mindestens auf das Jahr 700 v.\,Chr.\ 
zurückgeht, bezieht diese drei Farben auf “Elemente”:
\red{rot}\hspace{0.15mm}-\hspace{0.15mm}Feuer,
\white{weiss}\hspace{0.45mm}-\hspace{0.15mm}Wasser
und \black{schwarz}\hspace{0.15mm}-\hspace{0.15mm}Erde,
wahrscheinlich auch, weil Wasser transparenter (und damit “heller”) und
Asche “flüssiger” ist als Erde bzw.\ Kohle. Es scheint, dass
von einem gewissen Zeitpunkt an Rot stattdessen mit Luft
assoziiert wurde, und die Göttin wäre zur Repräsentantin
von Feuer und Mond als Herrscherin einer Dreifaltigkeit
von Luft-Wasser-Erde oder Himmel, Meer und Unterwelt
geworden. Die Farben, die Antiochus von Athen um das
zweite Jahrhundert n.\,Chr.\ den Elementen zuordnete,
waren entsprechend
\yellow{gelb}\hspace{0.15mm}-\hspace{0.15mm}Feuer,
\white{weiss}\hspace{0.45mm}-\hspace{0.15mm}Wasser,
\red{rot}\hspace{0.15mm}-\hspace{0.15mm}Luft und
\black{schwarz}\hspace{0.15mm}-\hspace{0.15mm}Erde,
und zumindest die heutigen Symbole sind Dreiecke.}

\vspace{-0.2mm}
\begin{center}
\includegraphics[scale=0.025]{i-pyramids-air.jpg}
\end{center}
\vspace{-1.7mm}

\en{Contrary to Aristotle’s model, which is a priori based on
touchable properties in the outer world, the present model
involves also things inside the mind. In the outer world, emo
and ero could be mapped via fire/earth to “energy/matter”
with matter split up into its 3 main states solid-liquid-gas.}%
\de{Im Gegensatz zu Aristoteles’ Modell, das a priori auf
berührbaren Eigenschaften in der äusseren Welt beruht,
bezieht das vorliegende Modell auch Dinge im Inneren des
Geistes mit ein. In der äusseren Welt könnten emo und ero
über Feuer/Erde auf “Energie/Materie” abgebildet werden
mit Materie aufgeteilt in ihre 3 Haupt-Aggregatzustände
fest-flüssig-gasförmig.}

\en{Maybe inside, since ‘states’ suggests resting, emi could
be ‘1’ and eri could be split into 3 ‘states of mind’, yielding
8 ‘elements’, similar to the trigrams of the I Ching\,?}%
\de{Könnte vielleicht im Inneren emi, da ‘Zustände’ ruhen
suggeriert, ‘1’ sein und könnte vielleicht eri in 3 ‘Geisteszustände’
aufgeteilt werden, woraus sich 8 ‘Elemente’
ergeben würden, ähnlich den Trigrammen des I Ging\,?}

\en{See the section \textsl{psyche} for a tentative explanation of
why in culture first just three “proto-elements”, \black{black} earth
(ero), \white{white} water (emi) and \red{red} fire (eri+emo).}%
\de{Siehe die Sektion \textsl{psyche} für eine mögliche Erklärung,
warum es in der Kultur zunächst nur drei “Proto-Elemente”,
\black{schwarze} Erde (ero), \white{weisses} Wasser (emi) und \red{rotes}
Feuer (eri+emo) gegeben hätte.}

\subsection{\en{leads}\de{fährten}}

\small
\begin{list}{$\bullet$}{\setlength{\leftmargin}{10pt}}

\item
\en{Could three main states of matter outside be derived from
the model or would that have to be added via experimental
facts, like the observation that freedom inside seems larger
than outside, which lead to passive/active, soft/hard, etc.\,?
And similarly for three states of mind inside\,?}%
\de{Könnten drei Haupt-Aggregatszustände der Materie aussen
aus dem Modell abgeleitet werden, oder müssten diese über
experimentelle Fakten hinzugefügt werden, wie die Beobachtung,
dass die Freiheit innen grösser zu sein scheint als aussen,
was zu passiv/aktiv, weich/hart usw.\ führte\,? Und in ähnlicher
Weise für drei Geisteszustände im Inneren\,?}

\en{Maybe mirror elements in$\leftrightarrow$out, plus somehow naturally group
what has been mirrored to ‘states’ with ero/eri\,?}%
\de{Vielleicht die Elemente innen$\leftrightarrow$aussen spiegeln, und irgendwie
das, was gespiegelt wurde, natürlich mit ero/eri zu ‘Zustän\-den’
gruppieren\,?}

\en{Outside fire practically never rests, while solids, liquids and
gases can rest in their appearance. Maybe inside emi could
be linked to what practically never rests\,?}%
\de{Aussen ruht Feuer praktisch nie, wohingegen Festkörper,
Flüs\-sig\-kei\-ten und Gase augenscheinlich ruhen können. Könnte
vielleicht innen emi mit dem verbunden werden, was praktisch
niemals ruht\,?}

\en{Outside solid-liquid-gas have increasing degrees of freedom.
Maybe inside eri could be split up similarly, like from memory
via various kinds of structured thoughts to free imagination\,?
Outside “fire/earth-water-air”, inside “water/earth-air-fire”\,?}%
\de{Aussen haben fest-flüssig-gasförmig zunehmende Freiheitsgrade.
Könnte vielleicht eri ähnlich aufgeteilt werden,
vielleicht von Erinnerungen via verschiedenen Arten von
strukturierten Gedanken zu freier Vorstellung\,? Aussen
“Feuer/Erde-Wasser-Luft”, innen “Wasser/Erde-Luft-Feuer”\,?}

\en{Inside many things can be recalled at will, but feelings usually
only indirectly by recalling the circumstances under which they
previously occurred. Conscious thinking as well as changing
things outside can calm feelings.}%
\de{Innen können viele Dinge willentlich abgerufen werden, aber 
Gefühle normalerweise nur indirekt, indem man die Umstände
abruft, unter welchen die Gefühle zuvor auftraten. Bewusstes
Denken sowie aussen etwas ändern kann Gefühle beruhigen.}

\item
\en{In the I Ching, poles of ‘1+3’ are father-mother resp.\ heaven-earth
plus their 3 daughters and 3 sons. Similarly, Cronos and
Rhea had 3 daughters and 3 sons, and their parents Ouranos
and Gaia were also heaven (or mountain) and earth.}%
\de{Im I Ging sind die Pole von ‘1+3’ Vater-Mutter bzw.\ Himmel-Erde
plus ihre 3 Töchter und 3 Söhne. Ähnlich hatten Kronos
und Rhea 3 Töchter und 3 Söhne, und ihre Eltern Ouranos
und Gaia waren ebenfalls Himmel (oder Berg) und Erde.}

% page
\item
\en{Mother Mary could be seen as water, plus a male trinity of air
(father-fire, son-earth, spirit-air\,?). And any cultural creation
thus maybe as trifold air inspired by a water muse, like in the
\textsl{Odyssey}, in essence the (male) “human condition”\,?}%
\de{Die Jungfrau Maria könnte als Wasser gesehen werden, plus
einer männlichen Luft-Trinität (Vater-Feuer, Sohn-Erde, Geist-Luft\,?).
Und jede kulturelle Schöpfung könnte somit als dreifaltige
Luft, inspiriert von einer Wassermuse, gesehen werden,
wie in der \textsl{Odyssee}, im Wesentlichen das, was auf Englisch
als (männliche)\,“human condition” bezeichnet wird\,?}

\item
\en{If you toss four coins, there is a 50\% chance to get ‘3+1’,
a 37.5\% chance to get ‘2+2’ and only 12.5\% to get ‘4+0’.
Even if coins are skew, ‘3+1’ is always more probable than
‘2+2’, and ‘4+0’ only becomes the most probable result once
they are about 1\,:\,4 skew. Thus, whenever there are 4 things
in nature, chances are a priori high that they come as ‘3+1’.}%
\de{Wenn man vier Münzen wirft, besteht eine 50\%\,ige Chance,
‘3+1’ zu erhalten, eine 37.5\%\,ige Chance, ‘2+2’ zu erhalten
und nur 12,5\%, ‘4+0’ zu erhalten. Selbst wenn die Münzen
statistisch schief sind, ist ‘3+1’ immer wahrscheinlicher als
‘2+2’, und ‘4+0’ wird erst dann zum wahrscheinlichsten
Ergebnis, wenn sie etwa 1\,:\,4 schief sind. Also wann immer es
4 Dinge in der Natur gibt, ist die Wahrscheinlichkeit a priori
hoch, dass sie als ‘3+1’ auftauchen.}

\vspace{1.2mm}
\hspace{24mm}
\includegraphics[scale=0.17]{i-coins.png}
\vspace{1.2mm}

\item
\en{Time and space come as ‘1+3’. Their homogeneity implies
preservation of energy and momentum. The isotropy of space
implies preservation of angular momentum, while time is not
invariant to reversal, as entropy never decreases.}%
\de{Zeit und Raum erscheinen als ‘1+3’. Ihre Homogenität impliziert
die Erhaltung von Energie und Impuls. Die Isotropie des
Raumes impliziert die Erhaltung des Drehimpulses, während
die Zeit nicht invariant gegenüber einer Umkehrung ist, da
die Entropie niemals abnimmt.}

\en{The 4 forces of nature known to date (at everyday energies),
electromagnetic, strong, weak and gravitational forces, come
as ‘3+1’, since gravitation is the only one closely intertwined
with spacetime and without quantum theory.}%
\de{Die vier bisher bekannten Naturkräfte (bei Alltagsenergien),
die elektromagnetische, die starke, die schwache und die
Gravitationskraft, erscheinen als ‘3+1’, da die Gravitation die
einzige ist, die eng mit der Raumzeit verwoben ist und keine
Quantentheorie besitzt.}

\item
\en{Does space rest as it ‘is there’ and time move as it ‘flows’\,?
Might 3 main states of matter outside (ero) and 3 dimensions
of space (rest outside, ero) thus both be fundamentally tied
to the experience of conscious life\,? And similarly inside\,?}%
\de{Ruht Raum, weil er ‘da ist’, bewegt sich Zeit, weil sie ‘fliesst’\,?
Könnten die 3 Haupt-Aggregatzustände der Materie aussen
(ero) und die 3 Raumdimensionen (ruhen aussen, ero) also
beide fundamental mit dem Erlebnis bewussten Lebens
verbunden sein\,? Und in ähnlicher Weise innen\,?}

\en{Definitely surreal, maybe even going in circles, and yet\,…}%
\de{Definitiv surreal, vielleicht sich im Kreis drehend, und doch\,…}

\en{I had this idea 12 July 2023 around 09:10 in Wollishofen while
driving to work in the morning, coming from Leimbach and
turning left into the main street with the tramline.}%
\de{Ich hatte diese Idee am 12.\ Juli 2023 um etwa 09:10 in
Wollishofen, als ich morgens zur Arbeit fuhr und von Leimbach
kommend links in die Hauptstrasse mit der Tramlinie einbog.}

\item
\en{In \textsl{The Animate and the Inanimate} (1925), William James
Sidis observes that inanimate processes can appear alive if
time is reversed, using the example of drops of mercury that
flow together on a metal surface and then amalgamate with
the surface. Reversed in time, drops of mercury would appear
to grow out of the metal surface and divide like living cells.}%
\de{In \textsl{The Animate and the Inanimate}
(\textsl{Das Belebte und das Unbelebte},
1925) beobachtet William James Sidis, dass unbelebte
Vorgänge lebendig erscheinen können, wenn die Zeit umgekehrt
wird, und verwendet dazu das Beispiel von Quecksilbertropfen,
die auf einer Metalloberfläche zusammenfliessen und
dann mit der Oberfläche verschmelzen. Zeitlich umgekehrt,
würden die Quecksilbertropfen scheinbar aus der Metalloberfläche
herauswachsen und sich wie lebende Zellen teilen.}

\item
\en{A trinity is both 3 parts and 1 unit, so 3 turns almost
automatically to 3+1, and 4 to 4+1, or a couple+baby, 2+1.}%
\de{Eine Dreifaltigkeit besteht aus 3 Teilen und einer Einheit, also
wird aus 3 fast automatisch 3+1 und aus 4 wird 4+1, oder
ein Paar+Baby wird 2+1.}

\en{See also the pythagorean tetractys further below. The image
for the universe may have developed from a hill via
tetrahedron/pyramid to the dodecahedron in Plato’s \textsl{Timaeus}, with
increasing focus on the number 5 related to Venus, due to
the 5 stations of Venus on a 5-pointed star, which is again
related to the golden ratio, to harmony, beauty, roundness.}%
\de{Siehe auch die pythagoreische Tetraktys weiter unten. Das
Bild für das Universum könnte sich von einem Hügel via
Tetraeder/Pyramide zum Dodekaeder in Platons \textsl{Timaios}
entwickelt haben, mit zunehmendem Fokus auf die Zahl 5, die
mit der Venus in Verbindung steht, aufgrund der fünf Stationen
der Venus auf einem fünfzackigen Stern, der wiederum
mit dem Goldenen Schnitt, mit Harmonie, Schönheit, Rundheit
verbunden ist.}

% page
\item
\en{Before writing, myths were only preserved if people kept
remembering and retelling them to younger generations. Thus
only stories people really cared about survived. This does,
however, not imply that they necessarily consciously understood
myths analytically. In a way, myths are sort of informal
laws of nature, condense all kinds of experiences into a story.
Exploring such unconscious or even intentionally veiled legacy
spans ages, is still unfolding, even after Freud and Jung.}%
\de{Bevor es die Schrift gab, blieben Mythen nur erhalten, wenn
sich die Menschen an sie erinnerten und sie jüngeren
Generationen weiter erzählten. So überlebten nur Geschichten, die
den Menschen wirklich am Herzen lagen. Das bedeutet jedoch
nicht, dass sie Mythen unbedingt bewusst und analytisch
verstanden hatten. In gewisser Weise sind Mythen eine Art
informelle Naturgesetze, die alle möglichen Erfahrungen in eine
Geschichte verdichten. Die Erforschung solcher unbewussten
oder sogar absichtlich verschleierten Hinterlassenschaften
erstreckt sich über Zeitalter und ist auch nach Freud und Jung
noch nicht abgeschlossen.}

\en{People living in ancient cultures were not wise just from that
connotation of ‘old’; they were rather young and fresh
compared to us who can look back so far into history. But they
were also still closer to the ‘source’ and knew things that
got lost or were maybe never explicitly written down. Some
things were also truly archaic then, simpler and more brutal;
see Homer’s Iliad and Odyssey, for example.}%
\de{Die Menschen, die in alten Kulturen lebten, waren nicht
wegen dieser Assoziation zu ‘alt’ einfach weise; sie waren eher
jung und frisch im Vergleich zu uns, die wir so weit in die
Geschichte zurückblicken können. Aber sie waren auch noch
näher an der ‘Quelle’ und wussten noch Dinge, die seither
verloren gegangen sind oder vielleicht nie ausdrücklich
niedergeschrieben wurden. Manche Dinge waren damals auch
wirklich archaisch, einfacher und brutaler; siehe zum Beispiel
Homers Ilias und Odyssee.}

\en{In life, situations shortly after birth and close to death are
often similar, so early answers might be closer to late ones than
expected, even without cyclic ‘rebirth’ or ‘immortal souls’.}%
\de{Im Leben sind sich Situationen kurz nach der Geburt und kurz
vor dem Tod oft ähnlich, so dass die frühen Antworten den
späten näher sein könnten als erwartet, auch ohne zyklische
‘Wiedergeburt’ oder ‘unsterbliche Seelen’.}

\item
\en{The complex of ‘1+3’ and basic colors is very rich and beautiful,
but also a ‘can of worms’ and only partially fits here, so
just some gist below, and see my article “\white{White}-\red{red}-\black{black}
and the \green{“green”} goddess” under \cometartemis\ for more.}%
\de{Der Komplex mit ‘1+3’ und den Grundfarben ist sehr
reichhaltig und schön, aber auch ein ‘Fass ohne Boden’ und passt
nur zum Teil hierher, daher hier nur paar wesentliche Punkte,
und siehe meinen Artikel
“\white{White}-\red{red}-\black{black} and the \green{“green”} goddess”
(“Weiss-rot-schwarz und die grüne Göttin”) unter \cometartemis\ für mehr.}

\item
\en{One of the oldest ancient Indian Upanishads, the Chandogya
Upanishad (around 700 BCE), speaks of three colors of fire:
fire-red, water-white and earth-black.}%
\de{Eine der ältesten altindischen Upanishaden, die Chandogya
Upanishad (um 700 v.\,Chr.), spricht von drei Farben des
Feuers: feuer-rot, wasser-weiss und erde-schwarz.}

\en{“The red colour of [gross] fire is the colour of [the original]
fire; the white colour of [gross] fire is the colour of [the original]
water; the black colour of [gross] fire is the colour of [the
original] earth. Thus vanishes from fire what is commonly
called fire, the modification being only a name, arising from
speech, while the three colours (forms) alone are true.”
(6.4.1, translated by Swami Nikhilananda)}%
\de{“Die rote Farbe des [groben] Feuers ist die Farbe des
[ursprünglichen] Feuers; die weisse Farbe des [groben] Feuers ist
die Farbe des [ursprünglichen] Wassers; die schwarze Farbe
des [groben] Feuers ist die Farbe der [ursprünglichen] Erde.
So verschwindet vom Feuer das, was gemeinhin als Feuer
bezeichnet wird, wobei die Veränderung nur ein Name ist, der
aus der Sprache entsteht, während die drei Farben (Formen)
allein wahr sind.” (6.4.1, übersetzt auf engl.\ von Swami Nikhilananda)}

\en{These three colors, which appear as first colors in apparently
all earliest cultures able to write them down, represent most
likely a more archaic concept of color as light/fire, as follows.}%
\de{Diese drei Farben, die anscheinend in allen frühesten Kulturen,
die in der Lage waren, sie aufzuschreiben, als erste Farben
auftauchen, repräsentieren höchstwahrscheinlich ein eher
archaisches Konzept von Farbe als Licht/Feuer, wie folgt.}

\en{Without light no colors; fire produces light; so color would be
heavily related to light; thus the basic opposites white (bright)
and black (dark), plus the color(s) of fire, red-orange-yellow.
Water is transparent, earth is often intransparent, ashes are
more “fluid” than coal, hence water-white and earth-black.}%
\de{Ohne Licht keine Farben; Feuer erzeugt Licht; also wäre
Farbe stark mit Licht verbunden; daher die grundlegenden
Gegensätze Weiss (hell) und Schwarz (dunkel), plus die
Farbe(n) des Feuers, rot-orange-gelb. Wasser ist durchsichtig,
Erde ist oft undurchsichtig, Asche ist “flüssiger” als Kohle,
daher Wasser-weiss und Erde-schwarz.}

\en{In ancient Greek, the words for black/white, mélas/leukós,
still had, maybe even primarily, the connotation of dark/bright;
the word for red, pyrrós, literally says color of fire.}%
\de{Im Altgriechischen hatten die Wörter für schwarz/weiss, mélas /leukós,
noch, vielleicht sogar in erster Linie, die Bedeutung
von dunkel/hell; das Wort für rot, pyrrós, bedeutet wörtlich
die Farbe des Feuers.}

\en{In other words, no fire would have been black, lighting it red,
and fire/light would have saturated at white.}%
\de{Mit anderen Worten, kein Feuer wäre schwarz gewesen, es
anzünden rot, und Feuer/Licht würde bei weiss gesättigt.}

% page
\item
\en{The first 3 of the 4 riders of the apocalypse have the colors
white-red-black. The color of the fourth is chlōrós in ancient
Greek, thus related to chlorophyll, the substance that makes
leaves green. Colorwise, it was most likely a pale green/yellow
color, like new shoots of plants or also the color of a corpse.}%
\de{Die ersten 3 der 4 Reiter der Apokalypse haben die Farben
weiss-rot-schwarz. Die Farbe des vierten ist chlōrós im
Altgriechischen, also verwandt mit Chlorophyll, der Substanz, die
Blätter grün macht. Farblich war es wahrscheinlich ein blasses
Grün/Gelb, wie neue Pflanzentriebe oder auch die Farbe
eines Leichnams.}

\en{In the fairy tale around Baba Yaga, three riders appear, white-day
at dawn, red-sun when the sun rises, black-night when it
gets dark. They are all explicitly servants of Baba Yaga, who
also has three pairs of helping hands, which identify her as the
triple moon goddess Hecate-Artemis, who is both a goddess
of death and of birth, acting also as midwife in mythology.}%
\de{Im Märchen um Baba Yaga tauchen drei Reiter auf, weiss-Tag
bei Tagesanbruch, rot-Sonne wenn die Sonne aufgeht,
schwarz-Nacht wenn es dunkel wird. Sie alle sind ausdrücklich
Diener von Baba Yaga, die auch drei Paar helfende Hände
hat, was sie als die dreifache Mondgöttin Hekate-Artemis
ausweist, die sowohl eine Göttin des Todes als auch der Geburt
ist und in der Mythologie auch als Hebamme fungiert.}

\en{The idea behind this would be that the moon would be the
ruling light in the sky because it alone can appear both at
day and night, and can even shadow the sun during a total
solar eclipse. In folklore, Baba Yaga’s house is mobile, stands
on chicken legs, the rooster being again a symbol of fire.}%
\de{Die Idee dahinter ist, dass der Mond das herrschende Licht am
Himmel ist, weil er allein sowohl am Tag als auch in der Nacht
erscheinen kann und sogar die Sonne während einer totalen
Sonnenfinsternis verdunkeln kann. In der Folklore ist das Haus
der Baba Yaga beweglich und steht auf Hühnerbeinen, wobei
der Hahn wiederum ein Symbol für Feuer ist.}

\item
\en{The Red Sparrow in Bukowski’s \textsl{Pulp} reminds of that, of fire
in the outside world that would be the spark of life and the
physical end of it, in its three states of matter (versus inside
as already suggested a triune airy poet plus water muse)\,?}%
\de{Der rote Spatz in Bukowskis \textsl{Pulp} erinnert daran, an das Feuer
in der Aussenwelt, das der Funke des Lebens und das physische
Ende desselben wäre, in seinen drei Zuständen der Materie
(im Gegensatz zu innen, mit, wie bereits suggeriert, einem
dreieinig-luftigen Dichter plus Wassermuse)\,?}

\item
\en{Near the end of Apuleius’ \textsl{The Golden Ass} (around 150 CE),
Apuleius encounters the goddess Isis at full moon at the sea
shortly after moonrise:}%
\de{Gegen das Ende von Apuleius’ \textsl{Der goldene Esel} (um 150
n.\,Chr.) begegnet Apuleius der Göttin Isis bei Vollmond am
Meer kurz nach Mondaufgang:}

\en{“Her many-coloured robe was of finest linen; part was glistening
white, part crocus-yellow, part glowing red and along the
entire hem a woven bordure of flowers and fruit clung swaying
in the breeze. But what caught and held my eye more
than anything else was the deep black lustre of her mantle.
[…$\!$] It was embroidered with glittering stars on the hem
and everywhere else, and in the middle beamed a full and fiery
moon.” (Chapter 17, translated by Robert Graves)}%
\de{“Ihr vielfarbiges Gewand war aus feinstem Leinen; ein Teil war
gleissend weiss, ein Teil krokusgelb, ein Teil leuchtend rot, und
um den ganzen Saum hing eine gewebte Bordüre aus Blumen
und Früchten, die sich im Winde wiegte. Doch was mir mehr
als alles andere ins Auge fiel, war der tiefschwarze Schimmer
ihres Umhangs. […$\!$] Er war am Saum und überall sonst mit
glitzernden Sternen bestickt, und in der Mitte strahlte ein
voller, feuriger Mond.” (Kapitel 17, übersetzt auf engl.\ von
Robert Graves)}

\en{Shortly afterwards she describes herself:}%
\de{Kurz darauf beschreibt sie sich selbst:}

\en{“[…$\!$] mother of nature, encompassing mistress of elements,
first progeny of times, highest power/deity/queen, queen of
the dead, first/best (sky) deity, uniform face of gods and
goddesses, whose heavenly shining summits, salty sea breezes
[and] the dead down below in earth, silently weeped, are still
ruled by me. A single/unique goddess in multiple guises, with
changing rites, many names, worshipped all over the world.”
(translated by me)}%
\de{“[…$\!$] Mutter der Natur, umfassende Herrin der Elemente,
er\-ste Nachkommenschaft der Zeiten, höchste Macht/Gottheit/ Königin,
Königin der Toten, erste/beste (Himmels-)Gottheit,
einheitliches Gesicht der Götter und Göttinnen, deren himmlisch
leuchtende Gipfel, salzige Meeresbrisen [und] die Toten
unten in der Erde, schweigend weinend, noch immer von mir
regiert werden. Eine einzige/einzigartige Göttin in vielfältigen
Gestalten, mit wechselnden Riten, vielen Namen, die überall
auf der Welt verehrt wird.” (übersetzt von mir)}

\en{Note that she may be saying that she rules over heaven, sea
and earth, as in Zeus, Poseidon and Hades, hence a trinity of
air-water-earth, which would make her potentially fire.}%
\de{Es könnte sein, dass sie damit sagen würde, dass sie über
Himmel, Meer und Erde herrscht, wie Zeus, Poseidon und
Hades, also eine Dreifaltigkeit von Luft-Wasser-Erde, was sie
potenziell zum Feuer machen würde.}

\en{Astrologer Antiochus of Athens and physician Galenus of
Pergamon attributed colors resp.\ body fluids (humors) to
elements around the time Apuleius lived, based on older roots
going back at least partially to Hippocrates: white to water
(phlegm, phlegmatic), black to earth (black bile, melancholic),
yellow to fire (yellow bile, choleric) and red to air
(blood, sanguine), the colors of Isis’ dress above, plus stars
and moon for the round fifth element in the sky.}%
\de{Der Astrologe Antiochus von Athen und der Arzt Galenus
von Pergamon ordneten um die Zeit des Apuleius Farben
bzw.\ Körpersäfte den Elementen zu, wobei sie sich auf ältere
Wurzeln stützten, die zumindest teilweise auf Hippokrates
zurückgehen: Weiss dem Wasser (Schleim, phlegmatisch),
Schwarz der Erde (schwarze Galle, melancholisch), Gelb dem
Feuer (gelbe Galle, cholerisch) und Rot der Luft (Blut,
sanguinisch), die Farben des obigen Kleides der Isis, plus Sterne
und Mond für das runde fünfte Element am Himmel.}

% page
\en{This suggests overall that maybe at some point in time air
took the place of fire in the fire trinity as in the Chandogya
Upanishad, maybe via breath as a mixture of air and fire, as
in pneuma, or maybe Indian Aum (Om), plus maybe water.}%
\de{Dies deutet insgesamt darauf hin, dass vielleicht irgendwann
einmal die Luft den Platz des Feuers in der Feuertrinität
eingenommen hat, wie in der Chandogya Upanishad, vielleicht
über den Atem als eine Mischung aus Luft und Feuer, wie
in Pneuma, oder vielleicht wie das indische Aum (Om), plus
vielleicht Wasser.}

\vspace{-0.5mm}
{\small\begin{center}
\begin{tabular}{|c|c|c|l|c|c|c|} \cline{1-3} \cline{5-7}
\en{
$\!$\textbf{\green{“green”}}$\!$ & moon & (rules) & $\!\!\!\!\!$ & “energy” & fire & \textbf{\yellow{yellow}} \\ \cline{1-3} \cline{5-7}
\textbf{\white{white}} & day & water & $\!\!\!\!\!$ & liquid & water & \textbf{\white{white}} \\ \cline{1-3} \cline{5-7}
\textbf{\red{red}} & sun & fire & $\!\!\!\!\!$ & gas & air & \textbf{\red{red}} \\ \cline{1-3} \cline{5-7}
\textbf{\black{black}} & night & earth & $\!\!\!\!\!$ & solid & earth & \textbf{\black{black}} \\ \cline{1-3} \cline{5-7}
}
\de{
$\!\!$$\!$\textbf{\green{“grün”}}$\!$$\!\!$ & $\!$mond$\!$ & $\!\!$(herrscht)$\!\!$ & $\!\!\!\!\!$ & $\!\!$“energie”$\!\!$ & $\!\!$feuer$\!\!$ & $\!\!$\textbf{\yellow{gelb}}$\!\!$ \\ \cline{1-3} \cline{5-7}
$\!\!$\textbf{\white{weiss}}$\!\!$ & $\!$tag$\!$ & $\!\!$wasser$\!\!$ & $\!\!\!\!\!$ & $\!\!$flüssig$\!\!$ & $\!\!$wasser$\!\!$ & $\!\!$\textbf{\white{weiss}}$\!\!$ \\ \cline{1-3} \cline{5-7}
$\!\!$\textbf{\red{rot}}$\!\!$ & $\!$sonne$\!$ & $\!\!$feuer$\!\!$ & $\!\!\!\!\!$ & $\!\!$gas$\!\!$ & $\!\!$luft$\!\!$ & $\!\!$\textbf{\red{rot}}$\!\!$ \\ \cline{1-3} \cline{5-7}
$\!\!$\textbf{\black{schwarz}}$\!\!$ & $\!$nacht$\!$ & $\!\!$erde$\!\!$ & $\!\!\!\!\!$ & $\!\!$fest$\!\!$ & $\!\!$erde$\!\!$ & $\!\!$\textbf{\black{schwarz}}$\!\!$ \\ \cline{1-3} \cline{5-7}
}
\end{tabular}
\end{center}}

\item
\en{In alchemy, also since about at least the time Apuleius lived,
the transition of materials toward the philosopher’s stone was
believed to be black-white-yellow-red, i.e.\ earth-water-fire-air,
which is roughly in order of lightness of the elements and
their relatively layered appearance on earth. It is apparently
also the order of elements in the four tasks that Venus gives
Psyche in \textsl{The Golden Ass}. All of this has ancient Egyptian
roots, with Osiris, Isis, Horus, Seth, Nephthys, etc., as well as
with ancient crafts of creating fake noble metals and gems.}%
\de{In der Alchemie, auch etwa seit mindestens etwa der Zeit, in
der Apuleius lebte, glaubte man, dass der Wandel der
Materialien zum Stein der Weisen schwarz-weiss-gelb-rot wäre,
d.h.\ Erde-Wasser-Feuer-Luft, was ungefähr der Reihenfolge
der Leichtigkeit der Elemente und ihrer relativ geschichteten
Erscheinung auf der Erde entspricht. Es ist offenbar auch die
Reihenfolge der Elemente in den vier Aufgaben, die Venus
Psyche in \textsl{Der Goldene Esel} gibt. All dies hat altägyptische
Wurzeln, mit Osiris, Isis, Horus, Seth, Nephthys usw., sowie
mit dem alten Handwerk der Herstellung von falschen
Edelmetallen und Edelsteinen.}

\item
\en{Fire must have made a great impression on humanity, as it
allowed to keep warm and have light at night, to grill, cook
and bake food, eventually to bake pottery and to forge metals.
It has even been speculated that easier to digest grilled meat
allowed humans to grow larger brains. At first presumably
people did not know how to make fire themselves, so trees
that were known or believed to attract lightning might have
been sacred. As lightning comes from the sky, the “fires” in
the sky, i.e.\ sun, moon, planets and stars, would have been
identified with deities in the sky that give fire. Hence the
main deity would have been in the sky, most likely the moon.
The moon can be round like fruits and berries, but also slim
and pointy like leaves, and it can grow from the shape of a
“catkin” to the round one of a ripe fruit. Attributes of such
a deity may thus have been the fruits ripening on such sacred
trees in the colors of fire, like mulberries, or similar.}%
\de{Feuer muss die Menschheit sehr beeindruckt haben, denn
es ermöglichte es, sich warm zu halten und nachts Licht zu
haben, Essen zu grillen, zu kochen und zu backen, mit der
Zeit Töpferwaren zu brennen und Metalle zu schmieden. Es
wurde sogar spekuliert, dass das leichter verdauliche gegrillte
Fleisch es den Menschen ermöglichte, grössere Gehirne zu
entwickeln. Vermutlich wussten die Menschen zunächst nicht,
wie sie selbst Feuer machen konnten, so dass Bäume, von
denen man wusste oder glaubte, dass sie Blitze anziehen,
heilig gewesen sein könnten. Da der Blitz vom Himmel kommt,
wären die “Feuer” am Himmel, d.h.\ Sonne, Mond, Planeten
und Sterne, mit feuerspendenden Himmelsgottheiten identifiziert.
Daher wäre die Hauptgottheit am Himmel gewesen,
höchstwahrscheinlich der Mond. Der Mond kann rund sein
wie Früchte und Beeren, aber auch schlank und spitz wie
Blätter, und er kann von der Form eines “Kätzchens” bis zur
runden Form einer reifen Frucht wachsen. Attribute einer
solchen Gottheit könnten also die Früchte gewesen sein, die an
solchen heiligen Bäumen in den Farben des Feuers reiften,
wie Maulbeeren oder ähnliche.}

\en{Anything in nature that was not white-red-black would have
been unnamed first: green, blue, brown, pale colors like the
moon, gleaming colors; often colors that signal something
that is not crucial for survival, neither food nor danger. This
could explain why green only entered languages late, despite
being so predominant in nature. Shapes and colors of fruits
may have adapted to preferences of its consumers and they,
in turn, their sexually attractive body parts to fruits.}%
\de{Alles in der Natur, was nicht weiss-rot-schwarz war, wäre
zunächst namenlos gewesen: grün, blau, braun, blasse Farben
wie der Mond, schimmernde Farben; oft Farben, die etwas
signalisieren, das nicht überlebenswichtig ist, weder Nahrung
noch Gefahr. Dies könnte erklären, warum die Farbe Grün
erst spät in die Sprachen einging, obwohl sie in der Natur so
vorherrschend ist. Formen und Farben von Früchten könnten
sich an die Vorlieben ihrer Konsumenten angepasst haben und
jene wiederum ihre sexuell attraktiven Körperteile an Früchte.}

\en{I imagine a child in prehistory in the arms of its mother on a
tree at night, trying to “pluck” the moon in the sky, just as it
used to pluck fruit and already earlier used to get food from
the similarly round breasts of its mother, signaled also by her
“red” nipples; thus the gentle, soft roundness of the mother
so intimately linked to the moon and the colors of life/fire.}%
\de{Ich stelle mir ein Kind in der Vorgeschichte vor, das nachts in
den Armen seiner Mutter auf einem Baum sitzt und versucht,
den Mond am Himmel zu “pflücken”, so wie es zuvor Früchte
pflückte und schon früher Nahrung von den ähnlich runden
Brüsten seiner Mutter zu bekommen pflegte, was auch durch
ihre “roten” Brustwarzen signalisiert wurde; also die sanfte,
weiche Rundheit der Mutter, die so eng mit dem Mond und
den Farben des Lebens/Feuers verbunden ist.}

% page
\item
\en{Robert Graves in the introduction of \textsl{The Greek Myths}:}%
\de{Robert Graves in der Einführung von \textsl{Griechische Mythologie}:}

\en{“Ancient Europe had no gods. The Great Goddess was
regarded as immortal, changeless, and omnipotent; and the
concept of fatherhood had not been introduced into religious
thought. She took lovers, but for pleasure, not to provide
her children with a father. Men feared, adored, and obeyed
the matriarch; the hearth which she tended in a cave or hut
being their earliest social centre, and motherhood their prime
mystery. Thus the first victim of a Greek public sacrifice
was always offered to Hestia of the Hearth. The goddess’s
white aniconic image, perhaps her most widespread emblem,
which appears at Delphi as the \textsl{omphalos}, or navel-boss, may
originally have represented the raised white mound of tightly-packed
ash, enclosing live charcoal, which is the easiest means
of preserving fire without smoke.”}%
\de{“Das vorzeitliche Europa kannte keine (männlichen) Göt\-ter.
Die Grosse Göttin wurde als unsterblich, unveränderlich und
allmächtig erachtet, und das Konzept der Vaterschaft war
noch nicht ins religiöse Denken eingeführt worden. Sie nahm
sich Liebhaber, aber zum Vergnügen, nicht um ihren Kindern
einen Vater zu geben. Die Männer fürchteten, verehrten
und gehorchten der Matriarchin; der Herd, den sie in einer
Höhle oder Hütte hütete, war das früheste soziale Zentrum,
und die Mutterschaft das wichtigste Geheimnis. So wurde
das erste Opfer eines öffentlichen griechischen Opfers stets
der Hestia des Herdes dargebracht. Das weisse anikonische
Bildnis der Göttin, das vielleicht ihr am weitesten verbreitetes
Emblem ist und in Delphi als \textsl{Omphalos} oder Nabelstein
erscheint, könnte ursprünglich den aufgehäuften weissen Hügel
aus dicht gepackter Asche dargestellt haben, der glühende
Holzkohle umschloss, was das einfachste Mittel ist, um Feuer
ohne Rauch aufzubewahren.”}

\en{Again a sequence white-red-black, ash-glow-coal, with almost
certainly roots far back into prehistory. The triangle as the
mountain on which deities lived, where lightning was more
likely to strike, not to speak of volcanoes, or as a pyramid or
the symbols for the elements, and so much more.}%
\de{Wiederum eine Abfolge weiss-rot-schwarz, Asche-Glut-Kohle,
deren Wurzeln mit ziemlicher Sicherheit weit in die
Vorgeschichte zurückreichen. Das Dreieck als Berg, auf dem die
Götter lebten, wo der Blitz eher einschlug, um nicht von
Vulkanen zu sprechen, oder der Berg als Pyramide oder die
Symbole für die Elemente und so vieles mehr.}

\en{See also 20.2 and 90.3 in \textsl{The Greek Myths} about omphalos,
tripods, white-red-black, Crete, the moon-cow Io, and more.}%
\de{Siehe auch 20.2 und 90.3 in \textsl{Griechische Mythologie} über
Omphalos, Dreifüsse, weiss-rot-schwarz, Kreta, die Mondkuh
Io und mehr.}

\item
\en{The fifth element is round like the moon and cyclic motion in
the sky; if the first element is fire, then so is the fifth in a circle
of elements, thus the moon goddess also a “higher octave”
of fire. Of the three goddesses Hera, Athena and Aphrodite,
Paris hands the apple to Aphrodite (Venus) because if you
cut an apple in half, you get a five-pointed star, like the five
stations of Venus over 8 years, where also sun and moon
return quite closely to the same positions.}%
\de{Das fünfte Element ist rund wie der Mond und die zyklische
Bewegung am Himmel; wenn das erste Element Feuer ist,
dann ist es auch das fünfte in einem Kreis der Elemente,
daher ist die Mondgöttin auch eine “höhere Oktave” des Feuers.
Von den drei Göttinnen Hera, Athene und Aphrodite übergibt
Paris den Apfel an Aphrodite (Venus), denn wenn man einen
Apfel halbiert, erhält man einen fünfzackigen Stern, wie die
fünf Stationen der Venus im Laufe von 8 Jahren, wo auch
Sonne und Mond ziemlich genau zu denselben Positionen
zurückkehren.}

\item
\en{In the article “Red, White, and Black in Symbolic Thought:
The Tricolour Folk Motif, Colour Naming, and Trichromatic
Vision” (Folklore, 123:3, 310-329, 2012), Jessica Hemming
mentions that red was typically a color that is darker than
fresh blood, more towards brown. Now, Menstrual blood can
often be darker (already oxidized) than blood from a fresh
wound, which would again link to the moon.}%
\de{In ihrem Artikel “Red, White, and Black in Symbolic Thought:
The Tricolour Folk Motif, Colour Naming, and Trichromatic
Vision” (“Rot, Weiss und Schwarz im symbolischen Denken:
Das dreifarbige Volksmotiv, die Farbbenennung und das
trichromatische Sehen”, Folklore, 123:3, 310-329, 2012)
er\-wähnt Jessica Hemming, dass Rot typischerweise eine
Farbe war, die dunkler war als frisches Blut, mehr in Richtung
braun. Nun kann Menstruationsblut oft dunkler (bereits oxidiert)
sein als Blut aus einer frischen Wunde, was wiederum
mit dem Mond in Verbindung gebracht werden könnte.}

\en{See also her article “Pale horses and green dawns. Elusive
colour terms in early Welsh heroic poetry” (North American
journal of Celtic studies, Vol 1, No.\ 2, 189-223, 2017).}%
\de{Siehe auch ihren Artikel “Pale horses and green dawns.
Elusive colour terms in early Welsh heroic poetry” (“Bleiche Pferde
und grüne Morgenröte. Schwer fassbare Farbnamen in der
frühen walisischen Heldendichtung”, North American journal
of Celtic studies, Band 1, Nr.\ 2, 189-223, 2017).}

\item
\en{Robert Graves in\,\textsl{The White Goddess} (1948):}%
\de{Robert Graves in \textsl{Die Weisse Göttin} (1948):}

\en{“I write of her as the White Goddess because white is her
principal colour, the colour of the first member of her moon-
trinity, but when Suidas the Byzantine [ca.\ 10th century CE]
records that Io was a cow that changed her colour from \white{white}
to \red{rose} and then to \black{black} he means that the New Moon is
the white goddess of birth and growth; the Full Moon, the red
goddess of love and battle; the Old Moon, the black goddess
of death and divination. Suidas’s myth is supported by
Hyginus’s fable [ca.\ 0 CE] of a heifer-calf born to Minos and
Pasiphae which changed its colours thrice daily in the same
way. In response to a challenge from an oracle one Polyidus
son of Coeranus correctly compared it to a mulberry—a fruit
sacred to the Triple Goddess.” (Chapter 4)}%
\de{“Ich schreibe von ihr als der Weissen Göttin, weil Weiss ihre
Hauptfarbe ist, die Farbe des ersten Teils ihrer Mondtrinität,
aber wenn der Byzantiner Suidas [ca.\ 10.\ Jahrhundert n.\,Chr.]
berichtet, dass Io eine Kuh war, die ihre Farbe von \white{Weiss} zu
\red{Rosa} und dann zu \black{Schwarz} wechselte, meint er, dass der
Neumond die weisse Göttin der Geburt und des Wachstums
ist; der Vollmond die rote Göttin der Liebe und des Kampfes;
der alte Mond die schwarze Göttin des Todes und der
Weissagung. Suidas’ Mythos wird durch die Fabel von Hyginus
[ca.\ 0 n.\,Chr.] über ein von Minos und Pasiphae geborenes
Färberkalb gestützt, das dreimal täglich auf dieselbe Art seine
Farbe wechselte. Auf die Aufforderung eines Orakels hin
verglich Polyidus, der Sohn des Coeranus, die Färbung
korrekt mit einer Maulbeere\,—\,einer der dreifachen Göttin heiligen
Frucht.” (Kapitel 4)}

% page
\en{To me, the colors of the goddess would not directly reflect the
change of visible colors of the moon during its phases, as one
might think at first, but rather represent the hidden powers
that make it change, which would confirm Graves above:}%
\de{Für mich würden die Farben der Göttin nicht direkt den
sichtbaren Farbewechsel des Mondes während seiner Phasen
spiegeln, wie man erst meinen könnte, sondern vielmehr die
verborgenen Kräfte, die ihn sich ändern lassen, was das, was
Graves oben schrieb, bestätigen würde:}

\en{The \white{white} goddess would be the power that makes the new
moon brighter (more “white”) again, towards full moon, from
little baby girl to maiden, growth.\ The \red{red} goddess would be
the fertile adult woman, who menstruates (red blood); she
would make the moon pregnant, the round “belly” of the
full moon. The \black{black} goddess would make the moon darker
(more “black”) again, towards new moon, withering towards
crone. The “red phase” would be somewhat abstract as the
blood would only come to light at menstruation if the bearer
did not get pregnant. I guess the idea would have been that
the child’s blood and body would have grown from that.}%
\de{Die \white{weisse} Göttin wäre die Kraft, die den Neumond
wieder heller (“weisser”) macht, zum Vollmond hin, vom kleinen
Mädchen zur Jungfrau, dem Wachstum. Die \red{rote} Göttin wäre
die fruchtbare erwachsene Frau, die menstruiert (rotes Blut);
sie würde den Mond schwanger machen, den runden “Bauch”
des Vollmondes. Die \black{schwarze} Göttin würde den Mond
wieder dunkler (“schwärzer”) machen, zum Neumond hin, zur
alten Frau hin verwelkend. Die “rote Phase” wäre etwas
abstrakt, da das Blut nur bei der Menstruation zum Vorschein
kommt, wenn die Trägerin nicht schwanger wird. Ich vermute,
dass die Vorstellung dabei war, dass Blut und Körper des
Kindes aus dem Blut gewachsen wären.}

\en{So the seed for a new child would be expected to grow each
month from sometime after new moon until ovulation around
full moon and, if the bearer did not get pregnant, would result
in menstrual bleeding around new moon. Note, however, that
most contemporary women do not have their individual cycles
correlated with moon phases. The average cycle is 28 days
(but varies quite a bit individually), which is closer to the
time it takes the moon to return to the same spot relative to
the fixed stars (27.3 days) than to new moon (29.5 days).}%
\de{Es wäre also erwartet worden, dass der Samen für ein neues
Kind jeden Monat von irgendwann nach Neumond bis zum
Eisprung um den Vollmond herum gewachsen wäre und, wenn
die Trägerin nicht schwanger wurde, zu einer Menstruationsblutung
um Neumond führte. Es ist jedoch zu beachten, dass
der individuelle Zyklus der meisten heutigen Frauen nicht mit den
Mondphasen korreliert. Der durchschnittliche Zyklus
beträgt 28 Tage (variiert aber individuell sehr stark), was näher
an der Zeit liegt, die der Mond braucht, um an dieselbe Stelle
im Verhältnis zu den Fixsternen zurückzukehren (27.3 Tage)
als zum Neumond (29.5 Tage).}

\item
\en{Note that \white{white} is the principal color of the goddess simply
because it is the principal color of the moon in the sky.}%
\de{Es sollte erwähnt werden, dass \white{Weiss} einfach deshalb die
wesentliche Farbe der Göttin ist, weil es die wesentliche Farbe
des Mondes im Himmel ist.}

\item
\en{First Greek philosophers lived in Ionia (\hspace{-0.3em}\lrarr Io), on the western
coast of Asia Minor, and the ones who first brought philosophy
to other parts of Greece were also from there.}%
\de{Die ersten griechischen Philosophen lebten in Ionien (\hspace{-0.3em}\lrarr Io),
an der westlichen Küste Kleinasiens, und diejenigen, die die
Philosophie als erste in andere Teile Griechenlands brachten,
stammten ebenfalls von dort.}

\item
\en{Minoan-style golden seal ring found in mainland Greece at
the Acropolis in Mycenae (ca.\ 1450 BCE). Seems to show a
mulberry tree, sun, moon and milky way, three women with
labrys and flowers, and more. Might the two smaller women
with bulged, striped skirts maybe even symbolize bees\,?}%
\de{Goldener Siegelring im minoischen Stil, gefunden auf dem
griechischen Festland bei der Akropolis in Mykene (ca.\ 1450
v.\,Chr.). Zeigt offenbar einen Maulbeerbaum, Sonne, Mond
und Milchstrasse, drei Frauen mit Doppelaxt und Blumen und
mehr. Könnten die beiden kleineren Frauen mit wulstigen,
gestreiften Röcken vielleicht sogar Bienen symbolisieren\,?}

\vspace{1.15mm}
\hspace{6.8mm}\includegraphics[scale=0.14]{i-mycenae.jpg}
\vspace{0.35mm}

\en{Minoan seals do not generally mirror the supposed goddess
galore so precisely and completely, some feature, for example,
two or four women, but maybe this one is more canonical\,?
The labrys, so prominently at the center of the seal, seems to
have been only used for religious purposes in Minoan culture,
besides maybe just for decoration, but not as a weapon or a
tool; the material was too weak and no traces of usage.}%
\de{Minoische Siegel spiegeln die vermutete Masse von Attributen
der Göttin im Allgemeinen nicht so genau und vollständig
wider wie hier; manche zeigen z.B.\ zwei oder vier Frauen,
aber vielleicht ist dieses hier kanonischer\,? Die Doppelaxt,
die so prominent in der Mitte des Siegels steht, scheint in der
minoischen Kultur nur für religiöse Zwecke verwendet worden
zu sein, neben vielleicht nur zur Dekoration, aber nicht als
Waffe oder Werkzeug; das Material war zu schwach und es
finden sich keine Gebrauchsspuren.}

% page
\item
\en{Empedocles would have been the first to speak of four elements,
according to Aristotle in \textsl{Metaphysics} (Book I 3) and
in \textsl{On Generation and Corruption} (Book I 1).}%
\de{Empedokles wäre der erste gewesen sein, der von vier Elementen
sprach, gemäss Aristoteles in \textsl{Metaphysik} (Buch I 3)
und in \textsl{Über Werden und Vergehen} (Buch I 1).}

\en{Since at least then, Empedocles is usually credited for having
first mentioned the four elements, in the following fragment
(DK31B6) of a poem usually called \textsl{On Nature}:}%
\de{Spätestens seit dieser Zeit wird die erste Erwähnung von vier
Elementen üblicherweise Empedokles in folgenden Fragment
(DK31B6) eines Gedichts zugeschrieben, das gewöhnlich
\textsl{Über die Natur} genannt wird:}

\vspace{1mm}
\hspace{5mm}\includegraphics[scale=0.13]{i-empedocles.jpg}
\vspace{0.8mm}

\en{It speaks of “fourtold roots” at the origin of all, and then
lists four deities with some attributes, in this order: Zeus
(flashing/shining), Hera (live-giving/-bearing), Aidoneus (no
attributes), Nestis (moisture, tears/dew).}%
\de{Er spricht von “vierfachen Wurzeln” am Ursprung von allem,
und zählt dann vier Gottheiten mit einigen Attributen auf, in
dieser Reihenfolge: Zeus (blitzend/leuchtend), Hera
(lebensspendend/gebärend), Aidoneus (keine Eigenschaften), Nestis
(Feuchtigkeit, Tränen/Tau).}

\en{Interpreting the deities as \textsl{roots} of the elements, Zeus with his
thunderbolt would be fire, pregnant Hera earth, Hades, who’s
name means “unseen”, air, and Nestis obviously water.}%
\de{Wenn man Gottheiten als \textsl{Wurzeln} der Elemente interpretiert,
wäre Zeus mit seinem Donnerkeil Feuer, die schwangere Hera
Erde, Hades, dessen Name “ungesehen” bedeutet, Luft und
Nestis offensichtlich Wasser.}

\en{The quote is from a work by Aetius (1st or 2nd century CE),
which has only indirectly survived in several later works
attributed to different authors. Mostly elements are attributed
the same way as me above, else earth and air are flipped.}%
\de{Das Zitat stammt aus einem Werk von Aetius (1.\ oder 2.\
Jahrhundert n.\,Chr.), das nur indirekt in mehreren späteren,
verschiedenen Autoren zugeschriebenen Werken überliefert
ist. Meistens werden die Elemente auf die gleiche Weise
zugeordnet wie oben von mir, sonst Erde und Luft vertauscht.}

\en{It is obviously tempting to interpret Zeus as \white{white}, pregnant
Hera as \red{red} and Hades as \black{black}, in the ancient order of a
ripening mulberry, plus Nestis as great goddess, especially
since Nestis might be the the same goddess as the Egyptian
Nephthys, who Robert Graves calls “the Egyptian Hecate” in
\textsl{The White Goddess} (in the chapter \textsl{Gwion’s Heresy}).}%
\de{Es ist natürlich verlockend, Zeus als \white{weiss}, die schwangere
Hera als \red{rot} und Hades als \black{schwarz} zu interpretieren, in der
antiken Reihenfolge einer reifenden Maulbeere, plus Nestis
als grosse Göttin, zumal Nestis dieselbe Göttin sein könnte
wie die ägyptische Nephthys, die Robert Graves in \textsl{Die Weisse Göttin}
(im Kapitel \textsl{Gwions Häresie}) als “die ägyptische Hekate”
bezeichnet.}

\en{Or, since Nepthys was at the spinning house in Sais, maybe in
essence basically the ancient creator goddess Neith of Sais in
the Nile Delta, with the Nile for water (as later Isis), the delta
for the female sex and a trinity, sort of reversely spinning the
Nile from the strands in the delta, hence also a creatrix\,?}%
\de{Oder, da Nepthys im Spinnhaus in Sais war, vielleicht im
Wesentlichen die antike Schöpfergöttin Neith von Sais im Nildelta,
mit dem Nil für Wasser (wie später Isis), dem Delta für
das weibliche Geschlecht und einer Dreifaltigkeit, die den Nil
quasi umgekehrt aus den Fäden im Delta spinnt, also auch
eine Schöpfungsgöttin\,?}

\en{(The ancient Greeks identified Neith with Athena, which may
also explain some influences in Plato’s works, for example.)}%
\de{(Die alten Griechen identifizierten Neith mit Athene, was auch
einige Einflüsse in Platons Werken erklären könnte).}

\item
\en{In ancient Egypt, Osiris stood for \black{black}, the fertile earth of
the Nile valley; his brother Seth for \red{red}, the desert East and
West of the valley. The mythological killing and dismembering
of Osiris by Seth presumably reflects that in ancient
times sometime after the annual flood the soil would dry up
and become fractured into a mosaic of slabs, or even into
sand and dust. Fortunately, every year the Nile, \white{white} Isis
(also like milk), would restore Osiris to life with water and
the fresh fertile black sediments carried along.}%
\de{Im alten Ägypten stand Osiris für \black{Schwarz}, die fruchtbare
Erde des Niltals, sein Bruder Seth für \red{Rot}, die Wüste
östlich und westlich des Tals. Die mythologische Tötung und
Zerstückelung von Osiris durch Seth spiegelt vermutlich
wider, dass in alten Zeiten irgendwann nach der jährlichen
Über\-schwem\-mung der Boden austrocknete und zu einem Mosaik
aus Schollen oder sogar zu Sand und Staub  zerfiel.
Glück\-li\-cher\-wei\-se hätte der Nil, die \white{weisse} Isis (auch wie Milch),
Osiris jedes Jahr mit dem Wasser und den frischen, fruchtbaren
schwarzen Sedimenten wieder zum Leben erweckt.}

\en{This is certainly an oversimplification of Egyptian mythologies
that evolved over millennia, but likely still captures a core.}%
\de{Dies ist sicherlich eine Übervereinfachung der ägyptischen
Mythologie, die sich über Jahrtausende entwickelt hat, aber
wahrscheinlich trifft sie doch noch einen wahren Kern.}

\item
\en{See this absolutely stunning article by the Ethiopian “Shakespeare”,
Tsegaye Gabre-Medhin:
\textsl{The Origin of the Trinity in Art \& Religion: Ethiopian Roots in the Egypto-Greek \& Hebrew},
on page 99-120 of \textsl{African Origins of the Major World Religions},
ed.\ Amon Saba Saakana, Karnak House, 1988.}%
\de{Siehe diesen absolut umwerfenden Artikel des äthiopischen
“Shakespeare”, Tsegaye Gabre-Medhin: \textsl{The Origin of the
Trinity in Art \& Religion: Ethiopian Roots in the
Egypto-Greek \& Hebrew} (Der Ursprung der Trinität in Kunst \&
Religion: Äthiopische Wurzeln im Ägyptisch-Griechischen \&
Hebräischen), auf Seite 99-120 von \textsl{African Origins of the Major
World Religions} (Afrikanische Ursprünge der grossen
Weltreligionen), ed.\ Amon Saba Saakana, Karnak House, 1988.}

\en{KaBaRa to Kabbalah and Kaaba, Egypt as Kamit (black
land), sacred tree, Osiris to Moses and others, and so much
more. I guess Fela’s song \textsl{Shakara} might fit in, too.}%
\de{KaBaRa zu Kabbala und Kaaba, Ägypten als Kamit (schwarzes
Land), heiliger Baum, Osiris zu Moses und anderen, und
so vieles mehr. Ich nehme an, Felas Song \textsl{Shakara} würde da
auch dazu passen.}

% page
\item
\en{First Corinthians 13:13: “And now these three remain: faith,
hope, and love. But the greatest of these is love.” The Greek
words are pístis, elpís and agápē, which are also goddesses,
and they occur prominently in Plato’s philosophy.}%
\de{1 Korinther 13,13: “Und nun bleiben diese drei: Glaube,
Hoffnung und Liebe. Doch am grössten unter ihnen ist die Liebe.”
Auf griechisch pístis, elpís und agápē, die auch Göttinnen sind
und in Platons Philosophie prominent auftreten.}

\en{Faith could be interpreted as sun, bright/white, Hera; hope as
moon, dark/black, Athena; love as Venus, fire/red, Aphrodite,
the one that got the apple from Paris. These planets are also
the order of dresses in the 19th century fairy tale \textsl{Three nuts
for Cinderella} by Božena Němcová: sun, moon, stars, which
reminds also of Isis’ dress. Actually, these three occur already
in Mesopotamia in the 12th century BCE on a stele: Venus
for Ishtar, moon and sun for the gods Sin and Shamash.}%
\de{Der Glaube könnte als Sonne, hell/weiss, Hera, interpretiert
werden; die Hoffnung als Mond, dunkel/schwarz, Athene; die
Liebe als Venus, feuer/rot, Aphrodite, diejenige, die den Apfel
von Paris bekam. Diese Planeten sind auch die Reihenfolge
der Kleider im Märchen \textsl{Drei Nüsse für Aschenbrödel} von
Božena Němcová aus dem 19.\ Jahrhundert: Sonne, Mond,
Sterne, was auch an das Kleid von Isis erinnert. Tatsächlich
tauchen diese drei bereits im 12.\ Jahrhundert v.\,Chr.\ in
Mesopotamien auf einer Stele auf: Venus für Ishtar, Mond und
Sonne für die Götter Sin und Schamasch.}

\vspace{0.3mm}\hspace{2mm}
\includegraphics[scale=0.1]{i-ishtar.jpg}
\vspace{0.3mm}

\item
\en{A closer look at the passage from the Chandogya Upanishad
mentioned further above shows that the word used for red,
\textsl{rohitam}, is also the word for a female red deer, as well as as
Rohini the name of the red star Aldebaran, one of the eyes of
the bull in the constellation Taurus. In ancient Greece deer
were sacred to the moon goddess Artemis, originally probably
because antlers resemble a fire. In ancient Egypt in the first
dynasties the Pharaoh used to run with the \white{white}-\red{red}-\black{black}
Apis bull at the beginning of spring when the constellation
of Taurus was rising. The moon goddess resides at birth and
death, as both midwife and goddess of death, when the new
or old moon look like a flame or a shoot, or later in history a
bow or a sickle, hence she is also celebrated at the beginning
of spring when nature starts to sprout again.}%
\de{Ein genauerer Blick auf die weiter oben erwähnte Passage aus
der Chandogya Upanishad zeigt, dass das Wort für rot, \textsl{rohitam},
auch das Wort für einen weiblichen Rothirsch ist, ebenso
wie Rohini der Name des roten Sterns Aldebaran, eines der
Augen des Stiers im Sternbild Stier. Im antiken Griechenland
waren Hirsche der Mondgöttin Artemis heilig, ursprünglich
wahrscheinlich, weil das Geweih einem Feuer ähnelt. Im
alten Ägypten rannte der Pharao in den ersten Dynastien zu
Frühlingsbeginn, wenn das Sternbild Stier aufging, mit dem
\white{weiss}-\red{rot}-\black{schwarzen} Apis-Stier zusammen. Die Mondgöttin
residiert bei Geburt und Tod, sowohl als Hebamme wie auch
als Todesgöttin, wenn der neue oder alte Mond wie eine Flamme
oder ein Spross aussieht, oder später in der Geschichte wie
ein Bogen oder eine Sichel, daher wird sie zu Frühlingsbeginn
auch gefeiert, wenn die Natur wieder zu spriessen beginnt.}

\en{The first version of \textsl{The White Goddess}, which Robert Graves
wrote after new moon in the third degree of Taurus in spring
1944, was titled \textsl{The Roebuck in the Thicket}. Isis as the only
woman in Isis-Seth-Osiris would be \white{white}, hence \textsl{The White Goddess}
a fitting settled title\,? Hail Artemis\,!}%
\de{Die Urfassung von \textsl{Die weisse Göttin}, die Robert Graves nach
Neumond im 3.\ Grad des Stiers im Frühjahr 1944 schrieb,
trug den Titel \textsl{The Roebuck in the Thicket} (Der Rehbock im
Dickicht). Isis als einzige Frau in Isis-Seth-Osiris wäre \white{weiss},
daher \textsl{Die weisse Göttin} ein passender Titel\,? Heil Artemis\,!}

\en{Note that in astrology the moon is exalted (a good guest) in
the 3rd degree of Taurus, or maybe around 3$^\circ$ (see Vettius
Valens, \textsl{Anthology}, book 3, chapter 4, 2nd century CE).}%
\de{Zu erwähnen wäre hier, dass in der Astrologie der Mond im
3.\ Grad des Stiers als erhöht (als guter Gast) betrachtet wird,
oder vielleicht um 3$^\circ$ herum (siehe Vettius Valens, \textsl{Anthologie},
Buch 3, Kapitel 4, 2.\ Jahrhundert n.\,Chr.).}

\item
\en{Note that way more could be said around these themes; for
example, water-white is also closely related to milk and the
“mound” it comes from, or cows and the milky way and Isis as
the Nile; the three Graeae (grey women) and their single eye,
the three Fates and their fabric; purple Io like mulberry juice
and the famous die, as well as drinking wine from amethyst
goblets in Greek antiquity; the three Indian gunas (strands,
chords) in the colors white, red and black, as well as the four
varnas (colors) of social classes, with additionally yellow, all
maybe related to ancient Egyptian Ma’at; Ra as a yellow cat
with donkey ears defeating the white-red-black Apophis snake
wrapped around a green tree with red fruits in a painting in
Theban Tomb 359 of the 20th dynasty (12th century BCE);
as just a few of millions of examples…}%
\de{Zu diesen Themen liesse sich noch viel mehr sagen; zum
Beispiel steht Wasser-Weiss auch in enger Beziehung zu Milch
und dem “Hügel”, aus dem sie kommt, oder zu Kühen und
der Milchstrasse und Isis als Nil; die drei Graeae (graue Frauen)
und ihr einzelnes Auge, die drei Schicksalsgöttinnen und
ihr Schicksalsstoff (engl.\ fabric of fate); die purpurne Io wie
Maulbeersaft und die berühmte Farbe, sowie das Trinken von
Wein aus Amethystkelchen in der griechischen Antike; die
drei indischen Gunas (Stränge, Saiten) in den Farben Weiss,
Rot und Schwarz sowie die vier Varnas (Farben) der sozialen
Klassen, mit zusätzlich Gelb, welche alle möglicherweise mit
der altägyptischen Ma’at in Verbindung stünden; Ra als gelbe
Katze mit Eselsohren, die die weiss-rot-schwarze
Apophis-Schlange besiegt, welche sich um einen grünen Baum mit
roten Früchten gewickelt hat, auf einem Gemälde im thebanischen
Grab 359 der 20.\ Dynastie (12.\ Jahrhundert v.\,Chr.);
um nur einige von Millionen von Beispielen zu nennen…}

% page
\vspace{0.3mm}
\includegraphics[scale=0.16]{i-ra.jpg}
\vspace{0.3mm}

\en{Note that it were possibly similar depictions of Ra that led to
medieval depictions of “killer rabbits” after the crusades.}%
\de{Man beachte auch, dass es möglicherweise ähnliche Darstellungen
von Ra waren, die nach den Kreuzzügen zu mittelalterlichen
Darstellungen von “Killerhasen” führten.}

\item
\en{A link from elements to fire is immediately easier to trace than
one to the moon, which may be because this would have been
a secret, the unspeakable real name of the goddess\,?}%
\de{Eine Verbindung von den Elementen zu Feuer ist unmittelbar
leichter nachzuvollziehen als eine zum Mond, was daran
liegen könnte, dass dies ein Geheimnis gewesen wäre, der
unaussprechliche wirkliche Name der Göttin\,?}

\item
\en{Plato talks about colors in the \textsl{Timaeus}, Aristotle in
\textsl{On Sense and the Sensible}. Both start with black and white
as basic colors, which is scientifically correct in the sense that by
selectively taking frequencies out of the full spectrum of white,
you get all colors, including black and white.}%
\de{Platon spricht im \textsl{Timaios} über Farben, Aristoteles in \textsl{Sinn
und Venunft}. Beide gehen von Schwarz und Weiss als
Grundfarben aus, was wissenschaftlich insofern korrekt ist, als man
durch selektives Ausblenden von Frequenzen aus dem
gesamten Spektrum von Weiss alle Farben erhält, einschliesslich
Schwarz und Weiss.}

\en{Democritus, one of the first atomists, explains colors from
microscopic structure, e.g.\ white as smooth, using primary
colors white-black-red-green (leukós-mélas-pyrrós-chlōrós), in
that order, of which all other colors would be composed—at
least as handed down by Priscian of Lydia in the 6th
century CE about how Theophrastus, a pupil and successor of
Aristotle, would have described it in \textsl{On Sense Perception}.}%
\de{Demokrit, einer der ersten Atomisten, erklärt Farben aus der
mikroskopischen Struktur, z.B.\ Weiss als glatt, unter
Verwendung der Primärfarben Weiss-Schwarz-Rot-Grün
(leukós-mélas-pyrrós-chlōrós), in dieser Reihenfolge, aus denen sich
alle anderen Farben zusammensetzen würden—zumindest
ge\-mäss dem, was Priskianos von Lydien im 6.\ Jahrhundert n.\,Chr.\
darüber weitergegeben hat, wie Theophrastus, Schüler und
Nachfolger von Aristoteles, es in \textsl{On Sense Perception}
(Über Sinneswahrnehmung) beschrieben hätte.}

\en{There are three kinds of color sensors in the human eye, for
red, green and blue, sorted from low to high frequency. None
triggered (no light) is black, plus red gives red, plus also green
gives yellow, plus also blue gives white, hence a sequence
black-red-yellow-white or earth-air-fire-water.}%
\de{Es gibt drei Arten von Farbsensoren im menschlichen Auge,
für Rot, Grün und Blau, geordnet von niedriger nach hoher
Frequenz. Keins davon ausgelöst (kein Licht) ist schwarz, plus
rot ergibt rot, plus auch grün ergibt gelb, plus auch blau
ergibt weiss, also eine Abfolge schwarz-rot-gelb-weiss oder
Erde-Luft-Feuer-Wasser.}

\en{In Plato’s \textsl{Critias} the stones of Atlantis’ architecture are won
locally and have the colors white, black and red.}%
\de{In Platons \textsl{Kritias} werden die Steine der Architektur von
Atlantis lokal gewonnen und haben die Farben weiss, schwarz
und rot.}

\item
\en{The Yangshao culture “Xishuipo M45 Tomb” in China, which
dates back to the 4th millennium BCE, features the mosaic
of a tiger opposite the mosaic of a dragon, as constellations
in the sky, exactly the animals that are traditionally
assigned to West and East in China. Ra’s nightly fight with
the Apep/Apophis snake reminds of the phoenix and snake
(plus turtle) standing for South and North in China.}%
\de{Das “Xishuipo M45-Grab” der Yangshao-Kultur in China, das
auf das 4.\ Jahrtausend v.\,Chr.\ zurückgeht, zeigt das Mosaik
eines Tigers gegenüber dem Mosaik eines Drachens, als
Konstellationen am Himmel, genau die Tiere, die in China
traditionell dem Westen und dem Osten zugeordnet werden. Der
nächtliche Kampf von Ra mit der Schlange Apep/Apophis
erinnert an Phönix und Schlange (plus Schildkröte), die in
China für Süden und Norden stehen.}

\item
\en{Could such attributions already have existed at the time people
walked to North America across the Bering Sea maybe
over 10’000 years ago\,? Medicine wheel symbols with colors
black-red-yellow-white seem to be a 1972 invention by a plastic
shaman. In a 2010 video, Chief Arvol Looking Horse of
Lakota Sioux Nation and keeper of the sacred pipe and bundle
in the 19th generation tells the story of the White Buffalo
Calf Woman who first appeared as a cloud, then as a woman
and changed into a buffalo calf that changed colors
black-red-yellow-white. She was found by two scouts who remind
of Seth and Osiris, and she is related to water and fertility of
the land, which reminds via Isis even more of ancient Egypt.}%
\de{Könnte es solche Zuordnungen bereits gegeben haben, als
Menschen vor vielleicht über 10’000 Jahren über die
Beringsee nach Nordamerika wanderten\,? Medizinrad-Symbole
in den Farben schwarz-rot-gelb-weiss scheinen eine Erfindung
eines Plastikschamanen aus dem Jahr 1972 zu sein. In einem
Video aus dem Jahr 2010 erzählt Häuptling Arvol Looking
Horse der Lakota Sioux Nation und Hüter der heiligen Pfeife
und des Bündels in der 19.\ Generation die Geschichte der
Weissen Büffelkalbfrau, die zunächst als Wolke, dann als Frau
erschien und sich in ein Büffelkalb verwandelte, das die Farben
schwarz-rot-gelb-weiss wechselte. Sie wurde von zwei
Kundschaftern gefunden, die an Seth und Osiris erinnern, und sie
ist mit Wasser und der Fruchtbarkeit des Landes verbunden,
was über Isis noch mehr an das alte Ägypten erinnert.}

% page
\item
\en{In Egypt, the four sons of Horus stood also for the four points
of the compass, guarding specific organs as canopic jars:
liver/man/south, stomach/jackal/east, lungs/baboon/north
and intestines/falcon/west. Might they have been direct
predecessors of the ancient Greek elements, also via medicine\,?}%
\de{In Ägypten standen die vier Söhne des Horus auch für die
vier Himmelsrichtungen und bewachten bestimmte Organe
als Kanopen: Leber/Mensch/Süden, Magen/Schakal/Osten,
Lunge/Pavian/Norden und Eingeweide/Falke/Westen.
Könn\-ten sie direkte Vorläufer der altgriechischen Elemente gewesen
sein, auch über die Medizin\,?}

\item
\en{Aristotle considers four “causes” in \textsl{Physics} and \textsl{Metaphysics},
which remind of the four elements. Matter reminds of earth,
form of air, primary source of fire and final goal of water.}%
\de{Aristoteles betrachtet in \textsl{Physik} und \textsl{Metaphysik} vier
“Ursachen”, die an die vier Elemente erinnern. Die Materialursache
erinnert an Erde, die Formursache an Luft, die Bewegursache
an Feuer und die Zielursache an Wasser.}

\item
\en{Some fragments of Heraclitus might suggest the same circle
as Aristotle. DK22B76 seems to mention all four elements
in the same circle, earth-fire-air-water-earth, but the original
text cannot be restored for sure, according to Diels/Kranz
(DK) in \textsl{Die Fragmente der Vorsokratiker}. See also fragments
B31 and B36; and B90 might suggest that Heraclitus would
have considered fire the primary substance.}%
\de{Einige Fragmente von Heraklit könnten auf denselben Kreis
wie Aristoteles hinweisen. DK22B76 scheint alle vier Elemente
im gleichen Kreis zu erwähnen, Erde-Feuer-Luft-Wasser-Erde,
aber der ursprüngliche Text kann laut Diels/Kranz (DK)
in \textsl{Die Fragmente der Vorsokratiker} nicht mit Sicherheit
wiederhergestellt werden. Siehe auch die Fragmente B31 und
B36; und B90 könnte darauf hindeuten, dass Heraklit das
Feuer als die primäre Substanz betrachtet hätte.}

\item
\en{See the pythagorean tetractys and oath. Pythagoras lived
in the 6th century BCE, before Empedocles and Hippocrates.
The tetractys is a triangle with four dots on each side:}%
\de{Siehe die pythagoreische Tetraktys und den pythagoreischen
Eid. Pythagoras lebte im 6.\ Jahrhundert v.\,Chr., vor Empedokles
und Hippokrates. Die Tetraktys ist ein Dreieck mit vier
Punkten auf jeder Seite:}

\begin{center}
\hspace{-2.5mm}%
\begin{tabular}{c|l|l|l|l|} \cline{2-4}
\en{\multirow{4}{*}[-1mm]{\includegraphics[scale=0.14]{i-tetractys.jpg}\ \ } & 1 & point & monad (unity) \\ \cline{2-4}
& 2 & line & dyad (power) \\ \cline{2-4}
& 3 & triangle/plane & triad (harmony) \\ \cline{2-4}
& 4 & tetrahedron/space & tetrad (cosmos) \\ \cline{2-4}}%
\de{\multirow{4}{*}[-1mm]{\includegraphics[scale=0.14]{i-tetractys.jpg}\ \ } & 1 & punkt & monade (einheit) \\ \cline{2-4}
& 2 & linie & dyade (kraft) \\ \cline{2-4}
& 3 & dreieck/ebene & triade (harmonie) \\ \cline{2-4}
& 4 & tetraeder/raum & tetrade (kosmos) \\ \cline{2-4}}
\end{tabular}
\end{center}

\en{It relates also to music via the ratios between each line, octave
(2:1), perfect fifth (3:2) and perfect fourth (4:3).}%
\de{Sie bezieht sich durch die Verhältnisse zwischen den einzelnen
Linien auch auf die Musik, auf die Oktave (2:1), die perfekte
Quinte (3:2) und die perfekte Quarte (4:3).}

\en{The list reminds of fire-air-water-earth (light to heavy).}%
\de{Die Liste erinnert an Feuer-Luft-Wasser-Erde (von leicht zu
schwer).}

\en{But the above is apparently even more than usually for early
Greek philosophers based on speculation, since Pythagoras
reportedly never wrote anything down himself, so that there
are even less credible sources about his views in his time,
often surrounded by legends bordering on religion/sect.}%
\de{Aber das oben Gesagte ist anscheinend noch mehr als sonst
üblicherweise bei frühen griechischen Philosophen auf
Spekulationen gebaut, da Pythagoras angeblich selbst nie etwas
aufgeschrieben hatte, so dass es noch weniger glaubwürdige
Quellen als sonst über seine Ansichten zu seiner Zeit gibt, oft
von Legenden umgeben, die an Religion/Sekte grenzen.
}

\en{DK note that practically the same word that Empedocles used
for “roots” in DK31B6 also appears in the pythagorean oath
in DK58B15, with both fragments dating back to Aetius in
early CE, so once more circular paths.}%
\de{DK bemerken, dass praktisch dasselbe Wort, das Empedokles
in DK31B6 für “Wurzeln” verwendet, auch im pythagoreischen
Eid in DK58B15 vorkommt, wobei beide Fragmente
auf Aetius Anfang CE zurückgehen, also wiederum eine
Suche, die im Kreis verlaufen sein könnte.}

\item
\en{To ancient Greeks, the ancient Egyptians were apparently sort
of like the ancient Greeks in modern perception, an admired
ancient culture. It appears that the ancient Egyptians might
have kept things more secret than the Greeks in their time,
maybe only passing it on from master to pupil\,?}%
\de{Für die alten Griechen waren die alten Ägypter offenbar so
etwas wie die alten Griechen in der modernen Wahrnehmung,
eine bewunderte antike Kultur. Es hat den Anschein, dass die
alten Ägypter die Dinge vielleicht noch geheimer hielten als
die Griechen zu ihrer Zeit und sie vielleicht nur von Meister
zu Schüler weitergaben\,?}

\item
\en{As a playful teaser, note that the pyramids have five corners,
an earthly base of four, plus one on top…}%
\de{Als spielerische Anregung sei angemerkt, dass Pyramiden fünf
Ecken haben, eine irdische Basis mit vier Ecken, plus eine
Ecke auf der Spitze…}

\vspace{0.8mm}
\includegraphics[scale=0.1975]{i-pyramids.jpg}
\vspace{0.8mm}

% page
\en{With maybe even opposites attached as below (or maybe with
dry/hot and wet/cold flipped, or attached to faces instead),
reflecting the heating and drying effect of the sun during the
course of a day, similarly to the original image for yin-yang in
China as the shady and sunny sides of a hill\,?}%
\de{Vielleicht sogar mit Gegensätzen wie unten zugeordnet (oder
vielleicht mit trocken/warm und feucht/kalt vertauscht, oder
stattdessen den Seitenflächen zugeordnet), die die wärmende
und trocknende Wirkung der Sonne im Laufe eines Tages
widerspiegelnd, ähnlich wie das ursprüngliche Bild für
Yin-Yang in China als schattige und sonnige Seiten eines Hügels\,?}

\en{\vspace{0.8mm}\hspace{20.5mm}%
\includegraphics[scale=0.115]{i-pyramid-elements.jpg}
\vspace{0.8mm}}%
\de{\vspace{2mm}\hspace{20.5mm}%
\includegraphics[scale=0.115]{i-pyramid-elements-de.jpg}
\vspace{1.3mm}}

\en{But how pyramids evolved from single “floor” mastabas via
step pyramids to their final form seems to be well researched.
Especially how Sneferu had the first three pyramids without
steps built and the first two attempts failed, does not suggest
that all that much elemental symbolism would have been in
the conscious minds of ancient Egyptians at that time.}%
\de{Aber wie sich Pyramiden von “einstöckigen” Mastabas über
Stufenpyramiden bis zu ihrer endgültigen Form entwickelten,
scheint gut erforscht zu sein. Vor allem die Tatsache, dass
Sneferu die ersten drei Pyramiden ohne Stufen bauen liess und
die ersten beiden Versuche scheiterten, lässt nicht vermuten,
dass den alten Ägyptern zu dieser Zeit derart viel elementare
Symbolik bewusst im Kopf gewesen wäre.}

\item
\en{Antiochus of Athens attributed elements to seasons the same
way I did with faces of the pyramids, if winter is north, etc.:
spring-air, summer-fire, autumn-earth and winter-water. The
symbols for the four elements are triangles, reminding of the
four faces of a pyramid, so whoever created those symbols
might maybe have related elements to pyramids.}%
\de{Antiochus von Athen ordnete die Elemente den Jahreszeiten
zu, auf die gleiche Weise, wie ich es oben mit Seitenflächen
der Pyramiden tat, wenn Winter Norden ist usw.: Frühling-Luft,
Sommer-Feuer, Herbst-Erde und Winter-Wasser. Die
Symbole für die vier Elemente sind Dreiecke, die an die vier
Seiten einer Pyramide erinnern, so dass derjenige, der diese
Symbole geschaffen hat, die Elemente vielleicht mit Pyramiden
in Verbindung gebracht hätte.}

\en{The symbols also stand for female and male sexes, overlaid
to a hexagram “as above so below” for intercourse between
Gaia (earth) and Ouranos (sky). \elfire \elair \elwater \elearth \rarr \elhex}%
\de{Die Symbole stehen auch für das weibliche und das männliche
Geschlecht, überlagert zu einem Hexagramm “wie oben so
unten” für den Verkehr zwischen Gaia (Erde) und Ouranos
(Himmel). \elfire \elair \elwater \elearth \rarr \elhex}

\item
\en{In the \textsl{Timaeus} Plato does not stop at the platonic solids for
the elements, but explicitly constructs all of them except the
dodecahedron (‘fifth element’) from right-angled triangles, actually
even as ‘1+3’ with ‘1’ being the cube (earth).}%
\de{Im \textsl{Timaios} hört Platon nicht bei den platonischen Körpern
für die Elemente auf, sondern konstruiert ausdrücklich alle
ausser dem Dodekaeder (‘fünftes Element’) aus rechtwinkligen
Dreiecken, eigentlich sogar als ‘1+3’, wobei die ‘1’ der
Würfel (Erde) wäre.}

\item
\en{Zeus, Poseidon and Hades ruled in heaven (air), sea (water)
and underworld (earth). Life can exist in all three of these
elements, but not in fire, except in legend fire salamanders.}%
\de{Zeus, Poseidon und Hades regierten im Himmel (Luft), im
Meer (Wasser) und in der Unterwelt (Erde). Leben kann in
allen drei Elementen existieren, aber nicht im Feuer, ausser
in der Legende der Feuersalamander.}

\item
\en{In “A few new discoveries in physics” of 2002, on page 20,
see \textsl{Discoveries revisited} under \cometartemis, I related numbers
(and elements) to increasing realms:}%
\de{In “A few new discoveries in physics” von 2002, auf Seite
20, siehe \textsl{Discoveries revisited} unter \cometartemis, hatte ich
Zahlen (und Elemente) mit wachsenden Einflussbereichen in
Verbindung gebracht:}

\begin{center}
\begin{tabular}{|l|l|l|}\hline
\en{1 & {\smaller(}fire{\smaller)} & individual imagination \\ \hline
2 & {\smaller(}air{\smaller)} & logical consequences \\ \hline
3 & {\smaller(}water{\smaller)} & collective wishes \\ \hline
4 & {\smaller(}earth{\smaller)} & reality \\ \hline}%
\de{1 & {\smaller(}feuer{\smaller)} & individuelle vorstellung \\ \hline
2 & {\smaller(}luft{\smaller)} & logische konsequenzen \\ \hline
3 & {\smaller(}wasser{\smaller)} & kollective wünsche \\ \hline
4 & {\smaller(}erde{\smaller)} & realität \\ \hline}
\end{tabular}
\end{center}

\en{In astrology and similar fields, 1+3 have more weight than
2+4: Collective, more or less mainstream views plus individual
pet theories outweigh logic and reality, which have mainly
assisting roles like to provide accurate calculations of planetary
positions. Conversely, in exact science, 2+4 have more
weight than 1+3: Theory plus experiments have precedence
over individual imagination and collective wishes.}%
\de{In der Astrologie und ähnlichen Gebieten hat 1+3 mehr
Gewicht als 2+4: Kollektive, mehr oder weniger gängige
Ansichten plus individuelle Lieblingstheorien sind wichtiger als Logik
und Realität, die hauptsächlich unterstützende Funktionen
haben, wie z.B.\ genaue Berechnungen von Planetenpositionen
zu liefern. Umgekehrt hat in der exakten Wissenschaft
2+4 mehr Gewicht als 1+3: Theorie und Experimente haben
Vorrang gegenüber individuellen Vorstellungen und
kollektiven Wünschen.}

% page
\en{See also Robert Pirsig’s static qualities nature-biology-social-intellect-mystic
in \textsl{Lila}, with maybe the main difference that
the ‘air’ layer would be ‘social’ instead of ‘logic’, which likely
reveals bias in me (and him) regarding how to interpret the
element air, more as communication and social relations or
more as abstract airy logic with premises and conclusions.}%
\de{Siehe auch Robert Pirsigs statische Qualitäten Natur-Bio\-lo\-gie-Sozial-Intellekt-Mystik
in \textsl{Lila}, vielleicht mit dem Hauptunterschied,
dass bei ihm die ‘Luft’-Ebene ‘Sozial’ statt
‘Logik’ wäre, was wahrscheinlich eine Voreingenommenheit in
mir (und ihm) offenbart, was die Interpretation des Elements
Luft angeht, mehr als Kommunikation und soziale Beziehungen
oder mehr als abstrakte luftige Logik mit Prämissen und
Schlussfolgerungen.}

\en{A five that transcends reality could also be added above, plus
6 to 9 for transformations of air-fire-earth-water, 10 for
transformation of transformation, etc. See e.g.\ the chapter about
Capricorn in my book \textsl{Elementary Star Signs} (2018), which
might have been easier to promote if titled \textsl{Transformations
between the Classical Elements Fire, Air, Water and Earth in
the Twelve Signs of the (Western Tropical) Zodiac}, if only
maybe I had planets in Air signs in my birth chart and thus
better developed promotional/marketing skills\,?\ ;)}%
\de{Eine Fünf, die die Realität transzendiert, könnte auch oben
hinzugefügt werden, plus 6 bis 9 für Transformationen von
Luft-Feuer-Erde-Wasser, 10 für Transformation von
Transformation usw. Siehe z.B.\ das Kapitel über den Steinbock in
meinem Buch \textsl{Elementare Sternzeichen} (2018), welches
vielleicht leichter zu vermarkten gewesen wäre, wenn es den Titel
\textsl{Transformationen zwischen den klassischen Elementen Feuer,
Luft, Wasser und Erde in den zwölf Zeichen des (westlichen,
tropischen) Tierkreises} getragen hätte, wenn ich nur vielleicht
Planeten in Luftzeichen in meinem Geburtshoroskop und so
besser entwickelte Werbe-/Marketingfähigkeiten hätte\,?\ ;)}

\item
\en{\textsl{Xuan} near the end of the first section of the Tao Te Ching
is literally a very dark red. Maybe black/white for earth/heaven
and xuan for the secret of life, symbolized by the female sex,
the ninth gate to the female body, even Tao itself\,?}%
\de{\textsl{Xuan} gegen das Ende des ersten Abschnitts des Tao Te King
ist unmittelbar ein sehr dunkles Rot. Vielleicht schwarz/weiss
für Erde/Himmel und Xuan für das Geheimnis des Lebens,
symbolisiert durch das weibliche Geschlecht, das neunte Tor
zum weiblichen Körper, sogar das Tao selbst\,?}

\en{The Han dynasty \textsl{Tai Xuan Jing}, a fusion of Tao Te Ching
and I Ching, an oracle with 81 tegragrams made of unbroken,
once and twice broken lines, standing for heaven, earth and
‘human being’, might also be worth a closer look.}%
\de{Das \textsl{Tai Xuan Jing} aus der Han Dynastie, eine Verschelzung
von Tao Te King und I Ging, ein Orakel mit 81 Tetragrammen
aus ungebrochenen, einmal und zweimal gebrochenen Linien,
die für Himmel, Erde und Mensch stehen, könnte auch eine
nähere Betrachtung wert sein.}

\item
\en{In August 2015, I assigned Greek goddesses to pairs of
elements and moon phases: Artemis/Hecate to (re-)birth/death
at new moon as fire around water, Hera (and Clotho) to
growth as a young woman or girl at the first quarter as air
around earth, Aphrodite (and Lachesis) to bloom as a mature
woman at full moon as water around fire, and Athena (and
Atropos) to withering as an old woman at the last quarter as
earth around air. Artemis/Hecate would thus contain both
first and fifth element, with 2-3-4 in between, and elements
would touch as on the Möbius Strip.}%
\de{Im August 2015 hatte ich griechische Göttinnen zu Paaren
von Elementen und Mondphasen zugeordnet: Artemis/Hekate
der (Wieder-)Geburt/dem Tod bei Neumond als Feuer um
Wasser, Hera (und Clotho) dem Wachstum als junge Frau
oder Mädchen im ersten Viertel als Luft um Erde, Aphrodite
(und Lachesis) dem Erblühen als reife Frau bei Vollmond als
Wasser um Feuer, und Athene (und Atropos) dem Verwelken
als alte Frau im letzten Viertel als Erde um Luft.
Artemis/Hekate würde also sowohl das erste als auch das fünfte
Element enthalten, mit 2-3-4 dazwischen, und die Elemente
würden sich wie auf dem Möbiusband berühren.}

\begin{center}
\en{\includegraphics[scale=0.07]{i-moon-goddesses.png}}%
\de{\includegraphics[scale=0.07]{i-moon-goddesses-de.png}}
\end{center}

\en{In the introduction of \textsl{The Greeks Myths}, Robert Graves flips
Athena and Hera in terms of assignment to maiden/crone, as
I also did intermittently, so maybe both, yin-yang-style\,?}%
\de{In der Einleitung von \textsl{Griechische Mythologie} vertauscht
Robert Graves Athena und Hera in Bezug auf die Zuordnung
zu Jungfrau/alte Frau, wie ich es auch zeitweise getan hatte,
also vielleicht beides wahr, wie Yin und Yang\,?}

% page
\item
\en{Zhuangzi’s famous butterfly dream:}%
\de{Zhuangzis berühmter Schmetterlingstraum:}

\en{“Once Chuang Tzu dreamt that he was a butterfly,
a fluttering butterfly who felt at ease and happy
and knew nothing of Chuang Tzu. Suddenly he woke
up: Then he was again really and truly Chuang Tzu.
Now I do not know whether Chuang Tzu dreamt that
he was a butterfly or whether the butterfly dreamt
that it was Chuang Tzu, even though there is
certainly a difference between Chuang Tzu and the
butterfly. This is how the change of things is.”
(translated by me from the Wilhelm translation to German)}%
\de{“Einst träumte Zhuangzu, dass er ein Schmetterling sei, ein
flatternder Schmetterling, der sich wohl und glücklich fühlte
und nichts wusste von Zhuangzu. Plötzlich wachte er auf: da
war er wieder wirklich und wahrhaftig Zhuangzu. Nun weiss
ich nicht, ob Zhuangzu geträumt hat, dass er ein Schmetterling
sei, oder ob der Schmetterling geträumt hat, dass
er Zhuangzu sei, obwohl doch zwischen Zhuangzu und dem
Schmetterling sicher ein Unterschied ist. So ist es mit der
Wandlung der Dinge.” (übersetzt von Richard Wilhelm)}

\en{The same day I had first quoted the dream here,
on the streets of Zürich, two butterflies on a truck,
21 Sep 2016 at 13:34.
White, red, black, a little yellow, even a little circle and her.}%
\de{Am selben Tag wo ich den Traum hier zum ersten Mal zitiert
hatte, auf den Strassen von Zürich, zwei Schmetterlinge auf
einem Lastwagen, 21. Sept.\ 2016 um 13:34 Uhr. Weiss, rot, 
schwarz, ein bisschen gelb, sogar ein kleiner Kreis und sie.}

\en{(In Apuleius’ encounter with Isis, it is left open whether
he was “just dreaming” or “it really happened”.)}%
\de{(Bei Apuleius’ Begegnung mit Isis bleibt offen, ob er das “nur
geträumt” hatte oder “es wirklich passiert war”.)}

\vspace{0.8mm}
\hspace{5.5mm}
\includegraphics[scale=0.08]{i-papillons.jpg}

\vspace{0.2mm}
\en{The image is by Elena Vizerskaya (Getty Images 108350631);
I bought the rights to use it, too, just to be safe.}%
\de{Das Bild ist von Elena Vizerskaya (Getty Images 108350631);
ich habe die Nutzungsrechte dazu zur Sicherheit erworben.}

\item
\en{(The walking cat of the \textsl{metamorphosis} section came to me
at Delphi in Greece on Tuesday, 4 September 2018 at about
13:09, ate some of my food, a dry pretzel and salmon jerky,
then, after a few burps (still a kid) and playing a little, took
a nap of about 20 minutes on my lap, then left roughly in the
direction of the Athena Pronoia temple, where I had been a
bit earlier. During these few minutes there were no doubts
what to do and felt so good, like having a child to care for.
Was the AC maybe even an oracle for the AC of $\pi$, with the
moon maybe late at glowing quincunxes, or early spring with
almost shared progressed moons\,? Or a triangle with wings
or maybe two lovers strolling through a secluded walk\,?)}%
\de{(Die laufende Katze der Sektion \textsl{metamorphose} kam in Delphi
in Griechenland am Dienstag, dem 4. September 2018 um ca.\
13:09 Uhr zu mir, ass etwas von meinem Essen, eine trockene
Brezel und Lachs-Jerky, dann, nach ein paar Rülpsern (immer
noch ein Junge) und ein wenig Spielen, machte sie ein Nickerchen
von etwa 20 Minuten auf meinem Schoss, dann lief sie
ungefähr in Richtung des Athena-Pronoia-Tempels weg, wo
ich etwas zuvor gewesen war. Während dieser wenigen
Minuten gab es für mich keine Zweifel, was zu tun war und
fühlte sich so gut an, als hätte ich ein Kind um mich drum zu
kümmern. War der AC vielleicht sogar ein Orakel für den AC
von $\pi$, mit dem Mond vielleicht spät in leuchtenden Quincunxen,
oder früh im Frühling mit fast gemeinsamen progressiven
Monden\,? Oder ein Dreieck mit Flügeln oder vielleicht zwei
Verliebte, die einen abgeschieden Weg entlangschlendern\,?)}

\vspace{1mm}\hspace{5.5mm}
\includegraphics[scale=0.12]{i-cat.jpg}

\end{list}
