\section{\en{star signs}\de{sternzeichen}}

\en{Star signs in the Western zodiac seem to reflect transitions
between elements within Aristotle’s circle.}%
\de{Die Sternzeichen im westlichen Tierkreis scheinen Über\-gän\-ge
zwischen den Elementen im Kreis von Aristoteles
widerzuspiegeln.}

\en{Fire signs seem to transform from earth via fire to air,
while water is missing, thus desired:}%
\de{Feuerzeichen scheinen sich von Erde via Feuer in Luft zu
verwandeln, wobei Wasser fehlt, also ersehnt wird:}

\noindent
\en{\includegraphics[scale=0.1198]{i-fire.jpg}}%
\de{\includegraphics[scale=0.1198]{i-feuer.jpg}}

\en{The archetypal image is simply a fire that transforms
wood (earth) to smoke (air). Aries as a young fire has
most earth, Leo most fire, Sagittarius most air.}%
\de{Das archetypische Bild ist einfach ein Feuer, das Holz
(Erde) in Rauch (Luft) verwandelt. Der Widder als junges
Feuer hat am meisten Erde, der Löwe am meisten Feuer,
der Schütze am meisten Luft.}

\en{In psychological astrology a wound is a central theme
for the two later fire signs Leo and Sagittarius, namely for
the fisher king in Perceval and Chiron in mythology.}%
\de{In der psychologischen Astrologie ist eine Wunde ein zentrales
Thema für die beiden späteren Feuerzeichen Löwe
und Schütze, nämlich für den Fischerkönig in Parzival und
Chiron in der Mythologie.}

\en{In the model that wound is simply the human body
(earth) that is wounded by the fire of life, as any human
body must die one day. Only what is learned in life can be
formulated in words (air) and can thus be passed on to later
generations, thus becomes immortal in a way. So there is a
transformation from mortal body to immortal mind, or
from animal via man/king to god.}%
\de{Im Modell ist diese Wunde einfach der menschliche
Körper (Erde), der durch das Feuer des Lebens verwundet
wird, da jeder menschliche Körper eines Tages sterben
muss. Nur das, was man im Leben gelernt hat, kann in
Worte gefasst werden (Luft) und kann so an spätere
Generationen weitergegeben werden, wird also in gewisser Weise
unsterblich. Es gibt also eine Transformation vom sterblichen
Körper zum unsterblichen Geist, oder von Tier über
Mensch/König zu Gott.}

\en{Learning and getting compassion—the element water
that is missing in the transformation of the fire signs—in
the process is a vital goal for older fire signs.}%
\de{Das Erlernen und Erlangen von Mitgefühl—das Element
Wasser, das bei der Transformation der Feuerzeichen
fehlt—ist dabei ein wichtiges Ziel für ältere Feuerzeichen.}

% page
\en{Air signs seem to transform from fire via air to water,
while earth is missing, thus desired:}%
\de{Luftzeichen scheinen sich von Feuer via Luft in Wasser
zu verwandeln, wobei Erde fehlt, also ersehnt wird:}

\noindent
\en{\includegraphics[scale=0.1198]{i-air.jpg}}%
\de{\includegraphics[scale=0.1198]{i-luft.jpg}}

\en{The archetypal image is a cloud (air), which emits both
lightning (fire) and rain (water). Gemini as young air has
most fire, Libra most air, Aquarius most water.}%
\de{Das archetypische Bild ist eine Wolke (Luft), die sowohl
Blitze (Feuer) als auch Regen (Wasser) aussendet. Zwillinge
als junge Luft haben das meiste Feuer, Waage die meiste
Luft, Wassermann das meiste Wasser.}

\en{Paris, who is associated with Libra, chose Aphrodite’s
offering of love and marriage with Helena, the most
beautiful woman in the world, hence love (water) and thus the
possibility for the missing element earth in the form of
children as fruits of love. Similarly, the opening of Pandora’s
Box, associated with Aquarius, symbolizes birth.}%
\de{Paris, der mit der Waage assoziiert wird, wählte das
Liebes- und Heiratsangebot der Aphrodite mit Helena, der
schönsten Frau der Welt, also die Liebe (Wasser) und
damit die Möglichkeit für das fehlende Element Erde in Form
von Kindern als Früchte der Liebe. In ähnlicher Weise
symbolisiert das Öffnen der Büchse der Pandora, die dem
Wassermann zugeordnet wird, eine Geburt.}

\en{Water signs seem to transform from earth via water to
air, while fire is missing, thus desired:}%
\de{Wasserzeichen scheinen sich von Erde via Wasser in
Luft zu verwandeln, wobei Feuer fehlt, also ersehnt wird:}

\noindent
\en{\includegraphics[scale=0.1198]{i-water.jpg}}%
\de{\includegraphics[scale=0.1198]{i-wasser.jpg}}

\en{The archetypal image is a river with Cancer as a source
and young river emerging from the mountains, maybe from
a glacier (earth), merging with more and more rivers and
becoming a stream as Scorpio (water) and finally flowing
into the sea as Pisces from where most water eventually
evaporates again (air), by the power of the sun (fire), the
missing element and goal for the water signs.}%
\de{Das archetypische Bild ist ein Fluss, mit dem Krebs als
Quelle und jungem Fluss, der aus den Bergen entspringt,
vielleicht aus einem Gletscher (Erde), sich mit immer mehr
Flüssen vereinigt und als Skorpion zu einem Strom wird
(Wasser) und schliesslich als Fische ins Meer fliesst, von
wo aus das meiste Wasser schliesslich wieder verdunstet
(Luft), durch die Kraft der Sonne (Feuer), das fehlende
Element und Ziel für die Wasserzeichen.}

\en{So, the transition is, like for the fire signs, from earth to
air, but this time for a passive, female element. The river
that flows down to the sea is more fated than fire, since it
is passive, it cannot resist the movement.}%
\de{Der Übergang ist also, wie bei den Feuerzeichen, von
der Erde zur Luft, aber dieses Mal für ein passives, weibliches
Element. Der Fluss, der zum Meer hinunterfliesst, ist
schicksalhafter als das Feuer, denn er ist passiv, er kann
sich der Bewegung nicht widersetzen.}

\en{But the way up in the end towards light is important,
like, for example, for the crab that bit Heracles into his
ankle while he was fighting the Hydra in the swamps, and
got its place in the sky as the constellation Cancer.}%
\de{Aber der Weg nach oben, hin zum Licht, ist wichtig, wie
zum Beispiel für die Krabbe, die Herakles in den Knöchel
biss, als er in den Sümpfen gegen die Hydra kämpfte, und
die ihren Platz am Himmel als Sternbild Krebs bekam.}

% page
\en{Earth signs seem to transform from fire via earth to
water, while air is missing, thus desired:}%
\de{Erdzeichen scheinen sich von Feuer via Erde zu Wasser
zu verwandeln, wobei Luft fehlt, also ersehnt wird:}

\noindent
\en{\includegraphics[scale=0.1198]{i-earth.jpg}}%
\de{\includegraphics[scale=0.1198]{i-erde.jpg}}

\en{The archetypal image is a tree with Taurus focussing on
the directly visible, but short-lived beauties of the tree that
grow with the power of the sun (fire), Capricorn restraining
himself to the parts of the tree that persist across seasons
and which keep it from falling down, namely trunk and
roots, which feed it with water and the substances diluted
in it, and Virgo in between, between beauty and fate.}%
\de{Das archetypische Bild ist ein Baum, wobei sich der
Stier auf die unmittelbar sichtbaren, aber kurzlebigen Schön\-hei\-ten
des Baumes konzentriert, die mit der Kraft der Sonne
(Feuer) wachsen, der Steinbock sich auf die Teile des Baumes
beschränkt, die über die Jahreszeiten hinweg bestehen und
ihn vor dem Umfallen bewahren, nämlich Stamm und
Wurzeln, die ihn auch mit Wasser und den darin verdünnten
Stoffen versorgen, und die Jungfrau ist dazwischen,
zwischen Schönheit und Schicksal.}

\en{It is this fate or necessity, which creates minimal structures 
like the branches and roots of a tree, thus order, the
abstract element air.}%
\de{Es ist dieses Schicksal oder diese Notwendigkeit, welche
minimale Strukturen wie Äste und Wurzeln eines Baumes
schafft, also Ordnung, das abstrakte Element Luft.}

\en{This solves also the riddle that even though Virgo is
often depicted as being very concerned about order, many
Virgos do not keep their lives and homes in strict order. It
is Virgo for whom order is an \textsl{issue}, for Capricorn it is a
\textsl{given} and for Taurus it is not that important, except a bit,
as Taurus is transforming from fire to earth.}%
\de{Das löst auch das Rätsel, dass, obwohl die Jungfrau oft
als sehr ordnungsliebend dargestellt wird, viele Jungfrauen
ihr Leben und ihr Zuhause nicht in strikter Ordnung halten.
Für die Jungfrau ist Ordnung ein \textsl{Thema}, für den Steinbock
ist sie eine \textsl{Gegebenheit} und für den Stier ist sie nicht so
wichtig, ausser ein bisschen, da der Stier sich von Feuer zu
Erde wandelt.}

\en{Persephone, who is associated with Virgo, was collecting
flowers as a maiden, looking at the sunny (fire) side
of life, but already starting to look down to earth, starting
to wonder about how things work, what makes the flowers
grow, etc., when the earth opened up, Hades abducted her
and she became his wife, the queen of the underworld.}%
\de{Persephone, die mit der Jungfrau assoziiert wird,
sammelte als Jungfrau Blumen und betrachtete die sonnige
(Feuer-)Seite des Lebens, aber sie begann bereits, auf die
Erde hinunterzublicken und sich zu fragen, wie die Dinge
funktionieren, was die Blumen zum Wachsen bringt usw.,
als sich die Erde öffnete, Hades sie entführte und sie seine
Frau wurde, die Königin der Unterwelt.}

\en{For all elements transitions start with a dry element and
end with a wet one. This mirrors that often when one gets
older, one realizes that things are not so clearly and reliably
what they appeared to be when first encountered.}%
\de{Für alle Elemente beginnen die Wandlungen mit einem
trockenen Element und enden mit einem feuchten. Darin
spiegelt sich wider, dass man mit zunehmendem Alter oft
feststellt, dass die Dinge nicht so klar und zuverlässig das
sind, was sie zu sein schienen, als man sie erstmals antraf.}

{\small\begin{center}
\begin{tabular}{|c|rcccl|c|c|}\hline
\en{\textbf{element} & \multicolumn{5}{|c|}{\textbf{transition}} & \textbf{desired} & \textbf{image} \\ \hline
fire & earth & \hspace{-3.3mm}$\rightarrow$ & \hspace{-3.3mm}fire & \hspace{-3.3mm}$\rightarrow$ & \hspace{-3.3mm}air & water & fire \\ \hline
air & fire & \hspace{-3.3mm}$\rightarrow$ & \hspace{-3.3mm}air & \hspace{-3.3mm}$\rightarrow$ & \hspace{-3.3mm}water & earth & cloud \\ \hline
water & earth & \hspace{-3.3mm}$\rightarrow$ & \hspace{-3.3mm}water & \hspace{-3.3mm}$\rightarrow$ & \hspace{-3.3mm}air & fire & river \\ \hline
earth & fire & \hspace{-3.3mm}$\rightarrow$ & \hspace{-3.3mm}earth & \hspace{-3.3mm}$\rightarrow$ & \hspace{-3.3mm}water & air & tree \\ \hline}%
\de{\textbf{element} & \multicolumn{5}{|c|}{\textbf{wandlung}} & \textbf{wunsch} & \textbf{bild} \\ \hline
feuer & erde & \hspace{-3.3mm}$\rightarrow$ & \hspace{-3.3mm}feuer & \hspace{-3.3mm}$\rightarrow$ & \hspace{-3.3mm}luft & wasser & feuer \\ \hline
luft & feuer & \hspace{-3.3mm}$\rightarrow$ & \hspace{-3.3mm}luft & \hspace{-3.3mm}$\rightarrow$ & \hspace{-3.3mm}water & erde & wolke \\ \hline
wasser & erde & \hspace{-3.3mm}$\rightarrow$ & \hspace{-3.3mm}wasser & \hspace{-3.3mm}$\rightarrow$ & \hspace{-3.3mm}luft & feuer & fluss \\ \hline
erde & feuer & \hspace{-3.3mm}$\rightarrow$ & \hspace{-3.3mm}erde & \hspace{-3.3mm}$\rightarrow$ & \hspace{-3.3mm}wasser & luft & baum \\ \hline}
\end{tabular}
\end{center}}

% page
\en{How about trying to synthesize the properties of the
star signs formally from the transition between the elements
defined by in/out and rest/move alone, without relying on
properties of actual fire, air, water and earth\,?}%
\de{Wie wäre es, zu versuchen, die Eigenschaften der Sternzeichen
formal aus dem Übergang zwischen den Elementen
definiert alleine durch innen/aussen und ruht/bewegt zu
synthetisieren, ohne sich auf die Eigenschaften von realem
Feuer, Luft, Wasser und Erde zu stützen\,?}

{\small\begin{center}
\begin{tabular}{|l|}\hline
ero\,$\rightarrow$\,\textbf{emo}\,$\rightarrow$\,eri\, ($\rightarrow\,$emi) \\ \hline
emo\,$\rightarrow$\,\textbf{eri}\,$\rightarrow$\,emi\, ($\rightarrow\,$ero) \\ \hline
ero\,$\rightarrow$\,\textbf{emi}\,$\rightarrow$\,eri\, ($\rightarrow\,$emo) \\ \hline
emo\,$\rightarrow$\,\textbf{ero}\,$\rightarrow$\,emi\, ($\rightarrow\,$eri) \\ \hline
\end{tabular}
\end{center}}

\en{Libra, for example, learns from observation of motion
outside (fire) and inside (water). Since Libra’s transition is
towards water, the gift of “inner vision” is given to Teiresias
by Zeus and outer vision is reduced by Hera, except for
observing omens, which are arguably just outer reflections
of collective inner intentions. The transformation would not
be exclusively in the direction shown above, rather there
would be some back and forth, but summed up, it would
be; it would essentially lead inward for all star signs.}%
\de{Die Waage zum Beispiel lernt durch die Beobachtung
der Bewegung aussen (Feuer) und innen (Wasser). Da die
Wandlung der Waage zum hin Wasser ist, wird Teiresias
von Zeus die Gabe des “inneren Sehens” verliehen und
Hera reduziert die äussere Sicht, mit Ausnahme der Beobachtung
von Omen, die wohl einfach äussere Widerspiegelungen
von kollektiven inneren Absichten sind. Die Wandlung
würde nicht ausschliesslich in der oben gezeigten Richtung
verlaufen, sondern es gäbe ein gewisses Hin und Her, aber
in der Summe würde sie für alle Sternzeichen im Wesentlichen
nach innen führen.}

\en{For earth signs, the transition would be to channel motion
outside into a fixed “vessel”\,and then to let it flow
again inside, desiring to learn something about nature. For
air signs, the transition would be to observe outside, learn
its laws inside and thus also to derive how things flow
inside, desiring to change outer reality for the better. For
water signs, the transition would be to let impressions of
the outer state flow inside and learn from them, desiring
to get things outside moving. For fire signs, the transition
would be to get things outside moving and then learning
inside how they work, desiring to feel the inner flow.}%
\de{Für die Erdzeichen würde der Übergang darin bestehen,
die Bewegung aussen in ein festes “Gefäss” zu kanalisieren
und sie dann wieder nach innen fliessen zu lassen, mit
dem Wunsch, etwas über die Natur zu lernen. Für die
Luftzeichen wäre der Übergang, aussen zu beobachten, diese
Gesetze innen zu lernen und so auch abzuleiten, wie die
Dinge innen fliessen, mit dem Wunsch, die äussere Realität
zum Besseren zu verändern. Für die Wasserzeichen wäre
der Übergang, Eindrücke des Zustandes aussen nach innen
fliessen zu lassen und daraus zu lernen, mit dem Wunsch,
die Dinge draussen in Bewegung zu bringen. Für die Feuerzeichen
würde der Übergang darin bestehen, die Dinge
aussen in Bewegung zu bringen und dann innen zu lernen,
wie sie funktionieren, und den Wunsch zu haben, den
inneren Fluss zu spüren.}

\en{As another example, the abduction of Kore into the
underworld as Persephone is her way into inner worlds, where
Hades is more deeply immersed, as Scorpio is already more
inside than Virgo, while both are still connected to outside:
Hades at least went out to abduct her; she in the end only
spends part of the year inside, down in the underworld.}%
\de{Als weiteres Beispiel ist die Entführung von Kore in die
Unterwelt als Persephone ihr Weg in innere Welten, in die
Hades bereits tiefer eingetaucht ist, da der Skorpion bereits
mehr innen ist als die Jungfrau, während beide noch mit
aussen verbunden sind: Hades ging zumindest hinaus, um
sie zu entführen; sie verbringt schliesslich nur einen Teil des
Jahres im Inneren, unten in der Unterwelt.}

\en{Of course, this was just a partial sketchy first view.}%
\de{Das war jetzt natürlich nur ein teilweise skizzenhafter
erster Ausblick.}

\subsection{\en{leads}\de{fährten}}

\small
\begin{list}{$\bullet$}{\setlength{\leftmargin}{10pt}}

\item
\en{For more detailed expositions, see the longer article \textsl{Elementary star signs}
under \cometartemis\ or my book \textsl{Elementary Star Signs},
which are both also available in German.}%
\de{Ausführlichere Erläuterungen finden sich im längeren Artikel
\textsl{Elementare Sternzeichen} unter \cometartemis\ oder in meinem
Buch \textsl{Elementare Sternzeichen}, welche beide auch auf
Englisch erhältlich sind.}

\item
\en{The four tasks of Psyche in Apuleius’ \textsl{The Golden Ass}
seem to mirror the same transitions very beautifully and precisely,
in the order earth-water-fire-air, with goals air-fire-water-earth.}%
\de{Die vier Aufgaben der Psyche in Apuleius’ \textsl{Der goldene Esel}
scheinen dieselben Übergänge sehr schön und genau zu spiegeln,
in der Reihenfolge Erde-Wasser-Feuer-Luft, mit den Zielen
Luft-Feuer-Wasser-Erde.}

\item
\en{Are there similar elemental transitions in the Chinese zodiac\,?}%
\de{Gibt es ähnliche elementare Übergänge auch im chinesischen
Tierkreis\,?}

\item
\en{In 2005 Alois Treindl told me that he had shown my ‘A few
new discoveries in physics’ document of 2002 at the time to
Liz Greene who had deemed it “too theoretical”, which now
reminds me of Plato’s cave, see ‘metamorphosis’ leads.}%
\de{2005 erzählte mir Alois Treidl, dass er mein Dokument ‘A few
new discoveries in physics’ von 2002 damals Liz Greene
gezeigt hatte, die es als “zu theoretisch” erachtet hatte, was
mich nun an Platons Höhlengleichnis erinnert, siehe
‘metamorphose’ Fährten.}

\end{list}
