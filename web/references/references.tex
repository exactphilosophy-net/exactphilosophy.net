\section{\en{References}\de{Referenzen}}

\small
\begin{list}{$\bullet$}{\setlength{\leftmargin}{10pt}}

\item
\en{Andreas Schöter. \textsl{Bipolar Change}. Journal of Chinese
Philos\-ophy. Volume 35, Issue 2, p.~297-317 (June 2008).}%
\de{Andreas Schöter. \textsl{Bipolar Change} (\textsl{Bipolarer Wandel}).
Journal of Chinese Philosophy. Band 35, Ausgabe 2, s.~297-317
(Juni 2008).}

\en{\textbf{Abstract}\,
I reconsider the natural characterization of change
and non-change that arises from the algebraic approach: this
sees change as yang in contrast to nonchange, which is yin.
Following a persuasive example from Alain Stalder, rather
than consider change solely in contrast to non-change, I
develop a formal characterization of different forms of change
considered relative to each other. This extension allows the
internal structure of a change to be made explicit in a new
way, bifurcating the change into yang parts and yin parts. I
call this extended definition of change \textsl{bipolar change}.}%
\de{\textbf{Kurzfassung}\,
Ich überdenke die natürliche Charakterisierung
von Wandel und Nicht-Wandel, die sich aus dem
algebraischen Ansatz ergibt: Dieser sieht Wandel als Yang im
Gegensatz zu Nicht-Wandel, der Yin ist. In Anlehnung an ein
überzeugendes Beispiel von Alain Stalder, betrachte ich den
Wandel nicht nur als Gegensatz zu Nicht-Wandel, sondern
entwickle eine formale Charakterisierung der verschiedenen
Formen des Wandels, die relativ zueinander betrachtet werden.
Diese Erweiterung ermöglicht es, die innere Struktur eines
Wandels auf eine neue Art und Weise explizit zu machen,
indem der Wandel in Yang- und Yin-Anteile aufgegabelt wird.
Ich nenne diese erweiterte Definition von Wandel \textsl{bipolaren
Wandel}.}

\en{\textbf{Links}\, [Preprint] [Publication]}%
\de{\textbf{Links}\, [Preprint] [Publikation]}

\item
\en{Thread \textsl{exactphilosophy.net 2018 (1 Nov)} at the Usenet
newsgroup alt.philosophy.taoism (Nov 2018).}%
\de{Faden \textsl{exactphilosophy.net 2018 (1 Nov)} in der Usenet
Newsgruppe alt.philosophy.taoism (Nov.\ 2018).}

\en{\textbf{Links}\, [Archive]}%
\de{\textbf{Links}\, [Archiv]}

\end{list}
