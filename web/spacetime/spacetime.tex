
\section{\en{space and time}\de{raum und zeit}}

\en{Imagine you just now started to look at the world.}%
\de{Stell dir vor, du hättest erst gerade jetzt damit begonnen,
die Welt anzuschauen.}

\vspace{-1.8mm}
\begin{center}
\includegraphics[scale=0.1198]{i-world.jpg}
\end{center}
\vspace{-1.8mm}

\en{One of the first things you notice is space. There is you
and an outside world you can see, and you can see more
than one thing. What separates you and what you can see,
and what separates the different things you see, is space in
its most immediate definition.}%
\de{Eins der ersten Dinge, die dir auffallen, ist der Raum. Es
gibt dich und eine Aussenwelt, die du sehen kannst, und du
kannst mehr als ein Ding sehen. Was dich und was du siehst
trennt, und was die verschiedenen Dinge, die du siehst,
trennt, ist Raum in seiner unmittelbarsten Definition.}

\en{Then you also quickly notice that some things move
and others do not. This is time, again in its most
immediate definition, as motion or being at rest.}%
\de{Dann bemerkst du auch rasch, dass sich manche Dinge
bewegen und andere nicht. Das ist die Zeit, wiederum in
ihrer unmittelbarsten Definition, als Bewegung oder Ruhe.}

\vspace{-1.8mm}
\begin{center}
\en{\includegraphics[scale=0.1198]{i-elements.jpg}}%
\de{\includegraphics[scale=0.1198]{i-elements-de.jpg}}
\end{center}
\vspace{-1.2mm}

\en{Things can rest or move outside and inside the mind.
Thus there would a priori be 4 different kinds of things:
What moves outside, what rests outside, what moves inside,
and what rests inside. Let me call them \textsl{elements}
and give them the following names: \textsl{emo},
\textsl{ero}, \textsl{emi }and \textsl{eri}.}%
\de{Dinge können ausserhalb und innerhalb des Geistes
ruhen oder sich bewegen. Daher gäbe es a priori 4 verschiedene
Arten von Dingen: Was sich aussen bewegt, was aussen
ruht, was sich innen bewegt und was innen ruht. Ich nenne
sie \textsl{Elemente} und gebe ihnen die folgenden Namen:
\textsl{emo}, \textsl{ero}, \textsl{emi} und \textsl{eri}.}

{\small\begin{center}
\en{\begin{tabular}{|l|l|l|}\hline
\textbf{emo} & \textbf{m}\hspace{0.1mm}oves & \textbf{o}\hspace{0.1mm}utside \\ \hline
\textbf{ero} & \textbf{r}\hspace{0.1mm}ests & \textbf{o}\hspace{0.1mm}utside \\ \hline
\textbf{emi} & \textbf{m}\hspace{0.1mm}oves & \textbf{i}\hspace{0.1mm}nside \\ \hline
\textbf{eri} & \textbf{r}\hspace{0.1mm}ests & \textbf{i}\hspace{0.1mm}nside \\ \hline}%
\de{\begin{tabular}{|l|ll|ll|}\hline
\textbf{emo} & bewegt & \hspace{-0.8em}(\textbf{m}\hspace{0.1mm}oves) & aussen & \hspace{-0.8em}(\textbf{o}\hspace{0.1mm}utside) \\ \hline
\textbf{ero} & ruht & \hspace{-0.8em}(\textbf{r}\hspace{0.1mm}ests) & aussen & \hspace{-0.8em}(\textbf{o}\hspace{0.1mm}utside) \\ \hline
\textbf{emi} & bewegt & \hspace{-0.8em}(\textbf{m}\hspace{0.1mm}oves) & innen & \hspace{-0.8em}(\textbf{i}\hspace{0.1mm}nside) \\ \hline
\textbf{eri} & ruht & \hspace{-0.8em}(\textbf{r}\hspace{0.1mm}ests) & innen & \hspace{-0.8em}(\textbf{i}\hspace{0.1mm}nside) \\ \hline}
\end{tabular}
\end{center}}

\en{Where “emo” is an acronym for $\!$“\textbf{e}lement that \textbf{m}oves
\textbf{o}utside”, and accordingly for the other three.}%
\de{Wobei “emo” ein Akronym ist für $\!$“\textbf{e}lement that \textbf{m}oves
\textbf{o}utside”, also Element, das sich aussen bewegt, und analog
für die anderen drei.}

% page
\subsection{\en{leads}\de{fährten}}

\normalsize
\en{Some literature quotes, ideas and different points of view.
Always also see ‘\cometartemis’ for eventually articles that may
expose some topics in a more contemporarily amenable way.}%
\de{Ein paar Literaturzitate, Ideen und andere Sichtweisen.
Siehe immer auch ‘\cometartemis’ für ev.\ Artikel, welche einige
Themen zeitgenössisch zugänglicher beleuchten könnten.}

\small
\begin{list}{$\bullet$}{\setlength{\leftmargin}{10pt}}

\item
\en{A priori there is just one experience of being, which encompasses
all that is. In that sense, space and time or the elements
as tentatively defined above, may already be all that is.
A conscious mind or self separate of the elements may a priori
not be necessary, nor would it have to be limited to only part
of the elements (like inside).}%
\de{A priori gibt es nur eine Erfahrung des Seins, die alles, was
ist, umfasst. In diesem Sinne könnten Raum und Zeit oder die
Elemente, wie oben vorläufig definiert, bereits alles sein, was
ist. Ein bewusster Geist oder von den Elementen getrenntes
Selbst müsste a priori nicht notwendig sein, noch müsste es
auf einen Teil der Elemente (wie auf innen) beschränkt sein.}

\en{But still some considerations related to an observing self
further below. Plus, most likely related, the definition of a fifth
element e5 in the next section.}%
\de{Aber trotzdem weiter unten noch einige Überlegungen zu
einem beobachtenden Selbst. Plus, sehr wahrscheinlich damit
zusammenhängend, die Definition eines fünften Elements e5
in der nächsten Sektion.}

\item
\en{Immanuel Kant. \textsl{The Critique of Pure Reason.} 1787.}%
\de{Immanuel Kant. \textsl{Kritik der reinen Vernunft.} 1787.}

\en{In the early chapters, Kant discloses that some observable
things cannot be isolated from the self, but instead appear
to be themselves \textsl{a priori} necessary for thinking and
observation. These a priori concepts include space and time in
their immediate sense—the structure in which things appear
in the mind and seem to exist outside of it.}%
\de{In den ersten Kapiteln zeigt Kant, dass einige beobachtbare
Dinge nicht vom Selbst isoliert werden können, sondern
stattdessen selbst \textsl{a priori} notwendig für das Denken und die
Beobachtung zu sein scheinen. Zu diesen apriorischen Begriffen
gehören Raum und Zeit in ihrem unmittelbaren Sinn—die
Struktur, in der die Dinge im Geist erscheinen und ausserhalb
von ihm zu existieren scheinen.}

\item
\en{“By means of the external sense (a property of the mind), we
represent to ourselves objects as without us, and these all in
space. Herein alone are their shape, dimensions, and relations
to each other determined or determinable. […$\!$]\, Space is not a
conception which has been derived from outward experiences.
For, in order that certain sensations may relate to something
without me (that is, to something which occupies a different
part of space from that in which I am); in like manner, in order
that I may represent them not merely as without, of, and near
to each other, but also in separate places, the representation
of space must already exist as a foundation. […$\!$]\, We never
can imagine or make a representation to ourselves of the non-existence
of space, though we may easily enough think that
no objects are found in it.” (translated by J.~Meiklejohn)}%
\de{“Vermittelst des äusseren Sinnes (einer Eigenschaft unsres
Gemüts) stellen wir uns Gegenstände als ausser uns, und diese
insgesamt im Raume vor. Darinnen ist ihre Gestalt, Grösse
und Verhältnis gegen einander bestimmt, oder bestimmbar.
[…$\!$]\, Der Raum ist kein empirischer Begriff, der von äusseren
Erfahrungen abgezogen worden. Denn damit gewisse
Empfindungen auf etwas ausser mich bezogen werden (d.i.\ auf
etwas in einem andern Orte des Raumes, als darinnen ich
mich befinde), imgleichen damit ich sie als ausser \textsl{und neben}
einander, mithin nicht bloss verschieden, sondern als in
verschiedenen Orten vorstellen könne, dazu muss
die Vorstellung des Raumes schon zum Grunde liegen. Demnach kann
die Vorstellung des Raumes nicht aus den Verhältnissen der
äussern Erscheinung durch Erfahrung erborgt sein, sondern
diese äussere Erfahrung ist selbst nur durch gedachte
Vorstellung allererst möglich. […$\!$]\, Man kann sich niemals eine
Vorstellung davon machen, dass kein Raum sei, ob man sich
gleich ganz wohl denken kann, dass keine Gegenstände darin
angetroffen werden.”}

\item
\en{“Time is not an empirical conception. For neither coexistence
nor succession would be perceived by us, if the representation
of time did not exist as a foundation a priori. […$\!$]\, With
regard to phenomena in general, we cannot think away time
from them, and represent them to ourselves as out of and
unconnected with time, but we can quite well represent to
ourselves time void of phenomena.”}%
\de{“Die Zeit ist kein empirischer Begriff, der irgend von einer
Erfahrung abgezogen worden. Denn das Zugleichsein oder
Aufeinanderfolgen würde selbst nicht in die Wahrnehmung
kommen, wenn die Vorstellung der Zeit nicht a priori zum
Grunde läge. […$\!$]\, Man kann in Ansehung der Erscheinungen
überhaupt die Zeit selbsten nicht aufheben, ob man zwar
ganz wohl die Erscheinungen aus der Zeit wegnehmen kann.”}

% page
\item
\en{Arthur Schopenhauer. \textsl{The World As Will And Idea}. 1819.}%
\de{Arthur Schopenhauer. \textsl{Die Welt als Wille und Vorstellung}. 1819.}

\en{“[…$\!$] that the world which surrounds him is there only as
idea, i.e., only in relation to something else, the consciousness,
which is himself. If any truth can be asserted \textsl{a priori},
it is this: for it is the expression of the most general form of
all possible and thinkable experience: a form which is more
general than time, or space, or causality, for they all presuppose
it; and each of these, which we have seen to be just
so many modes of the principle of sufficient reason, is valid
only for a particular class of ideas; whereas the antithesis of
object and subject is the common form of all these classes,
is that form under which alone any idea of whatever kind it
may be, abstract or intuitive, pure or empirical, is possible
and thinkable.” (translated by R.~Haldane and J.~Kemp)}%
\de{“[…$\!$] dass die Welt, welche ihn umgiebt, nur als Vorstellung
da ist, d.h.\ durchweg nur in Beziehung auf ein Anderes, das
Vorstellende, welches er selbst ist. – Wenn irgendeine Wahrheit
\textsl{a priori} ausgesprochen werden kann, so ist es diese: denn
sie ist die Aussage derjenigen Form aller möglichen und
erdenklichen Erfahrung, welche allgemeiner, als alle andern, als
Zeit, Raum und Kausalität ist: denn alle diese setzen jene
eben schon voraus, und wenn jede dieser Formen, welche
alle wir als so viele besondere Gestaltungen des Satzes vom
Grunde erkannt haben, nur für eine besondere Klasse von
Vorstellungen gilt; so ist dagegen das Zerfallen in Objekt und
Subjekt die gemeinsame Form aller jener Klassen, ist diejenige
Form, unter welcher allein irgend eine Vorstellung, welcher
Art sie auch sei, abstrakt oder intuitiv, rein oder empirisch,
nur überhaupt möglich und denkbar ist.”}

\en{The word “Vorstellung” (for “idea” in the original German)
means literally something “put in front of or before you”,
spatially or chronologically.}%
\de{Das Wort “Vorstellung” (engl.\ wird “idea” verwendet) heisst
unmittelbar “etwas vor einen stellen”, räumlich oder zeitlich.}

\item
\en{If I can imagine something, is it then really inside of me\,?
Isn’t there already a separation (space) between me and what
I imagine\,? Such an extreme definition of \textsl{self }or
\textsl{inside }would mean that the self cannot have any
(consciously accessible) attributes, no memory etc., because any
such attribute of the self would be something that can be considered
by the self and would thus, by definition, not be part of the self…}%
\de{Wenn ich mir etwas vorstellen kann, ist es dann wirklich in
mir\,? Gibt es nicht bereits eine Trennung (Raum) zwischen
mir und dem, was ich mir vorstelle\,? So eine extreme Definition
von \textsl{Selbst} oder \textsl{Innen} würde bedeuten, dass das Selbst
keine (bewusst zugänglichen) Attribute, kein Gedächtnis usw.\
haben könnte, weil jedes solche Attribut des Selbst etwas
wäre, das vom Selbst betrachtet werden könnte und somit
per Definition nicht Teil des Selbst wäre…}

\item
\en{This definition of \textsl{self} reminds of the \textsl{Tao} (“way”) in Taoism.
Lao Tzu starts the \textsl{Tao Te Ching} with “The Tao that can be
Tao’ed (trodden/spoken), is not the real (unchanging) Tao”.}%
\de{Diese Definition von \textsl{Selbst} erinnert an das \textsl{Tao} (“Weg”) im
Taoismus. Laotse beginnt das \textsl{Tao Te King} mit “Das Tao das
ge-$\!$Tao’t (gegangen/gesprochen) werden kann, ist nicht das
wahre (unveränderliche) Tao”.}

\item
\en{In today’s science, organs of perception wire back what is outside
to the brain, where also mind and self would be. Maybe
the self would even be considered “more inside than inside”,
looking out first at what else is inside and then even further
out at what is outside. But how much of that might be
paradigm, and could thus change again over centuries\,?}%
\de{In der Sicht der heutigen Wissenschaft, verdrahten die
Wahrnehmungsorgane das, was draussen ist, an das Gehirn zurück,
wo sich auch der Geist und das Selbst befinden würden.
Vielleicht würde das Selbst sogar als “mehr innen als innen”
erachtet, zuerst auf das schauend, was sonst noch innen ist,
und dann noch weiter hinaus auf das, was aussen ist. Aber
wie viel davon könnte vielleicht Paradigma sein und sich daher
im Laufe der Jahrhunderte wieder ändern\,?}

\item
\en{How would rest/move be defined for other senses than vision\,?
How could eri and emi be measured inside\,? Would the only
“objective” way be to measure brain activity outside\,? Would
that be fundamental enough in this context\,? Could the self
(observer) be measured\,?}%
\de{Wie würde ruhen/bewegen für andere Sinne als das Sehen
definiert werden\,? Wie könnte man eri und emi im Inneren
messen\,? Wäre der einzige “objektive” Weg, die Gehirnaktivität
aussen zu messen\,? Wäre das in diesem Kontext fundamental
genug\,? Könnte das Selbst (Beobachter) gemessen
werden\,?}

\item
\en{Would a female observer also consider what is seen as not
being part of herself or would she rather tend to identify with
what she sees\,? (Is the own body part of the self\,? And lovers,
family, friends, house, garden, etc.\,?) In other words, is the
distinction between in and out hard or soft (gradual)\,?}%
\de{Würde eine weibliche Beobachterin das, was sie sieht, auch
als nicht zu ihr gehörend betrachten oder würde sie eher dazu
neigen, sich mit dem zu identifizieren, was sie sieht\,? (Ist
der eigene Körper Teil des Selbst\,? Und Liebhaber, Familie,
Freunde, Haus, Garten usw.\,?) Andersrum, ist die Unterscheidung
zwischen innen und aussen hart oder weich (graduell)\,?}

\item
\en{What about sleep, dreaming, trance, drunkenness\,? Why only
have a fully conscious observer\,?}%
\de{Wie steht es mit Schlaf, Träumen, Trance, Trunkenheit\,?
Wieso nur einen vollständig bewussten Beobachter haben\,?}

\item
\en{Is there already in/out for a newborn\,? I have a memory from
the age of 3 to 4 when I started to have a consciousness with
then a memory from birth (hence a memory of a memory),
which was apparently the white ceiling with lights on it in the
hospital.}%
\de{Gibt es bereits innen/aussen für ein Neugeborenes\,? Ich habe
eine Erinnerung vom Alter von 3 bis 4 Jahren als ich begann
ein Bewusstsein zu haben mit damals einer Erinnerung von
der Geburt (also eine Erinnerung an eine Erinnerung), welche
anscheinend die weisse Decke im Spital mit Lichtern darauf
war.}

\end{list}
