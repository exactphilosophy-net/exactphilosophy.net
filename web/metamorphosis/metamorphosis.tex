
\section{\en{metamorphosis}\de{metamorphose}}

\en{The next thing that one notices is that motion can start
and stop, and that changes outside and inside seem not to
be independent of each other. In other words, the elements
change, maybe even metamorphose into each other.}%
\de{Das Nächste, was einem auffällt, ist dass Bewegung beginnen
und enden kann, und dass Veränderungen aussen und
innen nicht unabhängig voneinander zu sein scheinen. Mit
anderen Worten, die Elemente verändern sich, verwandeln
sich vielleicht sogar ineinander.}

\en{What causes or allows these changes\,? Whatever it is,
it must be something fundamental, like the four elements.
So let me simply call it the fifth element, \textsl{e5}.}%
\de{Was verursacht oder erlaubt diese Veränderungen\,? Was
auch immer es ist, es muss etwas Fundamentales sein, wie
die vier Elemente. Lasst mich es daher einfach das fünfte
Element, \textsl{e5}, nennen.}

\en{Free will seems to be a part of e5. It is possible to lift
a spoon and then to throw it away, i.e.\ to get something
outside that rests into motion (ero\rarr emo). However, free
will cannot be identical to e5, as some things are much
harder to control (try lifting a tree) and things transform
all the time without conscious influence.}%
\de{Freier Wille scheint Teil von e5 zu sein. Es ist möglich,
einen Löffel anzuheben und ihn dann wegzuwerfen, d.h.\ etwas
aussen, das ruht, in Bewegung zu setzen (ero\rarr emo).
Freier Wille kann jedoch nicht identisch zu e5 sein, da manche
Dinge viel schwieriger zu kontrollieren sind (versuche
mal, einen Baum anzuheben) und sich Dinge ständig ohne
bewusste Einflussnahme wandeln.}

\en{Freedom inside the mind seems larger than outside. It
is much easier to lift a tree in the mind than a real tree
outside. But let me tackle things from a different angle:
Outside on average more things rest than move, while inside
the mind, things are almost always more flowing.}%
\de{Freiheit im Innern des Geistes scheint grösser zu sein als
aussen. Es ist viel einfacher, einen Baum im Geiste hochzuheben
als einen echten Baum aussen. Aber lasst mich die
Dinge aus einem anderen Blickwinkel angehen: Aussen
ruhen im Schnitt mehr Dinge, als sich bewegen, während im
Inneren des Geistes die Dinge fast immer fliessender sind.}

\en{For example, a tree is at rest in most situations, except
for a little movement of leaves and maybe branches. But if
you close your eyes and try to imagine a tree at rest, it will
get very hard after a few seconds not to deviate to other
thoughts and to keep the tree at rest.}%
\de{Ein Baum zum Beispiel ist in den meisten Situationen
in Ruhe, abgesehen von ein bisschen Bewegung der Blätter
und vielleicht der Äste. Wenn man aber die Augen schliesst
und versucht, sich einen ruhenden Baum vorzustellen, wird
es nach ein paar Sekunden sehr schwer, den Baum in Ruhe
zu halten und nicht zu anderen Gedanken abzuschweifen.}

\vspace{2mm}\hspace{-2mm}
\noindent
\en{\includegraphics[scale=0.1405]{i-closeeyes.jpg}}%
\de{\includegraphics[scale=0.1405]{i-closeeyes-de.jpg}}

\en{In conclusion, on average outside activity is needed to
get things moving, while inside activity is needed to keep
things at rest. More abstractly, emo and eri are thus active,
ero and emi are passive. Also, what is outside resists motion
on average more than what is inside. So emo and ero are
hard (out), emi and eri are soft (in). What moves usually
does so in various directions. Hence what rests appears to
bind, what moves appears to release.}%
\de{Zusammenfassend lässt sich sagen, dass im Schnitt aussen
Aktivität erforderlich ist, um die Dinge in Bewegung zu
bringen, während innen Aktivität erforderlich ist, um
die Dinge in Ruhe zu halten. Abstrakter ausgedrückt, sind
emo und eri daher aktiv, ero und emi passiv. Ausserdem
widersetzt sich, was aussen ist, im Durchschnitt stärker der
Bewegung als was innen ist. Also sind emo und ero hart
(aussen), emi und eri sind weich (innen). Was sich bewegt,
tut dies gewöhnlich in verschiedene Richtungen. Was ruht,
scheint also zu binden, was sich bewegt, scheint zu lösen.}

% page
{\small\begin{center}
\begin{tabular}{|l|l|l|l|l|l|}\hline
\en{\textbf{emo} & moves & outside & active & hard & release \\ \hline
\textbf{ero} & rests & outside & passive & hard & bind \\ \hline
\textbf{emi} & moves & inside & passive & soft & release \\ \hline
\textbf{eri} & rests & inside & active & soft & bind \\ \hline
\textbf{e5} & \multicolumn{5}{l|}{transforms the above elements} \\ \hline}%
\de{\textbf{emo} & bewegt & aussen & aktiv & hart & lösen \\ \hline
\textbf{ero} & ruht & aussen & passiv & hart & binden \\ \hline
\textbf{emi} & bewegt & innen & passiv & weich & lösen \\ \hline
\textbf{eri} & ruht & innen & aktiv & weich & binden \\ \hline
\textbf{e5} & \multicolumn{5}{l|}{verwandelt die obigen elemente} \\ \hline}
\end{tabular}
\end{center}}

\en{A camera can only register ero and emo, and thus only
transitions ero\lrarr emo, while transitions that would cross
between in and out seem impossible. Personal experience
might be a bit different, albeit a bit paradox, as follows.}%
\de{Eine Kamera kann nur ero und emo aufzeichnen, und
damit nur Übergänge ero\lrarr emo, währenddem Übergänge
zwischen innen und aussen ganz unmöglich scheinen. Die
persönliche Erfahrung könnte aber eine etwas andere sein,
wenn auch ein wenig paradox, wie folgt.}

\en{If you leisurely observe a scene outside, like at the
beach, usually most things will be resting, but there will
be some movement. If you then close your eyes, in my
experience, what will be immediately visible after closing
your eyes will be the few things that moved, but frozen
in movement, hence apparently a transition emo\rarr eri, a
transition in which activity is preserved.}%
\de{Wenn man entspannt eine Szene draussen beobachtet,
wie etwa am Strand, ruht normalerweise das meiste, aber
es hat auch etwas Bewegung. Wenn man dann die Augen
schliesst, sind in meiner Erfahrung die wenigen Dinge, die
unmittelbar nach dem Schliessen der Augen sichtbar sind, die
wenigen Dinge, die sich bewegt hatten, aber eingefroren
in der Bewegung, also offenbar ein Übergang emo\rarr eri, ein
Übergang, bei dem die Aktivität erhalten bleibt.}

\vspace{2.6mm}\hspace{-0.5mm}
\noindent
\includegraphics[scale=0.1172]{i-delphi.jpg}

\vspace{2mm}
\en{Accordingly, passivity outside would then yield passivity
inside, ero\rarr emi. Actively created change outside, which
more often means to get something in motion than the
other way round, usually needs active focus inside first.
Hence transitions in\lrarr out would go both ways, emo\lrarr eri
and ero\lrarr emi. Motion outside can also come to be and
stop without much activity inside, like when an apple falls
from a tree. Similarly, such things can also happen inside
without much activity outside. Hence there would apparently
also be transitions emo\lrarr ero and emi\lrarr eri. All in
all, apparently a circle ero\lrarr emo\lrarr eri\lrarr emi\lrarr ero, while
other transitions would at least be less frequent.}%
\de{Dementsprechend würde Passivität aussen zu Passivität
innen führen, ero\rarr emi. Aktiv erzeugte Veränderungen
aussen, was öfters bedeutet, etwas in Bewegung zu setzen als
umgekehrt, braucht typischerweise zuerst einen aktiven Fokus
im Inneren. Daher würden Übergänge innen\lrarr aussen
in beide Richtungen vorkommen, emo\lrarr eri und ero\lrarr emi.
Bewegung aussen kann auch ohne viel Aktivität innen
entstehen und vergehen, z.B.\ wenn ein Apfel von einem Baum
fällt. In ähnlicher Weise können solche Dinge auch innen
geschehen ohne viel Aktivität draussen. Also gäbe es offenbar
auch Übergänge emo\lrarr ero und emi\lrarr eri. Alles in allem,
offenbar ein Kreis ero\lrarr emo\lrarr eri\lrarr emi\lrarr ero, während
andere Übergänge zumindest weniger häufig wären.}

\vspace{3.2mm}\hspace{13mm}
\noindent
\en{\includegraphics[scale=0.055]{i-circle.jpg}}%
\de{\includegraphics[scale=0.055]{i-circle-de.jpg}}

\vspace{1.9mm}
% page
\en{The elements could a priori interface in six ways:
emo-ero, emi-eri, emo-emi, ero-eri, emo-eri, emi-ero. Any
interface between elements must be unobservable, because
otherwise it would be something that is perceived inside
or outside, i.e.\ it would \textsl{be} one of the four elements.
The same argument can be made for e5, of course.}%
\de{Die Elemente könnten sich a priori an sechs Schnittstellen
berühren: emo-ero, emi-eri, emo-emi, ero-eri, emo-eri,
emi-ero. Jede Schnittstelle zwischen den Elementen muss
unbeobachtbar sein, denn sonst wäre sie etwas, das innen
oder aussen wahrgenommen würde, d.h.\ sie \textsl{wäre} eines
der vier Elemente. Das gleiche Argument lässt sich natürlich
auch für e5 anführen.}

\en{Let me imagine an interface in-out as an infinitely thin
membrane. And imagine, say, a blob of ero at the interface.
If it remained passive, it could start to flow while
permeating inside, becoming emi, or the other way round,
and similarly for emo and eri.}%
\de{Lasst mich mir die Schnittstelle innen-aussen als unendlich
dünne Membran vorstellen. Und stellen wir uns zum
Beispiel einen Klecks ero an der Grenzfläche vor. Wenn
der Klecks passiv bliebe, könnte er zu fliessen beginnen
während er nach innen durchdringt, zu emi werden, oder
umgekehrt rum, und analog wäre es mit emo und eri.}

\vspace{2mm}
\noindent\hspace{16mm}
\includegraphics[scale=0.15]{i-inout.jpg}

\vspace{1.5mm}
\en{Since interfaces between elements would be invisible,
just like e5, they might a priori have an arbitrarily complex
nature, so that the above picture is a priori maybe just one
of the simplest ways of seeing them.}%
\de{Da die Schnittstellen zwischen den Elementen unsichtbar
wären, genau wie e5, könnten sie a priori beliebig komplex
sein, so dass das obige Bild a priori vielleicht nur eine
der einfachsten Möglichkeiten wäre, sie zu sehen.}

\subsection{\en{leads}\de{fährten}}

\small
\begin{list}{$\bullet$}{\setlength{\leftmargin}{10pt}}

\item
\en{If free will or the observing self is a part of e5, what is the
rest\,? Cause and effect, fate, destiny, the free will of others,
the own or collective unconscious\,? Quantum mechanics has
relativized the first assumption somewhat, or maybe not.}%
\de{Wenn der freie Wille oder das beobachtende Selbst ein Teil
von e5 ist, was ist dann der Rest\,? Ursache und Wirkung,
Schicksal, Bestimmung, der freie Wille der anderen, das
eigene oder kollektive Unbewusste\,? Die Quantenmechanik hat
die erste Annahme etwas relativiert, oder vielleicht auch nicht.}

\item
\en{What property of the issue of free will or not leads to millions
of variations when thinking about it\,? Could it possibly even
be literally the effect of many “transformations” in the mind,
even in circles, whatever that may mean precisely\,?}%
\de{Welche Eigenschaft der Frage ob freier Willen oder nicht führt
zu Millionen von Variationen, wenn man darüber nachdenkt\,?
Könnte das vielleicht sogar tatsächlich die Auswirkung vieler
“Verwandlungen” im Kopf sein, sogar in Kreisen, was immer
das genau bedeuten mag\,?}

\item
\en{Freedom to lift a spoon does not automatically mean
freedom of choice whether to want to lift the spoon or not.}%
\de{Die Freiheit, einen Löffel anzuheben, bedeutet nicht automatisch
die Freiheit der Wahl, ob man den Löffel anheben
möchte oder nicht.}

\item
\en{When I say that outside more things rest than move,
I mean this in a very specific sense: Relative macroscopic
motion at time scales that human beings can register.}%
\de{Wenn ich sage, dass draussen mehr Dinge ruhen als sich bewegen,
dann meine ich das in einem ganz bestimmten Sinn:
relative makroskopische Bewegung auf Zeitskalen, die ein Mensch
wahrnehmen kann.}

\en{At long time scales, all things move; microscopically
everything is in motion, as heat is nothing but random motion of
atoms or molecules. When I turn my head, all objects move,
but relative motion between them remains small.}%
\de{Auf langen Zeitskalen bewegen sich alle Dinge; mikroskopisch
gesehen ist alles in Bewegung, denn Wärme ist nichts anderes
als die zufällige Bewegung von Atomen oder Molekülen.
Wenn ich meinen Kopf drehe, bewegen sich alle Objekte, aber
die relative Bewegung zwischen ihnen bleibt gering.}

\item
\en{Some things outside keep moving, but often in a way “that
rests by changing”, reminding of Heraclitus, like a river that
remains the same despite its water flowing, or often waves
in the sea that move sort of periodically and only drastically
change their average size and shape over longer periods of
time than immediately observable. Fast moving clouds,
however, can take on quite different shapes. And so on; all in all,
categorizing outside as “hard” is not absolute.}%
\de{Manche Dinge aussen bleiben in Bewegung, aber oft auf eine
Art und Weise, “die durch sich wandeln ruht”, was an
Heraklit erinnert, wie ein Fluss, der trotz seines fliessenden
Wassers gleich bleibt, oder wie oft Wellen im Meer, die sich
gewissermassen periodisch bewegen und ihre durchschnittliche
Grösse und Form nur über längere Zeiträume drastisch
verändern als dies unmittelbar wahrnehmbar ist. Sich schnell
bewegende Wolken hingegen können ganz unterschiedliche
Formen annehmen. Und so weiter; alles in allem ist die
Einstufung von aussen als “hart” nicht absolut.}

% page
\item
\en{The present approach to nature is consequently centered
on the human perspective, on direct experience of nature.
Modern science usually differs from that by trying to pick
a point of view from which a problem is easy to describe.}%
\de{Die Herangehensweise an die Natur hier ist konsequent
zentriert auf die menschliche Perspektive, auf die direkte
Erfahrung der Natur. Die moderne Wissenschaft unterscheidet sich
in der Regel davon, indem sie versucht, einen Standpunkt zu
wählen, von dem aus ein Problem leicht zu beschreiben ist.}

\en{The oldest example for this is astronomy that has been greatly
simplified by solar centered calculations instead of using many
arbitrary epicycles in geocentric calculations.}%
\de{Das älteste Beispiel dafür ist die Astronomie, die durch
sonnenzentrierte Berechnungen stark vereinfacht wurde, anstatt
viele willkürliche Epizyklen in geozentrischen Berechnungen
zu verwenden.}

\item
\en{Modern science is a very valuable companion for the present
approach, especially for helping to exclude naive mistakes.}%
\de{Die moderne Wissenschaft ist ein sehr wertvoller Begleiter für
den vorliegenden Ansatz, insbesondere um zu helfen, naive
Fehler auszuschliessen.}

\item
\en{Can my observations about motion, activity and hardness
outside and inside be formalized and thus proven\,?
How would such a mathematical representation look like\,?
What assumptions would it be based on\,?}%
\de{Können meine Beobachtungen über Bewegung, Aktivität und
Härte aussen und innen formalisiert und damit bewiesen werden\,?
Wie würde eine solche mathematische Darstellung
aussehen\,? Auf welchen Annahmen würde sie beruhen\,?}

\item
\en{In any closed system, \textsl{entropy}, roughly a measure of disorder,
can at best remain constant, but usually it increases. With
time, macroscopic directed motion and structures decay into
microscopic random motion, which is, by definition, heat.
Life manages to escape this fate by operating in \textsl{open} systems,
by exporting disorder into the environment. That way, living
beings can grow from microscopic seeds to complex structures
and animals can repeatedly create directed motion.}%
\de{In jedem geschlossenen System kann die \textsl{Entropie}, grob ein
Mass für die Unordnung, bestenfalls konstant bleiben, meist
nimmt sie jedoch zu. Mit der Zeit zerfallen makroskopisch
gerichtete Bewegungen und Strukturen in mikroskopische
Zufallsbewegungen, die per Definition Wärme sind. Das Leben
entgeht diesem Schicksal, indem es in \textsl{offenen} Systemen
operiert, indem es Unordnung in die Umwelt exportiert. Auf
diese Weise können Lebewesen aus mikroskopischen Samen zu
komplexen Strukturen heranwachsen, und Tiere können immer
wieder gerichtete Bewegung erzeugen.}

\en{Since science considers the outside world to be mainly inanimate
and the mind to be located in a piece of organic matter,
the brain, it predicts that outside motion tends to disappear,
while inside the conscious mind has a hard time focusing on
something, because lots of mostly unconscious activity in the
brain keeps stirring things up.}%
\de{Da die Wissenschaft die Aussenwelt als überwiegend unbelebt
und den Geist als in einem Stück organischer Materie, dem
Gehirn, angesiedelt ansieht, sagt sie voraus, dass Bewegung
aussen dazu neigt, zu verschwinden, während es dem bewussten
Geist innen schwer fällt, sich auf etwas zu konzentrieren,
weil viele meist unbewusste Aktivitäten im Gehirn die Dinge
immer wieder aufrütteln.}

\en{Science is thus essentially compatible with the considerations
presented so far, except for science’s qualitative notion
that creating motion inside the mind is active, requires energy, like
outside. This might, however, simply be due to the viewpoint
of science, which only considers facts in the outer, material
world and might thus not be able to describe inner processes
as experienced from the inside…}%
\de{Die Wissenschaft ist also im Wesentlichen mit den bisher
vorgestellten Überlegungen vereinbar, mit Ausnahme der qualitativen
Vorstellung der Wissenschaft, dass die Erzeugung von
Bewegung im Inneren des Geistes aktiv ist und Energie
erfordert, wie aussen. Das könnte aber auch einfach an der
Sichtweise der Wissenschaft liegen, die nur Sachverhalte in
der äusseren, materiellen Welt betrachtet und damit innere
Vorgänge, wie sie von innen erlebt werden, vielleicht gar nicht
beschreiben kann…}

\item
\en{In meditation, calmness of the mind (eri) is often sought by
actively focussing the mind on something, thus reducing emi.}%
\de{In der Meditation wird die Ruhe des Geistes (eri) oft dadurch
zu erreichen versucht, indem man den Geist aktiv auf etwas
fokussiert und so emi reduziert.}

\en{Is motion time\,? If you are just sitting outside at a calm place,
time does not stop despite no movement outside (emo), but
people who meditate and thus also reduce emi, often report
that they feel time to slow down or even stop.}%
\de{Ist Bewegung Zeit\,? Wenn man einfach aussen an einem
ruhigen Ort sitzt, bleibt die Zeit nicht stehen, obwohl es aussen
keine Bewegung (emo) gibt, aber Menschen, die meditieren
und so auch emi reduzieren, berichten oft, dass sie fühlen,
dass die Zeit sich verlangsamt oder gar stehen bleibt.}

\item
\en{In daily life, the outer world seems often bigger and stronger
than the inner one. If you look at a bicycle and then close your
eyes, you can quite quickly imagine the bicycle in your mind,
but if you then imagine, say, that you add wings, and open
your eyes again, you will usually not see a winged bicycle.}%
\de{Im täglichen Leben erscheint die äussere Welt oft grösser
und stärker als die innere. Wenn man ein Fahrrad betrachtet
und dann die Augen schliesst, kann man sich das Fahrrad
recht schnell innen vorstellen, aber wenn man sich dann z.B.\
vorstellt, dass man dem Fahrrad Flügel hinzufügen würde,
und die Augen wieder öffnt, wird man normalerweise kein
geflügeltes Fahrrad vor sich sehen.}

\en{Conversely, you can usually make everything outside disappear
by just closing your eyes (“turn black”, ero), or you can
turn your head or walk away, so that the influence on what
one sees outside is immediately very strong in that sense.}%
\de{Umgekehrt kann man normalerweise alles draussen verschwinden
lassen, indem man einfach die Augen schliesst (“schwarz
werden lässt”, ero), oder man kann den Kopf drehen oder
weggehen, so dass der Einfluss auf das, was man draussen
sieht, in diesem Sinne unmittelbar sehr gross ist.}

% page
\en{Adding wings to a bicycle outside is still possible, but harder,
because the outer world is harder. It requires several steps
involving eri (planning, focussing), which then lead, via emo,
to a different arrangement of ero, a winged bicycle.}%
\de{Einem Fahrrad draussen Flügel anzubauen, ist schon auch
möglich, aber härtere Arbeit, weil die Aussenwelt härter ist.
Es erfordert mehrere Schritte mit eri (Planung, Fokussierung),
die dann über emo zu einer anderen Anordnung von
ero, einem geflügelten Fahrrad, führen.}

\item
\en{In\,\textsl{The World as Will and Idea}, Schopenhauer puts will before
a distinction between subject and object:}%
\de{In\hspace{-0.1em} \textsl{Die Welt als Wille und Vorstellung}, stellt Schopenhauer
den Willen vor eine Unterscheidung von Subjekt und Objekt:}

\en{“[…$\!$] as feeling, a knowledge that his will is the real inner
nature of his phenomenal being, which manifests itself to him as
idea, both in his actions and in their permanent substratum,
his body, and that his will is that which is most immediate
in his consciousness, though it has not as such completely
passed into the form of idea in which object and subject stand
over against each other, but makes itself known to him in a
direct manner, in which he does not quite clearly distinguish
subject and object, yet is not known as a whole to the individual
himself, but only in its particular acts,–whoever, I say,
has with me gained this conviction will find that of itself it
affords him the key to the knowledge of the inmost being of
the whole of nature; for he now transfers it to all those
phenomena which are not given to him, like his own phenomenal
existence, both in direct and indirect knowledge, but only
in the latter, thus merely one-sidedly as idea alone.” (§\,21)}%
\de{“[…$\!$] als Gefühl besitzt, dass nämlich das Wesen an sich
seiner eigenen Erscheinung, welche als Vorstellung sich ihm
sowohl durch seine Handlungen, als durch das bleibende
Substrat dieser, seinen Leib, darstellt, sein Wille ist, der das
Unmittelbarste seines Bewusstseyns ausmacht, als solches aber
nicht völlig in die Form der Vorstellung, in welcher Objekt
und Subjekt sich gegenüber stehn, eingegangen ist; sondern
auf eine unmittelbare Weise, in der man Subjekt und Objekt
nicht ganz deutlich unterscheidet, sich kund giebt, jedoch
auch nicht im Ganzen, sondern nur in seinen einzelnen
Akten dem Individuo selbst kenntlich wird; – wer, sage ich, mit
mir diese Ueberzeugung gewonnen hat, dem wird sie, ganz
von selbst, der Schlüssel werden zur Erkenntniss des Innersten
Wesens der gesammten Natur, indem er sie nun auch
auf alle jene Erscheinung übertragt, die ihm nicht, wie seine
eigene, in unmittelbarer Erkenntniss neben der mittelbaren,
sondern bloss in letzterer, also bloss einseitig, als Vorstellung
allein, gegeben sind.” (§\,21)}

\item
\en{There is usually less emo than ero and less eri than emi, but
emo and eri are on average not disappearing. So transitions
would have to be balanced and/or to return in loops.}%
\de{Normalerweise gibt es weniger emo als ero und weniger eri als
emi, aber emo und eri verschwinden im Durchschnitt nicht.
Die Übergänge müssten also ausgeglichen sein und/oder in
Schleifen wiederkehren.}

\en{In today’s science, the source of recurring activity would be
the sun. The earth receives about the same amount of energy
as light from the sun as it radiates back into the universe,
which is why the temperature of earth is roughly constant.
But since the earth receives energy from a single point in
space and exports energy into all directions, it effectively
exports entropy into space, thus preserving life on earth.}%
\de{Nach dem heutigen Stand der Wissenschaft wäre die Quelle
der wiederkehrenden Aktivität die Sonne. Die Erde empfängt
von der Sonne etwa die gleiche Energiemenge wie das Licht,
das sie ins Universum zurückstrahlt, weshalb die Temperatur
der Erde ungefähr konstant ist. Da die Erde jedoch Energie
von einem einzigen Punkt im Weltraum empfängt und Energie
in alle Richtungen abgibt, exportiert sie effektiv Entropie
in den Weltraum, wodurch Leben auf der Erde bewahrt wird.}

\item
\en{If something rests inside and one remains passive, it starts to
flow and change, but being active inside does generally not
seem to create calm, often seems to do the opposite.}%
\de{Wenn innen etwas ruht und man passiv bleibt, beginnt es zu
fliessen und sich zu verändern, aber innen aktiv sein scheint
generell nicht Ruhe zu erzeugen, scheint oft das umgekehrte
zu tun.}

\en{I am not sure if that is true, but personally, I tend towards the
following perspectives that preserve the view of the main text,
and which, together with many things that would fit in from
human history across ages and cultures (see later sections),
would give a coherent picture of almost everything in life.}%
\de{Ich bin mir nicht sicher, ob das stimmt, aber persönlich tendiere
ich zu den folgenden Perspektiven, die die Sichtweise des
Haupttextes bewahren und die zusammen mit vielen Dingen,
die aus der menschlichen Geschichte über Zeitalter und Kulturen
hinweg dazu passen würden (siehe spätere Abschnitte),
ein kohärentes Bild von fast allem im Leben ergeben würden.}

\en{That view would appear to be very universal, something that
would apply to any form of self-conscious life anywhere in
the universe, and thus public preservation and some form of
promotion would seem paramount, or, if not certain, likely
still better to preserve, and promote a little, just in case.}%
\de{Diese Sichtweise scheint sehr universell zu sein, etwas, das
auf jede Form von Leben mit einem Bewusstsein seiner selbst
überall im Universum zutreffen würde, und daher scheint
öf\-fent\-li\-che Bewahrung und irgendeine Form der Bekanntmachung
von höchster Wichtigkeit zu sein, oder, wenn nicht
sicher, wahrscheinlich immer noch besser, sie zu bewahren
und ein bisschen bekannt zu machen, nur für den Fall.}

\en{Maybe one day, me or someone else would find an immediately
more convicing perspective to my idea, or something
partially or entirely different; I am not a god or goddess, just
someone thinking about these things.}%
\de{Vielleicht finde ich oder jemand anderes eines Tages eine
unmittelbar überzeugendere Perspektive zu meiner Idee, oder
etwas teilweise oder ganz anderes; ich bin kein Gott oder eine
Göttin, nur jemand, der über diese Dinge nachdenkt.}

% page
\item
\en{The ancient Greek word \textsl{psychê}, which, for example, Plato
used for “soul”, literally also meant \textsl{butterfly}. A butterfly
flies from flower to flower, sometimes rests longer, sometimes
only briefly, and apparently randomly flies on.}%
\de{Das altgriechische Wort \textsl{psychê}, welches u.a.\ auch Platon für
“Seele” verwendete, bedeutete unmittelbar auch \textsl{Schmetterling}.
Ein Schmetterling fliegt von Blume zu Blume, bleibt mal
länger, mal nur kurz, und fliegt anscheinend zufällig weiter.}

\en{The flowers could be interpreted as individual words or images
or other kinds of inner topics, which would emerge (emi), in
modern view from the unconscious, then the conscious mind
would dwell on them (eri) for a short while, keeping them
actively still, and then they would disappear again (emi), sink
back to unconsciousness.}%
\de{Die Blumen könnte man als einzelne Worte oder Bilder oder
andere Inhalte interpretieren, welche auftauchen würden (emi),
in moderner Sicht aus dem Unbewussten, dann würde der
bewusste Geist für ein kurze Zeit bei ihnen verweilen (eri), sie
aktiv still halten, and dann würden sie wieder verschwinden
(emi), wieder ins Unbewusste zurücksinken.}

\item
\en{In Homer’s \textsl{Odyssey}, Odysseus is traveling from island to
island back home to Ithaca. This epic poem was arguably the
first time a new kind of thinking was described, one that is
practically everywhere today, at least in cultures influenced
by the ancient Greeks. Hence is Odysseus with his ship like
a butterfly and the islands like flowers\,?}%
\de{In Homers \textsl{Odyssee}, reist Odysseus von Insel zu Insel zurück
nach Hause nach Ithaka. Dieses Gedichtsepos war vertretbar
das erste Mal, wo eine neue Art von Denken beschrieben
wurde, eine, die heute praktisch allgegenwärtig ist, zumindest
in Kulturen, die von den alten Griechen beeinflusst wurden.
Ist also Odysseus mit seinem Schiff wie ein Schmetterling und
die Inseln wie Blumen\,?}

\begin{center}
\includegraphics[scale=0.117]{i-islands.jpg}
\end{center}

\en{On his journey, Odysseus is helped by Athena (also in dreams)
and also subtly by Zeus, while the sea god Poseidon, Zeus’
apparently more archaic brother, is making the return to Ithaca
at least harder. In antiquity, Poseidon was also responsible
for earthquakes, thus probably also for islands to emerge out
of the sea, at least where of tectonic/volcanic origin.}%
\de{Auf seiner Reise, bekommt Odysseus Hilfe von Athene (auch
in Träumen) und auch subtil von Zeus, wohingegen der
Meeresgott Poseidon, Zeus’ scheinbar archaischerer Bruder, die
Rückkehr zumindest erschwert. In der Antike war Poseidon
auch verantwortlich für Erdbeben, also wahrscheinlich auch
für aus dem Meer auftauchende Inseln, zumindest wo
tektonischen/vulkanischen Ursprungs.}

\item
\en{If you want to use a word like “island” in conscious considerations,
how can you consciously retrieve it from memory\,? You
cannot use “island“ itself to retrieve itself, and if you would
retrieve “island“ via an image or other kind of “token” for it,
how would you retrieve that token in the first place\,? Hence
it seems that retrieval cannot be a fully conscious process.}%
\de{Wenn man ein Wort wie “Insel” in bewussten Betrachtungen
benutzen will, wie kann man es bewusst aus dem Gedächtnis
abrufen\,? Man kann nicht “Insel” selbst zum Abrufen von
sich selbst benutzen, und wenn man “Insel” via ein Bild oder
einen anderen “Platzhalter” abrufen würde, wie hätte man
denn den Platzhalter zuvor abrufen können\,? Es scheint, dass
Abrufen kein vollständig bewusster Vorgang sein kann.}

\en{Similarly, a chain of thought might not be a fully conscious
process. The chain can be perceived consciously, but maybe
not created that way, or only indirectly.}%
\de{Ähnlich wäre ein Gedankengang vielleicht kein völlig bewusster
Vorgang. Er kann bewusst wahrgenommen werden, aber
vielleicht nicht in der Art erzeugt werden, oder nur indirekt.}

\item
\en{Measurement in quantum mechanics comes to mind, where a
well-defined consciously registered result (something static?)
comes about by the “wave function collapse”.}%
\de{Der Messvorgang in der Quantenmechanik kommt einem da
in den Sinn, wo ein wohldefiniertes bewusst registriertes
Resultat (etwas statisches?) durch den “Kollaps der Wellenfunktion”
zustande kommt.}

\item
\en{Overall my approach is a self in a contemplating mode, where
it observes what happens outside and inside. Outside things
tend to rest, getting them to move is often hard; inside things
tend to move, getting them to rest is often hard. One could
object that this is overlooking that the self would be part of
what goes inside, that a lot of what goes on inside is caused by
unconscious processes, but that is not a primary observation,
that is a theory based already on multiple assumptions, on
more than immediate contemplating observation.}%
\de{Insgesamt ist mein Zugang über ein Selbst in einem kontemplativen
Zustand, wo es beobachtet was aussen und innen
vor sich geht. Aussen tendieren Dinge dazu, zu ruhen; sie in
Bewegung bringen ist oft schwer; innen tendieren Dinge dazu,
sich zu bewegen; sie zur Ruhe zu bringen ist oft schwer.
Man könnte einwenden, dass das übersieht, dass das Selbst
Teil wäre von dem, was innen vor sich geht, dass vieles von
dem, was innen vor sich geht, das Resultat von unbewussten
Vorgängen wäre, aber das ist keine unmittelbare Beobachtung,
das ist eine Theorie, die bereits auf mehreren Annahmen
beruht, auf mehr als unmittelbar kontemplative Beobachtung.}

% page
\item
\en{By the way, seems that Latin ‘contemplatio’ was the original
translation of ‘theōría’ in ancient Greek philosophy. I used
‘contemplation’ as observing in a calm, relaxed way that does
not try to interfere with what is going on outside or inside.}%
\de{Übrigens scheint das lateinische ‘contemplatio’ die ur\-sprüng\-li\-che
Übersetzung von ‘theōría’ der griechischen Philosophie
gewesen zu sein. Ich verwendete ‘Kontemplation’ als ruhige,
entspannte Art von Beobachten, die nicht versucht, mit dem
zu interferieren, was aussen oder innen vor sich geht.}

\en{For how ‘theōría’ evolved in ancient Greek philosophy, see e.g.\
Andrea Nightingale’s \textsl{Spectacles of Truth in Classical Greek
Philosophy} (Cambridge University Press, 2004) or her earlier
article \textsl{On Wandering and Wondering} (2001). It is interesting
that theōría first related to literal travels, typically to witness
a religious event in a different polis and report about the
experience at return, and then it “traveled inside”…}%
\de{Zur Entwicklung von ‘theōría’ in der antiken griechischen
Philosophie siehe z.B.\ Andrea Nightingale’s \textsl{Spectacles of Truth
in Classical Greek Philosophy} (Cambridge University Press,
2004) oder ihren früheren Artikel \textsl{On Wandering and Wondering}
(2001). Es ist interessant, dass sich theōría erst auf
tatsächliche Reisen bezog, üblicherweise in eine andere Polis
um ein religiöses Ereignis mitzuerleben und bei der Rückkehr
darüber zu berichten, und es dann “nach innen reiste”…}

\en{In antiquity, people would usually not read a text silently, but
read it out loud, hence also the hexameters im Homer’s work,
for example. In that sense, internalization would likely also
mirror further development of a self that learned to contemplate
silently and invisibly inside.}%
\de{In der Antike wurden Texte üblicherweise nicht still gelesen,
sondern laut hervorgelesen, daher auch zum Beispiel die
Hexameter in Homers Werk. In diesem Sinn, würde
Verinnerlichung wahrscheinlich auch eine Weiterentwicklung eines
Selbst spiegeln, welches gelernt hat, still und unsichtbar innen
zu kontemplieren.}

\en{Accordingly, early ‘sophoi’ (wise men) like Thales, Heraclitus
and Pythagoras where broadly involved in many practical
ways, were active in politics, economics, religion, and more,
Plato already less so, even though he tried to realize his views
of a perfect state abroad, and Aristotle wrote about all these
areas and was a teacher of young Alexander the Great, but
was himself not actively involved any more.}%
\de{Dementsprechend waren frühe ‘sophoi’ (Weisen) wie Thales,
Heraklit und Pythagoras in vielerlei Hinsicht praktisch tätig,
engagierten sich in Politik, Wirtschaft, Religion und mehr,
Platon schon weniger, obwohl er versuchte, seine Vorstellungen
von einem perfekten Staat im Ausland zu verwirklichen,
und Aristoteles schrieb über all diese Bereiche und war ein
Lehrer des jungen Alexander des Grossen, war aber selbst
nicht mehr aktiv beteiligt.}

\en{As Andrea Nightingale discovered, Plato’s cave, the probably
most famous allegory in philosophy, is in the immediate sense
an attempt to introduce and explain a virtual form of ‘theōría’
to his fellow Athenians, a journey to the world of ideas and
forms, with likely also the aim to prevent clashes with power,
like the ones that had cost his idol Socrates his life.}%
\de{Wie Andrea Nightingale herausfand, ist Platons Höh\-len\-gleich\-nis,
das wahrscheinlich berühmteste Gleichnis der Philosophie,
im unmittelbaren Sinn ein Versuch, seinen Mitathenern
eine virtuelle Form von ‘theōría’ zu zeigen und erklären, eine
Reise in die Welt der Ideen und Formen, mit wohl auch der
Absicht, Konflikte mit der Macht zu verhindern, wie denjenigen,
die seinem Idol Sokrates das Leben gekostet hatten.}

\en{Despairing, or maybe rather only almost despairing at one’s
contemporaries is, of course, also the fate of any discoverer
of something new, as I know all too well. I would certainly
follow Odysseus’ choice in Plato’s Republic, choose the fate
of a regular man, have a family, kids, and so on.}%
\de{Über meine Zeitgenossen verzweifeln, oder vielleicht eher nur
fast verzweifeln, ist natürlich auch das Schicksal jedes
Entdeckers von etwas Neuem, wie ich nur allzugut weiss. Ich
würde gewiss Odysseus’ Wahl in Platons Staat folgen, das
Schicksal eines gewöhnlichen Mannes wählen, eine Familie,
Kinder haben, und so weiter.}

\en{Hey, the Odyssey, much more so than the Iliad, seems to me
to be the first virtual theōría, the very source of almost all
that is in this world today, and the mother of this here…}%
\de{Hey, die Odyssee, viel mehr als die Ilias, scheint mir die erste
virtuelle theōría zu sein, die wahre Quelle von fast allem, das
heute in dieser Welt ist, und die Mutter von dem hier…}

\item[\yinyang]
\en{The original idea is from 2004, doubts around active/passive
inside from 2024, so what would it be in 2044\,? Personally, I
just still dream of fully recovering the original idea, since it is
so beautiful and would explain so many things from one\,…}%
\de{Die ursprüngliche Idee ist von 2004, Zweifel um aktiv/passiv
innen von 2024, wie sähe es denn 2044 aus\,? Ich persönlich
träume einfach immer noch davon, die ursprüngliche Idee
vollständig wiederzugewinnen, da sie so wunderschön ist und
so vieles aus einem erklären würde\,…}

\en{I have no element air in my birth chart, which seems to make
my thinking lack calm, except regarding my Saturn/Chiron in
Pisces, which would in my model of the star signs be mainly
made of air. In ancient Greece in the second half of the Age of
Aries, the opposite sign Libra gained importance, the middle
air sign, balancing fire and water, hence inner contemplation
emerged. Air must rest, else it runs the risk of remaining
truly invisible, thus air-rests-eri. And a “final edition” of this
website as a book in 2025, Sleeping Beauty Dreaming here,
and maybe room and leisure for other publications.}%
\de{Ich habe kein Element Luft in meinem Geburtshoroskop, was
es meinem Denken anscheinend an Ruhe mangeln lässt,
ausser bezüglich meinem Saturn/Chiron in den Fischen, welche in
meinem Modell der Sternzeichen hauptsächlich aus Luft
bestehen würden. Im antiken Griechenland gewann in der zweiten
Hälfte des Widderzeitalters das entgegengesetzte Zeichen
Waage an Bedeutung, das mittlere Luftzeichen, welches Feuer
und Wasser in der Balance hält, daher entstand da auch
innere Kontemplation. Luft muss ruhen, sonst läuft sie Gefahr,
wahrhaftigt unsichtbar zu bleiben, daher Luft-Ruhe-eri. Und
eine “final edition” dieser Website als Buch im Jahr 2025,
Dornröschen träumend hier, und vielleicht Raum und Musse
für andere Publikationen.}

\end{list}

\end{document}
