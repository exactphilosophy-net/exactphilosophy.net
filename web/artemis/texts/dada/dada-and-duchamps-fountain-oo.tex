\avantgarde

\en{\section{Dada and Duchamp’s Fountain}}%
\de{\section{Dada und Duchamps Fountain}}%
\fr{\section{Dada et la Fontaine de Duchamp}}

\vspace{-1.2mm}
\noindent
\href{https://www.exactphilosophy.net/fontainebleue.jpg}{\includegraphics[scale=0.1]{i-fontainebleue.jpg}}

\vspace{2mm}
\noindent
\en{Duchamp’s ‘Fountain’ has a very direct relation to Dada:
A ‘dada’ in french is a horse in children’s language,
also rocking and hobby horse and hobby,
and appears in ‘à dada sur mon bidet’,
the french version of playing gee-gees,
and ‘bidet’ is a little horse,
as well as the sanitary fitment,
which strongly resembles Duchamp’s ‘Fountain’,
not least because of the usual meaning of ‘fountain’.}%
\de{Duchamps ‘Fountain’ hat einen sehr direkten Bezug zu Dada:
Ein ‘dada’ im französischen ist ein Pferd in Kindersprache,
auch Schaukel- und Steckenpferd und Hobby,
und kommt in ‘à dada sur mon bidet’,
der französischen Version von ‘Hoppe, hoppe Reiter’,
vor,
und ‘bidet’ ist sowohl ein kleines Pferd
wie auch der sanitäre Gegenstand,
welcher stark Duchamps ‘Fountain’ ähnelt,
nicht zuletzt,
da das englische Wort ‘fountain’ oft Springbrunnen bedeutet.}%
\fr{La ‘Fontaine’ de Duchamp a une relation très directe avec Dada:
Un ‘dada’ en français est un cheval en language enfantin,
aussi cheval à bascule et cheval de bois et hobby,
et apparait aussi dans ‘à dada sur mon bidet’,
et ‘bidet’ est un petit cheval
aussi bien que l’object sanitaire,
qui ressemble beaucoup à la ‘Fontaine’ de Duchamp,
tout particulièrement dans le sens d’un jet d’eau.}

\en{\subsection{Details}}%
\de{\subsection{Details}}%
\fr{\subsection{Détails}}

\en{(For many facts surrounding Duchamp’s ‘Fountain’,
see \href{http://www.artbouillon.com/2014/11/pilfered-pissoire-response-to.html}%
{\color{xphi}\textsl{Pilfered Pissoire? A Response to the Allegation that Duchamp Stole his Famous Fountain}},
Jesse Prinz, artbouillon, 20 Nov 2014.)}%
\de{(Für viele Fakten um Duchamps ‘Fountain’ herum
siehe \href{http://www.artbouillon.com/2014/11/pilfered-pissoire-response-to.html}%
{\color{xphi}\textsl{Pilfered Pissoire? A Response to the Allegation that Duchamp Stole his Famous Fountain}},
Jesse Prinz, artbouillon, 20 Nov 2014.)}%
\fr{(Pour beaucoup de faits autour de la ‘Fontaine’ de Duchamp,
voir \href{http://www.artbouillon.com/2014/11/pilfered-pissoire-response-to.html}%
{\color{xphi}\textsl{Pilfered Pissoire? A Response to the Allegation that Duchamp Stole his Famous Fountain}}, Jesse Prinz, artbouillon, 20 Nov 2014.)}

\en{The name ‘Dada’ for the art movement
originated in 1916 in the Zürich flat of Hugo Ball and Emmy Hennings
in company of Richard Huelsenbeck
(Huelsenbeck, \textsl{transition}, No.~2 (May 1927), pp.~134-135):}%
\de{Der Name ‘Dada’ für die Kunstbewegung
entstand 1916 in der Zürcher Wohnung von Hugo Ball und Emmy Hennings
im Beisein von Richard Huelsenbeck
(Huelsenbeck, \textsl{transition}, No.~2 (Mai 1927), pp.~134-135):}%
\fr{Le nom ‘Dada’ pour le mouvement d’art
s’origine en 1916 dans l’appartement de Hugo Ball et Emmy Hennings à Zürich
en compagnie de Richard Huelsenbeck
(Huelsenbeck, \textsl{transition}, No.~2 (Mai 1927), pp.~134-135):}

\begin{quote}
\small\color{darkgray}
\begin{otherlanguage}{french}
\textsl{\mbox{\ \ \ \ }I was standing behind Ball looking into the dictionary on his knees.
%
Ball’s finger pointed to the first letter of each word descending the page.
%
Suddenly I cried halt.
%
I was struck by a word I had never heard before,
the word dada.\newline
%
\mbox{\ \ \ \ }‘Dada,’ Ball read,
and added: ‘It is a children’s word meaning hobby-horse’.
%
At that moment I understood what advantages the word held for us.\newline
%
\mbox{\ \ \ \ }‘Let’s take the word dada,’ I said. ‘It’s just made for our purpose.
%
The child’s first sound expresses the primitiveness,
the beginning at zero,
in our art.
%
We could not find a better word.’}
\end{otherlanguage}
\end{quote}

\de{\vspace{-2mm}
\begin{quote}
\small\color{xphi}
\textsl{\mbox{\ \ \ \ }Ich stand hinter Ball und schaute in das Wörterbuch auf seinen Knien.
%
Balls Finger zeigte auf den ersten Buchstaben von jedem Wort beim die Seite runterfahren.
%
Plötzlich rief ich Halt.
%
Ich war erschlagen von einem Wort,
das ich nie zuvor gehört hatte,
das Wort dada.\newline
%
\mbox{\ \ \ \ }‘Dada,’ las Ball,
und fügte hinzu: ‘Es ist ein Kinderwort das Steckenpferd bedeutet’.
%
In diesem Moment verstand ich welche Vorteile das Wort für uns enthielt.\newline
%
\mbox{\ \ \ \ }‘Lass uns das Wort dada nehmen,’ sagte ich. ‘Es ist gerade für unsere Zwecke gemacht.
%
Der erste Laut eines Kindes drückt die Primitivität,
das Beginnen bei Null,
in unserer Kunst aus.
%
Wir könnten kein besseres Wort finden.’}
\end{quote}}%
\fr{\begin{quote}
\small\color{xphi}
\textsl{\mbox{\ \ \ \ }J’étais debout derrière Ball en regardant dans le dictionnaire sur ses genoux.
%
Ball montrait la première lettre de chaque mot du doigt en descendant la page.
%
Soudainement, je criai halte.
%
J’étais foudroyé par un mot que je n’avais jamais entendu auparavant,
le mot dada.\newline
%
\mbox{\ \ \ \ }‘Dada,’ lut Ball,
et ajouta: ‘Cet un mot d’enfant qui signifie cheval de bois’.
%
En ce moment je compris quels avantages le mot contenait pour nous.\newline
%
\mbox{\ \ \ \ }‘Prenons le mot dada ,’ je dis. ‘Il est juste fait pour notre propos.
%
Le premier son d’un enfant exprime la primitivité,
le commencement a zéro,
dans notre art.
%
Nous ne pourrions pas trouver de meilleur mot.’}
\end{quote}}

\noindent
\en{Independently of whether things took part exactly that way,
the primary association of Dada seems to be with the french ‘dada’,
which is children’s language for horse,
including rocking and hobby horse,
and figuratively also means hobby.}%
\de{Unabhängig davon,
ob sich die Dinge genau so zutrugen,
scheint die ur\-sprüng\-li\-che Assoziation von Dada mit dem französischen ‘dada’ zu sein,
was Kindersprache für ein Pferd ist,
inklusive Schaukel- und Steckenpferd,
und im übertragenen Sinne auch Hobby bedeutet.}%
\fr{Indépendamment de si les faits se sont déroulés exactement comme ça, 
l’association primaire de Dada semble être avec le mot français ‘dada’,
qui est un mot d’enfant pour un cheval,
aussi pour cheval à bascule ou cheval en bois,
et qui figurativement signifie aussi hobby.}

\en{The nursery rhyme ‘à dada sur mon bidet’
corresponds to the english ‘to play gee-gees’,
hence where a child “rides” on the thighs of an adult.}%
\de{Der französische Kinderreim ‘à dada sur mon bidet’
entspricht dem deutschen ‘Hoppe hoppe Reiter’,
also wo ein Kind auf den Schenkeln eines Erwachsenen “reitet”.}%
\fr{Puis il y a le rime d’enfants ‘à dada sur mon bidet’,
alors ou un enfant “fait du cheval” sur les cuisses d’un adulte.}

\en{The word ‘bidet’ stands in French originally and until today for a kind of little horse.}%
\de{Das Wort ‘bidet’ steht auf Französisch ursprünglich und bis heute für eine Art kleines Pferd.}%
\fr{Le mot ‘bidet’ signifie en français originalement et jusqu’aujourd’hui un genre de petit cheval.}
%
\en{Today’s better known meaning as a sanitary fitment with some kind of “fountain” in it,
originates from its original appearance that resembled a little horse,
for example in ‘La toilette intime ou la fleur effeuillée’ by Louis-Léopold Boilly (1761-1845):}%
\de{Die heute bekanntere Bedeutung als sanitärer Gegenstand mit einer Art “Springbrunnen” darin
kommt kommt vom ursprünglich einem kleinen Pferd ähnlichen Aussehen her,
zum Beispiel in ‘La toilette intime ou la fleur effeuillée’ von Louis-Léopold Boilly (1761-1845):}%
\fr{Le sense plus connu aujourd’hui comme object sanitaire avec une sorte de “fontaine” dedans,
origine de son apparence originale qui ressemblait à un petit cheval,
par exemple dans ‘La toilette intime ou la fleur effeuillée’ de Louis-Léopold Boilly (1761-1845):}

\vspace{2mm}
\hspace{25mm}\noindent
\href{https://commons.wikimedia.org/wiki/File\%3ABoilly_La_Toilette_intime_ou_la_Rose_effeuill\%C3\%A9e.jpg}{\includegraphics[scale=0.20]{i-boilly.jpg}}

\vspace{2mm}
\noindent
\en{Duchamp had submitted ‘Fountain’ with help from his friends
towards 1 April 1917 for the New York art exhibition.}%
\de{Duchamp hatte ‘Fountain’ (‘Springbrunnen’) mit Hilfe seiner Freunde
gegen den 1.\ April 1917 für die New Yorker Kunstausstellung eingereicht.}%
\fr{Duchamp avait soumis ‘Fountain’ (‘Fontaine’) avec l’aide de ses amis
vers le 1$^\mathrm{\textsf{\footnotesize{er}}}$ Avril 1917 pour l’exposition d’art new-yorkaise.}
%
\en{For all that it appears as Dada in the sense not least of the french ‘dada’.}%
\de{Allem Anschein nach als Dada im Sinne nicht zuletzt vom französischen ‘dada’.}%
\fr{Selon toute apparence comme Dada dans le sense notamment du mot français ‘dada’.}

\vspace{2mm}
\textsl{%
\en{Hence the fountain was intended to represent some kind of a bidet and thus Dada,
as a sarcastic April Fool’s joke.}%
\de{Der Springbrunnen sollte also eine Art Bidet und damit Dada darstellen,
und zwar als sarkastischer Aprilscherz.}%
\fr{La fontaine était donc censée représenter une sorte de bidet et donc Dada,
comme poisson d’avril sarcastique.}
%
\en{Almost all of modern art derives from that:
This April Fool’s joke became the ‘object trouvé’ / ‘ready-made’.}%
\de{Daran hängt dann fast die ganze moderne Kunst:
Aus diesem Aprilscherz ist das ‘object trouvé’ / ‘ready-made’ entstanden.}%
\fr{La quasi-totalité de l’art moderne dérive alors de cette idée:
Ce poisson d’avril est devenu ‘objet trouvé’ / ‘ready-made’.}
%
\en{If you now think that these claims might be a bit exaggerated,
you know the Bohème only from the outside…}%
\de{Wenn ihr jetzt meint,
das wäre vielleicht doch etwas übertrieben,
dann kennt ihr die Bohème nur von aussen…}%
\fr{Si vous pensez maintenant que c’est peut-être un peu exagéré,
vous connaissez la bohème seulement de l’extérieur…}}
