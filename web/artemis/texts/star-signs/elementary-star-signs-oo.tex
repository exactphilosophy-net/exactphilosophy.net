\avantgarde

\en{\section{Elementary star signs}}%
\de{\section{Elementare Sternzeichen}}

\en{This text about the star signs is something very special…}%
\de{Dieser Text über die Sternzeichen ist etwas ganz besonderes…}

\en{…at first apparently quite novel
and some things might appear a bit strange at first,
but in the end it is still something already very familiar,
just described a bit differently,
in which everybody can mirror her- of himself
and also loved or otherwise close people.}%
\de{…zwar erst mal ganz neuartig
und einiges könnte daher in der ersten Sekunde etwas seltsam wirken,
aber am Ende ist es doch etwas schon sehr Vertrautes,
einfach nur etwas anders beschrieben,
wo jede und jeder sich selbst und auch geliebte oder sonst wie vertraute Menschen drin spiegeln kann.}

\en{I describe the 12 star signs of the (Western) zodiac—%
Aries, Taurus, Gemini, Cancer, Leo, Virgo,
Libra, Scorpio, Sagittarius, Capricorn, Aquarius and Pisces—%
in a novel, very elementary way,
which is, yet, very rich and familiar in the end,
and hopefully also conveys a lot of sympathy for and understanding of the star signs
and what moves them in their innermost cores.}%
\de{Ich beschreibe die 12 Sternzeichen des (westlichen) Tierkreises—%
Widder, Stier, Zwillinge, Krebs, Löwe, Jungfrau,
Waage, Skorpion, Schütze, Steinbock, Wassermann und Fische—%
auf eine neue, ganz elementare Weise,
die aber dennoch sehr reichhaltig und am Ende vertraut ist,
und hoffentlich auch viel Ver\-ständ\-nis für die Sternzeichen
und was sie in ihrem Innersten bewegt enthält.}

\en{The whole idea can hardly be skated around in advance in a generally accessible way,
so let me simply first give a concrete example (Aries)
and then expose my model for all star signs,
with some applications
and comparisons with what other astrologers already wrote about star signs,
in order to connect the idea to more familiar things.}%
\de{Das Ganze lässt sich vorab kaum allgemeinverständlich umschreiben,
daher erst mal ein konkretes Beispiel (Widder),
und dann mehr zu meinem Modell und allen Sternzeichen,
mit einigen Anwendungen
und Vergleichen,
zu dem was andere Astrologen bisher zu Sternzeichen schrieben,
um die Sache noch mehr an Bekanntes anzuknüpfen.}
%
\en{At the end follows a review with an attempt
to carve out the overall structure of the model
as simply and clearly as possible.}%
\de{Am Schluss folgt ein Rückblick mit einem Versuch,
die Gesamtstruktur ganz einfach und klar herauszuschälen.}

\en{Please take some time to let yourself immerse into this.}%
\de{Bitte etwas Zeit nehmen, um sich darin hineinversetzen zu lassen.}

\en{\subsection{Example (Aries)}}%
\de{\subsection{Beispiel (Widder)}}

\en{I am starting with Aries, the first star sign of the zodiac:}%
\de{Ich fange mit dem erstem Sternzeichen im Tierkreis, dem Widder, an:}

\begin{quote}
\en{In my model, \textbf{Aries} is the first stage or phase
of a transformation from \textbf{Earth via Fire to Air}.}%
\de{In meinem Modell ist der \textbf{Widder} die erste Phase
einer Umwandlung von \textbf{Erde via Feuer zu Luft}.}
\end{quote}

\vspace{-2mm}
\en{\includegraphics[scale=0.1625]{i-fire.jpg}}%
\de{\includegraphics[scale=0.1625]{i-feuer.jpg}}

\noindent
\en{I am sure this must now sound abstract and arbitrary,
but this is much, much less the case than it appears,
which I can only explain gradually,
step-by-step.}%
\de{Das klingt jetzt sicher erst mal sehr abstrakt und willkürlich,
ist es aber viel, viel weniger als es scheint,
wobei ich das nur schrittweise erklären kann.}

\en{Let me start with a real fire.}%
\de{Erst mal anhand von einem echten Feuer.}

\en{A fire needs two things to burn:
Some fuel like wood or coal,
hence something that often reminds of the element Earth,
as well as the oxygen from the air,
hence roughly the element Air.}%
\de{Ein Feuer braucht zwei Dinge zum Brennen:
Brennmaterial wie Holz oder Kohle,
also etwas, das oft an das Element Erde erinnert,
sowie den Sauerstoff aus der Luft,
also so in etwa das Element Luft.}
%
\en{In addition, when a fire burns, most of the fuel disappears practically invisibly into the air,
as gas and smoke, and in terms of solid matter only a small heap of ashes remains.}%
\de{Ausserdem verschwindet beim Brennen das meiste Material praktisch unsichtbar in der Luft,
als Gas und Rauch, an fester Materie bleibt sichtbar nur ein kleines Häufchen Asche übrig.}

\en{Thus ‘Earth’ is transformed by Fire into ‘Air’.}%
\de{Also wird ‘Erde’ durch Feuer in ‘Luft’ verwandelt.}
%
\en{And why Aries as the first stage of this transition?}%
\de{Und wieso der Widder in der ersten Phase davon?}
%
\en{Well, Aries is the first of the three Fire signs Aries-Leo-Sagittarius in the zodiac.}%
\de{Nun, der Widder ist das erste von drei Feuerzeichen Widder-Löwe-Schütze im Tierkreis.}
%
\en{Aries is still very much rooted in immediate reality, close to the ground of reality,
but already filled with lots of fire and aspirations to higher things,
like more reason, knowledge, justice, which would astrologically all be Air themes.}%
\de{Der Widder ist noch stark in der unmittelbaren Realität am Boden verankert,
aber schon mit viel Feuer und Aspirationen zu Höherem,
quasi zu mehr Vernunft, Erkenntnis, Gerechtigkeit, was alles astrologisch Luftthemen wären.}
%
\en{Sagittarius, as the third and last Fire sign,
is already like Zeus/Jupiter, who rules the sign,
high up in the air and considers the world from above,
only from time to time he lets fire speak,
when he throws a thunderbolt or seduces a maiden.}%
\de{Der Schütze, das dritte und letzte Feuerzeichen,
ist bereits wie Zeus/Jupiter, der das Sternzeichen beherrscht,
hoch oben in der Luft und betrachtet sich die Welt von oben,
nur noch selten lässt er Feuer sprechen,
wenn er einen Blitz schleudert oder eine Maid verführt.}
%
\en{Leo, as the second Fire sign, is in the middle,
where the fire burns with most intensity, is most focussed.}%
\de{Der Löwe, als zweites Feuerzeichen, ist in der Mitte davon,
wo das Feuer am intensivsten, am fokussiertesten brennt.}

\en{But maybe first a quote from well-known astrologer Liz Greene about Aries,
to make the matter a bit more vivid:}%
\de{Aber vielleicht erst mal ein Zitat der bekannten Astrologin Liz Greene zum Widder,
um das Ganze etwas anschaulicher zu machen:}

\begin{quote}
\textsl{\color{xphi}%
\en{The Ram was known to the Egyptians as the primeval god Ammon, or Amun,
whose name means ‘the hidden one’.}%
\de{Der Widder war den Ägyptern als urzeitlicher Gott Ammon oder Amun bekannt,
dessen Name ‘der Verborgene’ bedeutet.}
%
\en{This antique ram-headed deity was said to be
the force behind the invisible wind.}%
\de{Man sagte von dieser antiken, widderköpfigen Gottheit,
sie sei die Kraft hinter dem unsichtbaren Wind.}
%
\en{He was also called ‘he who abides in all things’,
and was imagined as the soul of all earthly phenomena.}%
\de{Er wurde auch ‘er, der in allen Dingen fortbesteht’ genannt,
und wurde als die Seele aller irdischen Phänomene betrachtet.}%
}
\vspace{1mm}
\newline
\en{— Liz Greene, \textsl{The Astrology of Fate} (Weiser 1984),
in the chapter about Aries}%
\de{— Liz Greene, \textsl{The Astrology of Fate} (Weiser 1984),
im Kapitel über den Widder}
\end{quote}

\noindent
\en{Liz Greene instinctively used the right words:
‘earthly’ comes from ‘Earth’, hence the element Earth,
and ‘Wind’ is moving air, hence the element Air.}%
\de{Liz Greene brauchte da instinktiv die richtigen Worte:
‘Irdisch’ kommt von ‘Erde’, also das Element Erde,
und ‘Wind’ ist bewegte Luft, also Element Luft.}
%
\en{Also the word ‘soul’ fits in very well, as follows.}%
\de{Auch das Wort ‘Seele’ passt sehr gut dazu, wie folgt.}

\en{The human physical body (Earth) is mortal, ephemeral,
but the mind (Air) can create ideas and theories,
which can be communicated and written down,
hence can become sort of immortal.}%
\de{Der menschliche irdische Körper (Erde) ist sterblich, vergänglich,
aber der Geist (Luft) kann Ideen und Theorien und Einsichten schaffen,
die kommuniziert und niedergeschrieben werden können,
also quasi unsterblich werden.}
%
\en{That is where the transformation of the Fire signs lies,
it is about the aspiration to something higher,
about transforming the mortal human body (Earth) with help of Fire,
with imagination and energy etc., into something immortal.}%
\de{Da liegt die Wandlung bei den Feuerzeichen,
es geht um das Streben zu Höherem,
den sterblichen Körper (Erde) mithilfe des Feuers,
also mit Imagination und Energie usw.\ umzuwandeln in etwas Unsterbliches.}

\en{And the soul is considered immortal—%
but the soul is not really (only) like the element Air,
it resonates also a lot with feeling and fate,
hence with the element Water.}%
\de{Und die Seele wird ja als unsterblich betrachtet—%
nur ist die Seele nicht wirklich (nur) wie das Element Luft,
da schwingt auch viel Gefühl und Schicksal mit,
also das Element Wasser.}
%
\en{And Water is exactly the one element of the four Greek elements
that lacks in the transformation of all Fire signs,
they actually often lack compassion and comprehension for their fellow human beings,
also Aries, despite his noble goals for all people.}%
\de{Und Wasser ist genau dasjenige Element,
von den vier griechischen Elementen,
das in der Verwandlung aller Feuerzeichen fehlt;
es fehlt ihnen durchaus oft an Mitgefühl und Verständnis für ihre Mitmenschen,
auch beim Widder, allen hehren Zielen für alle Mitmenschen zum Trotz.}

\en{Liz Greene often emphasizes this theme
in \textsl{The Astrology of Fate} for the Fire signs,
like for Aries, where Jason cheats on Medea,
or for Leo, where Parsifal has at first no comprehension of the purpose of the Grail,
but learns is later, or finally for Sagittarius,
where the centaur Cheiron finally sacrifices his immortality for someone else,
thus in a way fulfilling the “task of the Fire signs”
of learning to feel and show compassion.}%
\de{Gerade bei Liz Greene in \textsl{The Astrology of Fate}
kommt das oft zur Sprache bei den Feuerzeichen,
wie beim Widder wo Jason Medea betrügt
oder beim Löwen wo Parsifal erst kein Verständnis für den Zweck des Grals hat,
es aber später doch noch lernt, oder schliesslich beim Schützen,
wo der Zentaur Cheiron schliesslich seine Unsterblichkeit für einen Anderen opfert,
also quasi so die “Aufgabe der Feuerzeichen”,
Mitgefühl zu lernen und zu zeigen, erfüllt.}

\en{Another, much older source,
where the elementary transformation of the fire signs
(and also of the other star signs) is visible,
is in the four tasks of Psyche from \textsl{The Golden Ass}
by Lucius Apuleius from the second century CE.}%
\de{Eine andere, viel ältere Quelle,
wo die elementare Struktur der Wandlung der Feuerzeichen
(und auch der anderen Sternzeichen) sichtbar ist,
ist in den vier Aufgaben der Psyche aus \textsl{Der Goldene Esel}
von Lucius Apuleius aus dem 2.\ Jahrhundert unserer Zeitrechnung.}
%
\en{In the story, Lucius is transformed into an ass and experiences lots of adventures.}%
\de{In der Geschichte wird Lucius in einen Esel verwandelt und erlebt so viele Abenteuer.}
%
\en{As a story within the story,
the \textsl{Fairy Tale of Cupid and Psyche} is told,
where the god Cupid, son of the love goddess Venus,
falls in love with Psyche, the most beautiful woman of the world.}%
\de{Als Geschichte innerhalb der Geschichte
wird das \textsl{Märchen von Cupid und Psyche} erzählt,
wo sich der Gott Cupid, Sohn der Liebesgöttin Venus,
in die schönste Frau der Welt, Psyche, verliebt.}
%
\en{And, as another story within the story,
the story of the four tasks of Psyche is told,
which Venus, jealous of the already pregnant Psyche, poses to her,
in the ostensible intent of having her fail.}%
\de{Und als weitere Geschichte in der Geschichte,
wird von vier Aufgaben erzählt,
die Venus ganz eifersüchtig der bereits schwangeren Psyche stellt,
vordergründig um sie daran scheitern zu lassen.}
%
\en{The four tasks correspond to the four elements,
as I hope to show,
and as others already tried to assign them,
although, in my view, often the wrong way round.}%
\de{Die vier Aufgaben entsprechen den vier Elementen,
wie ich zu zeigen hoffe,
und wie es schon Andere zuzuordnen versuchten,
allerdings, finde ich, oft falsch.}
%
\en{But more about that later,
here the task which I relate to the element Fire,
the third task.}%
\de{Aber mehr dazu später,
hier erst mal die Aufgabe, die ich dem Element Feuer zuordne,
die dritte Aufgabe.}

\en{This task consisted of bringing back, before nightfall, a goblet filled with water
from the river Styx, which flows eternally in a circle.}%
\de{Diese Aufgabe bestand darin, bis zum Abend einen Kelch aus dem Wasser
des ewig im Kreis fliessenden Flusses Styx zu holen.}
%
\en{But the river is surrounded by steep, sharp rocks,
guarded by monsters and in addition its water is poisonous.}%
\de{Aber der Fluss ist umgeben von steilen, spitzen Felsen,
wird von Ungeheuern bewacht und zudem ist das Wasser auch noch giftig.}
%
\en{So, no getting through for an Aries
who likes to fight on the ground of facts,
sort of with his horned head through the wall;
a solution can be only be found with oversight from high up in the air,
in Apuleius’ story when Jupiter sends Psyche an eagle,
who flies from high in the air to the middle of the river
and fills the goblet with water.}%
\de{Also kein Durchkommen für einen Widder,
der gerne unmittelbar irdisch am Boden der Tatsachen kämpft,
quasi mit dem gehörnten Kopf durch die Wand,
erst mit Überblick von hoch oben in der Luft findet sich eine Lösung,
bei Apuleius indem Jupiter Psyche einen Adler schickt,
der aus der Luft zur Flussmitte fliegt
und den Becher mit Wasser füllt.}

\en{Here all four elements come together:
The Earth, where there is no getting through;
the Fire, which makes it possible by its visual, imaginative power
to get an overview of the situation
(in order to see something visually, light, Fire, is required);
the Air that had risen from the power of Fire,
which can then easily find a solution from above;
and finally the goal,
Water, the immortal soul,
which circles in everything,
like the eternal Styx,
to put it a bit poetically…}%
\de{Da finden sich nochmals alle vier Elemente:
Die Erde unten wo kein Durchkommen möglich ist;
das Feuer, das es ermöglicht sich durch bildliche Vorstellungskraft
von der Lage einen Überblick zu verschaffen
(um etwas bildlich zu sehen braucht es Licht, Feuer);
die durch die Kraft des Feuers aufgestiegene Luft,
die daraufhin von oben her einfach eine Lösung finden kann;
und schliesslich das Ziel,
das Wasser, die unsterbliche Seele,
die alles im Kreis durchfliesst wie der ewige Styx,
um es etwas poetisch auszudrücken…}

\en{So much for this example,
now first to the elements all to themselves
and then to step-by-step descriptions of all 12 star signs.}%
\de{Soweit mal das Beispiel,
jetzt erst zu den Elementen ganz für sich
und dann zu schrittweisen Beschreibungen aller 12 Sternzeichen.}

\en{(An overview of the structure of the model follows later,
but if you are too curious, maybe take a peek already earlier further back in this text
and then come back and read on?}%
\de{(Ein Überblick über die Struktur des Modells folgt erst später,
bei Interesse aber vielleicht einfach schon früher mal weiter hinten dazu hineinschauen
und dann wieder zurück und weiter lesen?}
%
\en{The theme of the Fire signs, and also the themes of the other star signs,
as will become visible later, are mirrored here:
Without having “lived”, so to speak, through all transformations of the star signs,
it is difficult to convey an overall view in a really accessible way.)}%
\de{Hier spiegelt sich eben auch das Thema der Feuerzeichen,
und auch die Themen der anderen Sternzeichen, wie sich noch zeigen wird:
Ohne die Wandlungen aller Sternzeichen quasi durchlaufen zu haben,
ist es schwierig eine Gesamtsicht wirklich zugänglich zu schildern.)}

\en{\subsection{The elements}}%
\de{\subsection{Die Elemente}}

\en{Astrologers are usually well aware of the four elements,
in most textbooks about astrology
they are presented with at least a few keywords.}%
\de{Die vier Elemente sind ja Astrologen meist wohl bekannt,
in den meisten Lehr\-bü\-chern zur Astrologie
werden sie auch jeweils mindestens mit ein paar Stichworten vorgestellt.}
%
\en{Hence only a relatively short description here,
although here and there substantiated in a maybe a bit unexpectedly specific way.}%
\de{Daher nur eine relativ knappe Beschreibung,
allerdings hier und da vielleicht schon etwas ungewohnt spezifisch begründet.}

\en{Fire and Air are described as male, active, warm, light elements,
Water and Earth as female, passive, cold, heavy elements.}%
\de{Feuer und Luft werden als männliche, aktive, warme, leichte Elemente beschrieben,
Wasser und Erde als weibliche, passive, kalte, schwere Elemente.}
%
\en{According to Aristotle, in addition Fire and Earth are dry, Air and Water wet,
which includes also more bendability, malleability and flexibility—%
this will be important later on regarding the structures in the model of the star signs,
but now briefly to the individual elements.}%
\de{Laut Aristoteles sind zudem Feuer und Erde trocken, Luft und Wasser feucht,
was auch mehr Biegsamkeit, Formbarkeit und Flexibilität beinhaltet—%
das wird später bezüglich der Strukturen im Modell der Sternzeichen noch wichtig sein,
aber jetzt kurz zu den einzelnen Elementen.}

\en{\textbf{Fire} has certainly immediately to do with sources of light and heat,
like with the sun, a fire, a light bulb or a thunderbolt, etc.}%
\de{\textbf{Feuer} hat sicher mal unmittelbar mit Quellen von Licht und Wärme zu tun,
wie mit der Sonne, einem Feuer, einer Glühbirne oder einem Blitz, usw.}
%
\en{In order to see something with the eyes,
a source of light, hence ‘Fire’, is necessary;
thus also figuratively visual imagination is associated with fire,
simply because also in the head sort of a form of Fire
is required to make an inner image visible.}%
\de{Um etwas mit den Augen überhaupt sehen zu können,
braucht es eine Lichtquelle, also ‘Feuer’,
daher wird auch im übertragenen Sinn bildliche Vorstellungskraft mit Feuer assoziiert,
einfach da es eben auch quasi im Kopf dann Licht, also eine Form von Feuer,
benötigt um sich eine Vorstellung innerlich sichtbar zu machen.}
%
\en{Thus any kind of visual imagination belongs to Fire,
as well as a bit more negatively projections.}%
\de{Daher gehören dann auch so Dinge wie Phantasie oder Imagination zum Element Feuer,
bis hin auch etwas negativer zu Projektion.}

\en{Of course such assignments are never really absolute
and also not always in accordance with findings in natural sciences,
but the things I described above are still very strong
and often the dominating associations
that Fire—of the four elements—clearly fits best.}%
\de{Natürlich sind solche Zuordnungen nie ganz absolut
und auch nicht immer im Einklang mit naturwissenschaftlichen Erkenntnissen,
aber die Dinge, die ich oben beschrieb, sind eben doch sehr starke,
oft die überwiegenden Assoziationen
wo Feuer—von den vier Elementen—klar am Besten passt.}

\en{\textbf{Air} also has a lot to do with thinking,
but rather abstract thinking, not in images, because air is invisible.}%
\de{\textbf{Luft} hat auch viel mit Denken zu tun,
aber eher abstrakter statt bildlich, da eben Luft unsichtbar ist.}
%
\en{Air can also be very changeful, like the wind,
or sort of unreachable, because it can neither be seen nor grasped,
which can appear arrogant as well as fleeting, depending on circumstances.}%
\de{Luft kann auch sehr wechselhaft sein wie der Wind,
oder quasi unerreichbar, da nicht sichtbar und nicht greifbar,
was je nachdem sowohl arrogant wie auch flüchtig wirken kann.}
%
\en{The association with language and communication comes about as follows:
A word like “dog” is an abstraction from the corresponding living animal,
in such terms it is then possible to think purely abstractly, logically,
even without necessarily having to imagine a dog,
and there are words which were abstract from the beginning,
like justice, or think of mathematics, etc.}%
\de{Der Bezug zu Sprache und Kommunikation kommt folgendermassen zustande:
Ein Wort wie “Hund” ist eine Abstraktion vom entsprechenden lebendigen Tier,
mit solchen Begriffen kann dann rein abstrakt logisch gedacht werden,
sogar ohne sich umbedingt einen Hund dabei vorzustellen,
bis hin zu von vornherein abstrakten Begriffen
wie Gerechtigkeit, oder denkt an Mathematik, usw.}

\en{\textbf{Water} can flow or move in waves,
which is associated with feelings.}%
\de{\textbf{Wasser} kann fliessen oder sich in Wellen bewegen,
was mit Gefühlen assoziiert wird.}
%
\en{The flowing of rivers into each other can be associated
directly with an increase of “influence”,
hence with an increase of power,
and for a river flowing downstream is unavoidable,
also since Water is a heavy element,
which both brings Water symbolically close to the forces of fate
(due to ‘influence/heavy/unavoidable’).}%
\de{Das Ineinanderfliessen von Flüssen kann direkt
mit einer Vergrösserung des “Einflusses” identifiziert werden,
also einer Machtzunahme,
und das Fliessen nach unten ist für einen Fluss unvermeidlich,
auch da Wasser ein schweres Element ist,
was dann zusammen das Wasser symbolisch auch nahe zu den Kräften des Schicksals bringt
(eben wegen ‘Einfluss/schwer/unvermeidlich’).}
%
\en{The flowing and moving of waves is certainly also close to music and dance and song,
and love, as well as maybe also close to a belief in something, 
because, shared with others, like music,
it creates a shared feeling of security within a community.}%
\de{Das Fliessende und Wogende ist dann sicher auch nahe bei Musik und Tanz und Gesang,
und bei Liebe, sowie vielleicht auch beim Glauben an etwas,
weil es geteilt mit anderen, wie Musik etc.,
eine gemeinsame Geborgenheit schafft.}

\en{Every year afresh, \textbf{Earth} recreates new plants and trees,
which also transform during the seasons.}%
\de{\textbf{Erde} erzeugt jedes Jahr wieder aufs Neue Pflanzen und Bäume,
die sich doch im Wechsel der Jahreszeiten wieder wandeln.}
%
\en{There is a certain stubbornness and hardness in this,
a certain inability to gain knowledge,
but I am getting ahead of myself,
especially with realistic Earth
it is maybe better to get to this with concrete examples.}%
\de{Darin liegt eine gewisse Sturheit und Härte,
ein gewisses Unvermögen zu Erkenntnis,
aber ich greife etwas vor,
gerade bei der realistischen Erde ist es vielleicht besser
bei den konkreten Beispielen darauf zu sprechen zu kommen.}

\en{But now for the star signs in detail.}%
\de{Aber nun zu den einzelnen Sternzeichen im Detail.}
%
\en{Und just to avoid too high expectations:
The star signs are age-old and have already been lived through by so many people
that I can by no means provide a complete and balanced description here.}%
\de{Und damit nicht eine zu hohe Erwartung entsteht:
Die Sternzeichen sind uralt und wurden schon von so vielen Menschen durchlebt,
dass ich hier keineswegs eine vollständige und ausgeglichene Schilderung bieten kann.}
%
\en{But what I can hopefully do,
is convey that these transformations of the elements
are actually present in the star signs,
sort of as an inner \textsl{skeleton},
as the innermost structure of the star signs,
but around that there is still a lot of flesh and spirit and soul
and simply many ways
in which to live with it and how to live it out.}%
\de{Aber was ich hoffentlich tun kann,
ist vermitteln, dass diese Wandlungen der Elemente
in der Sternzeichen tatsächlich vorhanden sind,
so quasi als innerstes \textsl{Skelett},
als die innerste Struktur der Sternzeichen,
aber da herum hat es dann noch sehr viel Fleisch und Geist und Seele
und einfach ganz viele Wege,
wie genau damit leben und wie es ausleben.}
%
\en{The discussion about the star signs does not end here,
quite to the contrary.}%
\de{Die Diskussion um die Sternzeichen endet hier keineswegs,
ganz im Gegenteil.}

\en{\subsection{The Fire signs}}%
\de{\subsection{Die Feuerzeichen}}

\en{What is common to all Fire signs in the model,
is that they consist foremost of Fire,
empowered or sometimes maybe even possessed by lots of imagination.}%
\de{Was den Feuerzeichen im Modell also allen eigen ist,
ist, dass sie natürlich erst mal vorwiegend aus Feuer bestehen,
beseelt oder vielleicht machmal sogar besessen von einer grossen Vorstellungskraft.}
%
\en{The energy behind that is strongest in youth, hence in Aries,
because he still has the biggest supply of Earth fuel, of physical power to waste.}%
\de{Die Energie, die dahinter steht, ist in der Jugend, also beim Widder, am Grössten,
da der noch den grössten Brennvorrat an Erde, an Körperkraft zum Verbrauchen hat.}
%
\en{In Leo the energy is already a bit more tamed, but still wild,
and Sagittarius has to use his energy often selectively demonstratively,
like Zeus his thunderbolts.}%
\de{Beim Löwen ist es oft schon etwas sanfter, aber immer noch wild,
und der Schütze muss seine Energie dann oft gewählt demonstrativ einsetzen,
wie Zeus seine Blitze.}

\en{The burning of Fire can also be seen
as the Fire signs learning the path to abstract insights
and thoughts and theories,
sort of experimentally from reality,
by at first failing repeatedly at reality,
like the proverbial Aries,
who runs so often headfirst into any obstacles
that one might suspect purpose.}%
\de{Das Verbrennen der Erde kann man auch so sehen,
dass die Feuerzeichen anhand der Realität lernen,
den Weg zu abstrakten Einsichten und Gedanken und Theorien
quasi experimentell zu finden,
indem sie erst an der Realität wiederholt scheitern,
wie der sprichwörtliche Widder,
der so oft wieder in alle Hindernisse kopfvoran hineinrennt,
dass man Absicht dahinter vermuten könnte.}

\en{\subsection{The Fire signs – Aries}}%
\de{\subsection{Die Feuerzeichen – Widder}}

\en{Aristotle also arranged the elements in a circle—%
Earth to Fire to Air to Water and then again to Earth—%
and in this circle, among the Fire signs,
Aries is still the farthest away from the element Water,
in the direction of his life,
thus for him learning compassion (i.e.\ Water) with his fellow human beings
is at first hardly a theme for him,
but it is still expected of him to learn this a little bit in his life.}%
\de{Aristoteles hatte die Elemente auch in einem Kreis angeordnet—%
Erde zu Feuer zu Luft zu Wasser und dann wieder zu Erde—%
und in diesem Kreis ist der Widder unter der Feuerzeichen,
da er ja aus viel Erde besteht,
vom Element Wasser noch am Weitesten entfernt,
in seiner Laufrichtung im Leben;
für ihn ist Mitgefühl (also Wasser) für seine Mitmenschen zu lernen
zumindest erst mal noch kaum ein Thema,
aber es wird von ihm durchaus erwartet,
dass er es doch ein klein wenig schon noch lernt in seinem Leben.}
%
\en{From Leo, this is already quite a bit more is expected,
also coming from the image of a good king for his people,
and definitely from Sagittarius.}%
\de{Beim Löwen ist es schon viel mehr erwartet,
auch aus der Vorstellung von einem guten König für das Volk heraus
und erst recht beim Schützen.}

\en{On the other hand, seen the other way round, Aries is still very close to Water,
it is simply “behind” him.}%
\de{Allerdings ist umgekehrt gesehen der Widder noch sehr nahe vom Wasser,
es ist einfach “hinter” im.}
%
\en{That would for example fit Jason in mythology,
who the magician Medea helped a lot in the beginning,
and he married her and had children with her,
until he forgot about all of that and cheated on her.}%
\de{Das würde zum Beispiel in der Mythologie zu Jason passen,
dem die Zauberin Medea am Anfang sehr weit hilft,
und er sie auch heiratet und Kinder mit ihr hat,
bis er das dann vergisst und sie betrügt.}

\en{Or where young Jason carried an old woman across a river
and lost a sandal.}%
\de{Oder wo der noch junge Jason eine alte Frau über einen Fluss trug
und dabei eine Sandale verlor.}
%
\en{The old woman was the goddess Hera in disguise
and the river symbolizes Water, of the elements, of course,
the element he thus symbolically left behind.}%
\de{Die alte Frau war die verkleidete Göttin Hera
und der Fluss symbolisiert von den Elementen natürlich das Wasser,
das er ja dann hinter sich liess.}
%
\en{But the river is symbolically also where the dead were carried across,
hence a symbolic rebirth,
which one could well also see as the transition from Pisces to Aries,
hence to the new astrological year.}%
\de{Aber der Fluss ist auch symbolisch wohin man die Toten hinübertrug,
also eine symbolische Wiedergeburt,
die man durchaus auch als den Übergang von Fischen zum Widder,
also zum neuen astrologischen Jahr sehen könnte.}

\en{In order to get a young fire like Aries burning,
it takes at first a lot of care,
you have to protect it from wind and,
yet, provide air, so that it can catch on.}%
\de{Um ein junges Feuer wie den Widder zum Brennen zu bringen,
braucht es erst mal sehr viel Umsicht,
man muss es vor Wind schützen
und doch für etwas Luft sorgen damit es anzieht.}
%
\en{The finesse of Pisces
and the wisdom of the whole zodiac contained within them would fit this very well.}%
\de{Dazu würde die Feinheit der Fische
und die darin enthaltene Weisheit aus dem ganzen Sternkreis sehr gut passen.}
%
\en{And, last but not least,
in my model Pisces would be composed mainly of Air,
similar to Sagittarius, as I will expose later on.}%
\de{Und nicht zuletzt,
würden in meinem Modell die Fische aus vorwiegend noch viel Luft bestehen,
ähnlich wie der Schütze, wie ich noch ausführen werde.}

\en{Returning to the myth of Jason and Medea,
she fits symbolically quite well with Pisces,
not least in her sort of boundless revenge on Jason,
where she did not even spare her own children.}%
\de{Da trifft sich der Mythos mit Jason und Medea wieder;
sie passt symbolisch sicher recht gut zu den Fischen,
nicht zuletzt in ihrer quasi masslosen Rache an Jason,
wobei sie sogar ihre eigenen Kinder nicht verschonte.}

\en{In \textsl{The Astrology of Fate},
Liz Greene writes a lot more about Jason,
and also about more Aries themes.}%
\de{Liz Greene schreibt in \textsl{The Astrology of Fate}
zu Jason noch einiges mehr,
und auch noch zu einigen anderen Themen zum Widder.}
%
\en{For example, she strongly emphasizes the theme of a fight against the father for Aries.}%
\de{Zum Beispiel thematisiert sie den Kampf gegen den Vater sehr stark beim Widder.}
%
\en{I am not sure whether she maybe exaggerates a little bit there,
but the statement
that in heaven there is only room for one god (Zeus/Jupiter/(Jahwe)
would fit to some degree;
in order to get up to heaven,
you would have to chase others from there.}%
\de{Ich weiss nicht so recht, ob sie da nicht vielleicht doch etwas übertreibt,
aber die Aussage,
dass es im Himmel eben dann nur Platz für einen Gott (Zeus/Jupiter/Jahwe) hat,
würde eben schon auch passen, 
so wäre es nötig, um nach oben in den Himmel zu kommen,
auch andere von dort zu verdrängen.}
%
\en{Then again, such a hard exclusion fits
rather with the element Earth than with the element Air.}%
\de{Andererseits passt so ein harter Ausschluss
eher zum Element Erde wie zum Element Luft.}

\en{\subsection{The Fire signs – Leo}}%
\de{\subsection{Die Feuerzeichen – Löwe}}

\begin{quote}
\textsl{\color{xphi}%
\en{But the lion is a stage in a process, as Jung suggests;
and it is this process or pattern
which brings us into the sphere of the ‘fate’ of Leo.}%
\de{Doch der Löwe ist eine Stufe in einem Prozess, wie Jung vorschlägt;
und es ist dieser Prozess oder dieses Muster,
welches uns in die Sphäre des ‘Schicksals’ des Löwen bringt.}
%
\en{It would seem […$\!$] that there is an alchemical work to be performed.}%
\de{Es scheint […$\!$], dass eine alchemistische Wandlung vollführt werden muss.}
%
\en{The Lion is not permitted to remain in its bestial form,
but must give way to something other.}%
\de{Dem Löwen ist es nicht gestattet in seiner tierischen Form zu bleiben,
sondern er muss etwas Anderem weichen.}}
\vspace{1mm}
\newline
\en{— Liz Greene, \textsl{The Astrology of Fate} (Weiser 1984),
in the chapter about Leo}%
\de{— Liz Greene, \textsl{The Astrology of Fate} (Weiser 1984),
im Kapitel über den Löwen}
\end{quote}

\noindent
\en{This fits, of course, very well with the model,
it describes again a transformation.}%
\de{Das passt natürlich sehr gut zum Modell,
es beschreibt wiederum eine Wandlung.}
%
\en{The body of a human being is rather the animal side, its spirit
rather the human or even divine side,
since thoughts can be passed on
und thus have a certain immortality
and immortality is also an essential property of gods.}%
\de{Der Körper eines Menschen ist eher seine tierische Seite,
sein Geist eher seine menschliche oder sogar göttliche Seite,
da eben Gedanken weitergegeben werden können
und so eine gewisse Unsterblichkeit haben,
und Unsterblichkeit ist eben auch eine wesentliche Eigenschaft von Göttern.}
%
% looks like I missed that sentence in the english translation,
% but maybe leaving as-is...
\de{Beim Löwen als dem König der Tiere ist das Thema sehr stark:
Er muss seine tierischen Instinkte überwinden,
um ein guter und gerechter Herrscher sein zu können.}

\en{Connected to that is also the theme of the \textsl{wound},
which Liz Greene emphasizes especially for Leo and Sagittarius
in \textsl{The Astrology of Fate}.}%
\de{Damit verbunden ist auch das Thema der \textsl{Wunde},
das Liz Greene besonders beim Löwen und beim Schützen
in \textsl{The Astrology of Fate} betont.}
%
\en{In the legend of the Holy Grail, the Grail king is wounded at the lower part of his body
and thus no longer potent, which fits, of course, again into the picture;
with growing age the body is no longer able to produce offspring.}%
\de{Bei der Gralssage ist der Gralskönig an der unteren Körperhälfte verwundet
und somit nicht mehr zeugungsfähig, was natürlich auch wieder zum Bild passt;
mit zunehmendem Alter ist der Körper nicht mehr fähig Nachwuchs zu zeugen.}
%
\en{Just read up what on Liz Greene writes about that theme for more details and insights.}%
\de{Einfach dort bei Liz Greene nachlesen für noch mehr Details und Einsichten.}

\begin{quote}
\textsl{\color{xphi}%
\en{Leo at heart is a religious sign,
using the word here as it was originally meant: to reconnect.}%
\de{Der Löwe ist im Herzen ein religiöses Zeichen,
im ursprünglichen Sinn des Wortes: Wiederverbinden.}
%
\en{It is the secret of his intensive need to create something
which mirrors his own essential substance –
be it a company, a book, a painting, a political movement, an airline,
a scientific achievement, an empire, a photograph.}%
\de{Es ist das Geheimnis seines intensiven Bedürfnisses etwas zu schaffen,
was seine essentielle Substanz spiegelt –
sei es eine Firma, ein Buch, ein Gemälde, eine politische Bewegung, eine Fluggesellschaft,
ein wissenschaftliches Resultat, ein Imperium, eine Photographie.}}
\vspace{1mm}
\newline
\en{— Liz Greene, \textsl{Star Signs for Lovers} (Stein and Day 1984),
in the chapter about Leo}%
\de{— Liz Greene, \textsl{Star Signs for Lovers} (Stein and Day 1984),
im Kapitel über den Löwen}
\end{quote}

\noindent
\en{Incidentally, she relativizes this statement in \textsl{The Astrology of Fate},
I presume rightly, as follows, arguing from the elementary transformation.}%
\de{In \textsl{The Astrology of Fate} relativiert sie diese Aussage übrigens,
ich vermute zu recht, wie folgt, argumentiert aus der elementaren Wandlung heraus.}

\en{A book, for example,
is a mix of Earth (you can touch it) and Air (the textual content);
hence even if a book mirrors maybe Leo in the middle of the transformation,
it does not really live up to his goals,
because in the end these are rather more immaterial than a book or a painting or a company.}%
\de{Ein Buch zum Beispiel,
ist ein Mix aus Erde (man kann es anfassen) und Luft (der sprachliche Inhalt),
also auch wenn es vielleicht den Löwen in der Mitte der Umwandlung spiegelt,
ganz seinen Zielen gerecht wird es nicht,
denn die sind am Ende eher noch immateriellerer Natur als ein Buch oder ein Gemälde oder eine Firma.}
%
\en{Also reconnecting, via belief and more,
has is a relation to Water;
Leo wants and must in the end feel something when he does something,
it must be in relation to something that others wish for,
which they dream of, sometimes including of the lion himself.}%
\de{Auch hat es in ‘Wiederverbinden’ eben auch via Glauben
und so einen Bezug zum Wasser,
der Löwe will und muss am Ende etwas fühlen bei dem was er tut,
es muss in Beziehung stehen auch zu Dingen,
die sich andere wünschen,
von denen sie träumen, manchmal inklusive vom Löwen.}

\en{\subsection{The Fire signs – Sagittarius}}%
\de{\subsection{Die Feuerzeichen – Schütze}}

\en{The centaur Cheiron, one of the centaurs
that had been lifted into the sky as the constellation Sagittarius in mythology
was immortal,
but once while hunting, he was accidentally wounded at the hip by a poisoned arrow
shot by his good friend Herakles.
Since Cheiron was immortal, he could not die of the wound,
but, from then on, suffered permanently deadly pain.}%
\de{Der Zentaur Cheiron, einer der Zentauren,
die in der Mythologie in den Himmel als das Sternbild Schütze erhoben wurden,
war unsterblich,
aber einmal bei der Jagd wurde er versehentlich von seinem guten Freund Herakles
mit einem vergifteten Pfeil an der Hüfte getroffen,
konnte aber nicht daran sterben, da er unsterblich war,
leidete von da an aber permanent unsterbliche Qualen.}

\en{Cheiron has a lower part of the body that is animal, of a horse,
so it is symbolically again very fitting with the model
that he was wounded there (‘Earth’) by an arrow
and not, say, in the head (‘Air’, because one thinks with the head).}%
\de{Cheiron hat hat eine untere Körperhälfte die tierisch ist, von einem Pferd,
es ist also symbolisch wieder völlig passend zum Modell,
dass er dort (‘Erde’) vom Pfeil getroffen wurde
und nicht zum Beispiel am Kopf (‘Luft’, da mit dem Kopf gedacht wird).}
%
\en{His wound made it possible for him to learn compassion,
no longer so much distracted by his animal instincts.}%
\de{Seine Wunde ermöglicht es ihm erst, wirklich Mitgefühl zu lernen,
nicht mehr so sehr abgelenkt von tierischen Instinkten.}

\en{In the end he sacrificed himself, so that Prometheus,
who had stolen Fire from the gods, could become immortal.}%
\de{Am Ende opferte er sich, damit Prometheus die Unsterblichkeit erlangte,
welcher von den Göttern das Feuer gestohlen hatte,
um den Menschen so mehr Erkenntnis (Luft) zu bringen.}
%
\en{Hence again the same theme:
People down on earth get Fire and through that knowledge (Air)—%
the cycle of the transformation of the Fire signs—%
and in the end it is completed to immortality,
just like at Apuleius with the goblet of water from the eternally circling river Styx,
by Cheiron who sacrifices himself in his wisdom.}%
\de{Also wieder das gleiche Thema:
Die Menschen unten auf der Erde bekommen Feuer und dadurch Erkenntnis (Luft)—%
der Zyklus der Wandlung der Elemente bei den Feuerzeichen—%
und am Ende wird er vervollständigt zur Unsterblichkeit,
wie bei Apuleius mit dem Becher Wasser aus dem ewig kreisenden Fluss Styx,
indem Cheiron sich in seiner Weisheit opfert.}

\en{\subsection{The Water signs}}%
\de{\subsection{Die Wasserzeichen}}

\en{The transformation of the Water signs
is similar to the one of the Fire signs,
it starts again with Earth and ends with Air,
only this time Water transforms these two elements
and this time the missing element is, conversely, Fire.}%
\de{Die Wandlung der Wasserzeichen ist sehr ähnlich wie bei den Feuerzeichen,
wieder beginnt sie mit Erde und endet mit Luft,
nur diesmal verwandelt das Wasser die zwei Elemente,
und das Element, das fehlt, ist diesmal umgekehrt das Feuer.}

\de{\vspace{1mm}
\includegraphics[scale=0.1625]{i-wasser.jpg}}

\en{There are maybe two descriptive images for this.}%
\de{\noindent
Es gibt hier vielleicht zwei anschauliche Bilder dazu.}
%
\en{The most elementary one is water
in its three aggregate states ice, water and gas/steam.}%
\de{Das elementarste ist Wasser
in seinen drei Aggregatzuständen Eis, Wasser und Gas/Dampf.}
%
\en{Hence a block of ice that melts and finally evaporates.}%
\de{Also ein Eisblock, der langsam schmilzt und dann verdunstet.}
%
\en{Another image is the life of a river,
from a source in the mountains,
maybe out of glacier ice,
to a river that flows down and grows
until it finally flows into the sea
and there evaporates again,
with the power of the sun (Fire),
in order to start the cycle again,
when it returns as rain or snow to the mountain.}%
\de{Ein anderes ist das Leben eines Flusses,
von einer Quelle in den Bergen,
vielleicht ja aus einem Gletscher aus dem Eis hervor,
zu einem Fluss, der herunterfliesst und immer wächst,
bis er schliesslich ins Meer fliesst
und dort mit der Kraft der Sonne (Feuer) wieder verdunstet,
um so den Zyklus wieder zu beginnen,
wenn er als Regen oder Schnee wieder zum Berg zurückkehrt.}

\en{\vspace{1mm}
\includegraphics[scale=0.1625]{i-water.jpg}}

\en{\noindent
Cancer is the first Water sign,
with its hard earthy shell (Earth),
Scorpio as the second Water sign is already more flexible—%
‘liquid’ like Water—%
but still able to isolate herself from case to case,
and Pisces as the third Water sign are then already hard to catch,
the borders have blurred to gas (Air).}%
\de{Der Krebs ist das erste Wasserzeichen,
mit seiner irdischen harten Schale (Erde),
der Skorpion als das zweite Wasserzeichen ist schon flexibler—%
‘flüssig’ wie Wasser—%
aber grenzt sich doch durchaus von Fall zu Fall noch ab,
und die Fische als das dritte Wasserzeichen
sind dann schon kaum mehr zu fassen,
die Grenzen verschwimmen da bis zum Gas (Luft).}

\en{Or if you consider a river:
A source has not yet mixed with other rivers,
its water seems uniform to it,
later when rivers flow together (Scorpio),
symbolically different influences come together
and must be processed in some way,
until everything comes together in the sea and is relativized (Pisces).}%
\de{Oder wenn man es für einen Fluss betrachtet:
Eine Quelle hat sich noch nicht mit anderen Flüssen vermischt,
ihr Wasser scheint ihr einheitlich,
beim Zusammenfliessen (Skorpion) von Flüssen
kommen dann symbolisch verschiedenartige Einflüsse zusammen
und müssen irgendwie verarbeitet werden,
bis dann im Meer alles zusammenkommt und sich relativiert (Fische).}
%
\de{Daraus entsteht durchaus wieder Erkenntnis (Luft):
Aus der Erfahrung mit all den Vermischungen mit den anderen Flüssen lernt der Fluss,
dass es ganz verschiedene Wege gibt, die Welt zu sehen,
sowie auch schliesslich die Erkenntnis, dass Grösse relativ ist,
denn auch wenn ein Fluss gegen alle anderen Flüsse gewinnt,
indem er der Grössere ist und so den Namen des Flusses behaupten kann,
gegenüber dem Meer ist er dann schliesslich sowieso nur noch klein.}

\en{That way again knowledge (Air) is created:
From the experience of all that mixing with the other rivers,
the river learns that there are different ways of seeing the world,
as well as the realization that size is relative,
because even if a river wins against all other rivers
by being the larger one and thus defending the name of the river,
compared to the sea, it is finally just small in any case.}

\en{In the end only an escape into imagination (Fire) helps here,
resp.\ it is probably not fair to speak of escape,
because,
for one,
it is a cycle,
and on the other hand Pisces are the last star sign,
the body sort of symbolically already close to death,
imagination is the only thing that remains there,
and that can certainly move even mountains
and with gentle blows of Air
maybe gently direct an Aries-style hero like Jason?}%
\de{Da hilft am Ende nur noch die Flucht in die Phantasie (Feuer),
bzw.\ es ist wohl nicht fair da von Flucht zu sprechen,
denn einerseits ist es ein Kreislauf,
andererseits sind eben Fische das letzte Sternzeichen,
der Körper dem Tode quasi symbolisch irgendwie schon nahe,
da bleibt eben nur noch die Phantasie,
und die kann dann durchaus wohl auch Berge bewegen
und dazu mit sanften Luftstössen
vielleicht einen Widder-Helden wie Jason sanft steuern?}

\en{\subsection{The Water signs – Cancer}}%
\de{\subsection{Die Wasserzeichen – Krebs}}

\en{In Apuleius’ story, the second task of Psyche consisted of
getting some fleece from wild rams.}%
\de{Die zweite Aufgabe der Psyche bei Apuleius bestand darin,
von wilden Widdern etwas Vlies zu beschaffen.}
%
\en{These rams with poisonous teeth were grazing near a river,
hence quite a similar situation as with the river Styx in the third task.}%
\de{Diese Widder mit giftigen Zähnen grasten an einem Fluss,
also eine recht ähnliche Szenerie wie beim Fluss Styx in der dritten Aufgabe.}
%
\en{Only this time, the solution was not visual imagination,
but rather of a female nature,
as to be expected for a female element like Water,
it was rather experience and adaptation to natural cycles,
or, put differently, getting a feeling of the situation
to the degree of becoming one with it.}%
\de{Nur lag diesmal die Lösung nicht in bildlicher Vorstellung,
sondern war eher weiblicher Natur,
wie für ein weibliches Element wie Wasser zu erwarten,
lag mehr in der Erfahrung und der Anpassung an natürliche Zyklen,
oder anders gesagt an Einfühlung in die ganze Situation.}
%
\en{Instead of trying anything when the rams were wild and awake,
Psyche simply waited until the rams were asleep
and also only took fleece
which the rams had stripped off by their own fiery power on thorny bushes.}%
\de{Anstatt es zu versuchen, als die Widder wach und wild waren,
wartete Psyche einfach bis die Widder schliefen
und nahm auch nur Vlies,
das die Widder mit ihrer eigenen feurigen Kraft an dornigen Büschen abgestreift hatten.
% missing in english text:
So eroberte Psyche symbolisch ein Stück Feuer im Form des Vlieses.}

\en{Similar to Aries, learning this is not so important, yet, for Cancer,
he moves rather still in his very own cycles
and thus interacts often also not so harmonically with his environment,
rather still has some rough edges.}%
\de{Ähnlich wie beim Widder ist es auch für das erste Wasserzeichen,
den Krebs, noch nicht so wichtig das zu lernen,
er bewegt sich eher noch in seinen eigenen Zyklen
und eckt dadurch dann auch eher an.}

\en{In \textsl{The Astrology of Fate},
Liz Greene associates one of the most famous heroes, Heracles, with Cancer,
because the story how the crab came up to the sky
is connected with one of the tasks of Heracles,
namely the one where a crab pinched Heracles into the foot
while he was fighting the Hydra in the swamps
and Heracles stomped the crab,
but Hera thankfully lifted the crab into the sky
as the constellation Cancer.}%
\de{Liz Greene assoziiert in \textsl{The Astrology of Fate}
einen der bekanntesten Helden, Herakles, mit dem Krebs,
da die Geschichte, wie der Krebs in den Himmel kam,
mit einer Aufgabe von Herakles zusammenhängt,
nämlich, dass der Krebs Herakles beim Kampf gegen die Hydra
in den Sümpfen in den Fuss zwickte
und Herakles den Krebs daraufhin zertrat,
aber Hera den Krebs als Dank dafür in den Himmel hob.}

\en{This shows an important theme related to Cancer:
The water of a source is the same as the water in the sea,
Cancer itself already contains many different streams of water,
which he does not understand and can hardly tame.}%
\de{Darin sieht man ein wichtiges Thema zum Krebs:
Das Wasser einer Quelle ist dasselbe wie das Wasser des Meeres,
der Krebs enthält selber schon sehr viele verschiedene Strömungen von Wasser,
die er nicht versteht und nur schwer bändigen kann.}
%
\en{In this sense,
the crab, Heracles, Hera, the Hydra and the swamps are all aspects
of one and the same Cancer,
which partially fight or reward each other.}%
\de{In dem Sinn sind der Krebs, Herakles, Hera, die Hydra und die Sümpfe
alles Aspekte ein und desselben Krebses,
die sich teilweise gegenseitig bekämpfen und belohnen.}
%
\en{The result of this often seems pointless at first,
on the other hand, Cancer learns through this and gains experience
and a story like the one of Heracles is also full of imagination (Fire),
otherwise such an old story would be told anymore today.}%
\de{Das Resultat scheint da oft erst mal sinnlos,
andererseits lernt der Krebs eben dadurch, er gewinnt an Erfahrung
und so eine Geschichte wie bei Herakles ist ja auch voll von Phantasie (Feuer),
sonst würde ich eine so alte Geschichte hier ja kaum mehr erzählen.}

\en{It is just more difficult with Water
to put these transformations rationally into words
than with Fire.}%
\de{Es ist eben mit Wasser etwas schwieriger,
diese Umwandlungen der Elemente rational in Worten zu beschreiben,
wie beim Feuer.}
%
\en{Take the USA (Cancer sun) for example.
There it is often very difficult to predict who the next president will be,
and, yet, the result of all these complex,
and often also quite strange forces in the USA
is creative and often has a positive image.}%
\de{Oder zum Beispiel in den USA (Sonne im Krebs);
da ist es oft sehr schwer vorherzusagen, wer der nächste Präsident wird,
und doch ist das Resultat von all den komplexen,
teilweise auch sehr seltsamen Kräften in den USA
am Ende kreativ und oft sehr bildhaft positiv.}
%
\en{(Of course, the USA cannot be reduced to Cancer alone,
at lot more plays into it, and, yet, what I just wrote
describes something essential already very beautifully—%
what comes out of the USA into the world
is often quite special things that definitely remind of Fire.)}%
\de{(Natürlich kann man die USA nicht nur auf Krebs reduzieren,
da spielt noch viel mehr mit, und doch, was ich gerade schrieb,
beschreibt eben doch schon etwas Wesentliches sehr schön—%
was aus den USA hinaus in die Welt kommt,
sind eben oft recht besondere Dinge,
die durchaus an Feuer erinnern.)}

\en{\subsection{The Water signs – Scorpio}}%
\de{\subsection{Die Wasserzeichen – Skorpion}}

\begin{quote}
\textsl{\color{xphi}%
\en{The time of the autumn floods had come
and hundreds of wild streams were pouring into the great Yellow River.}%
\de{Die Zeit der Herbstfluten war gekommen,
und hunderte von wilden Bächen ergossen sich in den grossen Gelben Fluss.}
%
\en{Its waters were swelling and swelling, so that from one bank to the other
you could no longer tell an ox from a horse.}%
\de{Seine Wasser schwollen und schwollen, so dass man von einem Ufer zum anderen
nicht mehr den Ochsen vom Pferd unterscheiden konnte.}
%
\en{The god of the Yellow River rejoiced; he had the proud feeling
that there was nothing in the world that was not at his command.}%
\de{Der Gott des Gelben Flusses freute sich; er hatte das stolze Gefühl,
dass nichts auf der Welt sei, das ihm nicht zu Gebote stand.}
%
\en{He let himself drift downstream on the glorious floods and thus arrived at the open sea.}%
\de{Er liess sich auf den herrlichen Fluten hinab treiben und kam so zum offenen Meer.}
%
\en{There he turned left and looked into the distance.}%
\de{Dort wandte er sich gen Osten und blickte in die Ferne.}
%
\en{But as far as his eye could reach, he could not see an end to the water.}%
\de{Aber so weit sein Auge reichte, konnte er doch kein Ende des Wassers sehen.}}
\vspace{1mm}
\newline
\en{— \textsl{Zhuangzi}, from the chapter \textsl{The Floods of Autumn}
(English translation of Richard Wilhelm’s translation to German)}%
\de{— Dschuang Dsi, \textsl{Das wahre Buch vom südlichen Blütenland},
Kapitel Herbstfluten (übersetzt von Richard Wilhelm)}
\end{quote}

\noindent
\en{The above had been written by taoist philosopher Zhuangzi about 300 BCE in China,
but it fits well here,
also because in the Western zodiac Scorpio is in the middle of autumn.}%
\de{Das schrieb der taoistische Philosoph Dschuang Dsi um etwa 300 vor Christus in China,
aber es passt hier gut dazu,
auch da der Skorpion ja im westlichen Sternkreis in der Mitte vom Herbst ist.}

\en{In the following, the river talks with the god of the sea
and learns from him about relativity.}%
\de{Der Fluss spricht dann im Folgenden mit den Meeresgott
und lernt von ihm was relativ ist.}
%
\en{During his entire journey downwards, he had been solely busy
with growing and seeing his apparent power grow
and missed the opportunity to reflect on what he experienced,
to learn and imagine things,
like that there might be things that are still much larger than him
and that his strife was maybe not be so important and ultimately necessary at all.}%
\de{Während seiner ganzen Reise hinab war er einzig damit beschäftigt gewesen,
zu wachsen und damit seine scheinbare Macht wachsen zu sehen,
und versäumte es das Erlebte zu reflektieren, um daraus zu lernen
und auch um sich etwas vorzustellen,
wie eben, dass es Dinge geben könnte, die noch viel grösser sind wie er selbst
und dass sein Streben vielleicht gar nicht so wichtig und umbedingt notwendig war.}
%
\en{As a smaller river, he would still have ended up in the very same sea.}%
\de{Auch als kleinerer Fluss wäre er in demselben Meer gelandet.}

\en{Just as with Cancer, in \textsl{The Astrology of Fate},
Liz Greene writes also for Scorpio about a fight with the mother.}%
\de{Wie beim Krebs schreibt Liz Greene in \textsl{The Astrology of Fate}
auch beim Skorpion vom Kampf mit der Mutter.}
%
\en{The flowing into the sea can in that sense
maybe also be seen as a liberation from the mother;
resp.\ symbolically that happens maybe rather
through gaining knowledge and a perspective of what is and what else could be
and what is really important in life.}%
\de{Das Fliessen ins Meer kann man durchaus auch in dem Sinn
als eine Befreiung von der Mutter sehen,
bzw.\ symbolisch geschieht dies eben vielleicht eher
durch Erkenntnis und Vorstellung davon was noch ist und sein könnte
und was wirklich wichtig ist im Leben.}

\en{But the other way round, of course,
such insights can also serve to expand one’s own power.}%
\de{Aber umgekehrt kann natürlich solche Einsicht auch dazu dienen,
die eigene Macht noch auszudehnen.}
%
\en{Especially with the Water signs,
this shows that a simple model like this one based on elements
does not destroy the complexity of the star signs at all,
in the end the same questions, themes and complexes remain,
even if now some things are more structured,
more made of a single gush,
even if this gush can again bring forth many different gushes,
thus in the end rather increasing the overall richness of the star signs than reducing it.}%
\de{Gerade bei den Wasserzeichen zeigt sich,
dass ein einfaches Modell, wie hier mit den Elementen,
die Komplexität der Sternzeichen durchaus nicht zerstört,
es bleiben am Ende die gleichen Fragen, Themen und Komplexe,
aber doch ist jetzt einiges doch strukturierter,
mehr aus einem Guss,
auch wenn dieser Guss wiederum viele verschiedene Ströme hervorbringen kann,
also insgesamt dann den Reichtum der Sternzeichen am Ende
eher noch erhöht als ihn zu reduzieren.}

\en{\subsection{The Water signs – Pisces}}%
\de{\subsection{Die Wasserzeichen – Fische}}

\en{If I understood Pisces.}%
\de{Wenn ich die Fische verstehen würde.}
%
\en{But still a little bit of something.}%
\de{Aber trotzdem ein wenig etwas.}

\en{One of the first computer programs written under the title \textsl{Artificial Intelligence}
was called ELIZA and it was as a very simple program
which just mirrored its human counterpart.}%
\de{Eines der ersten Computerprogramme,
das unter dem Titel \textsl{künstliche Intelligenz} geschrieben wurde,
hiess ELIZA und war ein ganz einfach gestricktes Programm,
das einfach nur sein menschliches Gegenüber spiegelte.}
%
\en{If someone told it (typed into a keyboard) something about his or her mother,
ELIZA asked then to tell it more about one’s mother
or more generally one’s family.}%
\de{Erzählte jemand (eingetippt auf einer Tastatur) etwas von seiner Mutter,
bat ELIZA dann zum Beispiel darum, etwas mehr über die Mutter
oder allgemein die Familie zu erzählen.}
%
\en{ELIZA simply deduced a question from what was said,
and this was often perceived as very compassionate on the other side.}%
\de{ELIZA leitete einfach immer eine Frage aus dem Gesagten ab,
und das kam oft als sehr mitfühlend an auf der Gegenseite.}

\en{But, what does this have to do with the elements?}%
\de{Nur, was hat das jetzt mit den Elementen zu tun?}
%
\en{Well, at first it is a purely abstract airy mirroring of received sentences.}%
\de{Na ja, es ist erst mal reines abstraktes lufthaftes Spiegeln von empfangenen Sätzen.}
%
\en{The imagination (Fire) was here rather on the other side,
who imagined some things
and afterwards maybe wrote even more interesting things
than might have been the case without a question back.}%
\de{Die Phantasie (Feuer) war da eher auf der Gegenseite,
die sich dann etwas darunter vorstellte,
und daraufhin wohl noch Interessanteres schrieb
als ohne Nachfrage.}
%
\en{Thus ELIZA created something related to Fire
without having to create much of an imaginary power of its own.}%
\de{So schuf etwas fische-artiges wie ELIZA etwas,
das mit Feuer zu tun hatte in anderen,
ohne selber gross eine Vorstellungskraft zu erschaffen.}

\en{Maybe that is generally the case;
the manipulation that Scorpio is blamed for,
in order to get others to do something for it;
maybe this ends with Pisces
with bringing others to imagining a lot about Pisces;
hence by creating a secret around themselves,
Pisces bring others to imagining lots of things,
of which Pisces, in turn, live within their souls?}%
\de{Vielleicht ist das ja generell so;
das Manipulieren, das Skorpion nachgesagt wird,
um andere dazu zu bringen, etwas für ihn zu tun;
vielleicht endet das bei den Fischen dabei,
dass sie andere dazu bringen, sich viel zu den Fischen vorzustellen,
also indem die Fische ein Geheimnis um sich machen,
bringen sie andere dazu sich viel vorzustellen,
wovon die Fische wiederum seelisch leben?}

\en{Interesting; when I started to write this section about Pisces,
I did not know this, yet,
the elementary model has now even proven itself
in a way that it “ran” ahead of me and created something,
which I myself was not expecting,
and thus also proves that the model is very rich,
despite its simple structure.}%
\de{Interessant, als ich diesen Abschnitt zu den Fischen zu schreiben begann,
wusste ich das noch nicht,
das elementare Modell hat sich jetzt also sogar derart bewiesen,
dass es mir “davonlief” und etwas schuf,
das ich selbst nicht erwartete,
also so auch beweist,
dass das Modell trotz seiner einfachen Struktur doch sehr reichhaltig ist.}

\en{So,
once more,
maybe put somewhat differently:}%
\de{Also nochmals vielleicht etwas anders gesagt:}
%
\en{Pisces like to create a secret around themselves
because they like to sunbathe in all the imaginations (Fire) of the others;
they create in this way,
quite abstractly like ELIZA (Air), the fourth element Fire,
which they would otherwise lack, like all Water signs.}%
\de{Fische mögen es ein Geheimnis um sich zu machen,
da sie es mögen, sich in all den Vorstellungen der Anderen zu sonnen (Feuer),
sie schaffen sich so recht abstrakt wie ELIZA (Luft) das vierte Element Feuer,
das ihnen sonst wie allen Wasserzeichen fehlt.}

\en{And with Cancer it is probably similar to Aries,
inside Cancer there is probably already quite a lot of Fire,
which maybe explains all those impulsive
and often also contradictory currents even better.}%
\de{Und beim Krebs ist es wohl auch etwas ähnlich wie vorher beim Widder,
im Krebs drin hat es wohl auch schon viel Feuer,
was vielleicht all die impulsiven
und auch oft gegensätzlichen Strömungen
noch etwas besser erklärt.}

\en{\subsection{The Air signs}}%
\de{\subsection{Die Luftzeichen}}

\en{For the Air signs (and afterwards also for the Earth signs),
the transformation is shaped a bit differently.}%
\de{Bei den Luftzeichen (und nachher auch bei den Erdzeichen)
ist die Umwandlung etwas anders gestaltet.}

\de{\vspace{-0.5mm}
\includegraphics[scale=0.1625]{i-luft.jpg}}

\en{Here, the transformation starts with Fire and ends with Water.}%
\de{\noindent
Und zwar beginnt die Umwandlung hier mit Feuer und endet mit Wasser.}
%
\en{In youth, there is sort of a lot of hot wildness (Fire),
which, through thinking and learning, ends up in old age
as a pleasant ripple (Water).}%
\de{In der Jugend quasi viel heisse Wildheit (Feuer),
die dann durch Nachdenken und Lernen im hohen Alter
zu einem eher gemütlichen Plätschern (Wasser) wird.}

\en{A good image is a thunderstorm or a cloud.}%
\de{Ein gutes Bild ist ein Gewitter oder eine Wolke.}
%
\en{A thunderstorm at first begins with lots of thunder and lightning,
which can be heard from far away, and some rain,
then follows often an intensive phase with lots of rain
and lots of thunder and lightning,
until the thunderstorm usually ends with almost only rain.}%
\de{Ein Gewitter beginnt oft erst mal mit viel Donner und Blitzen,
die man schon von weit weg hört, und etwas Regen,
dann folgt oft eine intensive Phase mit viel Regen und viel Donner und Blitzen,
bis dann das Gewitter meist fast mit nur noch Regen endet.}

\en{\vspace{-0.5mm}
\includegraphics[scale=0.1625]{i-air.jpg}}

\en{\noindent
A bit more abstractly formulated,
the focus of the Air signs is at first on something visual (Fire),
on looking at the world from all possible angles,
from which inside, by comparing and abstracting, a model of the world emerges,
like, for example, the 6 faces of a die that are fused in the head
to an object called “die”.}%
\de{Etwas abstrakter formuliert,
ist der Fokus bei den Luftzeichen erst viel auf etwas Visuellem (Feuer),
darauf, die Welt von allen möglichen Winkeln zu betrachten,
woraus dann im Inneren durch Vergleichen und Abstrahieren ein Modell der Welt entsteht,
zum Beispiel die 6 Seiten eines Würfels werden im Kopf
zu einem Objekt “Würfel” verschmolzen.}
%
\en{Hence from 6 or more images (Fire) of a cube seen from different angles,
emerges by thinking (Air) an abstract model of a cube with properties
like 6 faces, 8 corners, several symmetry axes
and other abstract properties.}%
\de{Also aus 6 oder mehr Bildern (Feuer) eines Würfels von verschiedenen Seiten
entsteht durch Nachdenken (Luft) ein abstraktes Modell eines Würfels
mit Eigenschaften wie 6 Seitenflächen, 8 Ecken, mehreren Symmetrieachsen
und noch anderen abstrakten Eigenschaften.}

\en{The same happens in principle also with all other visual phenomena (Fire) of the world,
even those becomes in principle—somewhat sloppily formulated—in the head a “cube”
with thousands and thousands of faces.}%
\de{Analog passiert das auch mit allen anderen sichtbaren Phänomenen (Feuer) auf der Welt,
auch das wird im Prinzip—etwas schlampig formuliert—im Kopf zu einem “Würfel”
mit Abertausenden von Seiten.}

\en{And the Air signs then often fall in love with these abstract symmetries,
lead by the middle Air sign, Libra, which is ruled by Venus,
the goddess of love and beauty.}%
\de{Und die Luftzeichen verlieben sich dann eben oft auch in diese abstrakten Symmetrien,
allen voran das mittlere Luftzeichen, die Waage, die von der Venus,
der Göttin der Liebe und der Schönheit, beherrscht wird.}
%
\en{Thus emerges love, hence the element Water, from reason (Air).}%
\de{So entsteht jedenfalls aus der Vernunft (Luft) Liebe und somit das Element Wasser.}

\en{And something else, also something abstract, emerges at the same time: \textsl{transparency}.}%
\de{Und etwas Weiteres, auch eher etwas Abstraktes, entsteht dabei ebenfalls: \textsl{Transparenz}.}
%
\en{The 6 or more views of a cube seen from the outside are intransparent,
you only see one image, then again another, but never the connections.}%
\de{Vom aussen betrachtet sind die 6 oder mehr Bilder von einem Würfel undurchsichtig,
man sieht immer jeweils nur ein Bild, dann wieder das andere, aber nie die Zusammenhänge.}
%
\en{But with a model of the cube in the head, it becomes sort of transparent,
like the element Water;
it becomes possible to perceive the cube’s structure.}%
\de{Aber mit einem Modell des Würfels im Kopf, wird dieser quasi durchsichtig,
wie das Element Wasser,
es wird möglich, seine innere Struktur wahrzunehmen.}

\en{I hope this was not too much abstract stuff at once,
even if it concerns Air signs here,
which are often exactly at home there.}%
\de{Ich hoffe das war jetzt nicht zu abstrakt auf einmal,
auch wenn es hier eben um die Luftzeichen geht,
die oft genau dort beheimatet sind.}

\en{\subsection{The Air signs – Gemini}}%
\de{\subsection{Die Luftzeichen – Zwillinge}}

\en{The flames of a fire are quite unpredictable,
sometimes they snake here, then they meander there.}%
\de{Die Flammen eines Feuers sind recht unberechenbar,
mal schlängeln sie sich hierhin, mal dorthin.}
%
\en{In the same way, also light and the eye of an observer (Fire)
look sometimes here, then there.}%
\de{So fällt auch das Licht und das Auge des Beobachters (Feuer)
mal da- und mal dorthin.}
%
\en{Gemini, as the first Air sign, have a lot of this,
always ready to move, hard to predict,
very interested in learning something,
very curious and also flexible.}%
\de{Davon haben die Zwillinge, als das erste Luftzeichen, sehr viel,
immer im Sprung, schwer vorherzuberechnen,
sehr interessiert daran etwas zu lernen,
sehr neugierig und auch flexibel.}

\en{But why then also the theme of dark and bright,
of good and evil twin,
like in Liz Greene’s \textsl{The Astrology of Fate};
can that also be seen somehow in the elementary model?}%
\de{Aber woher dann auch das Thema von hell und dunkel,
von gutem und bösem Zwilling,
wie auch bei Liz Greene in \textsl{The Astrology of Fate};
sieht man das auch irgendwie im elementaren Modell?}

\en{In order to fuse things, you must be able to unify apparent contradictions
like bright/dark, good/evil or true/false in an abstract way.}%
\de{Um Dinge zu verschmelzen, muss man auch scheinbare Gegensätze
wie hell/dunkel, gut/böse oder wahr/falsch abstrakt vereinigen können.}
%
\en{Quite generally, logic is based on true/false or in a computer on zero/one.}%
\de{Ganz generell beruht ja Logik auf wahr/falsch oder im Computer auf Null/Eins.}

\en{For Gemini, as the first Air sign,
these opposites often remain just side-by-side;
in one moment,
like when looking at one face of a coin or the other,
only one side is seen and in the next moment maybe the other.}%
\de{Bei den Zwillingen, als das erste Luftzeichen,
bleiben diese Gegensätze eben oft einfach nebeneinander bestehen,
es wird jeweils in einem Moment,
wie wenn man eine Münze mal von der einen, mal von der anderen Seite anschaut,
nur eine Seite gesehen, und im nächsten Moment dann vielleicht die andere.}

\en{For Libra,
opposites are already quite well balanced
and finally for Aquarius, they are often already mostly unified,
which, however, often is not possible without suppressing certain things,
but more about that later when writing about Aquarius.}%
\de{Bei der Waage sind die Gegensätze dann schon recht gut in der Waage
und schliesslich beim Wassermann dann schon oft weitgehend vereinigt,
was allerdings oft auch nicht geht ohne gewisse Dinge zu verdrängen,
aber dazu später beim Wassermann.}

\en{\subsection{The Air signs – Libra}}%
\de{\subsection{Die Luftzeichen – Waage}}

\en{In \textsl{The Astrology of Fate},
Liz Greene associates the verdict of Paris with Libra,
where three goddesses, Hera, Athene and Aphrodite (Venus),
ask a human, Paris, to judge who of them is the most beautiful goddess.}%
\de{Beim Parisurteil,
das Liz Greene in \textsl{The Astrology of Fate} mit der Waage assoziiert,
fragen drei Göttinnen einen Menschen,
Paris, um sein Urteil wer die schönste Göttin sei.
Es sind Hera, Athene und Aphrodite (Venus).}
%
\en{And Paris in the end chooses love resp.\ beauty, Aphrodite’s offer,
to get Helena, the most beautiful woman of the world, as his wife,
which then also causes the Trojan War.}%
\de{Und Paris wählt am Ende die Liebe bzw.\ die Schönheit, das Angebot der Aphrodite,
die schönste Frau der Welt, Helena von Troja, zur Frau zu bekommen,
was dann auch den trojanischen Krieg auslöst.}

\en{This shows, of course, again love, the element Water,
which grows from abstract considerations of the world,
mirrored in Paris’ mind as well as in the fact,
that the goddesses consider his judgement as a human being so important.}%
\de{Da zeigt sich natürlich wieder die Liebe, das Element Wasser,
das aus den abstrakten Betrachtungen der Welt erwächst,
sowohl bei Paris, wie auch in der Tatsache,
dass die Göttinnen seinem Urteil als Sterblichen so viel Bedeutung beimessen.}
%
\en{And, yet, Paris fails first at reality (Earth),
the element missing in the transformation of the Air signs,
which did not want to adapt to his abstract ideals (Air).}%
\de{Und dennoch scheitert Paris dann erst mal an der Realität (Erde),
dem Element das in der Umwandlung der Luftzeichen fehlt,
die sich seinen abstrakten Idealen (Luft) nicht anpassen wollte.}

\en{And, yet, the realization (Earth) of an Utopia, an idea for the future,
is in the end what Air signs desire and want to implement,
at least the two later Air signs, Libra and Aquarius.}%
\de{Und doch ist die Realisierung (Erde) eines Utopias, einer Idee für die Zukunft,
am Ende das, was sich Luftzeichen wünschen und umsetzen wollen,
mindestens die zwei späteren Luftzeichen, Waage und Wassermann.}

\en{The fourth task of Psyche in Apuleius’ story was the most complex one:
Get some of the beauty ointment from the underworld goddess Proserpina down in Hades.}%
\de{Die vierte Aufgabe der Psyche bei Apuleius war die komplexeste:
Ein wenig Schönheitssalbe von der Unterweltgöttin Proserpina aus dem Hades holen.}
%
\en{Psyche only succeeds at this
by not showing any compassion with individuals
she meets in the underworld.}%
\de{Psyche gelingt dies nur,
indem sie eben gerade kein Mitgefühl mit Individuen zeigt,
denen sie in der Unterwelt begegnet.}
%
\en{For example, she is not allowed to give a beggar a piece of bread,
because she needs both pieces, in order to divert the dog Cerberus
both on the way down and up.}%
\de{Zum Beispiel darf sie einem Bettler kein Stück Brot geben,
denn sie braucht beide Stücke um den Hund Cerberus
beim Hin- und beim Rückweg abzulenken.}

\en{This shows probably that Libra’s love to human beings is already very abstract,
it is less concerned with individual fates than with the whole,
with realizing (Earth) a plan.}%
\de{Das zeigt wohl, dass bei der Waage die Liebe zu den Menschen schon sehr abstrakt ist,
es geht weniger um Einzelschicksale wie um das grosse Ganze,
darum einen Plan zu verwirklichen (Erde).}

\en{But how exactly is the fourth task of Psyche connected with the elements?}%
\de{Aber wie genau hängt die vierte Aufgabe der Psyche jetzt mit den Elementen zusammen?}
%
\en{Well, for the plan at first a precise knowledge of the world was necessary,
including circumstances in the underworld,
hence the element Fire, also coming from Gemini,
which are ruled by Hermes/Mercury, the messenger of the gods,
who also comes to the underworld when carrying messages
and who saw and remembered everything then.}%
\de{Nun ja, für den Plan war erst mal eine genaue Kenntnis der Welt notwendig,
inklusive der Umstände in der Unterwelt,
also das Element Feuer,
auch von den Zwillingen her, die ja vom Götterboten Hermes/Merkur beherrscht werden,
der auch in die Unterwelt kommt bei seinen Botengängen
und dabei alles sah und sich merkte.}
%
\en{And then it required a mental effort
in order to plan ahead how much of what would be needed
and what one was allowed to do and what not, hence the element Air.}%
\de{Dann brauchte es eben eine gedankliche Leistung,
um vorauszuplanen, wie viel man für was brauchte
und was man tun durfte und was nicht, also das Element Luft.}
%
\en{And, almost last, but not least, a universal love
in order to be able to withstand individual temptations (Water);
Water signs can be very cold if it has to be,
Water is a cold element.}%
\de{Und nicht zuletzt eine universelle Liebe
um den individuellen Versuchungen zu widerstehen (Wasser),
und auch Wasserzeichen können sehr kalt sein, wenn es sein muss,
und Wasser ist ja ein kaltes Element.}
%
\en{Finally, the prize, an ointment,
was in the end something material (Earth),
and also connected to Venus,
because it was an ointment for beauty.}%
\de{Schliesslich der Gewinn, die Salbe,
war am Ende etwas materielles (Erde),
und auch im Bezug zur Venus,
da eine Schönheitssalbe.}
 
\en{\subsection{The Air signs – Aquarius}}%
\de{\subsection{Die Luftzeichen – Wassermann}}

\en{In Aquarius, the third Air sign,
opposites are already quite a bit fused together,
which is actually not possible from a purely logical point of view,
except if certain inconsistencies are suppressed into the unconscious,
just as Uranus, the ruler of Aquarius, did in mythology with some titans
he did not like, by banning them to the Tartaros.}%
\de{Beim Wassermann, dem dritten Luftzeichen,
sind die Gegensätze schon sehr stark verschmolzen,
was rein logisch gesehen gar nicht möglich ist,
ausser gewisse Ungereimtheiten werden ins Umbewusste verdrängt,
wie der Herrscher des Wassermannes, Uranus, es in der Mythologie mit einigen Titanen tat,
die ihm nicht passten und daher in den Tartaros verbannt wurden.}
 
\en{A tiny example how logic always contains contradictions,
at least as soon as you break up the separation of subject and object:
The sentence “this sentence has 5 words”
and the sentence “this sentence has not 5 words”
are both true and, yet, are the logical opposites of each other.}%
\de{Ein ganz kleines Beispiel, wie Logik immer Widersprüche enthält,
zumindest sobald man die Trennung von Subject und Objekt aufbricht:
Der Satz “dieser Satz hat 5 Worte”
und der Satz “dieser Satz hat nicht 5 Worte”
sind beide wahr und doch die logischen Gegensätze voneinander.}
%
\en{So, if the first sentence is true,
normally its opposite would have to be false,
but here it is not, hence there is a logical paradox.}%
\de{Also wenn der erste Satz wahr ist,
müsste doch normalerweise sein Gegenteil falsch sein,
ist es hier aber nicht, also ein logisches Paradox.}

\en{On the other hand,
it is exactly this consequence of unification,
despite a certain suppression,
which only makes some kinds of progress possible;
some things should be realized in the end,
if the advantages clearly overweigh.}%
\de{Andererseits ist es eben gerade diese Konsequenz des Vereinheitlichens
trotz der gewissen Verdrängung,
die gewissen Fortschritt erst möglich macht,
gewisse Dinge sollten am Ende eben doch realisiert werden,
wenn die Vorteile klar überwiegen.}
%
\en{Also there is a clear transition from learning to teaching
from Gemini via Libra to Aquarius,
to sharing what has been learnt,
which usually makes sense.}%
\de{Auch gibt es von den Zwillingen über die Waage zum Wassermann
auch einen Übergang weg vom Lernen zum Lehren,
dazu das Gelernte weiterzugeben,
was ja meist durchaus Sinn macht.}

\en{In \textsl{The Astrology of Fate},
Liz Greene also mentions Prometheus in the context of Aquarius.}%
\de{In \textsl{The Astrology of Fate} erwähnt Liz Greene
auch Prometheus im Zusammenhang mit dem Wassermann.}
%
\en{Now, Prometheus means roughly “forethought”
and his brother Epimetheus “afterthought”,
hence Prometheus thinks first and acts afterwards,
Epimetheus does the opposite.}%
\de{Nun heisst Prometheus so in etwa “Vorgedanke”
und sein Bruder Epimetheus “Nachgedanke”,
also Prometheus denkt zuerst und handelt danach,
Epimetheus tut das umgekehrt.}
%
\en{That mirrors the dilemma of Air signs,
because they usually think first,
which often keeps them from realizing something,
because there are always arguments that speak against.}%
\de{Da spiegelt sich das Dilemma der Luftzeichen,
denn sie denken ja im Prinzip immer zuerst
und das hindert sie oft daran, etwas zu realisieren,
da es immer Dinge gibt, die gegen etwas sprechen.}

\en{But in the myth it is exactly Epimetheus who,
via Pandora who he took as his wife,
gets things rolling.}%
\de{Im Mythos ist es aber dann eben genau Epimetheus,
der dann über Pandora, die er zur Frau nimmt,
die Dinge doch ins Rollen bringt.}
%
\en{There Pandora as lover (Water) probably played the mediating role,
which lead from Air via Water to Earth, hence to realization.}%
\de{Da spielte wohl dann Pandora als Geliebte (Wasser) die Vermittlerrolle,
die von Luft via Wasser zur Erde, also zur Realisierung führte.}

\en{The only thing which remained in Pandora’s box
after it had been opened was hope, as is generally known.}%
\de{Das Einzige, was in Pandoras Schachtel blieb,
nachdem sie geöffnet wurde, war ja die Hoffnung.}
%
\en{Where exactly I would put that now, I won’t tell here;
I don’t have to sort in everything, a little bit of hope has to remain there…}%
\de{Wo ich das jetzt symbolisch hintun würde, verrate ich hier nicht,
ich muss ja nicht alles Einordnen, etwas Hoffnung muss ja auch da noch bleiben…}

\en{\subsection{The Earth signs}}%
\de{\subsection{Die Erdzeichen}}

\en{For the Earth signs, the transition is also from Fire to Water,
this time via Earth, and the missing element is, conversely, Air.}%
\de{Auch bei den Erdzeichen ist der Übergang von Feuer zu Wasser,
diesmal via Erde, und das fehlende Element ist umgekehrt die Luft.}

\vspace{2mm}
\en{\includegraphics[scale=0.1625]{i-earth.jpg}}%
\de{\includegraphics[scale=0.1625]{i-erde.jpg}}

\noindent
\en{The simplest image here is likely a tree.}%
\de{Das einfachste Bild ist hier wohl ein Baum.}

\en{It needs two things to grow: Light (Fire) and water,
plus nutrients from earth, diluted in water.}%
\de{Der benötigt zwei Dinge zum Wachsen: Licht (Feuer) und Wasser,
plus Nährstoffe aus der Erde, aufgelöst im Wasser.}
%
\en{(Note that this is not complete,
although I am not sure if people knew anything about it in antiquity,
because a tree also takes in carbon dioxide via its leaves,
hence a tree would also require the element Air to live,
but I will ignore this in the following.)}%
\de{(Das ist jetzt nicht ganz voll\-stän\-dig,
wobei ich nicht weiss ob das in der Antike bekannt war;
denn ein Baum nimmt auch Kohlendioxid über die Blätter auf,
würde also so gesehen auch das Element Luft benötigen,
aber ich klammere das im Folgenden mal aus.)}

\en{The first task of Psyche in Apuleius’ story
was to sort out a heap of different seeds before the evening,
which Venus had just mixed together.}%
\de{Die erste Aufgabe der Psyche bei Apuleius war,
einen Haufen verschiedener Körner,
den Venus zuvor kurz zusammengeschüttet hatte,
bis zum Abend zu sortieren.}
%
\en{At first Psyche despaired at the task,
as she always did,
but then ants came to her help
and carried the seeds one-by-one to separate heaps.}%
\de{Zuerst verzweifelte Psyche einfach nur,
wie bei jeder ihrer Aufgaben,
aber dann kamen Ameisen und halfen ihr dabei,
die Körner eins nach dem anderen auf eigene Haufen zu tragen.}

\en{The result of that work was order (Air),
an abstract structure, which emerged from the transformation,
the missing element in the transformation of the Earth signs.}%
\de{Das Resultat der Arbeit war also Ordnung (Luft),
eine abstrakte Struktur, die aus der Wandlung entstand,
das fehlende Element in der Wandlung der Erdzeichen.}
%
\en{A former school colleague, who is a Capricorn,
once told me of a theory he had thought up,
according to which consciousness (hence roughly ‘Air’)
would simply be the result of structure (order),
that, for example, if you moved chairs around,
that could already create consciousness.}%
\de{Ein ehemaliger Schulkollege, der Steinbock ist, also das dritte Erdzeichen,
erzählte mir mal von einer Theorie, die er sich ausgedacht hatte,
nach der Bewusstsein (also in etwa ‘Luft’)
einfach das Resultat von Struktur (Ordnung) sein könnte,
dass zum Beispiel, wenn man Stühle herumschiebt,
schon allein dadurch im Prinzip Bewusstsein entstehen könnte.}
%
\en{I am not sure if that is true, also in light of quantum mechanics,
but it would definitely fit with Earth signs and maybe it is really so.}%
\de{Ich bin nicht sicher, ob das so ganz stimmt, auch wegen der Quantenmechanik,
aber zu den Erzeichen passen würde es sicher und vielleicht ist es sogar wirklich so.}

\en{\subsection{The Earth signs – Taurus}}%
\de{\subsection{Die Erdzeichen – Stier}}

\en{As the first Earth sign, Fire is still more important than Water to Taurus,
and also the distance to order is still far ahead,
resp.\ again rather unconsciously behind him.}%
\de{Als das erste Erdzeichen ist beim Stier noch das Feuer wichtiger als das Wasser,
und auch die Distanz zur Ordnung noch weit voraus,
bzw.\ wiederum eher unbewusst hinter ihm.}
%
\en{But first it is more about the beauty of things at hand,
like a tree or generally nature in spring,
when everything blossoms and grows and one can enjoy it
and often also should just do so.}%
\de{Aber erst mal geht es mehr um die Schönheit der Dinge,
wie bei einem Baum oder generell der Natur im Frühling,
wo alles blüht und wächst und man es geniessen kann
und oft auch sollte.}

\en{King Minos got a holy bull from Poseidon, the god of the sea,
which allowed him to prevail against his brothers for the throne of Crete,
but later, when he had become king, he betrayed the god
by sacrificing another bull instead of the holy bull.}%
\de{König Minos erhielt von Poseidon, dem Gott des Meeres, einen heiligen Stier,
um sich so gegen seine Brüder um den Thron von Kreta behaupten zu können,
aber er betrog den Gott später, als er König wurde,
indem er einen anderen Stier opfern liess und nicht den heiligen Stier.}
%
\en{Poseidon took revenge
by arousing sexual lust for a bull in Minos’ wife Pasiphaë.}%
\de{Poseidon rächte sich,
indem er in Minos’ Frau Pasiphaë die sexuelle Lust nach einem Stier weckte.}
%
\en{Daedalus built her a wooden cow in which she could hide
and so she became the mother of Minotaurus, the monster with a bull’s head and a human body,
which hid in the labyrinth and devoured human sacrifices.}%
\de{Daedalus baute ihr eine Kuh aus Holz, worin sie sich verbergen konnte
und so wurde sie Mutter vom Minotaurus, dem Monster mit Stierkopf und Menschenkörper,
das sich im Labyrinth versteckte und Menschenopfer frass.}

\en{For one, Minos failed
because he wanted to use outer appearances (light, Fire)
to trick the god of the sea (Water),
for the other, it was simply the animal in the bull (Taurus), the sexual lust,
which can be connected both to Fire and Water,
resp.\ the impulse to it rather to Fire;
the consequence, children, rather to Water, because the latter is more fated.}%
\de{Minos scheiterte einerseits daran,
dass er den äusseren Schein (Licht, Feuer) nutzen wollte,
um den Gott des Meeres (Wasser) zu überlisten,
andererseits war es auch einfach das Tier im Stier, die sexuelle Lust,
die man sowohl mit Feuer wie auch mit Wasser in Beziehung bringen kann,
bzw.\ den Impuls dazu eher mit Feuer,
die Konsequenz, die Kinder, hingegen eher mit Wasser, da schicksalhafter.}
%
\en{The Minotaurus came somehow out the wrong way round,
he did not create order (Air) but remained in the labyrinth, trapped in disorder,
and especially his head, which would foremost stand for reason (Air)
had remained the one of an animal.}%
\de{Der Minotaurus kam aber irgendwie falsch herum zur Welt,
er schuf keine Ordnung (Luft) sondern blieb im Labyrinth, in Unordnung gefangen,
und gerade sein Kopf, der am Ehesten für Vernunft (Luft) stehen würde,
war tierisch geblieben.}

\en{Similar to Heracles in the fight with Hydra, Hera, crab and swamps,
it is here also again the whole myth that mirrors Taurus,
hence here also Theseus, the hero who defeated the Minotaurus, is part of it.}%
\de{Ähnlich wie bei Herakles im Kampf mit Hydra, Hera, Krebs und Sumpf,
ist es hier aber auch wieder der ganze Mythos, der den Stier spiegelt,
also auch hier gehört der Held Theseus, der den Minotaurus besiegte dazu.}
%
\en{And definitely also Ariadne,
who taught Theseus the trick with the thread and gave it to him,
so that Theseus could bring a bit of order (Air)
into the labyrinth and she probably did this also out of love (Water).}%
\de{Und nicht zuletzt gehört da auch Ariadne dazu,
die Theseus den Trick mit dem Faden verriet und ihm den Faden gab,
so dass Theseus etwas Ordnung (Luft) in das Labyrinth bringen konnte,
und sie tat das wohl nicht zuletzt aus Liebe (Wasser).}

\en{Earth signs often take liberties with the truth,
Taurus also out of a need for harmony,
in order not to make others jealous of its properties (Earth).}%
\de{Erdzeichen nehmen es oft nicht so genau mit der Wahrheit,
beim Stier durchaus auch aus einem Harmoniebedürfnis heraus,
um andere zum Beispiel nicht eifersüchtig zu machen auf ihren Besitz (Erde).}

\en{\subsection{The Earth signs – Virgo}}%
\de{\subsection{Die Erdzeichen – Jungfrau}}

\en{Towards the end of summer, where Virgo, the second Earth sign, is,
many fruits and other things are ripe to pick.}%
\de{Gegen Ende Sommer, wo das zweite Erdzeichen, die Jungfrau ist,
sind in der Natur viele Früchte und andere Dinge reif zum Pflücken.}
%
\en{As with Psyche in her task, then often things are just “thrown” at your feet
to pick or collect and then sort in; this can hardly be controlled;
if something is ripe, it must simply be harvested then, not earlier or later.}%
\de{Wie bei Psyche in der Aufgabe wird da oft einfach etwas zum Einsammeln oder Pflücken
und dann Einsortieren “hingeworfen”, das lässt sich dann nur wenig steuern;
wenn etwas reif ist, muss es einfach dann geerntet werden, nicht früher oder später.}

\en{This is fate then, the element Water,
where Virgo more and more develops to.}%
\de{Das ist dann Schicksal, das Element Wasser,
wohin sich die Jungfrau mehr und mehr entwickelt.}
%
\en{But in between there are also always phases
where there is nothing to do, where you can still laze around
and enjoy the beauty of things (Fire).}%
\de{Aber dazwischen gibt es auch immer wieder Phasen,
wo es gerade nichts zu tun gibt, wo man noch sonnig faulenzen
und die Schönheit der Dinge geniessen kann (Feuer).}
%
\en{Also there are some fruits of nature
which then cannot be preserved immediately,
those are better given to others
or you invite guests and enjoy things together.}%
\de{Auch gibt es gewisse Früchte der Natur,
die sich nicht unmittelbar konservieren lassen,
die verschenkt man dann besser
oder lädt sich dazu Gäste ein und geniesst es gemeinsam.}

\en{But again something from mythology, about Persephone.}%
\de{Aber wieder zu etwas aus der Mythologie, zu Persephone.}
%
\en{As a young woman she walked around
and when she looked curiously down to certain flowers,
the earth opened up, the underworld god Hades came up with his chariot
and abducted her to the underworld,
where he made her his wife and the queen of the underworld, Proserpina.}%
\de{Als junge Frau spazierte sie herum
und als sie neugierig nach unten schaute zu gewissen Blumen,
da tat sich die Erde auf, der Unterweltgott Hades kam mit seinem Wagen hervor
und entführte sie in die Unterwelt,
wo er sie zu seiner Frau und zur Herrscherin der Unterwelt, Proserpina, machte.}

\en{The virgin, simply because she was curious
as part of the transformation of elements Fire-Earth-Water-Air,
wanted more than just see the beautiful appearance of nature,
but as she looked down,
hence aways from the sun (Fire) in the sky,
down to earth (where also water flows to),
in order to understand how and why the flowers grow
and what kind of flowers there are,
she became fatedly trapped,
fell in love, similar to the Air signs,
with something that at first also appeared terrible to her,
like maybe sexuality to a virgin,
with all the fascination that also lies within.}%
\de{Die Jungfrau wurde eben in der Wandlung der Elemente Feuer-Erde-Wasser-Luft neugierig,
wollte mehr als nur den Schein der schönen Natur sehen,
sondern als sie nach unten schaute,
also weg von der Sonne (Feuer) im Himmel
zur Erde hin (wohin auch das Wasser fliesst),
um zu verstehen wie und wieso die Blumen wachsen
und was für welche es gibt,
da wurde sie schicksalshaft gefangen,
verliebte sich eben analog wie auch bei den Luftzeichen in etwas,
das ihr erst mal schrecklich erschien,
wie vielleicht auch Sexualität einer Jungfrau,
bei aller Faszination,
die eben auch darin liegt.}

\en{\subsection{The Earth signs – Capricorn}}%
\de{\subsection{Die Erdzeichen – Steinbock}}

\en{And in the end, as underworld queen Proserpina,
she did of course also structure many things.}%
\de{Und am Schluss als Unterweltkönigin Proserpina
strukturierte sie natürlich auch viele Dinge.}
%
\en{This mirrors already a tree in winter,
the third Earth sign, Capricorn,
a tree that stands with meager, empty branches,
only still supplied via its roots (hidden from view)
with water and nutrients,
hence containing a lot more Water than Fire.}%
\de{Das spiegelt auch schon einen Baum im Winter,
wie beim dritten Erdzeichen, dem Steinbock,
der im Winter mit kargen leeren Ästen dasteht,
nur im Verborgenen durch seine Wurzeln noch versorgt wird
mit Wasser und Nährstoffen,
also daher viel mehr Wasser als Feuer erhält.}
%
\en{Also generally a lot in nature disappears under the ground in winter,
also animals that hibernate or do something similar.}%
\de{Auch verschwindet sonst vieles in der Natur im Winter unter dem Boden,
auch Tiere, die einen Winterschlaf oder etwas Ähnliches halten.}

\en{The abduction of Persephone can also be seen that way,
because in the myth she was finally allowed to spend a third or half of the year
again on the surface, also in accord with the model,
where Virgo contains Fire and Water still in similar amounts,
even though the path clearly leads to more Water and towards Air.}%
\de{Auch so kann man die Entführung der Persephone verstehen,
die ja auch im Mythos dann doch wieder einen Drittel oder die Hälfte des Jahres
auf der Oberfläche verbringen durfte, auch im Einklang damit,
dass eben bei der Jungfrau sowohl Feuer wie Wasser ähnlich stark vorhanden sind,
wenn auch der Weg klar zum Wasser und zur Luft hin geht.}

\en{The roots and meager branches of a tree in winter mirror Capricorn;
there order comes without any ornamentation before beauty.}%
\de{Die Wurzeln und kargen Äste eines Baums im Winter spiegeln den Steinbock,
da kommt Ordnung ohne irgendwelche Verzierungen vor der Schönheit.}
%
\en{And, yet, this is again also a source of lust for Capricorn,
out of reversal of sexual drive, \textsl{Triebumkehr}, as Freud called it,
thus drawing sexual lust out of not doing things one would like to do,
especially for the oldest Earth sign also out of restrictions
that emerge due to old age, when the own body at some point
becomes some kind of cage, which keeps trapped.}%
\de{Und doch ist das wiederum auch eine Quelle von Lust, auch für den Steinbock,
aus \textsl{Triebumkehr} heraus, wie Freud das nannte;
also Lust ziehen aus Dingen, die man tun möchte, aber dann doch nicht tut,
gerade beim ältesten Erdzeichen auch unmittelbar aus Einschränkungen,
die im Alter entstehen, heraus, wo der eigene Körper irgendwann mal
zu einer Art Käfig wird, der gefangen hält.}

\en{In \textsl{The Astrology of Fate},
Liz Greene describes a dream of a Capricorn,
where he was prisoner with his wife in a house,
but with open doors and windows,
and they still stayed there.}%
\de{Liz Greene beschreibt in \textsl{The Astrology of Fate}
einen Traum von einem Steinbock,
wo er mit seiner Frau in einem Haus gefangen war,
aber mit offenen Türen und Fenstern,
und sie trotzdem da blieben.}
%
\en{This reminds me again very much of the idea of moving chairs around,
the house with open doors and windows symbolizes again an order,
created from Earth, but very abstract, hence Air.}%
\de{Das erinnert mich wieder sehr stark an die Idee mit den Stühlen herumschieben;
das Haus mit offenen Türen und Fenstern symbolisiert wiederum eine Ordnung,
geschaffen aus Erde, aber doch sehr abstrakt, also Luft.}

\en{I will leave it at that with examples for the star signs.}%
\de{Damit belasse ich es jetzt mal zu den Sternzeichen an Beispielen.}
%
\en{As announced earlier, this was now of course not complete at all,
the whole topic is simply too rich;
I could write a few sentences to almost every sentence
of Liz Greene and many other authors about the star signs.}%
\de{Das war jetzt natürlich wie schon angekündigt überhaupt gar nicht vollständig,
das ganze Thema ist einfach zu reichhaltig;
ich könnte fast zu jedem Satz bei Liz Greene
und vielen andern Autoren zu den Sternzeichen ein paar Sätze sagen.}
%
\en{And I am sure, about all I said above,
many readers will have a slightly or strongly different opinion,
and even myself in a few days;
I only hope that the other opinions would often still rather confirm
than refute my elementary model as the inner \textsl{skeleton} of the star signs?}%
\de{Und auch bei allem was ich oben sagte,
werden sicher viele Leserinnen und Leser eine leicht oder stark andere Meinung haben,
und auch ich selbst spätestens in ein paar Tagen,
nur hoffe ich, auch die anderen Meinungen würden eben doch oft mein elementares Modell
als inneres \textsl{Skelett} der Sternzeichen eher bestätigen als widerlegen?}

\en{In any case,
I hope that I could convey my idea simply and evidently enough,
so that now many readers can apply it when desired or needed,
to themselves, lovers and friends, mundane themes, etc.}%
\de{Jedenfalls hoffe ich,
dass ich meine Idee insgesamt doch so einfach und einleuchtend herüberbringen konnte,
dass jetzt viele Leserinnen und Leser sie bei Lust oder Bedarf selber anwenden können,
bei sich, bei Geliebten und Freunden, bei mondänen Themen, usw.}

\en{It would be a pity if this idea was lost to the world again.}%
\de{Es wäre schade, wenn diese Idee der Welt wieder verloren ginge.}

\en{\subsection{Review and structures in the model}}%
\de{\subsection{Rückblick und Strukturen im Modell}}

\en{First a bit about the overall structure of the model in review.}%
\de{Erst mal etwas zur Gesamtstruktur des Modells im Rückblick.}

\en{For Fire and Water signs, the transformation started with Earth and ended with Air,
for Air and Earth signs, it began with Fire and ended with Water.}%
\de{Bei Feuerzeichen und Wasserzeichen begann die Umwandlung mit Erde und endete mit Luft,
bei Luftzeichen und Erdzeichen begann sie mit Feuer und endete mit Wasser.}
%
\en{Hence, the start, Earth/Fire, was always with a dry element,
the end, Air/Water, always with a wet element, according to Aristotle.}%
\de{Also war der Beginn, Erde/Feuer, immer mit einem trockenen Element,
das Ende, Luft/Wasser, immer mit einem feuchten Element, gemäss Aristoteles.}
%
\en{This would fit a general development in life,
where in youth a lot would still be appear
as given and clear, as fixed, hard, immutable,
but with growing age it would turn out
that many things are more flexible, can be bent, flow.}%
\de{Das würde schon zu einer allgemeinen Entwicklung im Leben passen,
dass in der Jungend noch vieles als gegeben und klar erscheint,
als fix, hart, unveränderlich,
sich dann aber mit zunehmendem Alter zeigt,
dass vieles eben doch flexibler ist,
sich biegen lässt, fliesst.}

\en{The circle of transformation of the elements in the model was always the same one;
Earth, for example, always only transformed to Fire or Water, never directly to Air.}%
\de{Der Kreis der Wandlungen der Elemente war im Modell immer der gleiche,
immer wandelte sich zum Beispiel Erde in Feuer oder Wasser, nie direkt Erde in Luft.}
%
\en{This also fits with the philosophical view of Aristotle,
because he saw the circle such that in a transformation between elements
rather just one of the properties would change at a time,
so Earth, which is dry and cold, would transform to wet and cold,
to Water, or to dry and warm, to Fire, but only more slowly directly to wet and warm,
to Air, and the same way with the other three elements.}%
\de{Auch das passt zur philosophischen Sicht von Aristoteles,
denn er sah den Kreis so, dass sich bei einer Umwandlung der Elemente
jeweils meist nur eine Eigenschaft der Elemente ändern würde,
also Erde, die trocken und kalt ist, konnte sich in feucht und kalt,
also Wasser, umwandeln oder in trocken und warm,
also Feuer, aber nur langsamer direkt in feucht und warm,
also Luft, und analog mit den anderen drei Elementen.}

\en{\vspace{2mm}}%
\de{\vspace{1.2mm}}
\hspace{19.5mm}
\en{\includegraphics[scale=0.30]{i-circle.jpg}}%
\de{\includegraphics[scale=0.30]{i-kreis.jpg}}

\en{\vspace{1mm}}%
\de{\vspace{0.2mm}}
\noindent
\en{Thus my elementary model of the transformations of the star signs
is strongly rooted in well-known philosophical traditions.}%
\de{Somit ist also mein elementares Modell der Wandlungen der Sternzeichen
durchaus stark in bekannten philosophischen Traditionen verankert.}
%
\en{At the moment there are, however, no indications known to me
that my \textsl{model} of the star signs was ever conscious to people in the past,
which would, for one, be astonishing,
for the other it could in the end
give astrologers a stronger stand against science.}%
\de{Im Moment gibt es allerdings keinerlei mir bekannten Hinweise,
dass mein \textsl{Modell} der Sternzeichen je den Menschen in der Vergangenheit bewusst gewesen wäre,
was einerseits erstaunlich wäre,
andererseits am Ende durchaus Astrologen einen stärkeren Stand
gegenüber der Wissenschaft geben könnte.}
%
\en{Accusing of charlatanry or illusion would finally hardly be justifiable any more.}%
\de{Da lässt sich ein Vorwurf der Scharlatanerie oder Illusion am Ende kaum mehr vertreten.}

\en{By the way, I often wrote “my model” here,
which it naturally was, coming from its discovery,
but it can only grow if it belongs to all,
and if many think and feel into it,
imagine things about it and realize them,
only then can it maybe approach
something like a balanced view of all attributes of the star signs with time.}%
\de{Übrigens, ich schrieb hier oft “mein Modell”,
was es von der Entdeckung her natürlich schon war,
aber wachsen kann es nur, wenn es allen gehört
und viele mitdenken und mitfühlen
und sich Dinge dazu vorstellen und realisieren,
nur dann kann es sich vielleicht auch mit der Zeit
an so etwas wie eine ausgewogene Sicht aller Eigenschaften der Sternzeichen annähern.}

\en{\subsection{Appendix – A more formal approach}}%
\de{\subsection{Anhang – Eine formalere Sicht}}

\en{I would like to append something more abstract, for completeness,
even though this might get too abstract for many readers.}%
\de{Ich möchte doch noch kurz etwas Abstrakteres anhängen, zur Vollständigkeit,
auch wenn das wohl vielen zu abstrakt werden könnte.}

\en{On my web site
I postulate the elements more abstractly
as composed of in/out and rest/move.}%
\de{Auf meiner Website postuliere ich die Elemente abstrakter
als zusammengesetzt aus innen/aussen und ruhen/sich bewegen.}
%
\en{Earth rests outside, Fire moves outside,
Air rests inside, Water moves inside.}%
\de{Erde ruht draussen, Feuer bewegt sich draussen,
Luft ruht innen, Wasser bewegt sich innen.}
%
\en{Thus the transition for all four elements in the zodiac is always
from the outside, from Earth or Fire, to the inside, to Air or Water.}%
\de{So ist der Übergang bei allen vier Elementen im Sternkreis immer von aussen her,
also von Erde oder Feuer, nach innen, also zu Luft oder Wasser.}

\en{For example, Teiresias, who Liz Greene, in \textsl{The Astrology of Fate},
associates with Libra,
was struck with blindness by Hera, out of anger about his statement,
regarding who of men or women have more fun during sex.}%
\de{Zum Beispiel wurde Teiresias, den Liz Greene in \textsl{The Astrology of Fate}
mit der Waage assoziiert,
von Hera mit Blindheit geschlagen, aus Wut über seine Aussage darüber,
wer von Männern oder Frauen beim Sex mehr Lust empfindet.}
%
\en{But Zeus gave him in exchange the gift of “inner vision”.}%
\de{Aber Zeus schenkte ihm im Gegenzug die Gabe des “inneren Sehens”.}
%
\en{This means, Libra changes its perspective from looking from outside,
where the light, the Fire, is,
towards looking from inside (Air and Water),
where it could then find more abstract insights (Air),
also about love and sex (Water),
which in the end could again change a state outside (Earth) for good.}%
\de{Das heisst, der Blick der Waage richtet sich von aussen,
wo das Licht, das Feuer, ist,
nach innen (Luft und Wasser),
wo sie dann abstraktere Einsichten (Luft)
auch zu Liebe und Sexualität (Wasser) finden konnte,
welche am Ende dann wieder draussen
einen Zustand (Erde) nachhaltig verändern können.}

\en{I hope this was now at least as comprehensible that it has become visible
that maybe one day this could be applied mathematically in a formal way,
partially beyond human interpretation,
but at the moment this is still very much Zukunftsmusik
(something worthwhile to realize, but far away, yet)…}%
\de{Ich hoffe, das war jetzt zumindest soweit verständlich, dass ersichtlich ist,
dass sich das vielleicht eines Tages auch mathematisch formal anwenden liesse,
teilweise jenseits von menschlichen Interpretationen,
aber das ist im Moment immer noch Zukunftsmusik…}

\en{\subsection{Sources}}%
\de{\subsection{Quellen}}

\en{I often mentioned \textsl{The Astrology of Fate} by Liz Greene, Weiser 1984,
as well as, also by her, \textsl{Star Signs for Lovers}, Stein and Day 1980,
later retitled to \textsl{Astrology for Lovers}.}%
\de{Oft erwähnt hatte ich \textsl{The Astrology of Fate} von Liz Greene, Weiser 1984,
das auch in einer recht guten deutschen Übersetzung als \textsl{Schicksal und Astrologie} erhältlich ist,
sowie ebenfalls von ihr \textsl{Star Signs for Lovers}, Stein and Day 1980,
das heute neu unter dem Titel \textsl{Astrology for Lovers} erhältlich ist,
und hier gibt es auch eine (allerdings unvollständige) deutsche Übersetzung,
\textsl{Sag mir dein Sternzeichen und ich sage dir wie du liebst}.}
%
\en{After that she unfortunately hardly ever wrote anything about the star signs again,
except indirectly via planets, which, in my view,
does not capture all of the richness of the star signs,
especially not all of the richness that lies
in the transformations of the star signs and their relations.}%
\de{Später hat sie leider kaum mehr etwas zu den Sternzeichen geschrieben,
ausser indirekt via Planeten, was eben, finde ich zumindest,
doch nicht den ganzen Reichtum der Sternzeichen erfassen kann,
vor allem nicht den ganzen Reichtum,
der eben gerade im Wandel der Sternzeichen
und auch in ihren gegenseitigen Beziehungen liegt.}

\en{I recommend to read Apuleius’ \textsl{The Golden Ass} in the translation by Robert Graves,
or in the latin original (which I unfortunately cannot read).}%
\de{Apuleius’ \textsl{Der Goldene Esel} empfehle ich auf deutsch
in der Übersetzung von Carl Fischer, dtv/Artemis 1990,
und auf englisch in der Übersetzung von Robert Graves \textsl{The Golden Ass} zu lesen,
oder natürlich im lateinischen Original (was ich selbst leider nicht kann).}

\en{And, of course, just pick out something on your own,
which appears to be typical for a star sign,
from your favorite books and experiences,
and then mirror it in the elementary model of the transformation of the star signs…}%
\de{Und natürlich einfach aus eigenen geliebten Büchern
und erlebten Ereignissen etwas raussuchen,
das typisch für ein Sternzeichen zu sein scheint,
und es dann im elementaren Modell der Wandlung der Sternzeichen spiegeln…}

\en{I wish you a lot of fun doing that and it would make me very, very happy
if this thing would simply evolve beautifully on its own\,!}%
\de{Viel Spass dabei, es würde mich sehr, sehr freuen,
wenn diese Sache sich einfach von selbst schön entwickeln würde\,!}
