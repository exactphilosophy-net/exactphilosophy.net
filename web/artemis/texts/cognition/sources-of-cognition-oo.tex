\avantgarde

\en{\section{Sources of cognition}}%
\de{\section{Quellen der Erkenntnis}}

\en{Ever since Galileo observed Jupiter’s moons circle Jupiter with his telescope,
observation of the outer world has become
virtually the only source of cognition
accepted as valid in science
and generally in mainstream mundane perceptions and decisions.}%
\de{Seit Galileo Galilei mit seinem Teleskop sah wie die Jupitermonde den Jupiter umkreisen,
ist die Beobachtung der äusseren Welt praktisch zur einzigen Quelle der Erkenntnis geworden,
die in der Wissenschaft
und allgemein in mainstream weltlichen Wahrnehmungen und Entscheidungen
als gültig anerkannt wird.}
%
\en{This reflects a fundamental asymmetry inherent in most parts of exact sciences
and generally in mainstream approaches to the world.}%
\de{Dies widerspiegelt eine fundamentale Asymmetrie,
die in den meisten Bereichen der exakten Wissenschaften
und allgemein in den gängigen Weltanschauungen vorherrscht.}
%
\en{This may not be an ideal choice of paradigms in the longer run.}%
\de{Langfristig gesehen mag dies keine ideale Paradigmenwahl sein.}

\en{One of the involved paradigms is that there would be just one “outer world” or “reality”,
which different people would perceive from different angles at different times.}%
\de{Eines der beteiligten Paradigmen ist,
dass es nur eine “äussere Welt” oder “Realität” gäbe,
die verschiedene Menschen zu verschiedenen Zeitpunkten
aus verschiedenen Blickwinkeln wahrnehmen würden.}
%
\en{That single “outer world” would be where measurements in science are done;
that would be the source of cognition.}%
\de{Diese einzige “äussere Welt” wäre,
wo Messungen in der Wissenschaft durchgeführt werden;
das wäre die Quelle der Erkenntnis.}
%
\en{In contrast,
each individual being would have their own “inner world” (also called “mind” or “imagination”),
something generally not considered worthy as a source of cognition in science.}%
\de{Im Gegensatz dazu würde jedes einzelne Wesen
seine eigene “innere Welt” haben (auch “Geisteswelt” oder “Vorstellung” genannt),
etwas, das allgemein in der Wissenschaft als nicht würdig als Quelle der Erkenntnis erachtet wird.}

\en{{\color{xphi}
But why not alter that paradigm,
and assume that there would (conversely or symmetrically) be just \textbf{\color{indigo}a single inner world},
into which each individual would look from their own personal angle\,?
}}%
\de{{\color{xphi}
Aber warum dieses Paradigma nicht abwandeln
und annehmen,
dass es (umgekehrt oder symmetrisch) nur \textbf{\color{indigo}eine einzige innere Welt} gäbe,
in die jedes Individuum aus seinem eigenen persönlichen Blickwinkel schauen würde\,?
}}

\en{As a physicist
I am well aware of the colossal advantages that exact sciences have brought humanity.}%
\de{Als Physiker bin ich mir der kolossalen Vorteile bewusst,
die die exakten Wissenschaften der Menschheit gebracht haben.}
%
\en{I am also aware that just a single new experimental result
could wholly change practically all theories
as far as scientific answers to fundamental questions are concerned.}%
\de{Ich bin mir auch bewusst,
dass schon ein einziges neues experimentelles Resultat
praktisch alle Theorien völlig verändern kann
soweit es wissenschaftliche Antworten auf fundamentale Fragen betrifft.}
%
\en{But considering the current view of the universe in 2020,
I am beginning to wonder
if maybe the approach of current science might be too asymmetric
with regards to “in” and “out” as sources of cognition.}%
\de{Aber wenn ich mir die gegenwärtige Sicht auf das Universum im Jahr 2020 vor Augen halte,
beginne ich mich schon zu fragen,
ob der Ansatz der gegenwärtigen Wissenschaft vielleicht zu asymmetrisch sein könnte,
bezüglich was “innen” und “aussen” als Quellen der Erkenntnis betrifft.}
%
\en{In the “outer” view,
the universe appears extremely huge,
full of solar systems that would superficially resemble our solar system,
and, yet, no signs of life outside our own planet earth.}%
\de{In der “äusseren” Sicht
ist das Universum extrem riesig,
voll von Sonnensystemen,
die unserem Sonnensystem oberflächlich betrachtet ähneln würden,
und dennoch gibt es keine Anzeichen für Leben ausserhalb unseres eigenen Planeten Erde.}
%
\en{A large part of the universe would have to be composed
of postulated dark matter and dark energy
and, yet, even remote stars and galaxies seem to be made
of the same matter as our own environment.}%
\de{Ein grosser Teil des Universums müsste aus der postulierten dunklen Materie und dunklen Energie bestehen,
und doch scheinen selbst weit entfernte Sterne und Galaxien
aus der gleichen Materie wie unsere eigene Umgebung zu bestehen.}

\en{What speaks for a shared inner world would be, for example,
that in dreams of different people the same universal themes keep reappearing,
what Jung called \textsl{archetypes}.}%
\de{Was für eine gemeinsame innere Welt spräche, wäre, zum Beispiel,
dass in Träumen verschiedener Menschen
immer wieder die gleichen universellen Themen auftauchen,
das was Jung als \textsl{Archetypen} bezeichnete.}
%
\en{Of course,
in the current paradigm that could often be explained
via exchange of information in the outer world,
but this may not be a good explanation in all situations,
and generally not the simplest one.}%
\de{Natürlich liesse sich das im gegenwärtigen Paradigma
oft durch den Austausch von Informationen in der äusseren Welt erklären,
aber das ist vielleicht nicht in allen Situationen eine gute Erklärung,
und allgemein nicht die einfachste.}
%
\en{Paradigms are by definition rather a choice than a necessity;
they may make some parts of being easier or more complicated to describe,
but it may not be possible to prove paradigm A more true than paradigm B,
simply because paradigms often also influence how you determine “more true”.}%
\de{Paradigmen sind definitionsgemäss eher eine getroffene Wahl als eine Notwendigkeit;
sie mögen einige Teile des Seins leichter oder komplizierter beschreibbar machen,
aber es mag nicht möglich sein, Paradigma A als wahrer als Paradigma B zu beweisen,
einfach da Paradigmen oft auch beeinflussen wie “wahrer” bestimmt wird.}

\en{Obviously also some ancient “esoteric” (=“inner”) traditions like astrology
would assume in a way that there is just one inner world
of which each individual would be a specific part.}%
\de{Offensichtlich würden auch einige uralte “esoterische” (=“innere”) Traditionen wie die Astrologie
in gewisser Weise davon ausgehen,
dass es nur eine innere Welt gibt,
von der jedes Individuum ein spezifischer Teil wäre.}
%
\en{For example, of the pair of opposites egoism/altruism
a Leo would initially rather tend to egoism
and the opposite sign Aquarius rather to altruism.}%
\de{Zum Beispiel im Gegensatzpaar Egoismus/Altruismus,
wo ein Löwe zunächst eher zum Egoismus
und das entgegengesetzte Sternzeichen Wassermann eher zum Altruismus neigen würde.}
%
\en{Overall, a lot in astrology is based on a balance of opposites,
which does maybe also relate to Plato’s world of ideas.}%
\de{Insgesamt basiert vieles in der Astrologie auf einem Gleichgewicht der Gegensätze,
was vielleicht auch mit Platons Welt der Ideen zu tun hat.}
%
\en{Such an abstract world of ideas could be
all there is to an inner world as source of cognition,
or maybe not.}%
\de{Eine solche abstrakte Ideenwelt könnte alles sein,
was es an einer inneren Welt als Quelle der Erkenntnis gibt,
oder vielleicht auch nicht.}
%
\en{In any case,
a future science that would give the inner world
just as much weight and attention as the outer world
might be superior to current science in many ways,
just out of an argument of symmetry.}%
\de{Auf jeden Fall könnte eine zukünftige Wissenschaft,
die der inneren Welt gleich viel Gewicht und Aufmerksamkeit wie der äusseren schenken würde,
der gegenwärtigen Wissenschaft in vielerlei Hinsicht überlegen sein,
rein vom Argument der Symmetrie her.}

\en{I got this idea essentially after reading Jung’s work on psychological types,
where I had misread some passages without full context
and indirectly also attributed him somewhat of a “medieval” mind,
like this:}

\begin{center}
\includegraphics[scale=0.392]{i-sources.jpg}\linebreak
\textsl{\footnotesize
\en{Sources of cognition before Enlightenment:\linebreak
Robert Fludd,
Utriusque cosmi maioris scilicet et minoris […$\!$] historia,
tomus II (1619),
tractatus I, sectio I, liber X,
De triplici animae in corpore visione.}%
\de{Quellen der Erkenntnis vor der Aufklärung:\linebreak
Robert Fludd,
Utriusque cosmi maioris scilicet et minoris […$\!$] historia,
tomus II (1619),
tractatus I, sectio I, liber X,
De triplici animae in corpore visione.}}
\end{center}

\de{Diese Idee kam mir im Wesentlichen nach dem Lesen von Jungs Buch über psychologische Typen,
wo ich einige Passagen ohne vollen Kontext falsch gelesen hatte
und ihm indirekt auch einen “mittelalterlichen” Geist zuordnete,
etwa so wie im Bild auf der vorherigen Seite.}

\en{There are likely other aspects regarding sources of cognition
in which science is asymmetric, like conscious versus unconscious.}%
\de{Es gibt wahrscheinlich noch andere Gesichtspunkte bezüglich Quellen der Erkenntnis,
bei denen die Wissenschaft asymmetrisch ist, wie bei bewusst versus unbewusst.}
%
\en{This reflects also in astrology:
At night the sky shows lots of stars and planets;
during the day,
when people are typically consciously awake,
the sun outshines them all,
symbolically chasing away a maybe important “occult” part of the world.}%
\de{Dies spiegelt sich auch in der Astrologie wider:
Nachts zeigt der Himmel viele Sterne und Planeten;
tagsüber,
wenn die Menschen typischerweise bewusst wach sind,
überstrahlt die Sonne sie alle
und verjagt symbolisch einen vielleicht wichtigen “okkulten” Teil der Welt.}
%
\en{It may be worth noting that the moon
can shine both during day and night,
and even shadow the sun during a total solar eclipse.}%
\de{Es mag erwähnenswert sein,
dass der Mond sowohl bei Tag wie auch bei Nacht scheinen kann
und sogar die Sonne während einer totalen Sonnenfinsternis überschatten kann.}
%
\en{Thus to get a fuller picture, science might have to become,
so speak, “more like the moon”…}%
\de{Um ein vollständigeres Bild zu erhalten, könnte es also sein,
dass die Wissenschaft sozusagen “mehr wie der Mond” werden müsste…}

\en{By the way, in terms of my definition of elements on my web site,
two elements would be outside, two inside, i.e.\ quite symmetric from the start.}%
\de{Übrigens wären in meiner Definition von Elementen auf meiner Website
zwei Elemente aussen, zwei innen, d.h.\ von Anfang an recht symmetrisch.}
%
\en{I wrote this text initially in just about an hour,
so, for example, some complications with a mind that is supposedly inside itself
observing an inner world are not explored,
immediately for the sake of carving out some asymmetries most prominently.}%
\de{Ich schrieb diesen Text ursprünglich in nur etwa einer Stunde,
so dass zum Beispiel einige Komplikationen, die entstehen mit einem Verstand,
der selbst innen wäre und gleichzeitig eine innere Welt beobachtet,
nicht erforscht wurden,
unmittelbar um einige Asymmetrien besonders prominent herauszuschälen.}

\en{There is also a maybe more fundamental asymmetry in science,
a focus on the largest common denominator, on general things,
as first proposed by Aristotle in his metaphysics,
which would be, in part, contrary to Jarry’s pataphysics.}%
\de{Es gibt noch eine vielleicht grundlegendere Asymmetrie in der Wissenschaft,
eine Fokussierung auf den grössten gemeinsamen Nenner, auf allgemeine Dinge,
wie zuerst von Aristoteles in seiner Metaphysik vorgeschlagen,
was zum Teil gegensätzlich zu Jarrys Pataphysik wäre.}

\en{And there would likely be more assymetries of that kind…}%
\de{Und es gäbe wohl noch mehr Asymmetrien in der Art…}
