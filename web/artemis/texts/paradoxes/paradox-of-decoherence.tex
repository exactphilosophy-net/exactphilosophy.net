\documentclass[letterpaper]{article}
\pagestyle{empty}
\paperheight=3700mm
\textheight=3700mm
%\textwidth set depending on pdflatex or (experimentally)lualatex
\topmargin=-20mm
\oddsidemargin=25mm

\usepackage[utf8]{inputenc}
\usepackage[spanish,italian,french,ngerman,english]{babel} % last is main
\usepackage{graphicx}
\usepackage{multirow}
\usepackage{xcolor}
\usepackage{contour}
\usepackage{pict2e}
\usepackage{relsize}
\usepackage{amsmath}

\usepackage{iftex}
\ifpdftex
  % stronger fonts

% Find modes.mf, e.g. /usr/local/texlive/2025/texmf-dist/fonts/source/public/modes/modes.mf
%
% $ sudo cp modes.mf modes.mf.orig
%
% Add the following at the start of modes:
%
% mode_def xphi =
%   mode_param (pixels_per_inch, 1200);
%   mode_param (blacker, 1.9); % only difference to 'lexmarkr' (2 there)
%   mode_param (fillin, 0);
%   mode_param (o_correction, 1);
%   mode_common_setup_;
% enddef;
%
% Finally:
%
% $ sudo fmtutil-sys --byfmt mf

\pdfpkresolution=1200
\pdfpkmode={xphi}
\pdfmapfile{}

  \newcommand{\coretextwidth}{85.5mm}
\fi

% experimental, not used to produce the live website...
\ifluatex
  % about same heaviness in pdfs when rasterized in photoshop,
  % but since, unlike the metafont mechanisms I use, fake bold, "bleeds" in all directions,
  % seems heavier at least in core web page images
  \newcommand{\fontbleed}{0.8}
  % paragraphs wider and font looks larger, tried to fix, but then other things change a bit,
  % especially for section headings would have to change back, are now more narrow...
  \newcommand{\fontscale}{0.985}
  \newcommand{\coretextwidth}{85.2mm}
  \usepackage{fontspec}
  % microtype does maybe help and not help, maybe if would allow wider spaces...
  \usepackage{microtype}
  \setsansfont{Latin Modern Sans}[Scale=\fontscale, FakeBold=\fontbleed]
  \setmonofont{Latin Modern Mono}[Scale=\fontscale, FakeBold=\fontbleed]
  % this would be for "new computer modern" (but has many limitations so far)
  %\usepackage[default]{fontsetup}
  %\renewcommand{\familydefault}{\sfdefault}
\fi

\renewcommand{\familydefault}{\sfdefault}

\setcounter{secnumdepth}{-1}

\newcommand{\en}[1]{\iflanguage{english}{#1}{}}
\newcommand{\de}[1]{\iflanguage{ngerman}{#1}{}}
\newcommand{\fr}[1]{\iflanguage{french}{#1}{}}

% a bit less than ~255/256
\definecolor{almostwhite}{gray}{0.996}
\definecolor{xphi}{rgb}{0.0,0.5,0.5}
\definecolor{avant}{rgb}{1,0.5,0.5}

\definecolor{frame}{gray}{0.9}
\definecolor{lightgray}{gray}{0.8}
\definecolor{gray}{gray}{0.5}
\definecolor{darkgray}{gray}{0.3}

\definecolor{darkred}{rgb}{0.8,0.0,0.0}
\definecolor{darkyellow}{rgb}{0.7,0.7,0.0}
\definecolor{darkgreen}{rgb}{0.0,0.55,0.0}
\definecolor{darkviolet}{rgb}{0.5,0,0.5}

\definecolor{darkblue}{rgb}{0,0,0.7}
\definecolor{odyssey}{rgb}{0,0,0.8}
\definecolor{indigo}{rgb}{0.29,0,0.51}
\definecolor{indigoblue}{rgb}{0.1,0,0.6}

\definecolor{saffronback}{rgb}{1.000,0.878,0.627}
\definecolor{saffronfront}{rgb}{0.376,0.125,0.000}

\DeclareRobustCommand{\cometartemisscale}[1]{\includegraphics[scale=#1]{\sourcepath/i-comet.jpg}\hspace{-0.028453em} artemis}
\newcommand\cometartemis{\cometartemisscale{0.018}}
\newcommand\cometartemissection{\cometartemisscale{0.0225}}

\DeclareRobustCommand{\moebius}[1]{\includegraphics[scale=#1]{\sourcepath/i-moebius.jpg}}
\newcommand{\yinyang}{\includegraphics[scale=0.135]{\sourcepath/i-yinyang.jpg}}

\newcommand{\rarr}{\,$\rightarrow$\,}
\newcommand{\lrarr}{\,$\leftrightarrow$\,}

% greek elements
\newcommand{\elfire}{
\begin{picture}(9,6)
  \thicklines
  \put(1,-0.5){\line(1,0){7}}
  \put(1,-0.5){\line(1,1.732){3.5}}
  \put(8,-0.5){\line(-1,1.732){3.5}}
\end{picture}}
%
\newcommand{\elair}{
\begin{picture}(9,6)
  \thicklines
  \put(1,-0.5){\line(1,0){7}}
  \put(1,-0.5){\line(1,1.732){3.5}}
  \put(8,-0.5){\line(-1,1.732){3.5}}
  \put(2.75,1.9){\line(1,0){3.5}}
\end{picture}}
%
\newcommand{\elwater}{
\begin{picture}(9,6)
  \thicklines
  \put(1,5){\line(1,0){7}}
  \put(1,5){\line(1,-1.732){3.5}}
  \put(8,5){\line(-1,-1.732){3.5}}
\end{picture}}
%
\newcommand{\elearth}{
\begin{picture}(9,6)
  \thicklines
  \put(1,5){\line(1,0){7}}
  \put(1,5){\line(1,-1.732){3.5}}
  \put(8,5){\line(-1,-1.732){3.5}}
  \put(2.75,2.5){\line(1,0){3.5}}
\end{picture}}
%
\newcommand{\elhex}{
\begin{picture}(9,6)
  \thicklines
  \put(1,0.5){\line(1,0){7}}
  \put(1,0.5){\line(1,1.732){3.5}}
  \put(8,0.5){\line(-1,1.732){3.5}}
  \put(1,5){\line(1,0){7}}
  \put(1,5){\line(1,-1.732){3.5}}
  \put(8,5){\line(-1,-1.732){3.5}}
\end{picture}}

% i ching trigrams
\newcommand{\trigram}[3]{
\begin{picture}(9,6)
  \linethickness{0.36mm}
  \put(0,5){\line(1,0){#1}}
  \put(5.5,5){\line(1,0){3.5}}
  \put(0,2.5){\line(1,0){#2}}
  \put(5.5,2.5){\line(1,0){3.5}}
  \put(0,0){\line(1,0){#3}}
  \put(5.5,0){\line(1,0){3.5}}
\end{picture}}
\newcommand{\triheaven}{\trigram{5.5}{5.5}{5.5}}
\newcommand{\triearth}{\trigram{3.5}{3.5}{3.5}}
\newcommand{\trithunder}{\trigram{3.5}{3.5}{5.5}}
\newcommand{\triwater}{\trigram{3.5}{5.5}{3.5}}
\newcommand{\trimountain}{\trigram{5.5}{3.5}{3.5}}
\newcommand{\triwind}{\trigram{5.5}{5.5}{3.5}}
\newcommand{\trifire}{\trigram{5.5}{3.5}{5.5}}
\newcommand{\trilake}{\trigram{3.5}{5.5}{5.5}}

% i ching hexagrams
% 1+2 trigrams, 3 rest of line (see e.g. dreams.tex)
\DeclareRobustCommand{\hexagram}[3]{\raisebox{-3pt}{$\overset{\text{${#1}$}}{#2}$\,}#3\vspace{3pt}}

% white-red-black etc.
\DeclareRobustCommand{\outline}[1]{\contour{black}{{\color{white}#1}}}
\DeclareRobustCommand{\white}[1]{\outline{\textbf{#1}}}
\DeclareRobustCommand{\red}[1]{{\color{darkred}\textbf{#1}}}
\DeclareRobustCommand{\black}[1]{\textbf{#1}}
\DeclareRobustCommand{\yellow}[1]{{\color{darkyellow}\textbf{#1}}}
\DeclareRobustCommand{\green}[1]{{\color{darkgreen}\textbf{#1}}}
\DeclareRobustCommand{\violet}[1]{{\color{darkviolet}\textbf{#1}}}
\DeclareRobustCommand{\indigoblue}[1]{{\color{indigoblue}\textbf{#1}}}
\DeclareRobustCommand{\indigo}[1]{{\color{indigo}\textbf{#1}}}

% ELEMENTAL
\newcommand{\ELEMENTAL}{%
\colorlet{contour}{.}\textbf{\color{white}%
\raisebox{+0.001em}{\contour{contour}{E}}%
\raisebox{+0.015em}{\contour{contour}{L}}%
\raisebox{+0.016em}{\contour{contour}{E}}%
\raisebox{+0.023em}{\contour{contour}{M}}%
\raisebox{+0.023em}{\contour{contour}{E}}%
\raisebox{+0.017em}{\contour{contour}{N}}%
\raisebox{-0.020em}{\contour{contour}{T}}%
\raisebox{-0.002em}{\contour{contour}{A}}%
\raisebox{+0.006em}{\contour{contour}{L}}%
}}

% artemis pdf+web icons
\newcommand{\ipdfen}{\includegraphics[scale=0.5]{i-pdf-en.png}}
\newcommand{\ipdfde}{\includegraphics[scale=0.5]{i-pdf-de.png}}
\newcommand{\ipdffr}{\includegraphics[scale=0.5]{i-pdf-fr.png}}
\newcommand{\iweb}{\includegraphics[scale=0.055]{i-web.png}}
\newcommand{\ipdfblueen}{\includegraphics[scale=0.5]{i-pdf-blue-en.png}}
\newcommand{\ipdfbluede}{\includegraphics[scale=0.5]{i-pdf-blue-de.png}}
\newcommand{\ipdfbluefr}{\includegraphics[scale=0.5]{i-pdf-blue-fr.png}}
\newcommand{\iwebblue}{\includegraphics[scale=0.055]{i-web-blue.png}}


\textwidth=\coretextwidth



\begin{document}

\avantgarde

\section{Paradox of decoherence}

{\color{xphi}I combine several well-known Gedankenexperiments,
namely the one by Einstein, Podolski and Rosen (EPR),
plus Bell’s Inequalities,
and Schrödinger’s Cat,
as well as Wigner’s Friend,
into a new Gedankenexperiment},
that I essentially first devised in January 2003
for a Usenet post to the sci.physics.research newsgroup.
%
Archived here:
\href{https://www.classe.cornell.edu/spr/2003-01/msg0047545.html}%
{\color{xphi}https://www.classe.cornell.edu/spr/2003-01/msg0047545.html}

{\small
\begin{verbatim}
From: Alain Stalder <astalder@exactphilosophy.net>
Newsgroups: sci.physics.research
Subject: Re: Some questions on decoherence and QM.
Date: Mon, 13 Jan 2003 22:30:49 +0000 (UTC)
Message-ID: <astalder-A850F5.13133713012003@news.bluewin.ch>

In article <3E1C9025.A2D5A6CB@uni-essen.de>,
 Urs Schreiber <Urs.Schreiber@uni-essen.de> wrote:

> Frank Hellmann wrote:
 
> > A measurement of the quantum system described by rho in generally still
> > has a propability for both classically exclusive states though, so we
> > still have a superposition of classically exclusive states.
> 
> The last phrase must read: "a *mixture* of classical states". 
> 
> Using the density operator one is bound to talk about
> statistics only. Decoherence cannot and does not explain "how"
> a system chooses from the possible outcomes a specific one
> when we measure it. Decoherence only explains how the "quantum
> probability" becomes a "classical probability", very roughly
> speaking, but it still only gives probabilities.

It is worthwhile to explain what exactly "classical" means in
this context. This is maybe most easily seen if Schroedinger's
Gedankenexperiment is combined with the experiment for testing
Bell's Inequality:

Two entangled photons fly in opposite directions and then each
pass through polarization filters. A photon detector after each
filter either kills or does not kill a cat on each side,
depending on whether the respective photon has passed through
the polarization filter.
\end{verbatim}

\newpage

\begin{verbatim}
Decoherence tells us that each cat quickly ends up in a state
with a density matrix that is practically diagonal. Or, more
loosely put, the cat is "either dead or alive, but not both".
Can we conclude that whether the cat is dead or alive is already
determined, that an experimentator who looks inside to discover
either a dead or a living cat will only note what was already
determined before ?

No, because Bell's Inequality excludes any local hidden variable
theories in which for both cats it would already be determined
whether the cats are dead or alive. In other words, "classical"
means in this context only that you cannot do interference with
Schroedinger's cats, i.e. that they statistically behave like
measured cats, but not that measurement has already occured
through decoherence.

Hence some of the "strangeness" of quantum mechanics remains,
especially if you modify the above Gedankenexperiment to include
what is typically called "Wigner's Friend". Replace each cat by
an experimentator who looks at the detector, and place two other
experimentators outside the respective labs.

Now, when does measurement occur ? When the inner experimentators
look at the detectors, or when the outer experimentators open the
doors to the respective labs and ask the guys inside about what
they have measured ? At least decoherence tells us that we cannot
distinguish experimentally between the two possibilities, because
in both cases all experimentators behave statistically classical.

In conclusion, decoherence is a big step towards understanding
measurement in quantum mechanics, but does not go all the way,
at least not yet.

Alain Stalder
\end{verbatim}
}

\noindent
The more recent article
“Quantum theory cannot consistently describe the use of itself”
by Frauchiger and Renner (2018)
shows that at least in some cases
quantum mechanics as a universal theory
of how the world evolves
can lead to logical inconsistencies regarding measured data
from the point of view of different observers.
%
In other words,
if that proved to be true,
decoherence could not explain measurement in quantum mechanics
in general.

In a way,
this would have already been clear from my Gedankenexperiment:
Just singling out some quantum coherence that would decay independently on both sides,
except the one that is bound to remain correlated,
does not make sense.
%
In my view,
since science generally assumes that there is one “reality”—
otherwise published theories and measured data would not be the same for all,
i.e.\ the whole setup would be inconsistent—%
the only remaining solution might be that there really are connections at a “speed” faster than light behind the scenes,
i.e.\ also that the future would have an influence on the past,
albeit only within the limits of the strange things that quantum theory permits.

But the previous sentence is, of course,
not really news in this generality.
%
In any case,
{\color{xphi}I hope that my Gedankenexperiment
might help future research in quantum theory a bit},
if only as inspiration.

\end{document}
