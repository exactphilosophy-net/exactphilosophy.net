\avantgarde

\en{\section{Elemental changes in the I Ching\,?}}%
\de{\section{Elementare Wandlungen im I Ging\,?}}

\en{\subsection{Abstract}}%
\de{\subsection{Inhalt}}

\en{A new way of arranging the trigrams of the I Ching in a circle
is presented for consideration and discussion.}%
\de{Ich stelle eine neue Weise,
die Trigramme des I Gings in einem Kreis anzuordnen,
zur Ansicht und Diskussion vor.}
%
\en{Here it is in advance:}%
\de{Hier ist sie vorab:}

\vspace{4mm}
\hspace{-5mm}
\en{\href{https://www.exactphilosophy.net/elements-iching-circle.jpg}{\includegraphics[scale=0.16]{i-elements-iching-circle.jpg}}}%
\de{\href{https://www.exactphilosophy.net/elemente-iging-kreis.jpg}{\includegraphics[scale=0.16]{i-elemente-iging-kreis.jpg}}}

\vspace{2mm}
\noindent
\en{The eight trigrams are tentatively interpreted
as the eight possible transitions between the four “Greek” elements
Earth, Water, Air and Fire
along Aristotle’s circle of the elements.}%
\de{Die acht Trigramme werden tentativ
als die acht möglichen Wandlungen zwischen den vier “griechischen” Elementen
Erde, Wasser, Luft und Feuer
entlang Aristoteles’ Kreis der Elemente interpretiert.}
%
\en{The new arrangement features also
some very interesting structures and symmetries,
which will be exposed in detail later on.}%
\de{Die neue Anordnung enthält auch
einige sehr interessante Strukturen und Symmetrien,
welche ich später im Detail darlegen werde.}

\en{However,
no common historical roots of the I Ching and the ancient Greek elements are known,
and also no roots of the above arrangement in Chinese history,
although it is similar to the traditional Earlier Heaven arrangement.}%
\de{Allerdings sind keine gemeinsamen geschichtlichen Wurzeln des I Ging
und der antiken griechischen Elemente bekannt,
und auch keine Wurzeln der obigen Anordnung in der chinesischen Geschichte,
wenn sie auch der traditionellen Anordnung des Früheren Himmels ähnelt.}
%
\en{So did maybe both cultures mirror nature independently,
even unknowingly,
or is this maybe just a coincidence,
or a bit of both,
or…?}%
\de{Haben also beide Kulturen die Natur unabhängig,
sogar unwissentlich,
gespiegelt,
oder ist dies vielleicht nur Zufall,
oder ein wenig von Beidem,
oder…?}

\en{Not being an expert on ancient history
(I am a physicist)
and considering worldwide interest in the I Ching,
I thought it would be best to simply present this structure
as carefully and minimally as possible.}%
\de{Da ich kein Experte in antiker Geschichte bin
(ich bin Physiker)
und in Anbetracht des weltweiten Interesses am I Ging,
dachte ich,
es wäre das Beste,
diese Struktur einfach so sorgfältig und minimal wie möglich zu präsentieren.}

\en{In the following,
first a quick overview of Aristotle’s circle of elements
and the ancient elements as natural phenomena,
then to the I Ching
and finally to the new arrangement of the trigrams with its symmetries.}%
\de{Im Folgenden erst ein kleiner Überblick über Aristoteles’ Kreis der Elemente
und über die antiken Elemente als Naturphänomene,
dann zum I Ging
und schliesslich zur neuen Anordnung der Trigramme mit ihren Symmetrien.}

\newpage

\en{\subsection{Aristotle’s circle of the elements}}%
\de{\subsection{Aristoteles’ Kreis der Elemente}}

\en{In \textsl{On Generation and Corruption} (around 350 BCE)
Aristotle defines elements by properties that can be felt by \textsl{touching}.}%
\de{In \textsl{Über Werden und Vergehen} (um 350 v.\,Chr.)
definiert Aristoteles Elemente durch Eigenschaften,
die durch \textsl{Tasten} erfühlt werden können.}
%
\en{He names cold-hot, wet-dry, fine-coarse, soft-hard, brittle-malleable
and reduces the last four opposites to wet-dry.}%
\de{Er benennt warm-kalt, feucht-trocken, fein-grob, weich-hart, brüchig-formbar
und reduziert die letzten vier Gegensätze zu feucht-trocken.}
%
\en{Then he defines the four elements as combinations
of these two pairs of opposites cold-hot and wet-dry:
Earth as dry and cold,
Water as cold and wet,
Air as wet and hot,
Fire as hot and dry.}%
\de{Dann definiert er vier Elemente als Kombinationen
dieser zwei Paare von Gegensätzen,
warm-kalt und feucht-trocken:
Erde als trocken und kalt,
Wasser als kalt und feucht,
Luft als feucht und warm,
Feuer als warm und trocken.}

\en{He arranges the elements in a circle,
in which at each transition only one of the properties cold-hot or wet-dry is inverted:
Earth gets wet, becomes Water,
gets warm, becomes Air,
gets dry, becomes Fire,
gets cold, becomes Earth again,
and also the other way round in the circle:}%
\de{Er ordnet die Elemente in einem Kreis an,
wobei bei jedem Übergang nur eine der Eigenschaften
kalt-warm oder feucht-trocken umgekehrt wird:
Erde wird feucht, wird zu Wasser,
wird warm, wird zu Luft,
wird trocken, wird zu Feuer,
wird kalt, wird wieder zu Erde,
und auch umgekehrt im Kreis herum:}

\hspace{11mm}
\en{\href{https://www.exactphilosophy.net/elements-aristotle-circle.jpg}{\includegraphics[scale=0.24]{i-elements-aristotle-circle.jpg}}}%
\de{\href{https://www.exactphilosophy.net/elemente-aristoteles-kreis.jpg}{\includegraphics[scale=0.24]{i-elemente-aristoteles-kreis.jpg}}}

\noindent
\en{There are thus 8 possible transitions along this circle,
the same number as the number of trigrams in the I Ching.}%
\de{Somit gibt es 8 mögliche Übergänge zwischen den Elementen in diesem Kreis,
dieselbe Anzahl wie die Anzahl der Trigramme im I Ging.}

\en{In the tradition of the Stoics and other schools that emerged in Hellenistic times,
Earth and Water were later usually considered passive, heavy and female
and Air and Fire active, light and male.}%
\de{In der Tradition der Stoiker und anderer Schulen,
die im Hellenismus aufkamen,
wurden Erde und Wasser später üblicherweise als weiblich, passiv und schwer betrachtet,
sowie Luft und Feuer als männlich, aktiv und leicht.}

\en{\subsection{The Greek elements as natural phenomena}}%
\de{\subsection{Die Griechischen Elemente als Naturphänomene}}

\en{According to today’s chemistry and physics,
considering the material world to be composed of earth, water, air and fire
would, of course,
be wrong.}%
\de{Gemäss heutiger Chemie und Physik,
wäre es natürlich völlig falsch,
die materielle Welt als aus Erde, Wasser, Luft und Feuer
bestehend anzuschauen.}
%
\en{But if you interpret Earth as \textsl{solid},
Water as \textsl{liquid},
and Air as \textsl{gas}%
—the most commonly encountered \textsl{states} of matter—%
and interpret Fire as a
\textsl{chemical reaction or physical phenomenon that creates light and possibly heat},
this makes more sense:
Practically everything you could encounter in daily life in antiquity
would fit into one of these four categories,
would be either solid, liquid, gas or some sort of Fire,
or a mixture of these four,
like,
say,
mud a mixture of Earth and Water.}%
\de{Aber wenn man Erde als \textsl{fest} interpretiert,
Wasser als \textsl{flüssig}
und Luft als \textsl{gasförmig}%
—die am häufigsten angetroffenen \textsl{Aggregatzustände} von Materie—
und Feuer als eine
\textsl{chemische Reaktion oder physikalische Erscheinung, die Licht und möglicherweise Wärme erzeugt},
macht das mehr Sinn:
Praktisch alles,
was man im täglichen Leben in der Antike antreffen konnte,
würde in eine dieser vier Kategorien passen,
würde entweder fest, flüssig, gasförmig oder eine Art von Feuer sein,
oder eine Mischung dieser vier,
wie z.B. Schlamm ein Gemisch von Erde und Wasser.}
%
\en{Transformations of the elements,
like,
say,
ice melting and then evaporating (Earth-Water-Air)
or a fire that transforms wood into smoke and gas (Earth-Fire-Air),
would often be part of Aristotle’s circle.}%
\de{Wandlungen der Elemente,
wie z.B.\ Eis, das schmilzt und dann verdampft (Erde-Wasser-Luft)
oder ein Feuer, das Holz in Rauch und Gas verwandelt (Erde-Feuer-Luft),
wären oft Teil von Aristoteles’ Kreis.}

\newpage

\en{\subsection{The trigrams as transitions between elements}}%
\de{\subsection{Die Trigramme als Übergänge zwischen den Elementen}}

\en{The following three sentences in the introduction
of the Wilhelm/Baynes translation of the I Ching
had been in the back of my mind for years,
but only in August 2016,
I found a concrete, specific way of applying them:}%
\de{Die folgenden drei Sätze aus der Einführung
von Richard Wilhelms Übersetzung des I Ging
hatte ich viele Jahre im Hinterkopf,
aber erst im August 2016 fand ich einen konkreten, spezifischen Weg,
sie anzuwenden:}

\begin{quote}
{\textsl{\color{xphi}%
\en{The eight trigrams are symbols standing for changing transitional states;
they are images that are constantly undergoing change.}%
\de{Die acht [Trigramme] sind Zeichen wechselnder Übergangszustände,
Bilder,
die sich dauern verwandeln.}
%
\en{Attention centers not on things in their state of being%
—as is chiefly the case in the Occident—%
but upon their movements in change.}%
\de{Worauf das Augenmerk gerichtet war,
waren nicht die Dinge in ihrem Sein%
—wie das im Westen haupt\-säch\-lich der Fall war—,
sondern die Bewegungen der Dinge in ihrem Wechsel.}
%
\en{The eight trigrams therefore are not representations of things as such
but of their tendencies in movement.}%
\de{So sind die acht [Trigramme] nicht Abbildungen der Dinge,
sondern Abbildungen ihrer Bewegungstendenzen.}%
}}
\end{quote}

\noindent
\en{Hence eight trigrams as 8 transitions between 4 elements\,?}%
\de{Also acht Trigramme als 8 Wandlungen zwischen 4 Elementen\,?}
%
\en{In order to approach the specific arrangement step-by-step,
let me start with a table of the trigrams,
as given by Wilhelm/Baynes:}%
\de{Um mich der konkreten Anordnung Schritt für Schritt zu nähern,
beginne ich mit einer Tabelle der Trigramme,
gemäss Richard Wilhelm:}

\small
\begin{center}
\begin{tabular}{|l|l|l|l|l|l|}\hline
\en{\triheaven & qián & heaven & strong & creative & father \\ \hline
\triearth & k\={u}n & earth & devoted/yielding & receptive & mother \\ \hline
\trithunder & zhèn & thunder & inciting movement & arousing & 1st son\\ \hline
\triwater & k\v{a}n & water & dangerous & abysmal & 2nd son \\ \hline
\trimountain & gèn & mountain & resting & keeping still & 3rd son \\ \hline
\triwind & xùn & wind/wood & penetrating & gentle & 1st daughter \\ \hline
\trifire & lí & fire & light-giving & clinging & 2nd daughter \\ \hline
\trilake & duì & lake & joyful & joyous & 3rd daughter \\ \hline}%
\de{\triheaven & qián & Himmel & stark & schöpferisch & Vater \\ \hline
\triearth & k\={u}n & Erde & hingebend & empfangend & Mutter \\ \hline
\trithunder & zhèn & Donner & bewegend & erregend & 1.\ Sohn\\ \hline
\triwater & k\v{a}n & Wasser & gefährlich & abgründig & 2.\ Sohn \\ \hline
\trimountain & gèn & Berg & ruhend & stillehaltend & 3.\ Sohn \\ \hline
\triwind & xùn & Wind/Holz & eindringend & sanft & 1.\ Tochter \\ \hline
\trifire & lí & Feuer & leuchtend & haftend & 2.\ Tochter \\ \hline
\trilake & duì & See & fröhlich & heiter & 3.\ Tochter \\ \hline}
\end{tabular}
\end{center}
\normalsize

\noindent
\en{Now let me tentatively group them into pairs of “Greek” elements:}%
\de{Nun gruppiere ich sie tentativ zu Paaren von “griechischen” Elementen:}

\small
\begin{center}
\begin{tabular}{|l|l|l|l|l|l|}\hline
\en{\triheaven & qián & heaven & \textbf{Air} & (rests) & male \\ \hline
\triwind & xùn & wind & \textbf{Air} & (moves) & female \\ \hline
\trimountain & gèn & mountain & \textbf{Earth} & (rests) & male \\ \hline
\triearth & k\={u}n & earth & \textbf{Earth} & (moves) & female \\ \hline
\trifire & lí & fire & \textbf{Fire} & (rests) & female \\ \hline
\trithunder & zhèn & thunder & \textbf{Fire} & (moves) & male \\ \hline
\trilake & duì & lake & \textbf{Water} & (rests) & female \\ \hline
\triwater & k\v{a}n & water & \textbf{Water} & (moves) & male \\ \hline}%
\de{\triheaven & qián & Himmel & \textbf{Luft} & (ruht) & männlich \\ \hline
\triwind & xùn & Wind & \textbf{Luft} & (bewegt sich) & weiblich \\ \hline
\trimountain & gèn & Berg & \textbf{Erde} & (ruht) & männlich \\ \hline
\triearth & k\={u}n & Erde & \textbf{Erde} & (bewegt sich) & weiblich \\ \hline
\trifire & lí & Feuer & \textbf{Feuer} & (ruht) & weiblich \\ \hline
\trithunder & zhèn & Donner & \textbf{Feuer} & (bewegt sich) & männlich \\ \hline
\trilake & duì & See & \textbf{Wasser} & (ruht) & weiblich \\ \hline
\triwater & k\v{a}n & Wasser & \textbf{Wasser} & (bewegt sich) & männlich \\ \hline}
\end{tabular}
\end{center}
\normalsize

\noindent
\en{And finally arrange them in Aristotle’s circle,
letting the transition end with the corresponding element in the table above
and start with a female element (Water or Earth)
for the female trigrams (mother and three daughters)
and with a male element (Fire or Air)
for the male trigrams (father and three sons):}%
\de{Und schliesslich ordne ich sie in Aristoteles’ Kreis an,
wobei der Übergang mit dem entsprechenden Element
gemäss der Tabelle oben endet
und für die weiblichen Trigramme (Mutter und drei Töchter)
mit einem weiblichen Element (Wasser oder Erde) beginnt,
sowie für die männlichen Trigramme (Vater und drei Söhne)
mit einem männlichen Element (Feuer oder Luft):}

\newpage

\hspace{-5mm}
\en{\href{https://www.exactphilosophy.net/elements-iching-circle.jpg}{\includegraphics[scale=0.16]{i-elements-iching-circle.jpg}}}%
\de{\href{https://www.exactphilosophy.net/elemente-iging-kreis.jpg}{\includegraphics[scale=0.16]{i-elemente-iging-kreis.jpg}}}

\vspace{-5mm}

\small
\begin{center}
\begin{tabular}{|l|l|l|rl|l|}\hline
\en{\triheaven & qián & heaven & \textbf{Air} & \textbf{$\!\!\!\!\!\!$\color{gray}{$\leftarrow$\,Fire}} & Air risen from fire \\ \hline
\triwind & xùn & wind & \textbf{Air} & \textbf{$\!\!\!\!\!\!$\color{gray}{$\leftarrow$\,Water}} & Air from evaporated water \\ \hline
\trimountain & gèn & mountain & \textbf{Earth} & \textbf{$\!\!\!\!\!\!$\color{gray}{$\leftarrow$\,Fire}} & Earth from solidified lava (Fire) \\ \hline
\triearth & k\={u}n & earth & \textbf{Earth} & \textbf{$\!\!\!\!\!\!$\color{gray}{$\leftarrow$\,Water}} & Earth from sediments deposited by water \\ \hline
\trifire & lí & fire & \textbf{Fire} & \textbf{$\!\!\!\!\!\!$\color{gray}{$\leftarrow$\,Earth}} & Fire from burning matter (Earth) \\ \hline
\trithunder & zhèn & thunder & \textbf{Fire} & \textbf{$\!\!\!\!\!\!$\color{gray}{$\leftarrow$\,Air}} & Fire as lightning from the sky (Air) \\ \hline
\trilake & duì & lake & \textbf{Water} & \textbf{$\!\!\!\!\!\!$\color{gray}{$\leftarrow$\,Earth}} & Water sprung from sources (Earth) \\ \hline
\triwater & k\v{a}n & water & \textbf{Water} & \textbf{$\!\!\!\!\!\!$\color{gray}{$\leftarrow$\,Air}} & Water fallen as rain from the sky (Air) \\ \hline}%
\de{\triheaven & qián & Himmel & \textbf{Luft} & \textbf{$\!\!\!\!\!\!$\color{gray}{$\leftarrow$\,Feuer}} & Luft aufgestiegen aus Feuer \\ \hline
\triwind & xùn & Wind & \textbf{Luft} & \textbf{$\!\!\!\!\!\!$\color{gray}{$\leftarrow$\,Wasser}} & Luft aus verdunstetem Wasser \\ \hline
\trimountain & gèn & Berg & \textbf{Erde} & \textbf{$\!\!\!\!\!\!$\color{gray}{$\leftarrow$\,Feuer}} & Erde aus erstarrter Lava (Feuer) \\ \hline
\triearth & k\={u}n & Erde & \textbf{Erde} & \textbf{$\!\!\!\!\!\!$\color{gray}{$\leftarrow$\,Wasser}} & Erde aus von Wasser abgelagerten Sedimenten \\ \hline
\trifire & lí & Feuer & \textbf{Feuer} & \textbf{$\!\!\!\!\!\!$\color{gray}{$\leftarrow$\,Erde}} & Feuer aus brennender Materie (Erde) \\ \hline
\trithunder & zhèn & Donner & \textbf{Feuer} & \textbf{$\!\!\!\!\!\!$\color{gray}{$\leftarrow$\,Luft}} & Feuer als Blitz vom Himmel (Luft) \\ \hline
\trilake & duì & See & \textbf{Wasser} & \textbf{$\!\!\!\!\!\!$\color{gray}{$\leftarrow$\,Erde}} & Wasser entsprungen aus Quellen (Erde) \\ \hline
\triwater & k\v{a}n & Wasser & \textbf{Wasser} & \textbf{$\!\!\!\!\!\!$\color{gray}{$\leftarrow$\,Luft}} & Wasser gefallen als Regen vom Himmel (Luft) \\ \hline}
\end{tabular}
\end{center}
\normalsize

\noindent
\en{Note that this circle is none of the two traditionally known ones,
neither the \textsl{Earlier Heaven} nor the \textsl{Later Heaven}
arrangement of the trigrams:}%
\de{Man beachte,
dass dieser Kreis keiner der zwei traditionell bekannten ist,
weder die Anordnung des \textsl{Frühen Himmels}
noch die des \textsl{Späten Himmels}:}

\vspace{2mm}
\hspace{12mm}
\en{\href{https://www.exactphilosophy.net/elements-iching-circles.jpg}{\includegraphics[scale=0.12]{i-elements-iching-circles.jpg}}}%
\de{\href{https://www.exactphilosophy.net/elemente-iging-kreise.jpg}{\includegraphics[scale=0.12]{i-elemente-iging-kreise.jpg}}}

\vspace{2mm}
\noindent
\en{\textsl{The trigrams seem to fit closely:}
Thunder as fire that has suddenly come down as lightning from the sky (Air),
in contrast to fire steadily clinging to the matter (Earth) it burns;
wind as air that gently evaporated from water,
in contrast to gases from a fire risen to heaven;
a lake as water sprung from sources (Earth),
in contrast to water fallen down as rain from the sky (Air);
a mountain as earth solidified from lava (Fire),
in contrast to softly yielding earth from sediments deposited by water.}%
\de{\textsl{Die Trigramme scheinen gut zu passen:}
Donner als Feuer, das plötzlich als Blitz vom Himmel (Luft) herunter kommt,
im Kontrast zu Feuer das kontinuierlich an der Materie (Erde) haftet, die es verbrennt;
Wind als Luft, die sanft aus Wasser verdampft ist,
im Kontrast zu Gasen von einem Feuer, die in den Himmel aufgestiegen sind;
ein See als Wasser entsprungen aus Quellen (Erde),
im Kontrast zu Wasser, das als Regen vom Himmel (Erde) heruntergefallen ist;
ein Berg als Erde erstarrt aus Lava (Feuer),
im Kontrast zu sanft nachgebender Erde aus Sedimenten, die von Wasser abgelagert wurden.}

\vspace{2mm}
\noindent
\en{The arrangement has also the following additional symmetries:}%
\de{Die Anordnung hat folgende zusätzliche Symmetrien:}

\vspace{4mm}
\noindent\begin{minipage}{5mm}
\mbox{\ \,\includegraphics[scale=0.18]{i-elements-iching-circle-symm-family.jpg}}
\end{minipage}%
\hfill%
\begin{minipage}{0.8\textwidth}
\en{Sons and daughters are grouped in order of birth
from mother to father
(gray dotted lines above).}%
\de{Söhne und Töchter sind in der Reihenfolge ihrer Geburt
von Mutter zu Vater angeordnet
(grau gepunktete Linien oben).}
\end{minipage}

\vspace{2mm}
\noindent\begin{minipage}{5mm}
\mbox{\ \,\includegraphics[scale=0.18]{i-elements-iching-circle-symm-opposites.jpg}}
\end{minipage}%
\hfill%
\begin{minipage}{0.8\textwidth}
\en{If you mirror the lines of each trigram at their middle line
(i.e.\ swap first and third line)
and invert all three lines of the trigram (yin\lrarr yang),
you get exactly the lines of the trigram opposite in the circle.}%
\de{Wenn man die Linien jedes Trigramms an ihrer mittleren Linie spiegelt
(also erste und dritte Linie vertauscht)
und alle drei Linien des Trigrams invertiert (Yin\lrarr Yang),
erhält man genau die Linien des im Kreis gegenüberliegenden Trigrams.}
\end{minipage}

\newpage

\vspace{4mm}
\noindent\begin{minipage}{5mm}
\mbox{\ \,\includegraphics[scale=0.18]{i-elements-iching-circle-symm-wetdry.jpg}}
\end{minipage}%
\hfill%
\begin{minipage}{0.8\textwidth}
\en{The middle line of each trigram that transforms to or from a dry element
is a broken line,
which would fit with Aristotle’s view that dry is brittle,
hence can be broken more easily than wet.}%
\de{Die mittlere Linie von jedem Trigram,
das zu oder von einem trockenen Element verwandelt,
ist eine gebrochene Linie,
was dazu passen würde,
dass trocken laut Aristoteles brüchig ist,
also einfacher als feucht gebrochen werden kann.}
\end{minipage}

\vspace{2mm}
\noindent\begin{minipage}{5mm}
\mbox{\ \,\includegraphics[scale=0.18]{i-elements-iching-circle-symm-alternate.jpg}}
\end{minipage}%
\hfill%
\begin{minipage}{0.8\textwidth}
\en{Excluding the middle line,
between adjacent trigrams in the circle
always exactly one line is inverted (yin\lrarr yang).}%
\de{Die mittlere Linie ausgenommen,
wird zwischen im Kreis anliegenden Trigrammen
jeweils immer genau eine Linie invertiert (Yin\lrarr Yang).}
\end{minipage}

\vspace{7mm}
\noindent
\en{Here is another way of representing the structure,
along a Möbius Strip,
inspired by the images on Billy Culver’s “Energy Language”
\href{https://web.archive.org/web/20201020213547/https://sites.google.com/site/energylanguage/}{\color{xphi}website}.}%
\de{Hier ist noch eine weitere Weise,
wie die Struktur dargestellt werden kann,
nämlich auf einem Möbiusband,
inspiriert von Billy Culver’s “Energy Language”
\href{https://web.archive.org/web/20201020213547/https://sites.google.com/site/energylanguage/}{\color{xphi}Website}.}
% newer site: https://sites.google.com/view/mappingcycles/
%
\en{Note that each yin line on one side of the strip
touches a yang line on the other side,
and vice-versa,
and that the symbols for the Greek elements overlap,
too.}%
\de{Man beachte,
dass jede Yin Linie auf einer Seite des Bandes
auf der anderen Seite eine Yang Linie berührt,
und umgekehrt,
und dass die Symbole für die griechischen Elemente sich ebenfalls überlappen.}

\vspace{3mm}
\hspace{-10mm}
\noindent
\en{\href{https://www.exactphilosophy.net/elements-iching-moebius.jpg}{\includegraphics[scale=0.126]{i-elements-iching-moebius.jpg}}}%
\de{\href{https://www.exactphilosophy.net/elemente-iging-moebius.jpg}{\includegraphics[scale=0.126]{i-elemente-iging-moebius.jpg}}}

\en{\subsection{Conclusion}}%
\de{\subsection{Fazit}}

\en{My conclusion is twofold.}%
\de{Mein Fazit ist zweifaltig.}
%
\en{As a physicist,
I find that the number of symmetries is a bit too high
for immediately assuming a pure coincidence.}%
\de{Als Physiker finde ich,
dass die Anzahl Symmetrien ein wenig zu gross ist,
um unmittelbar von einem reinen Zufall auszugehen.}
%
\en{Judging how well the trigrams correlate with elementary transitions in the proposed way,
however,
is more difficult,
first fundamentally,
because it is prose and not math
and second for me personally,
because I am not an expert on ancient Chinese culture,
so my overview is limited and based on translations.}%
\de{Beurteilen,
wie gut die Trigramme mit elementaren Wandlungen
in der vorgeschlagenen Weise korrelieren,
ist allerdings schwieriger,
erstens fundamental,
weil es Prosa und nicht Mathematik ist,
und zweitens für mich persönlich,
weil ich kein Experte der antiken chinesischen Kultur bin,
also meine Übersicht eingeschränkt ist und auf Übersetzungen beruht.}
%
\en{This is why I wrote this up,
as a starting point for anyone interested
to take a closer look…}%
\de{Daher habe ich dies hier aufgeschrieben,
als Startpunkt für alle,
die Interesse haben,
sich das genauer anzuschauen…}

\en{This is not to say that I have no further ideas regarding this
and the I Ching and the elements,
but nothing further that immediately fits the focus here;
for more see my website
\href{https://www.exactphilosophy.net/}{\color{xphi}exactphilosophy.net}.}%
\de{Das soll nicht heissen,
dass ich keine weiteren Ideen bezüglich dem hier
und dem I Ging und den Elementen hätte,
aber nichts weiter,
das unmittelbar in den Fokus hier passt;
siehe dazu stattdessen meine Website
\href{https://www.exactphilosophy.net/}{\color{xphi}exactphilosophy.net}.}

\newpage

\footnotesize
\noindent
\en{\textbf{PS:} Note that you can click on most illustrations in this text
for the individual image with higher resolution.}%
\de{\textbf{PS:} Auf die meisten Illustrationen in diesem Text kann man klicken
für das jeweilige Bild mit höherer Auflösung.}
%
\en{Feel free to share these images with whoever you think might be interested,
if possible preferably with a link to this article or to work that references it.}%
\de{Gerne dürfen diese Bilder weiter geteilt werden;
wenn möglich vorzugsweise mit einem Link zu diesem Artikel
oder zu Arbeiten,
die auf ihn verweisen.}

\vspace{4mm}
\noindent
\en{\textbf{Previous related work:}
I am not aware of previous work that features
the same arrangement of trigrams as presented here.}%
\de{\textbf{Frühere verwandte Werke:}
Mir sind keine früheren Werke bewusst,
die dieselbe Anordnung der Trigramme vorschlagen.}
%
\en{Considering the millennia of considerations of the I Ching,
it might however be very likely that someone did in the past.}%
\de{In Anbetracht von Betrachtungen des I Ging über Jahrtausende hinweg,
könnte es jedoch sehr gut sein,
dass jemand das in der Vergangenheit bereits tat.}

\en{That said,
this article features more than “just an arrangement of trigrams in a circle”,
namely a \textsl{model} in terms of elemental transitions behind it.}%
\de{Dieser Artikel beinhaltet jedoch mehr als “nur eine Anordnung von Trigrammen in einem Kreis”,
nämlich ein \textsl{Modell} von elementaren Wandlungen dahinter.}
%
\en{Quite generally,
this article is rather pan-cultural or even non-cultural,
relating to the I Ching as well as to ancient Greek and Indian views on the elements,
and probably to any other culture and its views on “elements”.}%
\de{Ganz allgemein
ist dieser Artikel eher kulturübergreifend oder sogar nicht kulturell,
indem er sowohl auf das I Ging Bezug nimmt,
wie auch auf antike griechische und indische Sichten der Elemente,
sowie wahrscheinlich auf jede andere Kultur und ihre Sicht auf “Elemente”.}
%
\en{And note that the main illustration in this article features 12 not 8 symbols in a circle,
hinting thus also at possible relations to various zodiacs,
most likely,
though,
the Chinese one.}%
\de{Nicht unwesentlich ist auch,
dass die Hauptillustration dieses Artikels nicht 8 sondern 12 Symbole in einem Kreis zeigt,
und so auch mögliche Beziehungen zu Tierkreisen in verschiedenen Kulturen suggeriert,
am ehesten jedoch wohl zum chinesischen.}

\en{Moreover,
note that I originally got to this model coming from a definition of “elements”
in terms of in/out and rest/move,
which you can find in the main part of my website.}%
\de{Darüber hinaus
entstand dieses Modell ursprünglich aus einer Definition von “Elementen”
durch innen/aussen und ruhen/sich bewegen,
welche sich im Herz meiner Website befindet.}

\en{What I am aware of is that at least two people
attributed trigrams in pairs to elements in the same way as I just presented here
(i.e.\ Fire to fire/thunder,
Air to heaven/wind,
Water to water/lake
and Earth to earth/mountain).}%
\de{Was mir bekannt ist,
ist dass mindestens zwei Leute
schon vor mir Trigramme genauso in Paaren zu Elementen zugeordnet hatten,
wie ich gerade hier vorgestellt habe
(also Feuer zu Feuer/Donner,
Luft zu Himmel/Wind,
Wasser zu Wasser/See
und Erde zu Erde/Berg).}
%
\en{These two are \textbf{Bradford Hatcher} (\href{http://www.hermetica.info/}{\color{xphi}hermetica.info})
and \textbf{Karen Witter} (\href{https://groups.yahoo.com/neo/groups/WheelofTarot/info}{\color{xphi}WheelOfTarot}).}%
\de{Diese zwei sind \textbf{Bradford Hatcher} (\href{http://www.hermetica.info/}{\color{xphi}hermetica.info})
und \textbf{Karen Witter} (\href{https://groups.yahoo.com/neo/groups/WheelofTarot/info}{\color{xphi}WheelOfTarot}).}
%
\en{Bradford Hatcher brought this to my attention in January 2017
and told me he had been using this “since 1976”.}%
\de{Bradford Hatcher hatte mich im Januar 2017 darauf aufmerksam gemacht
und sagte mir, dass er das schon “seit 1976” so benutzt hatte.}
%
\en{References are Bradford Hatcher’s books \textsl{Tarot as a Counseling Language} [1]
and \textsl{The Book of Changes: Word by Word, Vol.\ 1, Part 2} [2],
and Karen Witter in a post to her Yahoo Group of 1999 [3].}%
\de{Referenzen sind Bradford Hatchers Bücher \textsl{Tarot as a Counseling Language} [1]
und \textsl{The Book of Changes: Word by Word, Vol.\ 1, Part 2} [2],
und Karen Witter in einem Beitrag an ihre Yahoo Gruppe im Jahr 1999 [3].}

\begin{list}{$\bullet$}{\setlength{\leftmargin}{20pt}}

\item[\mbox{[1]}]
\en{This book is available on his website
\href{http://www.hermetica.info/Tarot.htm}{\color{xphi}here} (html).}%
\de{Das Buch ist auf seiner Website
\href{http://www.hermetica.info/Tarot.htm}{\color{xphi}hier} erhältlich (html).}
%
\en{Search the text for “I Ching” to find a subsection
where he explains the attribution of trigrams to elements,
referring also to Karen Witter and earlier roots by Aleister Crowley.}%
\de{Im Text nach “I Ching” suchen um zu einem Abschnitt zu gelangen,
wo er die Zuordnung von Elementen zu Trigrammen erklärt,
mit auch Referenzen auf Karen Witter und frühere Wurzeln bei Aleister Crowley.}
%
\en{Note that both BH and KW group trigrams also into two sets of four elements,
“archetypal” elements (heaven, water, earth, fire)
and “attributional” elements (wind, lake, mountain, thunder).}%
\de{Zu beachten ist auch, dass sowohl BH wie auch KW
die Trigramme in zwei Gruppen von vier Elementen gruppieren,
“archetypische” Elemente (“archetypal”; Himmel, Wasser, Erde, Feuer)
und “zugeordnete” Elemente (“attributional”; Wind, See, Berg, Donner).}
%
\en{See also \href{http://www.hermetica.info/Tarot.htm\#E}{\color{xphi}this table}
in the book for these attributions.}%
\de{Siehe auch \href{http://www.hermetica.info/Tarot.htm\#E}{\color{xphi}diese Tabelle}
im Buch für diese Zuordnungen.}

\item[\mbox{[2]}]
\en{This book is listed prominently on his website
\href{http://www.hermetica.info/}{\color{xphi}hermetica.info}
and can also be downloaded as a
\href{https://www.hermetica.info/Yijing1+2.pdf}{\color{xphi}pdf}.}%
\de{Dieses Buch ist prominent auf seiner Website
\href{http://www.hermetica.info/}{\color{xphi}hermetica.info} aufgelistet
und kann auch als
\href{https://www.hermetica.info/Yijing1+2.pdf}{\color{xphi}pdf}
heruntergeladen werden.}
%
\en{See the descriptions of the trigrams in part 2 (vol.\ 1),
or search for “Greater Earth” to find this maybe more easily.}%
\de{Siehe die Beschreibungen der Trigramme in Teil 2 (Band~1),
oder nach “Greater Earth” suchen, um das vielleicht etwas einfacher zu finden.}
%
\en{In that book he uses “greater” and “lesser”
instead of “archetypal” and “attributional”
to distinguish the two sets of trigrams.}%
\de{In diesem Buch benutzt er “greater” and “lesser” (in etwa “primär” und “sekundär”)
statt “archetypal” und “attributional”,
um die Trigramme in dieselben zwei Gruppen einzuteilen.}

\item[\mbox{[3]}]
\en{Here is a
\href{https://groups.yahoo.com/neo/groups/WheelofTarot/conversations/topics/5}{\color{xphi}link to her post}
and an excerpt from the post:}%
\de{Hier ist ein
\href{https://groups.yahoo.com/neo/groups/WheelofTarot/conversations/topics/5}{\color{xphi}Link zu ihrem Beitrag}
und ein Ausschnitt aus dem Beitrag, auf deutsch übersetzt (siehe die
\href{https://www.exactphilosophy.net/elemental-changes-i-ching.pdf}{\color{xphi}englische Version}
dieses Artikels für das Original auf englisch):}

{\scriptsize
\vspace{2.45mm}
\en{\texttt{1.~There are 4 archetypal elements and 4 attributional elements in the I-Ching.\newline
Shown below; 1=line, 0=broken line, left to right=top to bottom:\newline
\newline
Archetype Air (Heaven) 111\newline
Attribute Air (Wind) 110\newline
\newline
Archetype Water (Water) 010\newline
Attribute Water (Marsh) 011\newline
\newline
Archetype Earth (Earth) 000\newline
Attribute Earth (Mountain) 001\newline
\newline
Archetype Fire (Fire) 101\newline
Attribute Fire (Thunder) 001}\par}%
\de{\texttt{1.~Es gibt vier archetypische Elemente und 4 zugeordnete Elemente im I-Ging.\newline
Unten gezeigt; 1=Linie, 0=Unterbrochene Linie, links nach rechts=oben nach unten:\newline
\newline
Archetyp Luft (Himmel) 111\newline
Attribut Luft (Wind) 110\newline
\newline
Archetyp Wasser (Wasser) 010\newline
Attribut Wasser (Sumpf) 011\newline
\newline
Archetyp Erde (Erde) 000\newline
Attribut Erde (Berg) 001\newline
\newline
Archetyp Feuer (Feuer) 101\newline
Attribut Feuer (Donner) 001}\par}}
\footnotesize

\end{list}
