\avantgarde

\en{\section{Cargo Cult Astrology}}%
\de{\section{Cargo-Kult-Astrologie}}

\en{As of summer 2022,
nothing I ever did in astrology really “landed” there
in the sense that it would continue to spur interest without my direct involvement,
even after almost 20 years and many articles and a book,
while just one article about the I Ching created more response.}%
\de{Stand Sommer 2022,
ist nichts von dem, was ich jemals in und um Astrologie gemacht habe, dort wirklich “gelandet”,
in dem Sinn, dass es ohne mein direktes Zutun
weiterhin Interesse erwecken würde,
nicht einmal nach fast 20 Jahren und vielen Artikeln und einem Buch,
währenddem ein einziger Artikel über das I Ging mehr Resonanz hervorrief.}
%
\en{In the following some thoughts “in spirals”
as they unfolded since January 2022,
slightly \hspace{-0.2em}\textsl{polemic}\hspace{-0.1em} here and there, maybe;
preserved here as \textbf{\color{darkblue}Zeitzeugnis} of how things seemed in the world then…}%
\de{Im Folgenden einige Gedanken “in Spiralen”
– in den Spiralen, in denen sie sich ab Januar 2022 entfaltet hatten,
leicht \textsl{polemisch} hier und da, vielleicht;
bewahrt als \textbf{\color{darkblue}Zeitzeugnis} dafür, wie die Dinge zu der Zeit so schienen in der Welt…}
%
\de{So gut es ging im August 2022 aus dem Original auf Englisch übersetzt,
eher auf den Fluss im Original Rücksicht genommen
als auf übliche Formulierungen auf Deutsch…}

\begin{center}
\includegraphics[scale=0.474]{i-foxyfox.png}
\end{center}
\en{\vspace{-3mm}}%
\de{\vspace{-4.4mm}}

\en{Astrologers have remained 100\% faithful to their multitude of views on the world,
not favoring any one of them more than individually their own personal view,
usually embedded into various currents in astrology
and usually including most of the mainstream,
and essentially immune to any arguments or “proof”,
if only that some simplifications regarding how things are connected
would be of any worth, rather preferring to dream on,
of a connection with the “stars” or whatever they would fancy.}%
\de{Die Astrologen sind ihrer Vielzahl von Weltanschauungen 100\% treu geblieben,
bevorzugen keine daraus mehr als individuell ihre eigene persönliche Sicht,
normalerweise eingebunden in verschiedene Strömungen in der Astrologie
und normalerweise inklusive dem grössten Teil des Mainstream,
und im Wesentlichen immun gegen jegliche Argumente oder “Beweise”,
nur schon dass ein paar Vereinfachungen  bzgl.\ wie die Dinge miteinander verbunden sind,
irgendetwas wert wären, ziehen es vor, weiter zu träumen,
von einer Verbindung mit den “Sternen” oder worauf auch immer sie Lust haben.}

\en{Since late August 2021,
astro.com has a new slogan, “The Art of Astrology”.}%
\de{Seit Ende August 2021 hat astro.com einen neuen Slogan:
“Die Kunst der Astrologie”.}
%
\en{This mirrors the approach to the world at astro.com
and in much of the self-proclaimed “cargo cult” life science astrology
it supports, as well as of many of astro.com’s customers, quite well:
There are no criteria regarding what gets attention in astrology
except what astrologers like\,!}%
\de{Das spiegelt den Zugang zur Welt bei astro.com
und in einem Grossteil der selbsternannten “Cargo-Kult”-Lifescience-Astrologie,
die astro.com unterstützt, sowie die vieler Kunden von astro.com, recht gut wider:
Es gibt keine Kriterien dafür, was in der Astrologie Beachtung findet,
ausser dem, was Astrologen gefällt\,!}

\en{In June 2002,
I had not expected everybody to grasp and like what I published,
but I expected at least some to do so,
including also Alois Treindl, physicist and founder\,/\,CEO of Astrodienst AG
(the company that owns astro.com),
and Liz Greene, psychologist\,/\,astrologer
and then chief astrologer at Astrodienst AG.}%
\de{Im Juni 2002 hatte ich nicht erwartet gehabt,
dass jeder das, was ich veröffentlichte, verstehen und mögen würde,
aber ich erwartete, dass zumindest einige das tun würden,
darunter auch Alois Treindl, Physiker und Gründer\,/\,CEO der Astrodienst AG
(die Firma, der astro.com gehört),
und Liz Greene, Psychologin\,/\,Astrologin
und damals Chefastrologin der Astrodienst AG.}
%
\en{I was thinking that, especially since my model was directly derived
from how Liz Greene described the star signs (signs of the zodiac)
and the two together had created the computer-generated psychological horoscope,
that they would see the unique value in my simpler model,
as a key to better and more analytically understand the signs,
and also that this would be a door-opener for proving astrology,
reaching out beyond the “ghetto” of astrology.}%
\de{Ich dachte, dass die zwei, insbesondere da mein Modell direkt davon abgeleitet war,
wie Liz Greene die Sternzeichen beschrieben hatte,
und die beiden zusammen das computergenerierte psychologische Horoskop geschaffen hatten,
den einzigartigen Wert meines einfacheren Modells
als Schlüssel zu einem besseren und analytischeren Verständnis der Sternzeichen erkennen würden,
und auch dass dies ein Türöffner für den Nachweis der Astrologie sein würde,
der über das “Ghetto” der Astrologie hinausreichen würde.}

\en{It turned out differently.}%
\de{Es kam anders.}

\en{In early 2005,
AT told me that he had shown my original document to LG at the time (2002)
and that she had then deemed it “too theoretical”
while AT himself apparently even more underwhelmingly superficially
told me it was nothing new anyways,
referring to a relatively old Swiss book that,
after I had ordered an antiquarian copy for about 100\,\$,
turned out to simply describe the traditional attribution of elements
to seasons, in Aristotle’s circle,
and later on AT also had the nerve to call me eccentric
and my choice of domain name “exactphilosophy.net” exaggerated,
which I guess means that he would be the center of it all
and that “astro.com”
– covering both “astrology” and “astronomy” and more by name –
would reflect ultimate modesty.}%
\de{Anfang 2005 sagte mir AT,
dass er LG damals (2002) mein ursprüngliches Dokument gezeigt hatte
und sie es dann für “zu theoretisch” befunden hatte,
während AT selbst mir, anscheinend noch unbeeindruckenderweise oberflächlich,
mitteilte, dass es ohnehin nichts Neues sei,
und sich auf ein relativ altes Schweizer Buch bezog, worin,
nachdem ich ein antiquarisches Exemplar für ca.\ 100\,\$ bestellt hatte,
sich herausstellte, dass dort einfach die traditionelle Zuordnung der Elemente
zu den Jahreszeiten im Kreis des Aristoteles beschrieben wurde,
und später hatte AT auch noch den Nerv, mich als exzentrisch
und meine Wahl des Domainnamens “exactphilosophy. net” als übertrieben zu bezeichnen,
was, vermute ich,
wohl bedeutet, dass er der Mittelpunkt von Allem wäre
und dass “astro.com”
– was sowohl “Astrologie” als auch “Astronomie” und mehr namentlich abdeckt –
ultimative Bescheidenheit widerspiegeln würde.}
%
\en{What Astrodienst\,/\,astro.com is doing
for the immediate needs and desires of contemporary astrologers
is not a problem, that is usually of good to very good quality;
the problem is what they evade and escape,
along with practically all astrologers.}%
\de{Was Astrodienst\,/\,astro.com für die unmittelbaren Bedürfnisse und Wünsche
der zeitgenössischen Astrologen tut,
ist kein Problem, das ist in der Regel von guter bis sehr guter Qualität;
das Problem ist das, was sie umgehen und wovon sie flüchten,
zusammen mit praktisch allen Astrologen.}

\en{But back to why astrologers love this and not that:
In my general model this would mainly be due to unconscious collective feedback
that manifests in feelings,
in other words the Fates would apparently simply not want people
to look at my discoveries openly and more closely,
at least not yet,
and there would be nothing in my power that could change that
as long as the Fates do not want it to happen,
while, whenever they would want to in the future,
they could presumably do it in two shakes of a lamb’s tail…}%
\de{Aber zurück zu der Frage, warum Astrologen dieses lieben und jenes nicht:
In meinem allgemeinen Modell wäre dies hauptsächlich wegen unbewusstem kollektiven Feedback,
der sich in Gefühlen manifestiert,
mit anderen Worten, die Schicksalsgöttinnen würden offenbar einfach nicht wollen,
dass die Menschen sich meine Entdeckungen offen und von nahe betrachten,
zumindest noch nicht,
und es gäbe nichts in meiner Macht, was das ändern könnte,
solange die Schicksalsgöttinnen nicht wollen, dass dies geschieht,
währenddem sie es, wann immer sie es in der Zukunft wollen möchten,
vermutlich im Handumdrehen tun könnten…}

\begin{center}
\includegraphics[scale=0.474]{i-foxyfox.png}
\end{center}
\vspace{-3mm}

\en{If you maybe think I exaggerated above
with regards to how contemporary astrologers view astrology,
read the following quotes from the forum at astro.com in January 2022:}%
\de{Wenn ihr jetzt vielleicht denkt,
dass ich oben übertrieben hätte,
was die Sichtweise zeitgenössischer Astrologen angeht,
lest die folgenden Zitate
aus dem Forum auf astro.com im Januar 2022:}

\textbf{Waybread}:
\en{Astrology is basically a subjective art, anyway.}%
\de{Astrologie ist im Grunde genommen sowieso eine subjektive Kunst.}
%
\en{I think a good analogy is with painting as a fine art.}%
\de{Ich denke, eine gute Analogie ist mit der Malerei als bildende Kunst.}
%
\en{There are specific rules of perspective, color, and composition.}%
\de{Es gibt bestimmte Regeln für Perspektive, Farbe und Komposition.}
%
\en{But after that, the painter makes many subjective choices
and the viewer subsequently decides whether s/he thinks it is a good painting or not.}%
\de{Aber danach trifft der Maler viele subjektive Entscheidungen,
und der Betrachter entscheidet dann, ob sie/er das Bild gut findet oder nicht.}

\textbf{Linda Harris}:
\en{Meanwhile, {\small USE} your system.}%
\de{In der Zwischenzeit, {\small BENUTZT} Euer System.}
%
\en{And if it works, what do you care who else uses it\,?}%
\de{Und wenn es funktioniert, was kümmert es Euch, wer es sonst noch benutzt\,?}
%
\en{I don’t think the job of an astrologer
is to change the way other astrologers do their work
or the way they are comfortable doing that work.}%
\de{Ich glaube nicht, dass es die Aufgabe eines Astrologen ist,
die Art und Weise, wie andere Astrologen ihre Arbeit machen,
oder die Art und Weise, wie sie diese Arbeit gerne machen, zu ändern.}
%
\en{It’s my point of view that we are here to be of help
to those who don’t speak the language of “human behavior”
in the way that we do…
to be part of their “solution” and not part of their “problem”.}%
\de{Ich bin der Ansicht, dass wir hier sind, um denen zu helfen,
die die Sprache des “menschlichen Verhaltens” nicht so sprechen wie wir…
um Teil ihrer “Lösung” und nicht Teil ihres “Problems” zu sein.}

\textbf{zed234}
\en{(after reading my article “Elementary Star Signs”):
It’s an interesting discussion
about the origins of the signs
and how they relate to each other.}%
\de{(nach dem Lesen meines Artikel “Elementare Sternzeichen”):
Es ist eine interessante Diskussion über die Ursprünge der Sternzeichen
und wie sie sich zueinander verhalten.}
%
\en{But as a counseling astrologer, it’s simply too simplistic for me
to use in any way as a tool to assist a client.}%
\de{Aber als beratender Astrologe ist das für mich einfach zu grob vereinfachend,
um es in irgendeiner Weise als Hilfsmittel für einen Klienten zu verwenden.}
%
\en{I have a several books on the history and development of astrology,
and I’ve seen this discussion or in several related to it several times.}%
\de{Ich besitze mehrere Bücher über die Geschichte und Entwicklung der Astrologie,
und ich habe diese Diskussion oder mehrere, die damit zusammenhängen, schon mehrmals gesehen.}
%
\en{Very interesting.}%
\de{Sehr interessant.}
%
\en{But not very useful to a practicing astrologer.}%
\de{Aber nicht sehr nützlich für einen praktizierenden Astrologen.}
%
\en{But maybe {\small I’M NOT GETTING THAT}.}%
\de{Aber vielleicht {\small KAPIERE ICH DAS NICHT}.}
%
\en{If so, perhaps you could show me how it would work in analysis.}%
\de{Wenn ja, könntest Du mir vielleicht zeigen, wie das in der Analyse funktionieren würde.}

\en{You might think that the last two sentences would be an opportunity,
but note that what I had written in the article, in the thread
and in what is publicly readable of my book of the same name
(to which I had also posted a link)
did not ring any bells, and it even appears
that he completely misunderstood even basic things,
like “about the origins of the signs”
or that he would have “seen this discussion or in several related to it several times”,
plus he commented on Linda Harris’ post with “nice post, Lainie!”.}%
\de{Man könnte jetzt meinen, dass die letzten beiden Sätze eine Gelegenheit bieten würden,
aber man beachte, dass das, was ich in dem Artikel geschrieben hatte, sowie in dem Faden
und in dem, was öffentlich lesbar ist von meinem gleichnamigen Buch
(zu dem ich auch einen Link gepostet hatte),
keine Glocken läuten liess, und es scheint sogar,
dass er selbst grundlegende Dinge völlig missverstanden hatte,
wie “über die Ursprünge der Zeichen” oder dass er
“diese Diskussion oder in mehreren damit zusammenhängenden mehrmals gesehen hätte”,
ausserdem kommentierte er Linda Harris’ Beitrag mit “Schöner Beitrag, Lainie!”.}
%
\en{At some point you simply have seen enough to still sustain hope,
even in the more and more unlikely case
that anything I could do would make any difference in my lifetime,
except maybe if I used the most powerful weapon in the world, the mirror,
namely that I just let things rest,
then astrologers would be only faced with themselves…}%
\de{Irgendwann hat man einfach genug gesehen, um die Hoffnung aufrechtzuerhalten,
selbst für den immer unwahrscheinlicher werdenden Fall,
dass irgendetwas, was ich tun könnte, zu meinen Lebzeiten irgendetwas ändern würde,
ausser vielleicht, wenn ich die mächtigste Waffe der Welt, den Spiegel, einsetzen würde,
nämlich dass ich die Dinge einfach ruhen lasse,
dann wären die Astrologen nur mit sich selbst konfrontiert…}

\begin{center}
\includegraphics[scale=0.474]{i-foxyfox.png}
\end{center}
\vspace{-3mm}

\en{To be fair,
three of the six reviews of my book “Elementary Star Signs” in 2018
did in fact grasp the basic idea
(Christoph Schubert-Weller, Victor Olliver and Jutta Stemmer),
even though almost certainly nobody grasped the potential,
while at least Jutta Stemmer
told me by email in early 2019
that~she does recommend the book as lecture to her pupils
at her astrology school in Munich.}%
\de{Fairerweise muss ich erwähnen,
dass drei der sechs Rezensenten meines Buches “Elementare Sternzeichen”
im Jahr 2018 tatsächlich die Grundidee begriffen hatten
(Christoph Schubert-Weller, Victor Olliver und Jutta Stemmer),
auch wenn fast sicher niemand davon das Potenzial begriffen hat,
während zumindest Jutta Stemmer mir Anfang 2019 per E-Mail mitteilte,
dass sie das Buch ihren Schülern
an ihrer Astrologieschule in München als Lektüre empfiehlt.}
%
\en{But coming back to the beginning:
My core idea is not centered on astrology,
so if I simply go by doing that where my heart is,
there will at most be (if at all)
one more book about astrology that briefly teases
with many ideas I have about astrology,
but without interest and sustained support by astrologers,
unlikely further attempts to promote my “elemental zodiac signs”.}%
\de{Aber um auf den Anfang zurückzukommen:
Meine Kernidee ist nicht um die Astrologie herum zentriert,
also wenn ich einfach das weitermache, wo mein Herz ist,
wird es höchstens (wenn überhaupt) noch ein weiteres Buch über Astrologie geben,
das viele Ideen, die ich über Astrologie habe, kurz antönt,
aber ohne Interesse und nachhaltige Unterstützung durch Astrologen,
sind weitere Versuche, meine “elementalen Tierkreiszeichen” zu bewerben,
unwahrscheinlich.}

\en{One important reason behind that decision
is also that what I wrote about elemental star signs
may not be perfect,
not yet Hollywood-style accessible to everybody –
but I expect some basic abilities and knowledge from astrologers,
and those were overall clearly lacking%
—short,
I do not see it as my fate/destiny to fight Idiocracy,
obstacles that are \textbf{purely imagined} by contemporary astrologers,
where if fact there are practically none at all,
even if it was me who might have unknowingly created those obstacles
by the way I write.}%
\de{Ein wichtiger Grund für diese Entscheidung ist auch,
dass das, was ich über die elementaren Sternzeichen geschrieben habe,
vielleicht nicht perfekt und noch nicht in Hollywood-Manier für jedermann zugänglich ist
– aber ich erwarte von Astrologen einige grundlegende Fähigkeiten und Kenntnisse,
und daran mangelte es insgesamt eindeutig%
—kurz,
ich sehe es nicht als mein Schicksal an, gegen die Idiokratie zu kämpfen,
gegen Hindernisse, die sich zeitgenössische Astrologen \textbf{rein nur vorstellen},
wo es in Wirklichkeit praktisch gar keine gibt,
selbst wenn ich es gewesen wäre, der diese Hindernisse durch die Art,
wie ich schreibe, unwissentlich geschaffen hätte.}

\en{I just realized something via a post to the same thread as quoted twice above,
namely that psychological astrology would be mainly water/fire,
while scientific theories are mainly air/earth,
going back to an idea from the 2002 discoveries document,
which would, of course mirror, quite a bit of my past experiences
and I probably underestimated
how much energy would be needed to bring such things together,
and I might have done better than I imagined.}%
\de{Mir ist gerade etwas klar geworden, und zwar durch einen Beitrag in demselben Faden,
den ich oben zweimal zitiert habe,
nämlich dass die psychologische Astrologie hauptsächlich Wasser/Feuer wäre,
während die wissenschaftlichen Theorien hauptsächlich Luft/Erde sind,
was auf eine Idee aus dem Discoveries Dokument von 2002 zurückgeht,
was natürlich einige meiner Erfahrungen widerspiegeln würde,
und ich hatte wahrscheinlich unterschätzt,
wie viel Energie nötig wäre, um solche Dinge zusammenzubringen,
und es könnte besser gelaufen sein, als ich es mir vorgestellt hatte.}

\en{For example, regarding Astrodienst\,/\,astro.com it would thus make sense
that the company was founded by and lead by a physicist (AT)
and would have “service”$\!$/“serving” in its name,
as it would be air/earth in a serving role to astrology,
providing the tools for doing astrology (water/fire),
collective watery streams and individual fiery sparks of pet theories,
and it would also explain
why AT delegated judgement of my discoveries document of 2002
practically exclusively to LG
who judged it “too theoretical”,
in other words too much of the element air in it,
hence does not fit into astrology.}%
\de{Zum Beispiel, in Bezug auf Astrodienst\,/\,astro.com würde es Sinn machen,
dass das Unternehmen von einem Physiker (AT) gegründet wurde und geleitet wird
und “Dienst” im Namen trägt,
da es Luft/Erde in einer dienenden Rolle für die Astrologie wäre,
welche die Werkzeuge für die Astrologie (Wasser/Feuer),
kollektive wässrige Ströme und individuelle feurige Funken für Lieblingsmethoden,
zur Verfügung stellt,
und es würde auch erklären,
warum AT die Beurteilung meines Discovery Dokuments von 2002
praktisch ausschliesslich an LG delegiert hat,
die es als “zu theoretisch” beurteilt hatte,
mit anderen Worten, es enthält zu viel vom Element Luft,
passt also daher nicht zur Astrologie.}
%
\en{I am finally relieved.}%
\de{Ich bin endlich erleichtert/erlöst.}

\en{As of the evening of 20 January 2022,
still some interesting things have evolved at the astro.com forum
and maybe I will still present my posts or a selection from them at some point in time,
maybe even already by the time you are reading this.}%
\de{Heute am Abend des 20.\ Januar 2022
haben sich im astro.com Forum doch noch einige interessante Dinge entwickelt
und vielleicht werde ich meine Beiträge oder eine Auswahl daraus irgendwann noch präsentieren,
vielleicht sogar schon, wenn ihr dies hier lest.}
%
\en{A bit later tonight there will be an exact opposition of Saturn
in transit to my natal Sun in Leo
(the only one for this Saturn cycle),
and one of the results of that
might be a broader view by me of the element Air,
in a nutshell that maybe I will see it (besides much logic)
also as something social, inspired by Robert Pirsig,
not unlikely also because I myself have no Air in my chart,
so this seemingly tiny bit of information
might be helpful in the future
and maybe also shortly worth a lead
in the core content of my web site.}%
\de{Etwas später heute Abend wird es eine exakte Opposition
von Saturn im Transit zu meiner Geburtssonne in Löwe geben
(die einzige in diesem Saturn-Zyklus),
und eines der Ergebnisse davon
könnte eine breitere Sichtweise von mir auf das Element Luft sein,
kurz gesagt, dass ich die Luft vielleicht (neben viel Logik)
auch als etwas Soziales sehen werde,
inspiriert von Robert Pirsig, nicht unwahrscheinlich auch deshalb,
weil ich selbst keine Luft in meinem Horoskop habe,
so dass dieses scheinbar winzige bisschen Information
in der Zukunft hilfreich sein könnte
und vielleicht auch in Kürze einen Lead im Kerninhalt meiner Website wert.}

\en{Ah, at least something,
always beautiful to find out new things for a researcher in physics.}%
\de{Ah, wenigstens etwas,
immer schön für einen Forscher in der Physik, neue Dinge herauszufinden.}

\begin{center}
\includegraphics[scale=0.474]{i-foxyfox.png}
\end{center}
\vspace{-3mm}

\en{Last night,
most likely already today in the early morning,
I had the following dream:}%
\de{Letzte Nacht,
wahrscheinlich schon heute in den frühen Morgenstunden,
hatte ich den folgenden Traum:}

\vspace{2mm}
\noindent
\colorbox{saffronback}{
\begin{minipage}{123mm}

\small{\color{saffronfront}
\en{I was at Astrodienst,
but it was a different place in the dream,
sort of a large loft,
just one large room,
maybe 20\,m\,$\times$\,25\,m,
and there was a woman,
a bit like Dolly Parton,
even if I did not see her precisely,
in other words very intelligent and also funny
[she was like Dolly Parton because of that,
there was no defined hair color
nor a specific body shape
besides clearly female in the dream],
and we walked to one end of the room,
or rather almost a corner,
and there was also the founder\,/\,CEO of Astrodienst,
in some rather comfortable office chair,
leaning back,
almost lying already,
and the woman gave me a present,
an envelope that was maybe 2\,$\times$ the size of US letter,
brown,
and inside there was a yellow pullover,
with a v-neck and squares of about 2” or a bit less stitched onto it,
both not really my kind of pullover,
and behind that about 3 yellow shirts,
with buttons and so,
all not really my kind of clothes
(that would rather be round necks and t-shirts),
but I was very happy about the present in the dream,
and AT in the chair “mumbled” something like
that giving me the present would be exaggerated,
and that was I think because I am not working at Astrodienst,
but then he said nothing any more,
and then the woman was again walking across the room,
astonishingly in the same direction as before,
and this time clearly in a yellow dress
[all yellow clothes in the dream were sunny and warm,
not like citrus],
quite a special,
beautiful one,
with stitched ornaments…}%
\de{Ich war bei Astrodienst,
aber es war anders dort in dem Traum,
eine Art grosser Loft,
nur ein grosser Raum,
vielleicht 20\,m\,$\times$\,25\,m,
und da war eine Frau,
ein bisschen wie Dolly Parton,
auch wenn ich sie nicht genau gesehen hatte,
also sehr intelligent und auch lustig
[sie war wie Dolly Parton deswegen,
es gab weder eine bestimmte Haarfarbe
noch eine bestimmte Körperform
ausser eindeutig weiblich in dem Traum],
und wir gingen zu einem Ende des Raumes,
oder besser gesagt fast in eine Ecke,
und dort war auch der Gründer\,/\,CEO von Astrodienst,
in einem ziemlich bequemen Bürostuhl,
zurückgelehnt,
fast schon liegend,
und die Frau gab mir ein Geschenk,
einen Umschlag,
der vielleicht 2\,$\times$ so gross war wie US-Letter Format,
braun,
und darin war ein gelber Pullover,
mit V-Ausschnitt 
und aufgestickten Quadraten von ca.\ 2” oder etwas weniger,
beides nicht wirklich meine Art von Pullover,
und dahinter ca.\ 3 gelbe Hemden,
mit Knöpfen und so,
alles nicht wirklich meine Art von Kleidung
(das wären eher Rundkragen und T-Shirts),
aber ich hatte mich im Traum sehr über das Geschenk gefreut,
und AT im Stuhl “brummelte” etwas wie,
dass mir das Geschenk geben übertrieben wäre,
und das war,
glaube ich,
weil ich nicht bei Astrodienst arbeite,
aber dann sagte er nichts mehr,
und dann lief die Frau wieder durch den Raum,
erstaunlicherweise in die gleiche Richtung wie vorher,
und diesmal eindeutig in einem gelben Kleid
[alle gelben Kleider im Traum waren sonnig und warm, nicht Zitronengelb],
einem ganz besonderen,
schönen,
mit aufgestickten Ornamenten…}}

\end{minipage}}
\vspace{2mm}

\en{In the West, yellow is Fire,
in China Earth (Saturn), the middle (hence also China itself),
and the color of emperors, and it also reminds of the sun,
Psyche’s second task and the golden fleece (clothes, reminding also of Vesta),
and presiding over white-red-black
it is also the elusive color chlōrós of the Great Goddess (of many names) herself,
this time in a hue that evokes honey,
and maybe since the dream played at Astrodienst
also alluding to something more Greene…
in any case, there seems to be recognition by the Fates.}%
\de{Im Westen ist Gelb das Feuer,
in China die Erde (Saturn), die Mitte (also auch China selbst),
und die Farbe der Kaiser, und es erinnert auch an die Sonne,
Psyches zweite Aufgabe und das goldene Vlies (Kleidung, die auch an Vesta erinnert),
und über Weiss-Rot-Schwarz thronend
ist es auch die schwer fassbare Farbe chlōrós
der Grossen Göttin (mit vielen Namen) selbst,
diesmal in einem Farbton, der an Honig erinnert,
und vielleicht da der Traum bei Astrodienst spielte,
auch auf etwas Greeneres angespielt…
auf jeden Fall scheint es eine Anerkennung durch die Schicksalsgöttinnen zu geben.}
%
\en{(And,
of course,
this mirrored the transit of Saturn in opposition to my natal sun.)}%
\de{(Und natürlich spiegelte dies den Transit von Saturn
in Opposition zu meiner Geburtssonne wider.)}

\en{(In retrospect, seen from June 2022 almost exactly at the summer solstice,
see the maturity initiation ritual to Artemis as a she-bear at Bauron
where young girls (and apparently sometimes also boys)
wore \colorbox{saffronback}{\color{saffronfront}saffron-yellow} short skirts
(that had replaced bear skins earlier in history)
that were possibly dropped at the end of the ritual.}%
\de{(Im Nachhinein, von Juni 2022 aus gesehen, fast genau zur Sommersonnenwende,
siehe das Reife-Initiierungsritual der Artemis als Bärin in Bauron,
bei dem junge Mädchen (und offenbar manchmal auch Jungen)
\colorbox{saffronback}{\color{saffronfront}safrangelbe} kurze Röcke trugen
(welche früher in der Geschichte die Bärenfelle ersetzt hatten),
die möglicherweise am Ende des Rituals abgelegt wurden.}
%
\en{\colorbox{saffronback}{\color{saffronfront}Saffron-yellow}
was quite exactly the color in the dream,
and I even think to remember
that I accidentally first wrote ‘skirt’ instead of ‘shirt‘
in the post with the dream at the astro.com forum,
which in retrospect may have been an Artemisian Slip…}%
\de{\colorbox{saffronback}{\color{saffronfront}Safrangelb} 
war ziemlich genau die Farbe in dem Traum,
und ich glaube mich sogar daran zu erinnern,
dass ich in dem Beitrag mit dem Traum im astro.com Forum
zuerst versehentlich ‘skirt’ (Rock) statt ‘shirt’ (Hemd) geschrieben hatte,
was im Nachhinein vielleicht ein artemisianischer Versprecher war…}
%
\en{And ‘V’ in v-neck might stand for Vesta/Hestia…)}%
\de{Und das ‘V’ in V-Ausschnitt könnte für Vesta/Hestia stehen…)}

\en{Possibly it was my Saturn return of 1996 in Montréal
(late in the 5th house, trine late Cancer MC
and trine Pluto in early Sagittarius in the 2nd)
that created lots of ideas
(and was lots of work since close to the cusp of the 6th)
but also apparently does make them seem not relevant to the public (10th)
– maybe(!) also because the south node then
was where my moon was at birth,
in the 10th house about 5° ahead of my MC\,?}%
\de{Möglicherweise war es meine Saturn-Rückkehr von 1996 in Montréal
(Saturn spät im 5.\ Haus, Trigon zum MC spät im Krebs
und Trigon zu Pluto früh im Schützen im 2.),
welche viele Ideen hervorbrachte
(und viel Arbeit war, da nahe an der Spitze des 6.),
diese aber offenbar auch für die Öffentlichkeit (10.) nicht relevant erscheinen lässt
– vielleicht(!) auch, weil der s.Mondknoten damals dort war,
wo mein Mond bei der Geburt stand,
im 10.\ Haus etwa 5° weiter als mein MC\,?}
%
\en{At least the first return of three in 2025/2026 would have Saturn in the 10th house
(with Ceres, n.node and Neptune),
assuming I would then still be at home
“on my island of Ithaca” in Adliswil, Switzerland.}%
\de{Zumindest die erste Rückkehr von dreien 2025/2026 hätte Saturn im 10.\ Haus
(mit Ceres, n.Mondknoten und Neptun),
vorausgesetzt, ich wäre dann immer noch zu Hause
“auf meiner Insel Ithaka” in Adliswil, Schweiz.}
%
\en{The book {\footnotesize\ELEMENTAL}
I am preparing on and off since early January 2019
would indicate that the first of these three returns
might be most relevant for the 29 years after that
regarding things of eventually public interest in my findings\,—\,maybe\,? }%
\de{Das Buch {\footnotesize\ELEMENTAL},
das ich seit Anfang Januar 2019 immer wieder ab und zu eingleise,
würde darauf hindeuten, dass die erste dieser drei Rückkehren
für die 29 Jahre danach am relevantesten sein könnte
bezüglich Dingen von öffentlichem Interesse für meine Funde
– vielleicht\,?}

\en{(In retrospect, seen from June 2022,
it seems much more like a return influences
quite strictly the period to the next return,
so the third of three would be the one to look at
for the about 29 years that might maximally follow,
and that one would have interestingly houses almost as at my birth,
which had then only lasted until then next return in February 1967,
which had then placed my Saturn in the 11th house,
likely fitting well with my studies and work
integrated into mainstream physics at the time.)}%
\de{(Im Rückblick, von Juni 2022 aus gesehen,
scheint es eher so zu sein,
dass eine Rückkehr ganz genau den Zeitraum bis zur nächsten Rückkehr beeinflusst,
so dass die dritte von dreien diejenige wäre, die man für die etwa 29 Jahre,
die maximal folgen könnten, betrachten sollte,
und diese hätte interessanterweise die Häuser fast wie bei meiner Geburt,
wo es nur bis zur nächsten Rückkehr im Februar 1967 gedauert hatte,
die dann meinen Saturn im 11.\ Haus platziert hatte,
was wahrscheinlich gut zu meinem Studium und meiner Arbeit,
je in die Mainstream-Physik integriert, gepasst hatte.)}

\en{But how about then just leaning back instead,
and have faith that people will come,
both attributes of Pisces,
and maybe even more so if at the end of Pisces,
plus Saturn likes minimalism, and then simply wait…
until people come who would then finally be able to “see Saturn in the 10th house”,
augmented by Ceres with gentle growth,
with a refreshing Hollywood-style early Aries cardinal halo of Neptune to dream of,
and boosted by the n.node\,?}%
\de{Aber wie wäre es denn,
wenn ich mich stattdessen einfach zurücklehnen würde
und darauf vertrauen würde, dass die Leute schon kommen werden,
beides Attribute der Fische,
und vielleicht sogar noch mehr, wenn sie am Ende der Fische stehen,
plus Saturn, der Minimalismus mag, und dann einfach nur warten…
bis Leute kommen, die dann endlich in der Lage wären,
meinen “Saturn im 10.\ Haus zu sehen”,
verstärkt durch Ceres mit sanftem Wachstum,
mit einem erfrischenden Hollywood-Style kardinalen Widder-Halo
durch Neptun um davon zu träumen,
und geboostert durch den n.Mondknoten\,?}
%
\en{Might my dream of being able to just watch things grow
simply come true by waiting,
having faith and helping out here and there\,?}%
\de{Könnte mein Traum, den Dingen einfach beim Wachsen zuzusehen,
wahr werden, indem ich warte,
Vertrauen habe und hier und da aushelfe\,?}
%
\en{Wish me luck,
since my discoveries are very beautiful, rich and useful/helpful,
for almost anybody, at least in the future.}%
\de{Wünscht mir Glück,
denn meine Entdeckungen sind sehr schön, reichhaltig und nützlich/hilfreich, 
für fast jeden, zumindest in der Zukunft.}

\en{(Seen again from June 2022,
maybe a Saturn in the 9th might mirror mainly just waiting,
while maybe also dropping a few things here and there,
quite well, and even for that return,
Neptune would be just a little bit ahead in early Aries,
so there might be Hollywood,
also since the 9th house would fit, too, in several ways.}%
\de{(Wiederum vom Juni 2022 aus gesehen,
könnte ein Saturn im 9.\ Haus vielleicht hauptsächlich das einfach nur Warten sein,
plus vielleicht auch hier und da ein paar Dinge fallen lassen, gut spiegeln,
und selbst für diese Rückkehr wäre Neptun im frühen Widder,
nur ein kleines Stück voraus, also könnte es Hollywood werden,
auch weil das 9.\ Haus dazu in verschiedener Hinsicht passen würde.}
%
\en{Maybe what is already public would just do
and perception change\,?)}%
\de{Vielleicht würde das, was bereits öffentlich ist, einfach reichen
und sich die Wahrnehmung ändern\,?)}

\begin{center}
\includegraphics[scale=0.474]{i-foxyfox.png}
\end{center}
\vspace{-3mm}

\en{In conclusion,
the Uranian split of the world into “astrology” and “astronomy” is
– in retrospect not astonishingly – deeper that I naively assumed,
and since currently nobody really suffers from this “division of labor”,
there is little immediate pressure to change anything in that regard.}%
\de{Zusammenfassend
wäre die uranische Spaltung der Welt in “Astrologie” und “Astronomie”
– im Nachhinein nicht erstaunlich – tiefer, als ich naiverweise annahm,
und da derzeit niemand wirklich unter dieser “Arbeitsteilung” leidet
gibt es wenig unmittelbaren Druck, in dieser Hinsicht etwas zu ändern.}
%
\en{So, why should I worry all too much\,?}%
\de{Warum sollte ich mir also allzu grosse Sorgen machen\,?}
%
\en{Que sera, sera…}%
\de{Que sera, sera…}

\en{Does all not exclude maybe some other book about astrology some day,
or at least still archiving my posts,
but any of that rather just to preserve some things than to promote them.}%
\de{Das alles schliesst nicht aus,
dass ich vielleicht eines Tages ein weiteres Buch über Astrologie schreibe
oder zumindest meine Beiträge archiviere,
aber das alles eher, um einige Dinge zu bewahren, als um sie zu promoten.}
%
\en{I guess because I have no Air in my birth chart,
there is no Earth as a goal for me
(see my model of the elemental changes in the star signs),
so no urge to literally manifest things in the world.}%
\de{Ich schätze, weil ich keine Luft in meinem Geburtshoroskop habe,
gibt es für mich keine Erde als Ziel
(siehe mein Modell der elementaren Wandlungen in den Sternzeichen),
also keinen Drang, Dinge wortwörtlich in der Welt zu manifestieren.}
%
\en{But a few seeds that start to grow would still be nice,
since at least Ceres and 1969 Alain are in Gemini (Air sign),
which also hints at written publications.}%
\de{Aber ein paar Samen, die zu wachsen beginnen würden, wären trotzdem schön,
da zumindest Ceres und 1969 Alain in den Zwillingen (Luftzeichen) stehen,
was auch auf schriftliche Veröffentlichungen hindeutet.}
%
\en{And so on.}%
\de{Und so weiter.}

\en{Just in case:}%
\de{Nur im Fall:}
%
\en{After almost 20 years,
in my feeling the overall response by astrology has been \textsl{inexcusable},
and individually also at least by Alois Treindl, Liz Greene and Robert Hand,
while at least some others may have simply lacked the ability
and context to grasp much of what I present,
but overall there is simply no explanation
in which all of these people would have been genuinely nice and open towards me,
no matter how clumsy my attempts may have been$^{*}$.}%
\de{Nach fast 20 Jahren
ist meiner Meinung nach die Reaktion der Astrologie insgesamt \textsl{unentschuldbar},
und individuell ebenfalls, zumindest von Alois Treindl, Liz Greene und Robert Hand,
während zumindest einigen anderen vielleicht einfach die Fähigkeit und der Kontext fehlten,
um vieles von dem zu begreifen, was ich darlege,
aber insgesamt gibt es einfach keine Erklärung,
in der all diese Leute wirklich nett und offen mir gegenüber gewesen wären,
egal wie ungeschickt meine Versuche gewesen sein mögen$^{*}$.}
%
\en{Accordingly, I do not feel like I owe astrology anything,
if anything it owes me.}%
\de{Dementsprechend habe ich nicht das Gefühl, dass ich der Astrologie irgendetwas schulde,
wenn schon schuldet sie mir etwas.}
%
\en{(But, of course, a change of mind there would still be welcome…)}%
\de{(Aber ein Sinneswandel wäre natürlich trotzdem noch willkommen…)}

\de{\vspace{-6mm}}
\begin{center}
\includegraphics[scale=0.474]{i-foxyfox.png}
\end{center}
\en{\vspace{-3mm}}%
\de{\vspace{-7mm}}

\en{Going in circles…}%
\de{Drehen in Kreisen…}
%
\en{It is just so sad that nothing took on,
nothing started to grow among astrologers.}%
\de{Es ist einfach so traurig, dass nichts in Gang gekommen ist,
nichts unter Astrologen zu wachsen begann.}
%
\en{In 2002 I really expected
that most likely Liz Greene and Alois Treindl
would invite me to Astrodienst
so that I could show them my model of the star signs
and that from then on it would take off, would start to grow,
they would tell others about it
and many people would start exploring,
so that relatively quickly
also some attention outside of astrology would emerge.}%
\de{Im Jahr 2002 hatte ich eigentlich erwartet gehabt,
dass mich Liz Greene und Alois Treindl zu Astrodienst einladen würden,
damit ich ihnen mein Modell der Sternzeichen zeigen könnte,
und dass es von da an abheben würde, anfangen würde zu wachsen,
sie würden anderen davon erzählen
und viele Leute würden anfangen, es zu erforschen,
so dass relativ schnell auch ausserhalb der Astrologie
eine gewisse Aufmerksamkeit entstehen würde.}
%
\en{But instead… see above.}%
\de{Aber stattdessen… siehe oben.}

\en{There comes a point, when people just tell you to do this or that,
each time you have invested some work to present something,
to invest even more work, that you start to think that they are wrong.}%
\de{Irgendwann kommt der Punkt, an dem man denkt,
wenn die Leute einem einfach sagen, man solle dies oder jenes tun,
jedes Mal, wenn man etwas Arbeit investiert hat, um etwas zu präsentieren, 
einem sagen, man solle noch mehr Arbeit investieren,
dass man anfängt zu denken, dass sie falsch liegen.}
%
\en{And likely they are.}%
\de{Und wahrscheinlich tun sie das auch.}
%
\en{Some additional work may really still make sense.}%
\de{Etwas zusätzliche Arbeit könnte tatsächlich noch sinnvoll sein.}
%
\en{But overall,
given the fears and discomforts my stuff seems to stir unconsciously in people,
less is more.}%
\de{Aber angesichts der Ängste und des Unbehagens,
das mein Material unbewusst in den Menschen zu wecken scheint,
ist weniger mehr.}
%
\en{So, again back to maybe one day collecting some of my posts,
or maybe not,
and maybe still one or two books related to astrology,
but maybe more likely not,
and rather one or two other things,
but maybe hardly more.}%
\de{Also, wieder zurück zu vielleicht eines Tages einige meiner Beiträge zu sammeln,
oder vielleicht nicht,
und vielleicht doch noch ein oder zwei Bücher im Zusammenhang mit Astrologie,
aber vielleicht eher nicht,
und eher ein oder zwei andere Dinge,
aber vielleicht kaum mehr als das.}
%
\en{Or maybe something different…}%
\de{Oder vielleicht etwas anderes…}

\en{A lot of all the things I did over the years are beautiful.}%
\de{Viele der Dinge, die ich im Laufe der Jahre getan habe, sind wunderschön.}
%
\en{I do not want to become famous,
just would have liked to see some of the things I found grow.}%
\de{Ich will nicht berühmt werden,
ich hätte nur gerne gesehen, wie einige der Dinge, die ich gefunden habe, wachsen.}
%
\en{Maybe the latter is too much to hope for,
maybe best to just enjoy the former,
things for what they are
instead of what people see in them today.}%
\de{Vielleicht ist letzteres zu viel erhofft,
vielleicht ist es am besten, sich einfach an ersterem zu erfreuen,
an den Dingen, wie sie sind,
anstatt an dem, was die Leute heute in ihnen sehen.}

\de{\vspace{-0.5mm}}
\begin{center}
\includegraphics[scale=0.474]{i-foxyfox.png}
\end{center}
\vspace{-3mm}

\en{Hey, today, 29 January 2022,
someone at the forum, Phoebe,
wrote that she is trying to “make headway”
with my “Atomic Model of the Zodiac”,
but also needs “more time to absorb and digest”,
which is more positive feedback than maybe ever regarding the model.}%
\de{Hey, heute, am 29.\ Januar 2022,
schrieb jemand im Forum, Phoebe,
dass sie versucht, mit meinem “Atomaren Modell des Tierkreises voranzukommen”,
aber auch “mehr Zeit zum Aufnehmen und Verdauen” braucht,
was ein positiverer Feedback ist
als vielleicht jemals zuvor in Bezug auf das Modell.}
%
\en{(As of late July, I have not heard anything since then,
unlikely that anything evolved there, like so many times before.)}%
\de{(Stand Ende Juli habe ich seither nichts mehr gehört,
unwahrscheinlich, dass sich da etwas entwickelt hat, wie so oft.)}

\en{Maybe just some more faith that things will just turn out well
and a bit more patience, are all that is needed.}%
\de{Vielleicht braucht es nur etwas mehr Vertrauen,
dass sich die Dinge schon zum Guten wenden werden,
und ein bisschen mehr Geduld.}
%
\en{Liz Greene told me “You have time.”
at the brief and only encounter with me in 1998,
just after a talk by her about Saturn
at which I had handed her my chart for interpretation during a break before.}%
\de{Liz Greene sagte mir
bei der kurzen und einzigen Begegnung mit mir im Jahr 1998:
“Du hast Zeit.” (bzw.\ auf Englisch “You have time.”),
kurz nach einem Vortrag von ihr über Saturn,
bei dem ich ihr in einer Pause mein Horoskop zur Deutung gegeben hatte.}
%
\en{My Saturn is at the end of Pisces,
plus retrograde, and rather slowly so.}%
\de{Mein Saturn steht am Ende der Fische,
dazu noch rückläufig, und das eher langsam.}
%
\en{Maybe simply my moon in Aries in the 10th house
and still at the MC (5° ahead of it) is too impatient;
not always being frustrated because of that might be healthier,
also for my digestion, maybe.}%
\de{Vielleicht ist einfach mein Mond in Widder im 10.\ Haus
und noch am MC (5° darüber hinaus) zu ungeduldig;
deswegen nicht immer frustriert zu sein, wäre gesünder,
auch für meine Verdauung, vielleicht.}

\en{Or this could again be something that leads practically nowhere,
just another illusion,
with a trine between Neptune and my north node currently
(and also still with my natal Neptune around there).}%
\de{Oder dies könnte wieder etwas sein, das praktisch nirgendwo hinführt,
nur eine weitere Illusion,
mit einem Trigon zwischen Neptun und meinem n.Mondkno\-ten derzeit
(und auch noch zu meinem Geburtsneptun in der Nähe).}
%
\en{Then again, Neptune is likely an important planet
also regarding public waves, Hollywood, so maybe…}%
\de{Andererseits ist Neptun wahrscheinlich auch ein wichtiger Planet
in Bezug auf öffentliche Wellen, Hollywood, also vielleicht…}

\begin{center}
\includegraphics[scale=0.474]{i-foxyfox.png}
\end{center}
\vspace{-3mm}

\en{Venus, also related I guess a lot to Hollywood,
is just about to go forward again today,
after having been retrograde since 19 Dec 2021,
which is incidentally also quite exactly the period
in which I first created the “noindex” part of exactphilosophy.net
with old webs, discoveries documents,
all my Usenet and I Ching posts,
as well as (earlier versions of) this text,
without the actual astro.com forum posts,
at least not yet for this Venus retrograde phase.}%
\de{Venus, die wohl auch viel mit Hollywood zu tun hat,
ist gerade dabei, wieder vorwärts zu gehen, 
nachdem sie seit dem 19.\ Dezember 2021 rückläufig war,
was übrigens auch ziemlich genau der Zeitraum ist,
in dem ich den “noindex”-Teil von exactphilosophy.net erstellt hatte,
mit alten Webseiten, Discovery Dokumenten,
all meinen Usenet-Posts und Posts zum I Ging,
sowie (früheren Versionen) dieses Textes,
noch ohne astro.com Forenbeiträge,
zumindest noch nicht für diese Venus-Rückläufigkeitsphase.}

\en{Feels like a good moment these past weeks until today,
also reminding of the second task of Psyche (“Venus/Pisces”),
Water, patience, waiting for the right moment.}%
\de{Fühlt sich wie ein guter Moment an in den letzten Wochen bis heute,
erinnert auch an die zweite Aufgabe der Psyche (“Venus/Fische”),
Wasser, Geduld, Warten auf den richtigen Moment.}
%
\en{And also again would seem to fit maybe well with Saturn
at its next return for me 14 May 2025
in the 10th house together with Ceres (“gentle growth”) close by,
plus north node, too, as well as,
already in early Aries Neptune (“Hollywood”).}%
\de{Und auch wieder würde das vielleicht gut passen
zu Saturn bei seiner nächsten Rückkehr für mich am 14.\ Mai 2025
im 10.\ Haus zusammen mit Ceres (“sanftes Wachstum”) in der Nähe,
plus auch n.Mondknoten, sowie,
schon im frühen Widder Neptun (“Hollywood”).}
%
\en{I just hope that things will turn out well regarding these things in the world
if they start to grow, that they would grow gently  and create more good than bad,
both initially and in the medium and long-term future.}%
\de{Ich hoffe einfach, dass sich die Dinge in der Welt gut entwickeln werden;
wenn sie zu wachsen beginnen, dass sie sanft wachsen und mehr Gutes als Schlechtes schaffen,
sowohl anfangs als auch in der mittel- und langfristigen Zukunft.}

\en{(Again in retrospect from June 2022,
more likely the third return will matter for a longer time,
but who knows what maybe also a short period could change,
if maybe I would get the book {\footnotesize\ELEMENTAL} out then,
will see…)}%
\de{(Wieder im Rückblick von Juni 2022,
wird eher die dritte Rückkehr für eine längere Zeit von Bedeutung sein,
aber wer weiss, was sich vielleicht auch kurzfristig ändern könnte,
wenn ich dann vielleicht das Buch {\footnotesize\ELEMENTAL} herausbringen würde,
würde man sehen…)}

\en{I just read that the daughter Nala
of famous German-speaking singer Helene Fischer
was born two months early (but is perfectly healthy and well now),
18 Dec 2021, hence just before Venus went retrograde (just 1° ahead of Pluto),
and this was announced in the media today,
when Venus will go direct again.}%
\de{Ich habe gerade gelesen, dass die Tochter Nala
der berühmten deutschsprachigen Sängerin Helene Fischer
zwei Monate zu früh geboren wurde (aber jetzt völlig gesund und wohlauf ist),
am 18.\ Dezember 2021, also kurz bevor Venus rückläufig wurde (nur 1° vor Pluto),
und dies wurde heute in den Medien angekündigt,
wo Venus wieder direktläufig wird.}

\en{Well, a day later, with Venus direct again,
hopes for anything by me “landing” in astrology
have once again been delusional dreams;
for all that it appears, astrologers are back to their “daily businesses”,
just doing what they fancy.}%
\de{Nun, einen Tag später, mit Venus wieder direkt,
sind die Hoffnungen, dass irgendetwas von mir in der Astrologie “landet”,
wieder einmal ein Wunschtraum gewesen;
denn so wie es aussieht, gehen die Astrologen wieder ihrem “Tagesgeschäft” nach
und machen einfach nur, wozu sie gerade Lust haben.}

\en{Maybe I will just do “as above, so below; as below, so above”\,?}%
\de{Vielleicht werde ich einfach “wie oben, so unten; wie unten, so oben” machen\,?}
%
\en{In the future only say something if there is specific interest into my contributions
and otherwise let things mostly rest\,?}%
\de{Künftig nur noch etwas sagen, wenn konkretes Interesse an meinen Beiträgen besteht,
und ansonsten die Dinge weitgehend ruhen lassen\,?}
%
\en{Or maybe just do something else, or more of the same\,?}%
\de{Oder vielleicht einfach etwas anderes machen, oder mehr vom Gleichen\,?}
%
\en{Que sera, sera…}%
\de{Que sera, sera…}
%
\en{Just so very sad for me this outcome after almost 20 years,
maybe just post for fun and ignore the fate of the “world”\,?}%
\de{Ich bin nur sehr traurig über diesen Ausgang nach fast 20 Jahren,
vielleicht einfach zum Spass posten und das Schicksal der “Welt” ignorieren\,?}
%
\en{Maybe just do what Odysseus did
in Kafka’s parable “Das Schweigen der Sirenen”
and once more trust the “Anhang” given there\,?}%
\de{Vielleicht einfach das tun,
was Odysseus in Kafkas Parabel “Das Schweigen der Sirenen” tat
und einmal mehr dem dort gegebenen “Anhang” vertrauen\,?}

\vspace{2mm}
\noindent
\small
\begin{otherlanguage}{ngerman}
\textsl{\color{xphi}
Es wird übrigens noch ein Anhang hierzu überliefert.
%
Odysseus,
sagt man,
war so listen\-reich,
war ein solcher \colorbox{saffronback}{\color{saffronfront}Fuchs},
dass selbst die Schicksalsgöttin nicht in sein Innerstes dringen konnte.
%
Vielleicht hat er,
obwohl das mit Menschenverstand nicht mehr zu be\-grei\-fen ist,
wirklich gemerkt,
dass die Sirenen schwiegen,
und hat ihnen und den Göttern
den obigen Scheinvorgang
nur gewissermassen als Schild entgegengehalten.}
\end{otherlanguage}
\de{\vspace{3mm}}

\en{\vspace{2mm}
\noindent
\small{\textsl{\color{xphi}
By the way,
a codicil to the foregoing has also been handed down.
%
Odysseus,
it is said,
was so full of guile,
was such a \colorbox{saffronback}{\color{saffronfront}fox},
that not even the goddess of fate
could pierce into his innermost core.
%
Perhaps he really noticed,
even though this is beyond human understanding,
that the Sirens remained silent,
and held up to them and to the gods
the aforementioned dummy activity merely sort of as a shield.}\par}
\vspace{3mm}}

\noindent
\en{Adliswil, 8 February 2022 at around 12:40 noon}%
\de{Adliswil, 8.\ Februar 2022 um etwa 12:40 mittags}

\vspace{3mm}
\noindent
\includegraphics[scale=0.5]{i-fox.png}
\vspace{1mm}

\noindent
\en{\small{(even looks a bit like a shield…)}}%
\de{\small{(sieht sogar ein bisschen wie ein Schild aus…)}}

\vspace{1.7mm}
\noindent
{\footnotesize\color{xphi}
\en{August 2022:
A sabian oracle this spring was
an old sea captain rocking (13° Leo),
reminding also of Odysseus’ choice+lot
at the end of Plato’s \textsl{Republic},
and maybe,
just maybe,
I will still tell you some stories
about some of the ‘islands’
I discovered on my virtual journeys…}%
\de{August 2022:
Ein sabisches Orakel diesen Frühling war
ein alter Seekapitän,
der schaukelt (13° Löwe),
was auch an Odysseus’ Wahl+Los
am Ende von Platons \textsl{Staat} erinnert,
und vielleicht,
nur vielleicht,
werde ich euch noch ein paar Geschichten erzählen
über einige der ‘Inseln’,
dich ich auf meinen virtuellen Reisen entdeckt habe…}

\en{To be fair,
all I ever wanted is to be with her,
just have her around me in everyday life,
live together,
found a family,
the usual,
this is why I tried astrology,
because she had made a presentation about astrology in school,
the rest was not so important,
including this here.}%
\de{To be fair,
all I ever wanted is to be with her,
just have her around me in everyday life,
live together,
found a family,
the usual,
this is why I tried astrology,
because she had made a presentation about astrology in school,
the rest was not so important,
including this here.}
\par}

\vspace{1.7mm}
\noindent
{\footnotesize\color{xphi}
\en{* Epilogue (March 2023):
Around the closedown of the forum at astro.com,
I realized via some closer contact
that the proponents from the generation with Pluto in Leo
would really have heavy deficits when it comes to perception,
some kind of inability
to perceive things outside of the own predator focus.}%
\de{* Epilog (März 2023):
Rund um die Schliessung des Forums auf astro.com,
wurde mir durch näheren Kontakt klar,
dass die Verfechter aus der Generation mit Pluto in Löwe
wirklich grosse Defizite hätten,
was die Wahrnehmung angeht,
eine Art Unfähigkeit,
Dinge ausserhalb des eigenen Raubtierfokus zu wahrzunehmen.}
%
\en{Seemed only fair to point this out for once relatively directly.}%
\de{Es erschien mir nur fair, dies für einmal relativ direkt hervorzuheben.}
%
\en{And still in the end almost infinitely sad,
considering also that the people who continue it are younger.}%
\de{Und doch ist es am Ende fast unendlich traurig,
auch da diejenigen, die es weiterführen, jünger sind.}
\par}

\vspace{1.7mm}
\noindent
{\footnotesize\color{xphi}
\en{* December 2023:
In the end, they were and are still primarily responsible, not me.}%
\de{* Dezember 2023:
Am Ende waren und sind doch sie primär verantwortlich, nicht ich.}
\par}
