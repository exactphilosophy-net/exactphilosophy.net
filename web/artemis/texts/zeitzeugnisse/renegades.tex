\documentclass[letterpaper]{article}
\pagestyle{empty}
\paperheight=3700mm
\textheight=3700mm
%\textwidth set depending on pdflatex or (experimentally)lualatex
\topmargin=-20mm
\oddsidemargin=25mm

\usepackage[utf8]{inputenc}
\usepackage[spanish,italian,french,ngerman,english]{babel} % last is main
\usepackage{graphicx}
\usepackage{multirow}
\usepackage{xcolor}
\usepackage{contour}
\usepackage{pict2e}
\usepackage{relsize}
\usepackage{amsmath}

\usepackage{iftex}
\ifpdftex
  % stronger fonts

% Find modes.mf, e.g. /usr/local/texlive/2025/texmf-dist/fonts/source/public/modes/modes.mf
%
% $ sudo cp modes.mf modes.mf.orig
%
% Add the following at the start of modes:
%
% mode_def xphi =
%   mode_param (pixels_per_inch, 1200);
%   mode_param (blacker, 1.9); % only difference to 'lexmarkr' (2 there)
%   mode_param (fillin, 0);
%   mode_param (o_correction, 1);
%   mode_common_setup_;
% enddef;
%
% Finally:
%
% $ sudo fmtutil-sys --byfmt mf

\pdfpkresolution=1200
\pdfpkmode={xphi}
\pdfmapfile{}

  \newcommand{\coretextwidth}{85.5mm}
\fi

% experimental, not used to produce the live website...
\ifluatex
  % about same heaviness in pdfs when rasterized in photoshop,
  % but since, unlike the metafont mechanisms I use, fake bold, "bleeds" in all directions,
  % seems heavier at least in core web page images
  \newcommand{\fontbleed}{0.8}
  % paragraphs wider and font looks larger, tried to fix, but then other things change a bit,
  % especially for section headings would have to change back, are now more narrow...
  \newcommand{\fontscale}{0.985}
  \newcommand{\coretextwidth}{85.2mm}
  \usepackage{fontspec}
  % microtype does maybe help and not help, maybe if would allow wider spaces...
  \usepackage{microtype}
  \setsansfont{Latin Modern Sans}[Scale=\fontscale, FakeBold=\fontbleed]
  \setmonofont{Latin Modern Mono}[Scale=\fontscale, FakeBold=\fontbleed]
  % this would be for "new computer modern" (but has many limitations so far)
  %\usepackage[default]{fontsetup}
  %\renewcommand{\familydefault}{\sfdefault}
\fi

\renewcommand{\familydefault}{\sfdefault}

\setcounter{secnumdepth}{-1}

\newcommand{\en}[1]{\iflanguage{english}{#1}{}}
\newcommand{\de}[1]{\iflanguage{ngerman}{#1}{}}
\newcommand{\fr}[1]{\iflanguage{french}{#1}{}}

% a bit less than ~255/256
\definecolor{almostwhite}{gray}{0.996}
\definecolor{xphi}{rgb}{0.0,0.5,0.5}
\definecolor{avant}{rgb}{1,0.5,0.5}

\definecolor{frame}{gray}{0.9}
\definecolor{lightgray}{gray}{0.8}
\definecolor{gray}{gray}{0.5}
\definecolor{darkgray}{gray}{0.3}

\definecolor{darkred}{rgb}{0.8,0.0,0.0}
\definecolor{darkyellow}{rgb}{0.7,0.7,0.0}
\definecolor{darkgreen}{rgb}{0.0,0.55,0.0}
\definecolor{darkviolet}{rgb}{0.5,0,0.5}

\definecolor{darkblue}{rgb}{0,0,0.7}
\definecolor{odyssey}{rgb}{0,0,0.8}
\definecolor{indigo}{rgb}{0.29,0,0.51}
\definecolor{indigoblue}{rgb}{0.1,0,0.6}

\definecolor{saffronback}{rgb}{1.000,0.878,0.627}
\definecolor{saffronfront}{rgb}{0.376,0.125,0.000}

\DeclareRobustCommand{\cometartemisscale}[1]{\includegraphics[scale=#1]{\sourcepath/i-comet.jpg}\hspace{-0.028453em} artemis}
\newcommand\cometartemis{\cometartemisscale{0.018}}
\newcommand\cometartemissection{\cometartemisscale{0.0225}}

\DeclareRobustCommand{\moebius}[1]{\includegraphics[scale=#1]{\sourcepath/i-moebius.jpg}}
\newcommand{\yinyang}{\includegraphics[scale=0.135]{\sourcepath/i-yinyang.jpg}}

\newcommand{\rarr}{\,$\rightarrow$\,}
\newcommand{\lrarr}{\,$\leftrightarrow$\,}

% greek elements
\newcommand{\elfire}{
\begin{picture}(9,6)
  \thicklines
  \put(1,-0.5){\line(1,0){7}}
  \put(1,-0.5){\line(1,1.732){3.5}}
  \put(8,-0.5){\line(-1,1.732){3.5}}
\end{picture}}
%
\newcommand{\elair}{
\begin{picture}(9,6)
  \thicklines
  \put(1,-0.5){\line(1,0){7}}
  \put(1,-0.5){\line(1,1.732){3.5}}
  \put(8,-0.5){\line(-1,1.732){3.5}}
  \put(2.75,1.9){\line(1,0){3.5}}
\end{picture}}
%
\newcommand{\elwater}{
\begin{picture}(9,6)
  \thicklines
  \put(1,5){\line(1,0){7}}
  \put(1,5){\line(1,-1.732){3.5}}
  \put(8,5){\line(-1,-1.732){3.5}}
\end{picture}}
%
\newcommand{\elearth}{
\begin{picture}(9,6)
  \thicklines
  \put(1,5){\line(1,0){7}}
  \put(1,5){\line(1,-1.732){3.5}}
  \put(8,5){\line(-1,-1.732){3.5}}
  \put(2.75,2.5){\line(1,0){3.5}}
\end{picture}}
%
\newcommand{\elhex}{
\begin{picture}(9,6)
  \thicklines
  \put(1,0.5){\line(1,0){7}}
  \put(1,0.5){\line(1,1.732){3.5}}
  \put(8,0.5){\line(-1,1.732){3.5}}
  \put(1,5){\line(1,0){7}}
  \put(1,5){\line(1,-1.732){3.5}}
  \put(8,5){\line(-1,-1.732){3.5}}
\end{picture}}

% i ching trigrams
\newcommand{\trigram}[3]{
\begin{picture}(9,6)
  \linethickness{0.36mm}
  \put(0,5){\line(1,0){#1}}
  \put(5.5,5){\line(1,0){3.5}}
  \put(0,2.5){\line(1,0){#2}}
  \put(5.5,2.5){\line(1,0){3.5}}
  \put(0,0){\line(1,0){#3}}
  \put(5.5,0){\line(1,0){3.5}}
\end{picture}}
\newcommand{\triheaven}{\trigram{5.5}{5.5}{5.5}}
\newcommand{\triearth}{\trigram{3.5}{3.5}{3.5}}
\newcommand{\trithunder}{\trigram{3.5}{3.5}{5.5}}
\newcommand{\triwater}{\trigram{3.5}{5.5}{3.5}}
\newcommand{\trimountain}{\trigram{5.5}{3.5}{3.5}}
\newcommand{\triwind}{\trigram{5.5}{5.5}{3.5}}
\newcommand{\trifire}{\trigram{5.5}{3.5}{5.5}}
\newcommand{\trilake}{\trigram{3.5}{5.5}{5.5}}

% i ching hexagrams
% 1+2 trigrams, 3 rest of line (see e.g. dreams.tex)
\DeclareRobustCommand{\hexagram}[3]{\raisebox{-3pt}{$\overset{\text{${#1}$}}{#2}$\,}#3\vspace{3pt}}

% white-red-black etc.
\DeclareRobustCommand{\outline}[1]{\contour{black}{{\color{white}#1}}}
\DeclareRobustCommand{\white}[1]{\outline{\textbf{#1}}}
\DeclareRobustCommand{\red}[1]{{\color{darkred}\textbf{#1}}}
\DeclareRobustCommand{\black}[1]{\textbf{#1}}
\DeclareRobustCommand{\yellow}[1]{{\color{darkyellow}\textbf{#1}}}
\DeclareRobustCommand{\green}[1]{{\color{darkgreen}\textbf{#1}}}
\DeclareRobustCommand{\violet}[1]{{\color{darkviolet}\textbf{#1}}}
\DeclareRobustCommand{\indigoblue}[1]{{\color{indigoblue}\textbf{#1}}}
\DeclareRobustCommand{\indigo}[1]{{\color{indigo}\textbf{#1}}}

% ELEMENTAL
\newcommand{\ELEMENTAL}{%
\colorlet{contour}{.}\textbf{\color{white}%
\raisebox{+0.001em}{\contour{contour}{E}}%
\raisebox{+0.015em}{\contour{contour}{L}}%
\raisebox{+0.016em}{\contour{contour}{E}}%
\raisebox{+0.023em}{\contour{contour}{M}}%
\raisebox{+0.023em}{\contour{contour}{E}}%
\raisebox{+0.017em}{\contour{contour}{N}}%
\raisebox{-0.020em}{\contour{contour}{T}}%
\raisebox{-0.002em}{\contour{contour}{A}}%
\raisebox{+0.006em}{\contour{contour}{L}}%
}}

% artemis pdf+web icons
\newcommand{\ipdfen}{\includegraphics[scale=0.5]{i-pdf-en.png}}
\newcommand{\ipdfde}{\includegraphics[scale=0.5]{i-pdf-de.png}}
\newcommand{\ipdffr}{\includegraphics[scale=0.5]{i-pdf-fr.png}}
\newcommand{\iweb}{\includegraphics[scale=0.055]{i-web.png}}
\newcommand{\ipdfblueen}{\includegraphics[scale=0.5]{i-pdf-blue-en.png}}
\newcommand{\ipdfbluede}{\includegraphics[scale=0.5]{i-pdf-blue-de.png}}
\newcommand{\ipdfbluefr}{\includegraphics[scale=0.5]{i-pdf-blue-fr.png}}
\newcommand{\iwebblue}{\includegraphics[scale=0.055]{i-web-blue.png}}


\textwidth=\coretextwidth



\begin{document}

\avantgarde

\section{Renegades}

Between May and 10 September 2025
many ruminations grew in the $\!$\textsl{Timeline} document,
spawning twelve pages from 3 to 14.
%
They are preserved here,
with identical page numbers
and the end of page 2 for context.
%
Plus
I have added a section “Prague Fringe”
on page 15
at the end of summer,
a series of Sabian Symbols oracles,
and then even a few more pages
of quite untamed ruminations
with I guess still interesting elements
and a conclusion likely only afterwards,
in the following lead of the ‘dreams’ section,
thus more of a dream$\,$…

\color{odyssey}

\vspace{3mm}
\hspace{5mm}\begin{minipage}[t]{85.5mm}

\footnotesize
\begin{list}{$\circ$}{\setlength{\leftmargin}{10pt}}

\item
Since 2025 I dream of close to completely recovering the way
so that I could also convey my other findings going that way,
more naturally and with less resistance in people.

\end{list}

\end{minipage}

\color{black}

\normalsize

\vspace{4mm}
\noindent
I guess the elements are still central to recovery and more, as follows,

Even though the four elements in psychological astrology
seem not to go in circles at first sight,
in my model they do also in the zodiac
for each triple of the same element.
%
Earth would be about realism, hence tied to an outer world;
air and water would be about thinking and feeling, which happens in the mind;
not certain about fire, but maybe because would be about imagination,
and images with light and warmth of the sun and fire would be outside\,?

Could be a start,
and thus a focus on “way” might not be the best way to convey things,
better this website or ‘\white{Space} \red{Time} \black{Elements}’
or even more to the point on \ELEMENTAL.
%
$\pi$ would have been right once more.

In any case,
returning also to the zodiac and astrology
might provide crucial clues to move on;
I am much more hopeful now.

An exactphilosophy.net series of pocket books
does have quite a bit of appeal.
%
Would be preserved, more accessible,
could be presented and referenced,
would make it easier to approach people with it.
%
In the sense of the image in exactphilosophy
of a magic carpet settling down on “reality” or something like that,
also the pocket books would in style
best be introductions to new ideas.
%
The series could include anything except \ELEMENTAL.

All in all,
this would not be a dramatic change,
keep doing research,
pocket books instead of articles,
and at least at first more to myself.

But I also want to keep dreaming, including of\, \includegraphics[scale=0.070]{i-wayfox.jpg}

10 October 2025 {\smaller(10/10/25, book published 5/5/25)}.

\small
\begin{itemize}

\newpage

\vspace*{164mm}

\item[2024]
Publication of the book \textsl{exactphilosophy.net} (edition 2024).
%
January.

Conscious insight that my approach around the core idea
is generally via a \textbf{\color{xphi}contemplating self},
for which outside things tend to rest and making them move tends to require energy,
while inside things tend to move and making them rest tends to require energy,
whereas for a self that gets drawn into a chain of thought
that chain of thought would usually appear both moving and active.
%
October.

Via ‘contemplatio’ and ‘theōría’,
Plato’s cave seems to be more closely related
to my perspective on active/passive inside
than expected
and maybe the best way to start telling the world about my findings.
%
I guess,
I might cultivate the “three mounds on my forehead”:
x$\varphi$,
the book twins ‘\white{Space} \red{Time} \black{Elements}’\,/ ‘\white{Raum} \red{Zei} \black{Elemente}’
and of course the one I keep dreaming of.
%
29 October.

Sleeping Beauty Dreaming of \ELEMENTAL\,…

\item[2025]
{\color{darkgray}%
Core idea most likely unrecoverably gone, no future.
%
Maybe a little bit of future for smaller ideas on the side.
%
But they are already published.}

Publication of the book \textsl{exactphilosophy.net} (final edition 2025).
%
May.

\newpage

Turning the page on \textsl{exactphilosophy.net} (final edition 2025),
I doubt that the core idea would be “unrecoverably gone”.
%
I rather see this potential “flaw”
in an otherwise at least qualitatively almost all-encompassing model of life and the world
as a valuable asset to make it interesting for others,
a “riddle” to be solved
or maybe even to remain puzzling for a long time.
%
Around noon on 23 May I got 3 copies of \ELEMENTAL\
with just about $\pi$ pages of text and this time,
even though even there some changes seem necessary,
I could really feel it,
could feel that this book can grow now
and is in many ways the successor to the past works at exactphilosophy.net.
%
But for all that it appears,
the book pair ‘\white{Space} \red{Time} \black{Elements}’\,/ ‘\white{Raum} \red{Zei} \black{Elemente}’
will also just evolve,
sort of reformulated as a first glimpse at a new idea,
hence no need to be perfect,
just expose the basic ideas;
I noticed that evolutions around these two books
usually had something to do with Aquarius,
which is probably good for promotion
at least within a group of people who read such textbooks,
compared to \ELEMENTAL,
which has potential for a much larger audience.
%
And, last but not least,
I will write a series of books titled ‘Reports for Future Academies’,
inspired  by the Matt Helm movies of the 1960s and Kafka,
with topics around ideas from exactphilosophy.net and forum/Usenet posts;
each report maybe at most 20 pages or something like that,
primarily addressing the future academies,
but also with concessions for contemporary readers with a free mind.
%
24 May.

PS: How \ELEMENTAL\ has evolved today
even reminds a lot of the oracles mentioned in the article ‘Mysteries of life’.

Maybe my observation at the end of the leads of the ‘metamorphosis’ section
(as of the book of 2025)
would really be the key to recover my core idea,
namely that I myself have no planets in air signs in my birth chart,
hence lack some calm or poise inside,
even despite my IC in the middle of Libra…
%
5 June.

\includegraphics[scale=0.02]{i-foxyfox.jpg}

A day later,
I see things again more differentiated.
%
In the cycle of elements inspired by their physical properties,
water flows down, fire rises up, air rests high up, earth rests low down.
%
Water and earth would be heavy and passive,
hence water flowing down would match that.
%
Whether light fire and air would thus be active when moving up or resting up,
is a bit more doubtful,
but provides a nice symmetry
that seems often in line with astrology and similar cultural systems.

However,
whether eri defined as what rests inside,
would be active,
while emi defined as what moves inside would be passive,
is somewhat of a different question.
%
What would a rational (air) perspective on that say\,?
%
Well,
one could argue that elements would exist both outside and inside the mind for a start.

But going back to the cycle of elements:
Focus inside is needed to change something outside
and since outside most things tend to rest or come to rest,
focus inside would usually create movement outside,
i.e.\ a transition from eri to emo.
%
And so on.
%
Hence in the end,
not all is lost,
rather to the contrary.

I think this just needs time for now,
probably some years of mostly non-public contemplation,
while still writing a few books on the side,
as listed further above.
%
And never say never.

\includegraphics[scale=0.02]{i-foxyfox.jpg}

The elements with their natural properties or also views based on Aristotle,
as well as elements as defined by me here,
do quite naturally transform in circles.
%
But the elements as currently used in (psychological) astrology do not.
%
A thought (air) can trigger to imagine (fire) something,
to consider how realistic (earth) it is and how it feels (water),
and also the other ways round.
%
That is not a circle,
that is a mutual interconnection between all four.

Hence maybe splitting up eri into memory (earth),
logical/abstract thoughts (air) and free imagination (fire)
would provide a closer match to the psychological view on elements,
especially since they would by definition be inside the mind.
%
And probably relate eri to what is conscious,
since the above three states of eri can be recalled at will,
and emi mainly to what is unconscious.
%
Other than memories, thoughts that follow some logic and free imaginations,
feelings cannot be directly analyzed,
because they do not directly reveal their inner (unconscious) workings.
%
Well, this seems to fit already quite well together.
%
9 June.

\includegraphics[scale=0.02]{i-foxyfox.jpg}

There is one thing about the idea of internal states that is maybe a flaw:
earth-air-fire (memory-logic-imagination) would not be part of Aristotle’s circle of elements
(earth-air flips cold+dry to hot+wet).

And regarding cycles,
there seems to be a cycle ero-emo-eri-emi-ero,
a cycle that goes in and out,
maybe more important than one inside the mind,
and, yes, outside with “physical” elements there seems to be a cycle.

Well,
all in all,
maybe really just publish with all the flaws mentioned
as I first brought up 24 May,
the day before Saturn entered Aries for the first time
in the lifetime of exactphilosophy.net:
Maybe I can sort it out later,
maybe someone else can find something useful in it later or even very much later.

And, yes, as Cynthia stresses in \ELEMENTAL,
only if and as long as you focus,
things stay put inside,
making the core idea worth sharing as it is.
%
14 June.

\includegraphics[scale=0.02]{i-foxyfox.jpg}

I think at the moment doing my research more privately and also more leisurely than in the years since 2016 is good,
because new ideas and perspectives often only come in time,
while writing a little more towards specific publications,
also rather leisurely,
would seem to be more healthy for me personally:
visible and tangible output instead of nothing,
but never say never, in all aspects.

But what I have also been observing in the past few weeks
with Neptune and especially Saturn in a new cycle,
is that people seem to at least try to drop previous commitments or relations,
also to me,
thus maybe I could also be happy now
by doing nothing about many or all the things
that emerged since the turn of the millennium at exactphilosophy.net,
and could maybe do other things,
maybe not so fancy ones,
more like Odysseus in Plato’s Republic.

Well, on verra, never say never, relax\,…
%
14 June.

\includegraphics[scale=0.02]{i-foxyfox.jpg}

I guess I will keep contemplating the core idea.
%
A contemplating self is maybe the key.
%
It would not be drawn into a tunnel of thoughts that chase each other,
would rather see it happen,
itself rather effortlessly,
while focussing on something to keep it put would require energy.

Looking back,
so far not all that much new and significant since the 2025 book.

That psychological elements seem rather not to go naturally in a circle
would have maybe been worth a lead.

The basic idea of just writing about the core idea
with mention of potential or even likely flaws
certainly has some appeal,
but: Really do that or still wait for maybe a new idea\,?
%
But maybe at least the book pair that is now explicitly
intended to be just a first glimpse at a new way of seeing the world,
why not just do that there\,?
%
And the reports for future academies,
why not just do some of them\,?
%
Going in circles,
better for me personally to gently, leisurely, but still quite regularly
to write down some stuff in books and then publish them
(no promotion or marketing needed),
and to keep ruminating on the core idea,
maybe some new perspectives will evolve in time,
actually maybe only via the process of writing the books
or even from feedback of readers,
or, of course, they themselves eventually carrying on
and me maybe just leaning back and watching things grow,
as I always dreamed of\,?

I am also thinking back of Petrarca’s account of how he climbed Mont Ventoux
with his brother and two servants,
an event that marked the beginning of the Renaissance.
%
There Petrarca recounts
that he was often first trying to walk around the mountain
to find a good way up,
while his brother would often just go up,
also since in the end
you have to go all the way to the top anyways.
%
To me it seems that it would still be a pity
if my findings would only be published
in forms that seem rather hard to access,
instead of at least having given it some effort
to make them more accessible.
%
15 June.

\includegraphics[scale=0.02]{i-foxyfox.jpg}

I made sort of a Gedankenexperiment today.
%
I ‘asked’ my Saturn of after its latest return about a month ago about this
(this time around Saturn had returned in the 10th house of also public status and career).
%
The ‘answer’ was rather astonishing to me:
{\color{xphi}No need to write/publish more;
things would already be sufficiently published;
and also no need to continue to research it.}
%
I occasionally made similar ‘queries’ in the past,
but their results were often along the lines of something
I would have expected to come from ‘myself’.
%
This seems different.
%
Would be nice if I could just essentially let this here be
and do completely other things,
I guess maybe like Odysseus rather than like Petrarca.
%
Time will tell,
but maybe I just got lucky…
%
And I guess I would still be free to write and publish,
if fun to me,
maybe most fun the reports for future academies,
with monographs rather more leisurely on the side,
but I am again going in circles.
%
15+16 June.

\includegraphics[scale=0.02]{i-foxyfox.jpg}

Space seems to have 3 dimensions, time only one.
%
Space seems to be static in a way (not what it contains), while time seems to move in a way.
%
Is that a mere coincidence or maybe related to the core idea in a fundamental way\,?

Of course, in general relativity space and time are bent by masses
and similarly there are more than three states of matter,
but in both cases this is largely not part of everyday life,
especially was not in antiquity.

Can you really observe your own thinking\,?
%
Can you contemplate your own inner self while it does what it usually does,
often sort of “following the white rabbit”,
going from clue to clue\,?
I guess this is not usually done,
but at least for short moments it can be done,
and maybe for longer time spans with some practice,
and you can, of course, contemplate in retrospect.
%
17 June.

\includegraphics[scale=0.02]{i-foxyfox.jpg}

I guess I made the same mistake as in 2019 again in 2024/2025,
namely to sort of model an observing, contemplating self as separate of what happens outside and inside.
%
There is just a single experience of living.
%
Things are happening in what appears to be an outside and an inside.
%
Inside things generally only rest as long as you focus,
and where you focus next is not something you really control.
You can sort of consciously imagine things that move inside,
but at least in my experience that is then rather a sequence of images if you look more closely.

In other words,
at its core,
the core hypothesis of this site would appear to be still be as valid
as I thought it was since 2004.

But would it still be as beautiful and natural as before sometime in 2024\,?
%
Would I want to “refactor” the core site once more,
and not be afraid to lose some essential and very beautiful things in it,
which I maybe even forgot about\,?

Thus only continue if/when I am sure,
otherwise,
which seems more likely,
just let the site be,
except maybe tiniest rather formal fixes,
and leisurely do some of the other things mentioned,
if/when it feels like it,
and otherwise do other things,
or just enjoy life…?

But hey, exactphilosophy.net is my lifeblood,
or at least apparently more so than most other related projects,
so why not take the risk to try again;
worst case there is still the 2025 book…

Well, in a nutshell,
to me beauty is important;
if the idea does regrow to be as beautiful as it used to be to me,
I will probably go all the way in with exactphilosophy.net,
but otherwise maybe really rather do other things…

But there is maybe one certainty:
At this time it seems wiser to proceed slowly,
to take several years time
with the core idea.
%
Other things on the side may grow.

18 June.

PS: Die Zauberperle bei Dschuang Dsi in der Übersetzung von Richard Wilhelm.
%
If you lost something very dear,
the only way to find it again,
would be to do nothing,
to forget about yourself.

18 June around 21:50.

\includegraphics[scale=0.02]{i-foxyfox.jpg}

What if I would write
‘\white{Space} \red{Time} \black{Elements}’\,/ ‘\white{Raum} \red{Zei} \black{Elemente}’
in a very modest way,
just describe and illustrate my thoughts,
including all that is not so sure\,?
%
Seems like a good idea, also a very Swiss approach.
%
Does not exclude doing other things,
writing other books,
but that in itself would be a nice and presentable contribution
to the world.

\vspace{-0.15em}
What if I would just write some \includegraphics[scale=0.1]{i-reports.png}
whenever it feels to me like it,
or at least make a compilation of titles and topics,
as well as technically prepare the format\,?
%
Why not,
just be careful to maybe generally favor the above twin books.
%
These reports have the advantage of being somewhat tailored for the “US approach”,
would allow to spread ideas quasi-continuously in social media,
hence maybe create some attention.
%
But maybe write several or even many in advance,
but do not publish them, yet.
%
Something along those lines,
progressing very gently at first,
would seem also to be fun.

What about just continuing with \ELEMENTAL\
in the same way as since 2019,
which is extremely slowly,
goes back and forth,
but also does not take much of my time\,?
%
Maybe,
in a way just wait for things maybe to unfold there.
%
That book is the only one with immediate “Hollywood” qualities,
with a chance to reach almost everyone,
hence worth pursuing.

And, yes,
hopefully in time maybe recover the core idea in at least almost its original beauty,
and then go back to \textbf{\color{xphi}exactphilosophy.net}
and maybe write that down in the style of the website,
or maybe just dream of it…

Sounds like a beautiful, realistic, fruitful and leisurely plan.
%
19 June.

\includegraphics[scale=0.02]{i-foxyfox.jpg}

The plan has admittedly at lot that speaks for it.
%
Not sure if I will be able to motivate myself to do much according to it,
as long as the core idea is not relatively certainly again something really fundamental,
but maybe that will come in time.
%
Presumably the dialogue in \ELEMENTAL\ could prove to be the best way to get closer to the core again.
%
19 June at about 22:20.

\includegraphics[scale=0.02]{i-foxyfox.jpg}
\vspace{-0.01mm}

But doesn’t being modest with the twin books
mirror Zhuangzi’s forgetting about oneself in order to retrieve the magic pearl again,
hence maybe really just do it that way\,?
%
Similarly with the reports,
just write them for anyone who would read them
with a mind largely free from (at least contemporary) chains and dogmas\,?
%
Only with \ELEMENTAL\ really try to push the envelope.
%
And finally on the website judge maybe every year
if updates or even rearrangements make sense\,?

I guess that way,
if all done leisurely,
the plan would be viable and usually be fun to do,
as well as have rewarding results in at least the first two segments.

For the twin books,
I could start by rewriting blurb and preface in a more modest way;
for the reports,
I could start by collecting titles plus short sketches of what they would contain;
with \ELEMENTAL\ I could just continue;
for the website,
I could maybe collect new findings from time to time,
maybe even from here,
like that psychological elements do apparently not go naturally in a circle,
and when it makes sense adapt the website,
also keeping in mind that it will be an appendix in the twin books,
in English resp.\ German.

In time,
there would also be lots of material emerging
that could be promoted,
at least more easily than what is present now,
especially make things more approachable to people only open to some content
that stays within fairly common boundaries.
%
That would be sustainable progress,
something I could be proud of,
and maybe from some point on just watch evolve by itself.

Sounds still at least as good to me as at the end of the previous page.

20 June at 08:09.

\includegraphics[scale=0.02]{i-foxyfox.jpg}
\vspace{-0.01mm}

Once more:

‘\white{Space} \red{Time} \black{Elements}’\,/ ‘\white{Raum} \red{Zei} \black{Elemente}’ very modestly and leisurely, and include the core website as appendix.

\includegraphics[scale=0.1]{i-reports.png} very surrealistically and leisurely, start by defining titles and sketches of the content, plus general design.

\ELEMENTAL\ push the envelope, again leisurely.

\textbf{\color{xphi}exactphilosophy.net} seldomly update gently and leisurely, usually just maybe a few leads or tiny fixes in the core content, larger changes to the core only if a real breakthrough.

And do other things that are fun, enjoy life.

Note that the previous section for 08:09\,–\,when I decided to follow the plan for good\,–\,had the AC in the first degree of Leo (where my Mercury is and close to my AC), as well as MC and moon very similarly to at my own birth…

Maybe back on the next page to the ‘normal’ timeline.

20 June.

\includegraphics[scale=0.02]{i-foxyfox.jpg}
\vspace{-0.01mm}

Not sure if the core idea in its current state is worth such big plans, yet;
Wei Chi and \ELEMENTAL\ until worth bothering others (and me) with this.
%
21 June.

% note: Wei Chi means that can leisurely pursue the projects,
% but better be reluctant to publish, except probably some reports

\newpage

First of all, I still think that the plan is a good one.

However, for the time being, there are some restrictions, as follows.

The book \textsl{exactphilosophy.net} (final edition 2025)
is clearly labeled as final and as intended as a reference.
%
In that sense,
starting again to modify the website \textbf{\color{xphi}exactphilosophy.net} arbitrarily,
even if only with small changes,
seems not ideal,
even knowing that there are hardly any visitors at the moment.
%
Hence keep new findings essentially private
or maybe at some point publish on the website in a pdf like this one.
%
Should there be a major new discovery regarding the core idea,
including recovery of its original beauty,
things would, of course, be somewhat different,
but even then would have to consider how exactly to handle.
%
All in all,
the website will essentially remain unchanged for the time being,
except for what I am writing here
(which may find an end in the not so far future,
or maybe not,
does not really matter all that much).

This has some consequences for the twin books
‘\white{Space} \red{Time} \black{Elements}’\,/ ‘\white{Raum} \red{Zei} \black{Elemente}’,
since the idea there was to include the core website as an appendix.
%
That would have to be coordinated, too.
%
But there is time.
%
First write the book leisurely,
then worry about the appendix.

The \includegraphics[scale=0.1]{i-reports.png} can be written freely,
they are not so directly tied to the core idea.
%
More precisely,
some more important ones might be,
could fit very well into that overall idea,
but is also not a necessity.

And \ELEMENTAL, well, just continue,
and if it comes to a conclusion,
it can also be published and promoted,
in that sense it is much more independent of previous work,
would be a dream come true.

And finally,
I am free to more philosophical thoughts,
and writing some things might also lead to a few new insights,
or maybe not.
%
Maybe my first Saturn cycle was about learning,
the second about both learning and already publishing,
and now the third maybe mostly
about conveying (or even teaching) my findings.

And there will probably other minor adjustments to the plan,
rather to confirm the plan than contradict it,
it would seem.

Finally,
as of today \textsl{exactphilosophy.net} (final edition 2025)
is available also at various bookstores besides the big online stores,
maybe already since a few days,
but hardly earlier than that.

Sounds like a beautiful, realistic, fruitful and leisurely plan.

21 June.

\includegraphics[scale=0.02]{i-foxyfox.jpg}

What about Artemis/Hecate and the moon, and mulberries, and so on, the whole theater-moon story\,?
%
Just one of the reports and a footnote around elements, the three colors\,?
%
Sorry, but maybe really best to let things be for the moment,
except maybe \ELEMENTAL\ and generally contemplating things\,?

Sort of, but I will probably still do some things related to the other parts,
but likely not very much.

Hm, just do nothing and fail to completely achieve that goal\,?

Maybe the theater-moon story would be the key to recover the core idea\,?

21 June.

\includegraphics[scale=0.02]{i-foxyfox.jpg}

But the moon stuff is an integral part of \ELEMENTAL,
and a report even if one of many would shine exactly because of that,
and, yes, maybe key to recovery;
anyways, I guess the plan survived once more.
%
Still 21 June.

The following plan emerged over the past six pages:

\begin{list}{$\bullet$}{\setlength{\leftmargin}{10pt}}

\item
Twin books ‘\white{Space} \red{Time} \black{Elements}’\,/ ‘\white{Raum} \red{Zei} \black{Elemente}’ very modestly and leisurely, and include the core website as appendix.

\item
\includegraphics[scale=0.1]{i-reports.png} series of short books, very surrealistically and leisurely, create a beautiful but playful design and style early on.

\item
\ELEMENTAL\ push the envelope, again leisurely.

\item
\textbf{\color{xphi}exactphilosophy.net} gently preserve, avoid any changes, at most collect them and maybe integrate them as a separate article, unless there is a really breakthrough, but even then remember that the book \textsl{exactphilosophy.net} (final edition 2025) is explicitly intended as a non-moving reference.

\item
And do other things that are fun, enjoy life.

\item
More details on the past 6 pages and all over the place\,…

\item
Decided to follow the plan 20 June 2025 at 08:09 in the morning in Adliswil.

\end{list}

Note that 6 is Artemis’ number. 22 June.

And worry less, be like water, forget about yourself a bit…

\includegraphics[scale=0.02]{i-foxyfox.jpg}

Lead about spring point and 1+3 elements inside and outside,
inspired by Bukowski’s Red Sparrow.
%
A lead that may be crucial in my feeling.
%
Idea ca.\ 23 July.

You may say that the human condition was something of the 20th century,
no longer relevant in current times.
%
But so far all living beings composed of multiple cells still die,
including humans,
including me
and probably also you who are reading this.
%
\ELEMENTAL\ is closest to that theme of all the projects above,
as I remembered when re-reading \textsl{Pulp} in hospital,
and the theme is one that is central to all humans.
%
The dog walks down the street.
%
30 July.


I think this settles it:
%
As “daily business”\hspace{-0.1em},
I can write reports for future academies,
in a narrow format that is easy to read on cell phones,
published on my website and eventually in other places online,
and at times in a single pocket book,
since there is no incentive on my side
to go through the hassle of publishing a whole series of books;
also as part of “daily business”\hspace{-0.1em},
write the twin pocket books about space time elements;
and as more of a personal project \ELEMENTAL\ as fate;
and also maybe find new clues on the core website,
but large refactorings there rather unlikely,
except never say never;
and do other things,
enjoy life;
I am not entirely a slave to the above,
even though it is certainly fun of a kind
I might not be able to recreate again in my lifetime,
hence worth keeping what is to a certain degree.
%
The dog is still walking down the street.

But maybe I am happy the way things are,
except to continue thinking about the core idea,
and maybe privately write \ELEMENTAL,
if it wants to grow.
%
In any case,
there’s time again now.
%
1.~August.

{\color{odyssey}%
What I desire is to be remembered as someone who made an important discovery.
%
In my lifetime some recognition of my work or at least the prospect of it later on would be very much desired.
%
What I do not desire is to become broadly famous in my lifetime
nor to have to invest much work into getting some recognition.
%
Not being consciously sure any more if my core idea is really that universal
and given that other discoveries on the side do to me not really merit
to promote them any more than already happened,
I guess only new findings can convince me
to invest much time into further publications and promotion.
%
Being afraid to destroy the beauty of this site due to getting older,
the prime candidate for making progress regarding the core idea is the book \ELEMENTAL,
while I cannot exclude for sure that the other book projects would be completely in vain in that respect,
maybe even due to eventual readers eventually responding.
%
7.~August.}

\newpage

Maybe just as important (or maybe not important at all)
is that yesterday I had the idea to a book simply titled “way”.
%
Now you may say that is just another way of saying “tao” (dao)
or that this is also the title of the first section after the welcome page on my website.
%
But “way” would simply be my new way of looking at life and the world,
would be a title in a way closer to what I am actually doing than “elements”,
even though they occur very early in my way
via immediate experience of space and time.
%
Also 7.~August.

What the idea with “way” shows is maybe more that “elements” are not central,
that \ELEMENTAL\ and ‘\white{Space} \red{Time} \black{Elements}’ would not really be on topic,
maybe even shy people away
more than the core approach of looking at life and the world as experienced by individuals,
while a book or website titled “way”
might still not necessarily be a way I could go in public,
as follows.

The way I see people still shy away from the core idea
makes me wonder if the way I am already presenting it
and have been for more than 20 years now
would really be at the core of the lack of public reception,
whether it would not instead simply be
that none of my contemporaries would be ready to approach my stuff,
except best case feel that there is something to it,
but not want to get into it.

I have no answer at the moment regarding what would be a next best step,
except that doing nothing and still waiting for inspiration
might be the best thing to do or rather not visibly do
for the time being\,…

Again,
it is not certain so far if my approach is really worth much trouble,
either because there is not all that much to it
or else because there is nothing I could do to give it any attention now
nor more attention into the future.

Wei Chi, but if the little fox\,…

But if you never tried,
you might never find out that something would have worked,
if you just had tried,
even if only very naively,
reminding of Kafka’s Odysseus in \textsl{Das Schweigen der Sirenen}.
%
Conversely,
I still have the feeling that I already did enough
and doing more would not make any difference,
or that I might not be naive enough any more to try.
%
Then again,
the question is also whether I would have something better to do
in the coming years as hobbies or simply pastime\,?
%
Maybe, maybe not.
%
Circles over circles.
%
9 August.

(Side remark:
The idea with “way” is I think from 6 August,
when my solar return was at 11:14 local time,
not sure if the idea was earlier or later,
maybe rather earlier,
but not sure,
and it was originally the name of a new USB stick
for my source of \ELEMENTAL,
a stick that I spray painted silver
in the afternoon the same day.
%
Maybe this says in a way (pun!)
more than the sentences above,
at least if I was just able to read the signs
or instead feel into them\,?
%
\textbf{way}.)

Sounds nice,
but maybe too straightforward to really mirror life.
%
My heart does hang on to \ELEMENTAL,
even though it is emerging so slowly;
I am now only at the 4th page of the main text,
and there is no preface.
%
Does not matter that the blurb on the back cover
is rather focused on the implications than what the book is factually about,
but maybe that is exactly because any selection would be biased.
%
But maybe a book titled “way” could still become part of a set of books,
who knows.
%
I was born with a retrograde Mercury in early Leo,
only about 20.5 hours away from going direct again,
and as I am writing this,
retrograde Mercury is again in early Leo,
not quite as early as at birth,
but still quite early,
and about a day and a half away from going direct again.
%
My heart hangs on to \ELEMENTAL.
%
Seems to be fate.
%
9 August 2025 around 22:10.

Mirroring my Saturn in late Pisces
seems hardly if at all all possible
in a streamlined single publication that would be accessible to many.
%
And several rather independent publications would rather be Sagittarius,
I guess.
%
I guess this is in the end
why any projects I try to start do not want to grow.
%
Maybe I should essentially just be proud and happy
with what I already created.
%
And any further projects,
except maybe \ELEMENTAL,
would just be paths (ways) to it\,?

And there is also my moon in the 10th house in Aries,
still conjunct my MC.
%
So why did the book project “Artemis – Die geheime surreale Welt des Mondes” disappear\,?
%
One reason was that it is so omnipresent around art,
especially around theater,
that I thought it would not be necessary
nor would I be able to produce all that much new,
which is maybe exaggerated
because I would have made general statements
that would fit many concrete manifestations.
%
The other reason is probably because \ELEMENTAL,
again,
would be a good fit for that.
%
And there is also the idea
that ‘space time elements’ would sort of represent my IC in Libra,
the wish of my father with his MC in Libra,
while the MC in Aries and the moon there,
and hence also \ELEMENTAL,
would sort of represent the wish of my mother with her MC in Scorpio.
%
There is certainly something too it,
but not all black and white,
a small pocket book would fit both,
and also me,
to be honest,
even though the exactphilosophy.net book of this year was large,
as also previous editions.
%
It is just so that the format of \ELEMENTAL\ would fit many themes,
and also that it would be sort of a novel with a dialogue and setup, at least,
hence also many things at once.
%
Finally,
$\pi$ seems to clearly favor \ELEMENTAL,
while not objecting to do other things leisurely, too.
%
Do I have much of a choice\,?
%
Not really,
this website was core,
but I admit that continuing it seems difficult,
I think I could hardly fit all complex thoughts
that went into it again completely in my head and refactor,
plus,
as mentioned,
there is one part where I am not sure
if it does keep the whole thing as fundamental
as I thought until about a year ago.
%
I guess that leaves no choice.
%
Things will continue to grow only very slowly,
including \ELEMENTAL\
and probably some other projects
whenever the time is right for each to grow a bit,
except \ELEMENTAL\
where I am still aiming at and hoping for some steadier progress,
even of overall still very slowly.
%
A rushed result would also not be desired at all.
%
I guess the world does sort of have a plan for me still,
also let me survive this spring/summer so far,
could have done differently.
%
Let me rest things at that,
from now on again just report things that went public here or in related publications,
assuming that new findings would usually still get onto this site in minimal forms.
%
10 August.

Hey:
Maybe “way” and maybe also in German “weg”
would be a way to reach an audience\,—\,%
just that, nothing more, nothing less, but already a big step.

What I like to do is complex, difficult things,
and find and mirror the beauty in them.
%
This website does fit that description.
%
Most book projects only partially,
hence at most do them part-time,
or to fill time,
but \ELEMENTAL\ does fit the description,
except maybe still lacking the beauty part a bit in the text inside,
but that could emerge in time.
%
Today I was at the Rhine Falls with a good friend
and he took a few pictures of me:
I looked old in my perception;
if I would live to the same age as my father,
I would now only have about 25 years left,
hence not really all that much time to waste
on projects that might maybe not even make me happy
if they were successful.
%
{\color{odyssey}%
This website and \ELEMENTAL\ would both make me happy
even if completely ignored.}
%
This is not the complete truth regarding the other projects,
they contain beautiful and desirable things,
but still\,…
%
10 August in the evening,
{\color{odyssey}the human condition}\,…

Hey again:
I think the plan of 20 June is good enough,
with the later modification of maybe publishing the reports
first online and then in a single book,
but that is a detail.
%
And, yes,
preserving this website and writing \ELEMENTAL\
will be what really matters to me,
while the other two projects might matter more to others,
and would still be enough fun to write for me.
%
Mercury still slowly rx.

Mercury is just about to station and go direct again
in early Leo already in my first house (4°15').
%
The plan of 20 June is good enough.
%
The website will be essentially preserved
with eventual tiny additions or changes
that keep it still stable as a reference.
%
The reports will allow me
to add some more free elements
that were in articles or posts.
%
The twin books will allow me
to present the core idea
in a respectable and quite accessible manner.
%
And \ELEMENTAL\,…

Turning yet another page\,…

Pisces is about refining and finishing things,
about preparing the ground for new things to grow,
besides also a great diversity and abundance
fueled by all lives ever lived.
%
In that sense,
the current state of this website is one of refinement,
it should at most change minimally from now on,
also preserving the book of 2025 as a viable reference.
%
New beginnings should instead be its children.

\ELEMENTAL\ is very beautiful,
is something new to refine over the years;
with Saturn (and Chiron) such things
simply take time and are not so easy to do.

The twin books
‘\white{Space} \red{Time} \black{Elements}’\,/$\!$ ‘\white{Raum} \red{Zei} \black{Elemente}’
allow to present the core idea to a somewhat wider,
even though rather intellectual audience
for the first time in a more amenable manner.

The \includegraphics[scale=0.1]{i-reports.png}
would present and preserve all kinds of ideas
about nature, life, the world, you name it,
and thus make sense,
and also fit the piscean theme.

Other books like ‘way’
or another one about the elemental zodiac,
or even astrological scientific work related to Saturn returns,
a Möbius strip of elements,
and more,
would also generally fit the theme if not too homogenous.

My remaining life expectancy is something like 25 years
and unsure as for everybody in what shape if that long at all,
hence not really all that much time to ponder
what to do and what not to do,
unless I would prefer the pondering,
which has some truth to it,
especially as much of my work
cannot be accelerated much
by investing a larger percentage of every day,
but still,
it has taken a long time now,
since at least early 2019,
when I started \ELEMENTAL.

Did I mention how beautiful \ELEMENTAL\ is,
and that it is also the only project besides/after this website
with at least comparable depth and universality,
if not way more,
the potential to reach a very large audiences\,?

Any more questions\,?
%
Hopefully at least by myself not so many any more\,…

This page has been written on the tiny MacBook
on which I am also essentially trying to write \ELEMENTAL.

\vspace{5mm}
{\smaller%
The timeline was saved yesterday
13 August at 15:49:57 on this MacBook
and it wrote itself almost by itself.
%
Likely Artemis (near my native Lilith and $\pi$’s sun) 
and the north node near the IC,
mirroring also that Pisces is also about dreaming,
that some things are more beautiful
if only imagined or kept to oneself.
%
You can also see the open envelope
with a grand trine pointing to Sedna/Uranus
in early Gemini,
also near the combined north node with $\pi$,
or the AC of saving near ax’s new moon,
and obviously a relation to the chart
of when I bought this MacBook (see my article \textsl{Mysteries of life}),
and more\,…
%
But the core message is
that the mentioned books and projects
might well materialize over the coming years,
with \ELEMENTAL\ being the one closest to my heart,
but if they did not or would never be published (or not much),
it would also be ok or even more satisfying,
at any point in time from now on.%
\par}

\begin{center}
\includegraphics[scale=0.127]{i-timeline-saved.jpg}
\end{center}

\item[2025]
Apparently recovered the core idea
on the evening of 14 August 2025 at 22:25:

What moves inside (emi),
even conscious thinking,
is mostly an automatic process
(and thus passive from the point of view of an observing self),
with just tiny fully conscious glimpses
where things are static (eri),
like the flowers for a butterfly
or the islands Odysseus visited.
%
The identification of emi via water with feelings
and of eri via air with logic/communication
of psychological astrology, however, does not fit,
also because in psychological astrology
there are four elements in the psyche,
not just two,
while it might still make sense
to split eri into three different inner states.
%
Anyways,
this should allow the core idea to grow again\,…

\begin{center}
\includegraphics[scale=0.42]{i-core-idea-recovered.jpg}
\end{center}

Added
‘Changes since the 2025 book from\,‘way’\,to\,‘dreams’\,in {\color{odyssey}blue}.’
to the main page.
%
The idea is to keep also online the state in the 2025 book as the reference,
while evolving things rather outside of the website
in child publications,
like the ones insinuated in the ‘Seeds’ section,
and likely in the years to come first rather privately\,—\,%
all unless there would be crucial new findings
or ‘never say never’…

Posted current web (including noindex)
to the Usenet news group alt.binaries.misc
with subject “Der Student von Prag (1913) (exactphilosophy.net 2025)”
and inside both the web with a short readme
and a copy of the movie
as downloaded a day before from the Internet Archive.
%
Posted 28 August $\sim$19:14:50 {\smaller(UTC+2)}.

The same day in the afternoon,
I had the idea to just “forget” about exactphilosophy.net
and all the sort of planned books
(like a fish swimming away after spawning,
fitting maybe finally with my Saturn in late Pisces)
and instead just do a website way.exactphilosophy.net,
this time as plain html,
not images or pdfs,
and just do it essentially as if I had never published my core idea before.
%
I suspect maybe even the gods/goddesses
might not be so sure all the time where my core idea should go,
or at least they would not tell me,
and maybe they would even be interested in my opinion from time to time,
since they lack the perspective of the human condition,
like they asked Paris and Teiresias in mythology\,?

Maybe just doing nothing would be best.

\newpage

\smaller

To me the core idea is still not really recovered
and until it would be,
they way I know myself,
practically nothing will happen,
besides a feeling that my life was largely in vain…

But maybe the fates are happy with what emerged?

Maybe look at Chiron (and Pallas)
in my Chiron returns of 2018 and especially%
—when I will almost certainly already be gone—%
in 2068,
at least that gives me some (delusional?) hope
for the future of the world
thanks to my findings…

You may be aware
that what lead me into this
was the love for a woman
who had given a talk at school about astrology,
possibly more as a joke then,
I don’t know for sure,
and initially all was done in the hope to get together with her,
in vain, of course,
and then I felt obligated to share what I found for humanity,
all in vain so far, of course,
but maybe not ultimately.

Maybe I found a trick to circumvent my apparent fate at least a little bit, via “reincarnation”,
a two-tailed kitsune of some way of a fox in Switzerland (30 Aug 2025 22:11:33), will see…

12 July 2025 at 17:10,
while swimming at Kilchberg,
I had the feeling that she was departing,
and the next day at 12:45,
while walking in Zürich,
she ‘said’ to me that she would now be here ‘for good’.
%
Maybe that would be the real reason for this here stopping now.

(17 July 2025 I had an operation to remove my gallbladder,
which according a CT of 2016 would have ruptured under pressure already about 10 years ago
and by 2025 was surrounded with a large protective bladder.
%
Now you might say that I maybe mistook that for a connection.
%
This may well be,
as usual around any ‘psi phenomena’ there is plausible deniability also here,
but note the placement of Mars in the 12 July moment relative to her birth chart.
%
Fates can be connected,
maybe some sort of an Artemis-Apollon twin theme,
that when one suffers so does the other,
what do I know.
%
Just before the operation also emerged
that I have usually too much sugar in my blood,
maybe completing my stellium in the 12th house around 20° Cancer
with gallbladder-Mars, sugar-Venus and liver-Jupiter.
%
Anyways,
in my feeling,
at least as long as there is no literal sign of life of $\pi$ visible to me,
x$\varphi$ will probably just rest,
Sleeping Beauty Dreaming,
a theme that came up in spring 2021
after first still quite energetically working on \ELEMENTAL,
but will see,
maybe a bit further into the future would make more sense to produce things again,
at least besides the one thing hinted at above.)

\larger

\begin{center}
*
\end{center}

This website is very beautiful as it is,
its content is often phenomenal,
and well-rounded.
%
Thus it just has to rest as it is,
{\color{xphi}Sleeping Beauty Dreaming}…

Whether the planned books and other projects would have any chance,
first for me to write them,
then for me to get them some visibility in these adverse times,
then find some interested folks who would not be afraid of all that is attached to it,
which seems to have scared everybody I met so far beyond approach,
well,
all of this seems rather unlikely to manifest,
except that maybe such material
might help further in the future,
even though it might not be a necessity of any kind.

Saturday 30 August 2025 at 22:11:33
(or maybe a few seconds earlier)
I pressed a button to register a new website
that would be designed to be independent of the complex around x$\varphi$ here,
sort of a kitsune with two tails,
a way of the fox.

Here’s a preliminary sketch:

\begin{center}
\includegraphics[scale=0.3]{i-wayfox-sketch.jpg}
\end{center}

Come to think of it,
reminds me of Kafka’s \textsl{Das Schweigen der Sirenen},
which starts with
“Beweis dessen, dass auch unzulängliche, ja kindische Mittel zur Rettung dienen können”,
in English
“Proof that inadequate, even childish measures, may serve to rescue one from peril".
%
Yes, that’s exactly it, love it.

\begin{center}
$\diamond$
\end{center}

Que sera, sera.
%
This website was virtually the product of a relation,
hence what I can do alone is limited.
%
The beauty must be preserved,
that is paramount.

\end{itemize}
\normalsize

\subsection{Prague Fringe}

As it turned out,
my ‘departed’ encounter was at most purely virtual.

My quintessence
from the ruminations
is
that it is paramount
to preserve this website
in the state of the 2025 book,
without knowing exactly why.
%
Thus changes since then,
except tiny fixes,
will be clearly marked
for the time being.

The following
references things from my life
that I had brought up online before
and to which you should now have
quasi-permanent access,
since their publication
in late August 2025.
%
I will not explain all associations in detail.

The name of this website
goes back to my first question ever to the I Ching,
“What about Prague?”$\!$,
at the time clearly thinking about $\pi$
who was presumably living there at the time.
%
The result,
64 Wei Chi,
is the reason why this website features a fox,
and,
also in analogy
to seemingly contradictive portmanteaus
like “velvet revolution”,
then “exactphilosophy”
came up as the name.

Kafka,
in his silence of the sirens
refers to Odysseus as a fox,
and also my last name,
Stalder,
seems to have some relations to that,
and Kafka also insinuates
that childish, insufficient+inadequate means
can yield salvation.

This may have actually happened
with sort of a virtual ‘reincarnation’
when Saturn was just about to reenter Pisces from Aries,
as Prague means threshold,
thus also related to rebirth,
when I started something
with sort of a Kitsune fox with two tails,
but now looking maybe more
like a naive dog
that wags its tail.

Accordingly,
maybe all the book
and other projects around this site
would also largely just sleep
and wait for new princes to come along
and waken this sleeping beauty.
%
Talking about beauties,
here it comes…

On the afternoon of Saturday 13 September 2025
there was my second Saturn return of the second cycle,
a retrograde one,
while I was “evaluating” public table tennis tables in Adliswil
for a short test game on Sunday with a friend,
where in the end astonishingly the tables I knew from way back at Kopfholz won,
because they had apparently been ground down the week before
after having become almost unusable over the years.

After that I drove to town and when driving down in Wollishofen on the street with the tram, towards the church, there were two people with bicycles ahead of me, in front a man and behind a woman, and I assumed they were together, but then the man took a left just before the curve to the right, which she and I followed. I noticed the woman was very special, rather slim except her ass which just slightly showed its contours unter a skirt that was in itself rather special and/or what she wore on top, something with a metallic reflection, and both with some patterns on it, a bit reminding of the 1960s, and long reddish hair sort of in a fishtail, but let me keep the details, the moment was 15:40 and the woman made a big impression on me, something I will never forget, a typical signature of “Prague”, nothing blunt, just of unparalleled beauty ……………

Student of Prague Usenet noindex,
way of the fox started again with two tails
(see my first question to the I Ching)
and then in town buying a copy “by chance” of the secret of secrets
with a scene with a black sun on the threshold early on
(see me in about 1988 seeing $\pi$ near central in Zürich),
“but who would claim that a sign ceases to be a sign only because its cause has been revealed”,
as Věra Linhartová $\sim$ wrote in the short story “room”, and so on\,…

What seems to have made a huge difference to me
was to cast my weight
by publishing also the noindex part
of my work
in a way that
would be quasi-permanent,
maybe also in analogy
to my gallbladder etc.
\includegraphics[scale=0.07]{i-wayfox.jpg}

\newpage

\subsection{Sabian Symbols Oracle}

Answers by my Sabian Symbols Oracle to ten questions:

\begin{itemize}

\item[(1)] {\color{xphi}Make wayfox\,?}\newline
Sagittarius 9 – A mother with her children on stairs. {\color{gray}Education}

\vspace{-2mm}
\item[(2)] {\color{xphi}Reports for future academies\,?}\newline
Gemini 25 – A man trimming palms.  {\color{gray}Enhancement}

\vspace{-2mm}
\item[(3)] {\color{xphi}Make \ELEMENTAL\,?}\newline
Libra 18 – Two men placed under arrest.  {\color{gray}Consequence}

\vspace{-2mm}
\item[(4)] {\color{xphi}Just let exactphilosophy.net grow on (core, articles, German)\,?}\newline
Aquarius 18 – A man unmasked.  {\color{gray}Analysis}

\vspace{-2mm}
\item[(5)] {\color{xphi}Simply do nothing, or close to nothing\,?}\newline
Capricorn 20 – A hidden choir singing.  {\color{gray}Worship}

\vspace{-2mm}
\item[(6)] {\color{xphi}Book about Artemis\,?}\newline
Capricorn 12 – A student of nature lecturing.  {\color{gray}Explanation}

\vspace{-2mm}
\item[(7)] {\color{xphi}Book about zodiac signs\,?}\newline
Gemini 1 – A glass-bottomed boat in still water.  {\color{gray}Curiosity}

\vspace{-2mm}
\item[(8)] {\color{xphi}Somehow promote my Möbius strip of elements\,?}\newline
Aquarius 18 – A man unmasked.  {\color{gray}Analysis}

\vspace{-2mm}
\item[(9)] {\color{xphi}Continue researching the core idea\,}\newline
Pisces 10 – An aviator in the clouds.  {\color{gray}Observation}

\vspace{-2mm}
\item[(10)] {\color{xphi} Make the book Space Time Elements / Raum Zeit Elemente\,?}\newline
TAURUS 21 – A finger pointing in an open book.  {\color{gray}Confirmation}

\end{itemize}

\noindent
Synthetically,
it seems that there was only a {\color{xphi}clear “Yes” to (5),
thus doing nothing or close to nothing seems paramount}.

Several projects sound more like embellishments or refinements,
like (2) and maybe also (7),
but also the rest does not immediately indicate any necessity.
%
It also seems regarding (9)
that more research would likely not yield new findings.
%
Several like (6) or (10)
could mean that doing them would just make things clearer,
without indicating any necessity.
%
One could read (3)
as that even though the main protagonists would be a man and a woman,
it would be written by a man
and thus would not be able to live up to what it aspires,
helas!
%
I am not so sure what to think about the same answer to (4) and (8),
but it would seem
that they would maybe only allow me or others
to understand my ways of thinking a bit more precisely,
with again no necessity.

{\color{xphi}The only answer that is maybe in addition to (5) is maybe (1)},
because a mother and her children are,
of course,
bound to each other with a certain necessity,
including to move on.
%
Thus,
even though Saturn in late Pisces
would be about spawning and then swimming away,
there would be parts of me
(like my Mars/Venus/Jupiter stellium in Cancer in the 9th house)
that would suggest
that carrying and holding on a bit
would not be wrong,
but apparently just not in context of exactphilosophy.net
and its ruminated seeds.

Maybe (5 Leo) as higher elemental quattrave of (1 Aries),
{\color{xphi}akin to e5/fire}\,…

\vspace{2.2mm}
\noindent
{\footnotesize%
{\color{xphi}One more thing:}
I am only at about chapter 52 of \textsl{The Secret of Secrets},
but so far it seems to me that Prague played Dan Brown,
that he did not notice
that she planted the idea of a dream in the novel
(like Athena in the Odyssey)
into his brain
such that the Mórrígan would walk exactly across the threshold\
…\ unless Dan Brown would be as much a fox as Odysseus\,…%
\par}

\vspace{1mm}
\noindent
{\footnotesize{%
I am played,
too,
of course,
“way” is the natural name,
thus (3) and (10) are wrong titles for a continuation of the core idea,
the rest is also not immediately core forward,
except (5).%
\par}

\newpage

\small
\begin{itemize}

\color{odyssey}

\item[ ]
The core idea
seems to be at least somewhat restored
since this summer,
which is why
I \textsl{\color{darkgray}grayed}
the second last paragraph
on the previous page
since the book,
see also the new article\,\textsl{Timeline ruminations}
of this spring+summer,
while I am not sure if the core idea would be recoverable in its original universal beauty.

Additions since the 2025 book
are now marked in blue
from ‘way’ to ‘dreams’,
including in articles like this one,
except for tiny rather formal fixes,
in order to keep the book as reference,
which feels paramount to me in this time,
while I do not expect more than a tiny bit to grow and change%
\,—\,unless the core idea would miraculously recover completely or as good as completely.

I am still sad
that none of my findings
has found open ears
in more than 20 years
and would have started to grow
on its own,
maybe also too lonely
to continue$\,$…

But I also see
the “Prague Fringe”
in all of this,
see also\,\textsl{Timeline ruminations}:

Maybe the things I found
are simply too much for just one person to manage\,?
%
Thus I would be very lucky
that the fates are apparently largely protecting me –
and the world –
from any harm
that premature public recognition
could cause\,?
%
But I am still very sad to have lost the beauty of the core idea in this time.

\includegraphics[scale=0.7]{i-wayfox.jpg}

Saturday 30 August 2025 at 22:11:33
(or maybe a few seconds earlier)
I pressed a button to register a new website
that would be designed to be independent of the complex around x$\varphi$ here,
sort of a kitsune with two tails,
a way of the fox.

Maybe childish to assume that a virtual “reincarnation” could bring salvation,
but maybe Kafka would have been right about the silence of the sirens…

But at least here and in the previously considered projects around the ideas here,
it seems that nothing or close to nothing would grow
unless the core idea was fully or as good as full recovered,
but maybe either would be good in the end.

Sleeping Beauty Dreaming here\,…

But: At the moment I am still working 80\% and doing this as a hobby,
finding less and less time for this as I am getting older,
while that situation would presumably end in a few years,
if all goes well,
and then what would I like to do
if not maybe present some of the things I found here more amenably,
assuming the kitsune would not take up most of my time,
see also the very revealing Sabian Symbols Oracles in, once more,\,\textsl{Timeline ruminations}…?

Thus the conclusion remains Wei Chi of the I Ching:

\vspace{1.2mm}
\mbox{\ \ \ Before Completion. Success.}\newline
\mbox{\ \ \ But if the little fox, after nearly completing the crossing,}\newline
\mbox{\ \ \ Gets his tail in the water,}\newline
\mbox{\ \ \ There is nothing that would further.}
\vspace{1.2mm}

That is fate, I presume\,…

The “masterplan” would thus be to preserve this website,
with only tiny changes that are clearly marked as additions/changes since the 2025 book
(unless there would be another breakthrough regarding the core idea),
and to continue working on related projects outside,
well aware that maybe all except way of the fox
could at most be embellishments and (more amenable?)\ clarifications.
%
28 September.

That way I could at least depart in “company” of many beautiful creations.

\end{itemize}
\normalsize

\newpage

\noindent
The previous page was another page 3 of the \textsl{Timeline}
that was only briefly up,
removed again already this next morning.

I do not know what will be and can be,
but just dreaming about such projects
and also doing or dreaming other things
would still also be beautiful possibilities,
maybe more beautiful than realizing any,
as far as I can see\,…

The oracles were pretty clear about what really makes sense
(do nothing or close to nothing)
and what would seem to be “passe-temps”…

Me doing nothing would also not necessarily mean
that nothing would evolve,
even though that seems a gazillion times more likely to me in my time.

I have no clue,
except that maybe really doing nothing or almost nothing would be best,
if I interpreted the oracle the way it might have been meant,
while it would certainly seem to mirror my Saturn in late Pisces in 9.

Keep changes to this website minimal and clearly marked in blue,
and see if anything else would want to grow, as meaningful continuation or pastime\,…

My hope now is still on the way of the fox.

\begin{center}
\includegraphics[scale=1.35]{i-wayfox.jpg}
\end{center}

\noindent
I am reminded of developers from Prague who were working for the company I worked at 20 years ago. As they told me later on, they would privately speak about building a new “tower” whenever they got a new project to implement, as I understand not trying to do the same as before, but doing something special, beautiful, something they liked and I guess also implicitly to please the goddess (of the city of 100 spires).

Why not do different things in public, make them beautiful and special?

Many of my ideas seem already to mirror that, then let the public decide\,?

Personally, my hope now is still on the way of the fox.

My hope now is still on the way of the fox.

\begin{center}
\includegraphics[scale=0.02]{i-foxyfox.jpg}
\end{center}

\noindent
The way the way is described on this website
is perfectly fine for reading and understanding it,
there is factually no need to add or change anything,
at least not in the foreseeable future,
anything more is thus just for fun.

\newpage

\noindent
This was yet another short-lived version of page 3 of the timeline:

\small
\begin{itemize}

\item[ ]
Changes since the 2025 book from ‘way’ to ‘dreams’ in {\color{odyssey}blue}.

Article\,\textsl{Timeline 2025 ruminations} with a personal breakthrough in the end:

The way the core idea and general approaches are described on this site
is perfectly fine for reading and understanding them;
there is factually no need to add or change anything,
at least as long as there are no important new findings.

As long as there is no significant public interest
into particular topics of this site addressed at me,
I see no reason to publish more about them,
except eventually purely for my personal fun, pastime or leisure\,…

Maybe future readers would appreciate.
%
29 September.

\end{itemize}

\normalsize
\noindent
I guess in the end things are too complex
to encompass with conscious prose.
%
Overall it looks like it is not clear
if there is really a universal discovery with the “core idea”,
while there are some other interesting ideas,
and especially my model of the star signs (zodiac signs)
seems to be something that is really so.
%
Also not so clear is if my findings got no recognition
because they would be nothing special
or because people would be afraid of their consequences for their mindsets;
I guess from reactions of friends
rather the latter plus an uncertainty of the former.
%
As it turned out on the previous pages is that the “core idea”
seems to be rather the way,
the way of looking at life and the world
from immediate personal experience
than something with “elements”,
that would rather be one of the potential consequences,
and even if not the logical similarities
would be of a certain interest.
%
The Sabian Symbols oracles a few pages back
seem to suggest that doing nothing or almost nothing
would be the best way to continue,
even though the choir would remain invisible,
even maybe the website and its ideas,
but I am trying to solve things by thinking again.

Let me turn the page and instead muse, consider, dream\,…

\subsection{way}

I want to let this website rest essentially as it is
as long as the full beauty of the way has not been restored
and would be great if any of the other projects would grow into beautiful islands,
while keeping them as dreams seems also beautiful,
for me and maybe for others reading this in the future\,…

PS: I noticed that there is a difference between “core idea” and “way”:
What made the “core idea” so special was exactly “that with the elements”.

Thus publishing under the title “way” might not necessarily be better.

\newpage

\subsection{Where I would probably gravitate to …}

A series of pocket books, also for free on this website,
in the same format as the previously published \textsl{Space, Time, Elements at exactphilosophy.net} books,
thus inside also largely text in the same way and width as on this website.

Keep it all at one place under the umbrella exactphilosophy.net,
which I think is a strong “brand” without really being a brand,
and under my real name,
which is unique in the sense
that no very famous person has a similar one.

Each book would present one of the ideas of this website
as simply and directly as possible,
with as little context as possible required,
but still hint at related books towards the end of each book.

I would not advertise any of these pocket books
until several of them would have settled
and been published as printed pocket books.

And I would continue to do research,
especially regarding the core idea,
which would not exclude a book that would present the status quo,
most likely ‘\white{Space} \red{Time} \black{Elements}’\,/ ‘\white{Raum} \red{Zei} \black{Elemente}’
and maybe also in addition a pocket book titled ‘way’$\!$, which would be
more focussed on the approach to life and the world than on elements.

Any ideas from this website could be published that way,
including also many of the ideas for \includegraphics[scale=0.1]{i-reports.png}
and also ‘Artemis’, of course.

Maybe the pocket books would be under “easy”.
The rest of the website would remain close to as is for the time being,
with changes from ‘way’ to ‘dreams’ in marked in \textbf{\color{odyssey}blue}.

And $\pi$ wants \ELEMENTAL\,…

\color{odyssey}

\subsection{dreams}

In the end, I dream like Pisces …

Of the way coming back in its original beauty,
if ever maybe most probably via writing or trying to write \ELEMENTAL\ …

And of the way of the fox kitsune finding a public,
and of all the books about all the beautiful ideas finding at least a spot in my bookshelf …

\subsection{renegades}

I make art, special things, avantgarde, new things, not mainstream.\newline
Maybe $\pi$ would be right again. {\smaller(velocybird. chinese ink. 2025.10.02.)}

\begin{center}
\includegraphics[scale=0.1]{i-velocybird.jpg}
\end{center}

\end{document}