\documentclass[letterpaper]{article}
\pagestyle{empty}
\paperheight=3700mm
\textheight=3700mm
%\textwidth set depending on pdflatex or (experimentally)lualatex
\topmargin=-20mm
\oddsidemargin=25mm

\usepackage[utf8]{inputenc}
\usepackage[spanish,italian,french,ngerman,english]{babel} % last is main
\usepackage{graphicx}
\usepackage{multirow}
\usepackage{xcolor}
\usepackage{contour}
\usepackage{pict2e}
\usepackage{relsize}
\usepackage{amsmath}

\usepackage{iftex}
\ifpdftex
  % stronger fonts

% Find modes.mf, e.g. /usr/local/texlive/2025/texmf-dist/fonts/source/public/modes/modes.mf
%
% $ sudo cp modes.mf modes.mf.orig
%
% Add the following at the start of modes:
%
% mode_def xphi =
%   mode_param (pixels_per_inch, 1200);
%   mode_param (blacker, 1.9); % only difference to 'lexmarkr' (2 there)
%   mode_param (fillin, 0);
%   mode_param (o_correction, 1);
%   mode_common_setup_;
% enddef;
%
% Finally:
%
% $ sudo fmtutil-sys --byfmt mf

\pdfpkresolution=1200
\pdfpkmode={xphi}
\pdfmapfile{}

  \newcommand{\coretextwidth}{85.5mm}
\fi

% experimental, not used to produce the live website...
\ifluatex
  % about same heaviness in pdfs when rasterized in photoshop,
  % but since, unlike the metafont mechanisms I use, fake bold, "bleeds" in all directions,
  % seems heavier at least in core web page images
  \newcommand{\fontbleed}{0.8}
  % paragraphs wider and font looks larger, tried to fix, but then other things change a bit,
  % especially for section headings would have to change back, are now more narrow...
  \newcommand{\fontscale}{0.985}
  \newcommand{\coretextwidth}{85.2mm}
  \usepackage{fontspec}
  % microtype does maybe help and not help, maybe if would allow wider spaces...
  \usepackage{microtype}
  \setsansfont{Latin Modern Sans}[Scale=\fontscale, FakeBold=\fontbleed]
  \setmonofont{Latin Modern Mono}[Scale=\fontscale, FakeBold=\fontbleed]
  % this would be for "new computer modern" (but has many limitations so far)
  %\usepackage[default]{fontsetup}
  %\renewcommand{\familydefault}{\sfdefault}
\fi

\renewcommand{\familydefault}{\sfdefault}

\setcounter{secnumdepth}{-1}

\newcommand{\en}[1]{\iflanguage{english}{#1}{}}
\newcommand{\de}[1]{\iflanguage{ngerman}{#1}{}}
\newcommand{\fr}[1]{\iflanguage{french}{#1}{}}

% a bit less than ~255/256
\definecolor{almostwhite}{gray}{0.996}
\definecolor{xphi}{rgb}{0.0,0.5,0.5}
\definecolor{avant}{rgb}{1,0.5,0.5}

\definecolor{frame}{gray}{0.9}
\definecolor{lightgray}{gray}{0.8}
\definecolor{gray}{gray}{0.5}
\definecolor{darkgray}{gray}{0.3}

\definecolor{darkred}{rgb}{0.8,0.0,0.0}
\definecolor{darkyellow}{rgb}{0.7,0.7,0.0}
\definecolor{darkgreen}{rgb}{0.0,0.55,0.0}
\definecolor{darkviolet}{rgb}{0.5,0,0.5}

\definecolor{darkblue}{rgb}{0,0,0.7}
\definecolor{odyssey}{rgb}{0,0,0.8}
\definecolor{indigo}{rgb}{0.29,0,0.51}
\definecolor{indigoblue}{rgb}{0.1,0,0.6}

\definecolor{saffronback}{rgb}{1.000,0.878,0.627}
\definecolor{saffronfront}{rgb}{0.376,0.125,0.000}

\DeclareRobustCommand{\cometartemisscale}[1]{\includegraphics[scale=#1]{\sourcepath/i-comet.jpg}\hspace{-0.028453em} artemis}
\newcommand\cometartemis{\cometartemisscale{0.018}}
\newcommand\cometartemissection{\cometartemisscale{0.0225}}

\DeclareRobustCommand{\moebius}[1]{\includegraphics[scale=#1]{\sourcepath/i-moebius.jpg}}
\newcommand{\yinyang}{\includegraphics[scale=0.135]{\sourcepath/i-yinyang.jpg}}

\newcommand{\rarr}{\,$\rightarrow$\,}
\newcommand{\lrarr}{\,$\leftrightarrow$\,}

% greek elements
\newcommand{\elfire}{
\begin{picture}(9,6)
  \thicklines
  \put(1,-0.5){\line(1,0){7}}
  \put(1,-0.5){\line(1,1.732){3.5}}
  \put(8,-0.5){\line(-1,1.732){3.5}}
\end{picture}}
%
\newcommand{\elair}{
\begin{picture}(9,6)
  \thicklines
  \put(1,-0.5){\line(1,0){7}}
  \put(1,-0.5){\line(1,1.732){3.5}}
  \put(8,-0.5){\line(-1,1.732){3.5}}
  \put(2.75,1.9){\line(1,0){3.5}}
\end{picture}}
%
\newcommand{\elwater}{
\begin{picture}(9,6)
  \thicklines
  \put(1,5){\line(1,0){7}}
  \put(1,5){\line(1,-1.732){3.5}}
  \put(8,5){\line(-1,-1.732){3.5}}
\end{picture}}
%
\newcommand{\elearth}{
\begin{picture}(9,6)
  \thicklines
  \put(1,5){\line(1,0){7}}
  \put(1,5){\line(1,-1.732){3.5}}
  \put(8,5){\line(-1,-1.732){3.5}}
  \put(2.75,2.5){\line(1,0){3.5}}
\end{picture}}
%
\newcommand{\elhex}{
\begin{picture}(9,6)
  \thicklines
  \put(1,0.5){\line(1,0){7}}
  \put(1,0.5){\line(1,1.732){3.5}}
  \put(8,0.5){\line(-1,1.732){3.5}}
  \put(1,5){\line(1,0){7}}
  \put(1,5){\line(1,-1.732){3.5}}
  \put(8,5){\line(-1,-1.732){3.5}}
\end{picture}}

% i ching trigrams
\newcommand{\trigram}[3]{
\begin{picture}(9,6)
  \linethickness{0.36mm}
  \put(0,5){\line(1,0){#1}}
  \put(5.5,5){\line(1,0){3.5}}
  \put(0,2.5){\line(1,0){#2}}
  \put(5.5,2.5){\line(1,0){3.5}}
  \put(0,0){\line(1,0){#3}}
  \put(5.5,0){\line(1,0){3.5}}
\end{picture}}
\newcommand{\triheaven}{\trigram{5.5}{5.5}{5.5}}
\newcommand{\triearth}{\trigram{3.5}{3.5}{3.5}}
\newcommand{\trithunder}{\trigram{3.5}{3.5}{5.5}}
\newcommand{\triwater}{\trigram{3.5}{5.5}{3.5}}
\newcommand{\trimountain}{\trigram{5.5}{3.5}{3.5}}
\newcommand{\triwind}{\trigram{5.5}{5.5}{3.5}}
\newcommand{\trifire}{\trigram{5.5}{3.5}{5.5}}
\newcommand{\trilake}{\trigram{3.5}{5.5}{5.5}}

% i ching hexagrams
% 1+2 trigrams, 3 rest of line (see e.g. dreams.tex)
\DeclareRobustCommand{\hexagram}[3]{\raisebox{-3pt}{$\overset{\text{${#1}$}}{#2}$\,}#3\vspace{3pt}}

% white-red-black etc.
\DeclareRobustCommand{\outline}[1]{\contour{black}{{\color{white}#1}}}
\DeclareRobustCommand{\white}[1]{\outline{\textbf{#1}}}
\DeclareRobustCommand{\red}[1]{{\color{darkred}\textbf{#1}}}
\DeclareRobustCommand{\black}[1]{\textbf{#1}}
\DeclareRobustCommand{\yellow}[1]{{\color{darkyellow}\textbf{#1}}}
\DeclareRobustCommand{\green}[1]{{\color{darkgreen}\textbf{#1}}}
\DeclareRobustCommand{\violet}[1]{{\color{darkviolet}\textbf{#1}}}
\DeclareRobustCommand{\indigoblue}[1]{{\color{indigoblue}\textbf{#1}}}
\DeclareRobustCommand{\indigo}[1]{{\color{indigo}\textbf{#1}}}

% ELEMENTAL
\newcommand{\ELEMENTAL}{%
\colorlet{contour}{.}\textbf{\color{white}%
\raisebox{+0.001em}{\contour{contour}{E}}%
\raisebox{+0.015em}{\contour{contour}{L}}%
\raisebox{+0.016em}{\contour{contour}{E}}%
\raisebox{+0.023em}{\contour{contour}{M}}%
\raisebox{+0.023em}{\contour{contour}{E}}%
\raisebox{+0.017em}{\contour{contour}{N}}%
\raisebox{-0.020em}{\contour{contour}{T}}%
\raisebox{-0.002em}{\contour{contour}{A}}%
\raisebox{+0.006em}{\contour{contour}{L}}%
}}

% artemis pdf+web icons
\newcommand{\ipdfen}{\includegraphics[scale=0.5]{i-pdf-en.png}}
\newcommand{\ipdfde}{\includegraphics[scale=0.5]{i-pdf-de.png}}
\newcommand{\ipdffr}{\includegraphics[scale=0.5]{i-pdf-fr.png}}
\newcommand{\iweb}{\includegraphics[scale=0.055]{i-web.png}}
\newcommand{\ipdfblueen}{\includegraphics[scale=0.5]{i-pdf-blue-en.png}}
\newcommand{\ipdfbluede}{\includegraphics[scale=0.5]{i-pdf-blue-de.png}}
\newcommand{\ipdfbluefr}{\includegraphics[scale=0.5]{i-pdf-blue-fr.png}}
\newcommand{\iwebblue}{\includegraphics[scale=0.055]{i-web-blue.png}}


\textwidth=\coretextwidth



\begin{document}
\selectlanguage{ngerman}

\avantgarde

\section{Mondfaden}

Hier die ersten drei Beiträge von über tausend des Mondfadens,
\textsl{\color{xphi}Der Mond und die Astrologie (und die Welt)},
den ich am 26. Dezember 2014 um 10:24 morgens
im astro.com Forum gestartet hatte,
plus einer Übersichtszeichnung,
die ich in den Tagen zuvor gemacht hatte,
basierend auf einer Liste mit Stichworten vom 9.\ Mai 2012,
aber erst im August 2015 im Faden veröffentlichte.
%
Daneben gab es noch einige weitere Fäden,
auch mit einigen Beiträgen von mir.
%
Einfach dort nachschauen,
falls Astrodienst das weiterhin aufbewahrt hätte;
der Mondfaden lohnt sich trotz allem wirklich,
ist praktisch der Umfang eines Buches,
mit sehr vielen recht langen und illustrierten Beiträgen,
die meisten von mir.
%
Vielen Dank an alle,
die im Forum mitgeschrieben und/oder mitgelesen hatten!

\subsection{Der Mond und die Astrologie (und die Welt)}

Was ich hier zum Mond sagen werde,
kann man natürlich direkt anwenden auf den Mond im Horoskop
(oder auch in üblicher Weise teilweise auf Krebs oder 4./10.\ Haus, usw.)
– aber es geht auch noch deutlich weiter:

\begin{quote}
\textsl{\color{xphi}%
Wie ich zu zeigen hoffe,
ist der Mond mit dem Wesen der Astrologie
viel tiefer und direkter verwoben
als bisher wahrgenommen wird.%
}
\end{quote}

\noindent
Bevor ich das kann,
muss ich aber erst die Dinge etwas entfalten…

\subsubsection{Runde Dinge und in Kreisen sich drehen}

Der Mond ändert sein “Gesicht” periodisch im Verlauf von jedem Monat:
Von leer zu voll und wieder zurück.
%
Daher und auch weil der Mond sichtbar rund ist,
kommt u.\,A.\ die Identifikation mit allem Runden/Zyklischen.
%
Gerade weil sich beim Mond immer alles dreht,
kann man es auch nicht so logisch linear beschreiben und analysieren,
daher hier einfach mal ein Katalog von mondhaften Dingen,
sie werden sich bald verweben…

\subsubsection{Spiegelndes, reflektierendes “Nichts”}

Der Mond reflektiert “nur” das Licht der Sonne;
wenn der Mond z.\,B.\ im Löwen ist,
dann ist es nicht wie bei der Sonne messbar Hochsommer,
aber doch hat es eine Wirkung.
%
Wenn man eine anscheinend mysteriöse komplexe Frau als Mann verstehen möchte,
so sollte man oft halt nicht zu kompliziert denken,
nicht “zu viel Licht hineinprojizieren”.
%
Mit ganz einfach denken kommt man oft schon nahe heran,
auch wenn man dann doch keine Chance hat auf das “Nichts” am Ende.
%
Oscar Wilde hat das in “Die Sphinx ohne Geheimnis” auf die Spitze getrieben,
mit der Frau die sich einfach nur ein Zimmer für ein paar Stunden in der Woche gemietet hatte
um dort ihr so schön “leeres” Geheimnis zu pflegen
indem sie einfach niemandem sagte was sie da tat
und auch nichts besonderes tat…

\hspace{-11mm}
\noindent
\includegraphics[scale=0.176]{i-artemis-2014.jpg}
\vspace{-8mm}

\subsubsection{Artemis/Hekate und der Frühlingspunkt}

Die Mondgöttin Artemis wurde kurz vor ihrem Zwillingsbruder Apollon geboren
und half gleich mit als Hebamme bei seiner Geburt.
%
Als Hebamme ist sie dort wo neues Leben entsteht
oder generell wo Neues erschaffen wird.
%
Hekate ist eher die Totengöttin,
und sie ist die einzige der alten Götter bei den Griechen,
die nach der Revolution durch Zeus und seiner Generation ihren Platz behielt.
%
In dem Sinne ist sie die unsterblichste Göttin,
älter als alle Götter die waren, sind und sein werden.
%
Oft wurden die beiden als eine Einheit gesehen,
am Punkt zwischen Tod und (Wieder?-)Geburt.
%
Astrologisch ist dieser Übergang zwischen Ende Fische und Anfang Widder,
am Frühlingsanfang.

\subsubsection{Die Liebe zur Kunst}

Wenn man wie Hekate ewig lebt,
so werden andere Dinge wichtig als bei Sterblichen.
%
Menschen kommen und vergehen so schnell,
aber Kunstwerke bleiben Artemis über Jahrhunderte und mehr
als “Gefährten” erhalten.
%
Alles was den Menschen im Leben wichtig erscheint
hat sie schon abermals gesehen,
nur was an Intensität nahe herankommt an eine Geburt oder den Tod
hat für sie noch einigermassen eine Bedeutung.
%
Umgekehrt ist sie als Artemis immer auf der Suche nach dem Neuen,
voll die wilde ungezähmte Avantgarde.
%
Auf der Suche nach irgendwas das neu und intensiv genug ist,
um in ihrem ewigen Leben noch Sinn zu machen.

\subsubsection{Symmetrie und Asymmetrie und die Gegensätze}

Jetzt beginne ich schon mehr die Dinge zu verweben.
%
Für Männer sollte ich jetzt vielleicht noch erwähnen,
dass das auf eine gewisse Weise sehr unlogisch wird,
da sich Gegensätze beim Mond oft nicht widersprechen sondern befruchten…

Die Schönheit der Frauen beruht oft auf passiv spiegeln und rund sein.
%
Das ist durchaus auch rein physisch optisch
das Zeigen oder Verbergen der runden, gespiegelten Attribute
wie Augen, Brüsten mit Brustwarzen und Hintern.
%
Dort aber fast im gleichen Zug wieder das Spiel mit der Asymmetrie,
in der Frisur oder in der Kleidung,
besonders bei Künstlerinnen und so ist oft dort wieder viel Asymmetrie zu finden,
manchmal auch direkt körperlich mit verschiedenen Augenfarben usw.
%
Und natürlich ist das Ganze nicht nur auf das visuell sichtbare beschränkt,
sondern eine schöne Frau ist auch innerlich “rund”.
%
N.B.:
Um eine Frau zu interessieren kann man z.\,B.\ diese passive Rundheit
versuchen zum Drehen zu bringen,
also nicht voll frontal auf sie zu,
sondern irgendwie immer wieder “tangential” und sie so zum Drehen bringen.
%
Was nicht heissen soll,
dass dann der Mann das Spiel kontrollieren würde,
natürlich.
%
Das schwarz-weisse Chinesische Yin-Yang Symbol mit viel Rundem,
das wohl alle kennen,
geht historisch u.\,A.\ auf das Bild eines Hügels zurück,
wo sich im Laufe eines Tages die sonnige und schattige Seite abwechseln,
also Gegensätze und doch ein Ganzes und periodisch in Bewegung.
%
Gerade daher ist neben dem Runden
durchaus auch das betont zackig-eckige auch wieder sehr mondhaft, usw.

\subsubsection{Schwarz-\red{Rot}-\white{Weiss}}

Der Mond ist “weiss” bei Vollmond,
schwarz-weiss die meiste Zeit
und bei Neumond leicht rötlich dunkel.
%
Da ja die weibliche Periode mit dem Mondzyklus assoziiert wird,
ist das rot auch das rot des Menstruationsblutes (oder der Geburt).
%
Das Wachsen zum Vollmond ist auch
wie das Wachsen des schwangeren Bauches bis zur Geburt.
%
In der Mythologie ist das mondhaft weibliche oft dreifaltig,
auch für drei Lebensphasen,
als Mädchen (noch nicht zeugungsfähig),
erwachsene Frau und dann als alte Frau (nicht mehr zeugungsfähig).

\subsubsection{Die Ästhetik in der Kunst}

In der Kunst sieht man oft die drei Farben,
aber auch eben oft Rundes und die Symmetrie.
%
Auch in der Photographie,
die (vor der digitalen Photographie) auf Silber beruhte,
dem Metall des Mondes.
%
Man muss sich einfach mal ein paar berühmte Bilder oder Skulpturen in Erinnerung rufen
und ja,
oft ist es halt dann genau ein mondhaftes Element was es ausmacht,
bzw.\ wenn man es sich wegdenkt dem Kunstwerk
oft einen grossen Teil seiner Berechtigung stehlen würde.
%
Darin liegt auch eine grosse Vertrautheit mit dem Mond,
mit dem Mütterlichen, das man nach der Geburt sieht,
wie z.\,B.\ unmittelbar die Augen und Brüste der Mutter.
%
Umgekehrt sucht Artemis in der Kunst
(zumindest seit der Moderne)
oft die Ästhetik eines neugeborenen Kindes,
das nüchtern betrachtet im Vergleich zu dem,
was man sonst als “schön” bezeichnet,
oft sehr hässlich ist—%
nicht glatt, rund, sondern faltig, verklebt mit allen Möglichem—%
und dennoch die Eltern in Liebe binden kann wie sonst nichts.
%
Genau das versucht Artemis in der Kunst zu erreichen.

\subsubsection{Golem, Roboter, Grossstädte und Prag}

Artemis,
die wilde naturnahe Jägerin ist aber auch Göttin der Städte,
von Zeus selber bekam sie 50 Städte.
%
Prag gilt mystisch als die Stadt der Städte,
und ja,
da findet sich sehr viel mondhaftes,
aber eben,
es ist halt eher im Schatten als im Licht.
%
Der Name “Prag” steht anscheinend für “Übergang”,
möglicherweise für eine Fuhrt in früher Zeit,
wozu es aber anscheinend keine historischen Hinweise gibt,
symbolisch denkt man aber sofort an den Frühlingspunkt,
den Übergang zwischen Tod und Leben.
%
Im Bild “La Plume” vom Tschechischen Jugendstil Maler Alfons Mucha,
hat Mucha als Kunstgriff den Sternkreis versteckt auf 14 Zeichen erweitert,
zwischen Fischen und Widder,
versteckt im Hals der Schönen,
hat er zwei neue Sternzeichen hingetan,
symbolisch also auch für den Übergang zwischen Tod und Wiedergeburt
wie bei den alten Ägyptern.
%
Es waren in Ägypten Heqet und Khnum,
die am Oberlauf des Nils neues Leben schufen aus Lehm,
so wie in der Sage in Prag der Golem am Ufer der Moldau aus Lehm geschaffen wurde.
%
Auch Roboter sind,
wenn man so will,
ein Versuch künstlich Lebewesen zu schaffen,
und das Wort “Roboter” kommt aus einem Theaterstück von Karel Capek
(das Wort ist die Erfindung von seinem Bruder Josef).
%
Es hat sehr viel Kunst in Prag,
das meiste aber unsichtbar für all die Touristen.

\subsubsection{Schlaf, Traum, Rausch}

Künstler oder vielleicht die “Bohème” (Prag liegt in Böhmen)
schaffen ihre Kunst oft nicht analytisch bewusst und linear,
sondern suchen halt oft die Kreise,
in Träumen und auch im Rausch,
eben auch um den Geheimnis des Mondes so nahe zu kommen.
%
Das “automatische Schreiben” der Surrealisten kommt einem da in den Sinn.

\subsubsection{Theater}

Auch wenn Artemis 50 Städte hat,
ist sie selbst sicher keine “Bürgerin”,
sie lebt eher im Grossstadtdschungel,
vielleicht in Abbruchhäusern die sie sich wieder einrichtet.
%
Aber im Theater da kommen die zwei Welten
wie schwarz-weiss im Mond nahe zusammen.
%
Das Publikum im Halbdunkel,
die Bühne im Licht.
%
Die Bourgeoisie im Publikum,
die Bohème auf der Bühne.
%
Und da spiegeln die Künstler auf der Bühne alles,
was in der “bürgerlichen” Welt so abläuft,
auf ganz dichte, mondhafte Weise.
%
Und die Bürger applaudieren dann noch dazu.
%
Das dient wohl (ich sollte das ev.\ noch weiter begründen, aber kein Platz mehr) dazu,
dass die Bourgeoisie unbewusst in Wallung gerät,
ihr Handeln irgendwie unbewusst erkennt und danach unmerklich etwas ändert.
%
Aber Zeit, um zum Versprochenen zu kommen:

\subsubsection{Der Mond und die Astrologie}

In der Astrologie bestimmt sich das Schicksal aus dem Moment der Geburt,
also genau aus dem Punkt wo die Artemis ist,
der Punkt worum ihr Interesse an der Welt kreist.
%
Und das Schicksal entfaltet sich nicht linear sondern in Kreisen,
nämlich in den Kreisen,
die die Planeten im Himmel machen
(geozentrisch, mit all den Kristallsphären und Epizyklen).
%
Das war’s schon,
wie so oft eben mit dem Mond am Ende fast leer und irgendwie banal
und doch die ganze Welt darin enthalten.

Seit der Entdeckung des Uranus
wurden die Abschnitte in der Geschichte bis zur nächsten Entdeckung
mit dem Planeten in Verbindung gebracht.
%
Wie Aufklärung, Revolutionen und Eisenbahn mit dem Uranus,
Photographie und Öl/Automobil mit dem Neptun,
Atombomben und mehr mit dem Pluto.
%
Nun ist seit Sommer 2006 am astronomischen Weltkongress in Prag (!)
Pluto offiziell-astronomisch kein Planet mehr
(was rein astronomisch-wissenschaftlich gesehen unvermeidbar war).
%
Damit endete wohl die Periode mit Pluto.
%
Also was dann?
%
Nun,
ich behaupte,
genauso wie Pluto als erstes Objekt in seiner Entfernung prägend war,
ist es heute das erste entdeckte Objekt einer Sphäre in weiterer Ferne,
nämlich Sedna,
welche im Herbst 2003 entdeckt wurde.
%
Wenn man Planetenherrschaften erweitert,
so bekam Uranus ein Sternzeichen vom Saturn,
Neptun vom Jupiter,
Pluto vom Mars.
%
Also würde Sedna ein Sternzeichen von der Venus bekommen;
ich vermute, dass Venus eher das weibliche Zeichen Stier behalten würde
und Sedna also die Waage bekäme
(ganz oder teilweise oder wie immer man das sehen will).
%
Und Sedna ist im Moment im letzten Dekan des Stiers
(und das für eine Weile, da sehr weit weg und daher langsam,
auch wenn im Moment der Sonne relativ nahe).

Das suggeriert,
dass in den heutigen Zeiten Geld sehr wichtig sein würde,
stärker als ob man Atombomben hat oder nicht.
%
Das würde für mich gut passen.
%
Auch dass (wie Sedna in der Inuit Mythologie)
man irgendwie keine Hände hat mit denen man etwas bewirken kann in der Gesellschaft,
dass man der ganzen etwas sturen
und doch oft auch wohlwollenden Fürsorge
irgendwie halt passiv sich fügen muss in gewissen Bereichen.
%
Ich sehe das eben auch aus Sicht der Artemis oder von Prag aus:
All das Leid dass Pluto mit dem 2. Weltkrieg und Holocaust
und danach dem kalten Krieg nicht zuletzt in Prag gebracht hatte,
“wollte” Prag/Artemis ändern durch die Schaffung
einer mehr und mehr matriarchalischen Welt,
wo passive Macht stärker ist.
%
Es liegt ja in der Natur der Sache,
dass auch wenn ich richtig läge mit meiner “Sedna Epoche der Weltgeschichte”,
das eventuell mondhaft halt im Verborgenen bliebe.

\newpage

\noindent
Hier noch die erwähnte Liste,
aus Gedanken ab 2010:

\noindent
% left lower right upper
\includegraphics[trim=37mm 140mm 200mm 36mm]{artemis-list-2012.pdf}

\end{document}
