\avantgarde

\en{\section{How astrology might really work\,?}}%
\de{\section{Wie Astrologie wirklich funktionieren könnte\,?}}

\en{Nowadays scientists and astrologers live in almost completely separated worlds.}%
\de{Wissenschaftler und Astrologen leben heute in fast völlig getrennten Welten.}
%
\en{I am a physicist and versed in both.}%
\de{Ich bin Physiker und kenne mich in beiden aus.}
%
\en{From where I am standing, the following would seem to be the most plausible,
as I will expose step-by-step afterwards:}%
\de{Aus meiner Warte würde Folgendes am plausibelsten scheinen,
wie ich danach Schritt für Schritt ausführen werde:}

\vspace{-2mm}
\begin{itemize}

\item
\en{All people, even those who consciously do not believe in astrology,
would be noticeably influenced in their behavior by astrology.}%
\de{Alle Menschen, auch die, die bewusst überhaupt nicht an Astrologie glauben,
würden merklich in ihrem Verhalten durch Astrologie beeinflusst.}
%
\en{Nowadays, it should also be possible to experimentally confirm this.}%
\de{Das müsste heutzutage auch experimentell nachweisbar sein.}

\item
\en{The effect of astrology, at least the way it is used today,
would in an immediate sense have practically nothing to do with the planets and stars in the sky.}%
\de{Die Wirkung der Astrologie, zumindest wie sie bis heute angewandt wird,
hätte unmittelbar praktisch nichts mit den Planeten und Sternen im Himmel zu tun.}
%
\en{Astrology would rather be a collective effect,
unconsciously created by practically all people on earth.}%
\de{Astrologie wäre eher ein kollektiver Effekt,
unbewusst erschaffen von praktisch allen Menschen auf der Erde.}

\item
\en{This would imminently be difficult to conceive for many astrologers and scientists,
since each party would in the end have to give up a basic assumption
in order to return to a jointly accepted world view.}%
\de{Das wäre für viele Astrologen und Wissenschaftler unmittelbar schwierig anzunehmen,
da beide am Ende je eine Grundannahme aufgeben müssten,
um wieder zu einem gemeinsam akzeptierten Weltbild zu finden.}
%
\en{Conversely, it would, of course, also be a chance.}%
\de{Umgekehrt läge darin natürlich auch eine Chance.}

\item
\en{On the path to the above view of astrology,
I can also make other concepts a bit more amenable to science again:
Love, religion and deities, telepathy, world soul, collective unconscious, etc.}%
\de{Auf dem Weg zur obigen Sicht der Astrologie
kann ich auch andere Begriffe der Wissenschaft wieder etwas zugänglicher machen:
Liebe, Religion und Gottheiten, Telepathie, Weltseele, kollektives Unbewusstes, usw.}
%
\en{In a way, the path is even more significant than the goal in this text.}%
\de{In gewisser Weise ist der Weg hier sogar bedeutsamer als das Ziel.}

\item
\en{In the end, fortunately a lot remains fundamentally open,
also whether there might maybe still be direct correlations between “heaven and earth”,
as basically presumed in astrology.}%
\de{Am Ende bleibt zum Glück vieles weiterhin fundamental offen,
auch ob es vielleicht doch direkte Korrelationen zwischen “Himmel und Erde” gäbe,
wie grundsätzlich in der Astrologie vermutet.}

\end{itemize}

\noindent
\en{I will first sketch how the human brain mirrors the world inside itself,
resp.\ in the network of its neurons.}%
\de{Erst werde ich skizzieren, wie das menschliche Gehirn die Welt in seinem Innern,
bzw.\ im Netzwerk seiner Neuronen, spiegelt.}
%
\en{Building on that, I will describe what happens when two people love each other,
and then expand this to more people, and many different concepts
which have emerged over millennia, until I get to astrology.}%
\de{Darauf aufbauend  werde ich dann beschreiben, was bei Liebe zwischen zwei Leuten passiert,
und das dann ausdehnen auf mehr Leute, und auf verschiedene Konzepte,
die über die Jahrtausende aufgekommen sind, bis ich bei der Astrologie anlange.}
%
\en{Finally, I will briefly explore further possibilities a bit more freely.}%
\de{Am Ende werde ich weitergehende Möglichkeiten noch kurz etwas freier betrachten.}

\en{\subsection{Mirrors}}%
\de{\subsection{Spiegel}}

\en{In the head of every human being there is a copy of the world,
or at least of part of the world.}%
\de{Im Kopf jedes Menschen befindet sich eine Kopie der Welt,
oder zumindest von einem Teil der Welt.}
%
\en{It contains fellow humans, other living beings and many things,
plus how they behave, also in interaction with oneself.}%
\de{Darin enthalten sind Mitmenschen, andere Lebewesen und viele Dinge,
plus wie sie sich verhalten, auch in Wechselwirkung zu einem selbst.}
%
\en{Everyone can imagine, say, an acquaintance inside,
even if that person in currently not in view,
and often also how that person would behave in certain situations.}%
\de{Jede und jeder kann sich z.B.\ einen Bekannten oder eine Bekannte innerlich auch dann vorstellen,
wenn derjenige oder diejenige gerade nicht sichtbar sind,
und auch oft wie er oder sie sich in gewissen Situationen verhalten würden.}

\en{This mirroring of the world to the inside, into the human brain,
is what essentially allows people (and animals) to live,
to deal with the world, without e.g.\ quickly falling down somewhere.}%
\de{Dieses Spiegeln der Welt ins Innere, ins menschliche Gehirn,
ist das, was es Menschen (und Tieren) im Wesentlichen erlaubt, zu leben,
mit der Welt umzugehen, ohne z.B.\ gleich irgendwo hinunterzufallen.}

\en{\vspace{1mm}}%
\de{\vspace{0.5mm}}%
\hspace{5mm}
\noindent
\includegraphics[scale=0.12]{i-mirror.jpg}

\en{A key point regarding this mirroring is now
that it is often essentially only mirroring,
but not consciously understanding how that which is mirrored exactly ticks%
—let me explain this sentence more thoroughly in the following.}%
\de{Ein wesentlicher Punkt bei dieser Spiegelung ist nun aber,
dass im Wesentlichen oft nur gespiegelt wird,
aber nicht bewusst verstanden, wie das Gespiegelte genau tickt%
—diesen Satz werde ich gleich noch genauer erklären.}

\en{In the brain,
billions of nerve cells (neurons) are connected to each other.}%
\de{Im Gehirn sind Milliarden von Nervenzellen (Neuronen) miteinander verbunden.}
%
\en{Today, such neural networks can be replicated also on computers,
to a certain degree.}%
\de{Solche neuronalen Netzwerke kann man heute zu einem gewissen Grad
auch in Computern nachbilden.}
%
\en{For example, such a virtual neural network can be “fed”
with millions of digital photos and drawings of the digits from 0 to 9.}%
\de{Zum Beispiel kann man ein solches virtuelles neuronales Netzwerk “füttern”
mit Millionen von digitalen Fotos und Zeichnungen der Ziffern 0 bis 9.}
%
\en{That way you can train the network, until becomes able
to often correctly name the digit on an image it had never seen before.}%
\de{So kann man das Netzwerk trainieren, bis es in der Lage ist,
oft in einem bisher noch nie gesehenen Bild die Ziffer korrekt zu nennen.}

\hspace{11mm}
\noindent
\includegraphics[scale=0.12]{i-network.jpg}

\en{Does that now mean that the network has understood
what it is doing and how it is doing it\,?}%
\de{Bedeutet das nun, dass dieses Netzwerk verstanden hat,
was es tut und wie es das tut\,?}
%
\en{Or that it would even be able to explain that\,?}%
\de{Oder dass es sogar fähig wäre, das zu erklären\,?}
%
\en{This seems rather unlikely to be the case,
it is probably rather as colloquially with riding a bicycle.}%
\de{Das scheint eher kaum der Fall zu sein,
es ist wohl eher so wie umgangssprachlich mit Fahrrad fahren.}
%
\en{You can learn it, but afterwards you do not really know what you do.}%
\de{Man kann es lernen, aber danach weiss man nicht wirklich was man tut.}

\en{A concrete example:
Ride quickly on a bicycle
and then—very carefully and gently\,{\color{avant}(!)}—%
pull a little bit on the left side on the handlebar,
but really pull only \textsl{horizontally}.}%
\de{Ein konkretes Beispiel:
Mit dem Fahrrad schnell fahren
und dann—ganz vorsichtig und sanft\,{\color{avant}(!)}—%
ein wenig links am Lenker ziehen,
aber wirklich nur \textsl{horizontal} ziehen.}
%
\en{This is what one would naively think what one does
when one wants to take a left turn.}%
\de{Das ist ja, was man naiv als das betrachten würde, was man tut,
wenn man eine Kurve nach links machen will.}
%
\en{But this is not what happens experimentally,
instead rather a force results
that wants to tilt the bicycle to the right\,{\color{avant}(!)}%
—hence please take caution if you try!}%
\de{Aber das ist nicht, was experimentell passiert,
sondern es resultiert eher eine Kraft,
die das Fahrrad nach rechts\,{\color{avant}(!)} kippen will%
—daher Vorsicht beim Ausprobieren!}

\en{What you have to do instead to take a left turn,
is to also press somewhat onto left side of the handlebar \textsl{vertically from the top},
which has physically to do
with the fact that the rolling wheels are also a spinning top.}%
\de{Was man stattdessen tun muss, für eine Linkskurve,
ist links auch etwas \textsl{vertikal von oben} auf den Lenker drücken,
was physikalisch damit zu tun hat,
dass die rollenden Räder physikalisch rotierende Kreisel sind}
%
\en{But what is essential in this example,
is that a trained neural network does not imply
that the laws of the outer world
are somehow analytically accessibly stored in the head.}%
\de{Aber wesentlich bei diesem Beispiel ist,
dass ein trainiertes neuronales Netzwerk nicht bedeutet,
dass die Gesetze der Aussenwelt
irgendwie analytisch zugänglich im Kopf gespeichert sind.}
%
\en{In the head there is thus rather
an often just as incomprehensible \textsl{copy} of the world,
not an analytical model of it.}%
\de{Im Kopf existiert eben auch eher eine
oft genauso unverständliche gespiegelte \textsl{Kopie} der Welt,
nicht ein analytisches Modell davon.}

\en{More psychologically speaking, unconscious content in the brain would
often not be present in analytically resolved form.}%
\de{Etwas psychologischer gesprochen, wären unbewusste Inhalte im Gehirn
oft gar nicht wirklich analytisch aufgelöst vorhanden.}
%
\en{A trauma would have rather simply “burnt” itself into the structure of the brain
than that the brain would have understood its structure.}%
\de{Ein Trauma hätte sich eher einfach in die Struktur des Gehirns “eingebrannt”,
als dass seine Struktur verstanden wäre.}
%
\en{In this sense, it is probably often not correct to speak
about bringing up unconscious content into consciousness.}%
\de{In diesem Sinn kann man dann wohl oft auch nicht davon sprechen,
dass man unbewusste Inhalte ins Bewusstsein hervorholen kann.}
%
\en{I would rather be so that hypotheses about the inner structure
would lead to an inner reaction
whenever they mirror the inner structure well.}%
\de{Es wäre dann eher so, dass Hypothesen über die innere Struktur
zu einer inneren Reaktion führen würden,
wenn sie sich in den inneren Struktur gut spiegeln.}
%
\en{That would thus not be much different
from how a scientist postulates hypotheses about the outer world
and then compares them experimentally with it.}%
\de{Das wäre also nicht unähnlich dazu,
wie ein Wissenschaftler Hypothesen über die äussere Welt postuliert
und sie dann experimentell damit vergleicht.}

\en{I hope this was now not to complicated to understand.}%
\de{Ich hoffe, das war jetzt nicht zu kompliziert zu verstehen.}
%
\en{Brief, the brain often rather mirrors the world,
creates a copy, than it really understands it.}%
\de{Kurz gesagt, spiegelt eben das Gehirn meist eher die Welt,
macht sich eine Kopie, als dass es sie wirklich versteht.}
%
\en{That way also structures get into the brain
which the person cannot consciously understand.}%
\de{Damit kommen auch Strukturen ins Gehirn,
die der oder die Betreffende nicht bewusst verstehen.}
%
\en{This could, by the way, even go so far
that laws of nature of which no scientist is yet aware
would be mirrored inside, too.}%
\de{Das könnte übrigens sogar so weit gehen,
dass auch Naturgesetze, die heute noch kein Wissenschaftler kennt,
auf diese Weise auch mit hinein gespiegelt würden.}

\en{But isn’t one person alone and abstractly “the world” rather boring\,?}%
\de{Aber ist eine Person allein und abstrakt “die Welt”$\!$ nicht langweilig\,?}
%
\en{Let’s look at two lovers instead, and what maybe goes on in their heads.}%
\de{Schauen wir doch zwei Verliebte an, und was da vielleicht in deren Köpfen so abläuft.}

\en{\subsection{Love}}%
\de{\subsection{Liebe}}

\en{Is love a real connection between two people\,?}%
\de{Ist Liebe eine echte Verbindung zwischen zwei Menschen\,?}

\en{Of course, it often appears to be so, for example,
when the loved one calls you exactly when you think about him or her.}%
\de{Natürlich erscheint es oft so, zum Beispiel,
wenn der oder die Geliebte genau dann anruft, wann man an ihn oder sie denkt.}
%
\en{Only, scientifically no connection is possible
when, for example, the two lovers work
at different places in the city during the day,
and they do not use technical devices (e.g.\ cell phones)
in order to communicate with each other.}%
\de{Nur,
naturwissenschaftlich ist keine Verbindung möglich,
wenn zum Beispiel die zwei Verliebten tagsüber
an verschiedenen Orten in einer Stadt arbeiten
und sie keine technischen Hilfsmittel (z.B.\ Handy) verwenden,
um miteinander zu kommunizieren.}

\vspace{2mm}\hspace{5mm}
\noindent
\includegraphics[scale=0.12]{i-lovers.jpg}

\en{It could of course be that today’s science is wrong in that respect,
resp.\ that such connections really exist but could not be confirmed, yet.}%
\de{Es könnte natürlich sein, dass die heutige Naturwissenschaft da falsch liegt,
bzw.\ etwas, das tatsächlich existiert, noch nicht nachweisen konnte.}
%
\en{But I will totally exclude this for the moment,
since how brains work alone can already explain a lot.}%
\de{Aber das werde ich im Moment erst mal ganz ausklammern,
da schon rein wie Gehirne funktionieren vieles erklären könnte.}
%
\en{But I will come back to such possibilities towards the end of this text.}%
\de{Ich komme aber gegen Ende dieses Textes wieder kurz auf solche Möglichkeiten zurück.}

\en{In any case, the two lovers of the example above
will usually still feel clearly in love and connected
when physically separated during the day.}%
\de{Jedenfalls fühlen sich die zwei Verliebten im obigen Beispiel
doch meist auch tagsüber physisch getrennt noch ganz klar
verliebt und miteinander verbunden.}
%
\en{Is their love thus only purely an illusion,
which only exists inside the respective head of each lover\,?}%
\de{Ist ihre Liebe also nur eine reine Illusion,
die je nur im jeweiligen Kopf der Verliebten vorhanden ist\,?}
%
\en{Would love maybe even only be an individual illusion,
in each of them separately, when they are physically together\,?}%
\de{Wäre die Liebe vielleicht selbst dann nur eine individuelle Illusion,
je in jedem der beiden separat, wenn sie physisch zusammen sind\,?}

\en{Well, when you love somebody, you usually like to fill your brain
with any available impressions from that person.}%
\de{Nun, wenn man jemanden liebt, dann saugt man sich gerne sozusagen das Gehirn voll
mit allen verfügbaren Eindrücken zu dieser Person.}
%
\en{That way inside a mirror image of the person emerges,
which probably even also encompasses a lot
which oneself does not consciously understand,
and also the loved person not necessarily consciously knows or understands,
but which will be stored in the structures in the brain
in a rather unconsciously mirrored way.}%
\de{So entsteht im Innern ein Spiegelbild der Person,
das eben wohl sogar auch sehr vieles umfasst,
das man selbst nicht bewusst versteht,
und auch die geliebte Person selbst nicht umbedingt bewusst weiss oder versteht,
das sich aber in den Gehirnstrukturen eher unbewusst gespiegelt ablegt.}
%
\en{That way one could thus, for example,
possibly sometimes also instinctively predict in total isolation
when the loved one will call.}%
\de{So könnte man dann z.B.\ eben,
je nachdem, auch rein isoliert instinktiv vorausberechnen,
wann der oder die Geliebte anruft.}

\en{In network technology there is the notion of a “store and forward” network,
a network in which information cannot flow all the time,
but only at certain times, and is stored locally in between,
just like when the two lovers meet again in the evening after work
and talk to each other, and so on.}%
\de{Es gibt in der Netzwerktechnik den Begriff eines “store and forward” Netzwerks,
von einem Netzwerk, wo die Informationen nicht jederzeit fliessen können,
sondern nur zu bestimmten Zeiten, und dazwischen lokal zwischengespeichert werden,
wie eben, wenn sich zwei Verliebte am Abend nach der Arbeit wieder sehen
und miteinander sprechen, und so weiter.}
%
\en{But it remains a network,
as long as the two keep exchanging information again and again.}%
\de{Aber es bleibt ein Netzwerk,
solange die zwei sich immer wieder austauschen.}

\en{But so far this does maybe not fully mirror what happens with lovers, yet%
—or also in families, and less intensively with friends and acquaintances.}%
\de{Aber soweit spiegelt das vielleicht noch nicht ganz, was bei Verliebten geschieht%
—oder auch in der Familie, und weniger intensiv mit Freunden und Bekannten.}
%
\en{In principle, the two brains of the two lovers connect and form a single brain.}%
\de{Im Prinzip verbinden sich bei Verliebten zwei Gehirne zu einem Gehirn.}
%
\en{Hence almost certainly also superordinate structures emerge,
which overlap between the physical vessels in the two heads,
thus forming a larger neural network
than could exist in a single brain.}%
\de{Es entstehen also fast sicher dabei auch übergeordnete Strukturen,
die übergreifen zwischen den physischen Behältern in den Köpfen,
also ein grösseres neuronales Netzwerk
als in einem einzigen Gehirn existieren könnte.}

\en{Such a larger compound of nerve cells could in principle be able
to develop independent wishes, dreams, thoughts, etc.,
hence a relationship could go beyond
what the two lovers would be able to fully capture individually.}%
\de{Ein solch grösserer Verbund von Nervenzellen könnte im Prinzip fähig dazu sein,
eigenständige Wünsche, Träume, Gedanken, Ängste, usw.\ zu entwickeln,
also dass eine Beziehung über das hinausginge,
was die beiden Verliebten einzeln vollständig erfassen könnten.}
%
\en{This mirrors maybe already often how it is in a relation:
often beautiful, but analytically often not fully seizable.}%
\de{Das spiegelt vielleicht schon oft, wie es in einer Beziehung ist:
oft schön, aber analytisch oft nicht voll erfassbar.}
%
\en{In a way, you can only decide
whether you want to stay in a relation or not,
but not fundamentally change its nature.}%
\de{In gewisser Weise, kann man sich nur entscheiden,
ob man in einer Beziehung bleiben will oder nicht,
aber ihre Natur nicht wesentlich verändern.}

\en{This has now, of course, been quite speculative in detail.}%
\de{Das war jetzt natürlich im Detail unmittelbar recht spekulativ.}
%
\en{The brains of the lovers would still be
comparably more separated from each other that the nerve cells
in the individual brains from each other.}%
\de{Die Gehirne der zwei Verliebten wären ja doch noch
vergleichsweise viel stärker voneinander getrennt als die Nervenzellen
in den einzelnen Gehirnen untereinander.}
%
\en{And yet, as a “store and forward” network,
and by storing most shared information in parallel on both sides,
the above possibility still seems, to a certain degree, most plausible to me.}%
\de{Und doch, eben als “store and forward” Netzwerk,
und indem die meiste geteilte Information auf beiden Seiten parallel gespeichert wäre,
erscheint mir das obige eben doch zu einem gewissen Grad am plausibelsten.}

\en{If two lovers were separated a long time from each other,
many things could develop separately into different directions,
but not necessarily, if the two really hang on to it.}%
\de{Wenn zwei Verliebte lange Zeit voneinander getrennt wären,
könnte sich einiges getrennt anders entwickeln,
aber auch nicht notwendigerweise, wenn beide sehr daran hängen.}
%
\en{Hence it would be difficult in practice to distinguish experimentally
whether the two are really permanently connected or not,
since both possibilities would manifest almost identically.}%
\de{Daher wäre es wohl in der Praxis schwierig, experimentell zu unterscheiden,
ob die zwei tatsächlich permanent miteinander verbunden sind oder nicht,
da sich beide Möglichkeiten fast gleich äussern würden.}

\en{\subsection{Collective beings}}%
\de{\subsection{Kollektive Wesen}}

\en{\hspace{5mm}
\noindent
\includegraphics[scale=0.12]{i-nested.jpg}}

\en{\noindent
If you now extrapolate such connections created from mutual mirroring,
like between two lovers, to more people,
like family, acquaintances, village, city, region, country,
even the whole earth, including also many animals,
different “collective brains” would emerge
at nested scales.}%
\de{Wenn man nun solche durch gegenseitige Spiegelung entstandene Verbindungen
wie zwischen zwei Verliebten ausdehnt auf mehr Leute,
wie Familie, Bekanntenkreis, Dorf, Stadt, Region, Land,
ja die ganze Erde, auch inklusive vieler Tiere,
dann entstünden daraus verschiedene “kollektive Gehirne”
in verschachtelten Grössen.}
%
\en{What would hold these compounds of brains together
would be, depending on how you look at it,
the power of love or mutual mirroring of each other,
just like two lovers.}%
\de{Zusammengehalten würden diese Verbunde von Gehirnen,
je nachdem wie man es betrachtet,
durch die Kraft der Liebe oder durch ein gegenseitiges Spiegeln untereinander,
also genauso wie zwei Verliebte.}
%
\en{There would thus be a collective brain for each family,
then, building on that, one per community,
and so on, up to country and earth,
while, of course, these entities would overlap
in many and diverse ways in practice.}%
\de{Es gäbe dann kollektive Gehirne zu jeder Familie,
daraus aufbauend eins pro Gemeinde,
und so weiter bis zu Land und Erde,
wobei sich die Dinge in der Praxis natürlich
auch vielfach und vielfältig überschneiden würden.}

\de{\vspace{1.5mm}%
\hspace{5mm}
\noindent
\includegraphics[scale=0.12]{i-nested.jpg}}

\en{The idea is now again
that such collective brains would a priori be quite able
to have independent thoughts and feelings,
hence could feel joy, fear and anger,
could have plans, dreams and a will, etc.%
—simply everything that also a single human being is able to think and feel.}%
\de{\vspace{-0.5mm}
Die Idee ist nun wiederum,
dass solche kollektiven Gehirne a priori durchaus fähig
zu eigenständigen Gedanken und Gefühlen sein könnten, 
also Freude, Angst und Wut empfinden könnten,
Pläne, Träume und einem Willen haben könnten, usw.%
—einfach alles was auch ein einzelner Mensch denken und empfinden kann.}
%
\en{But it could also go beyond that,
because more connected nerve cells with more stored information
would potentially be, just like in a relation of two lovers, a “superbrain”,
which would be able to have thoughts which a single human could never grasp,
just like a single nerve cell in the human brain
would hardly ever be able to really grasp
the thoughts which it helps to process in the human brain.}%
\de{Aber es könnte auch darüber hinaus gehen,
denn mehr verknüpfte Nervenzellen mit mehr gespeicherter Information
wäre potentiell eben, wie bei einer Beziehung von zwei Verliebten, ein “Superhirn”,
welches zu Gedanken fähig sein könnte, die ein einzelner Mensch nie erfassen könnte,
ganz ähnlich wie eine einzelne Nervenzelle im menschlichen Gehirn
kaum je in der Lage sein kann,
die Gedanken, die es zu verarbeiten hilft, wirklich zu erfassen.}

\en{This is maybe best conveyed as follows.}%
\de{Das kann man sich vielleicht wie folgt am Besten veranschaulichen.}
%
\en{Ants often form trails,
which connect sources of food with their nest.}%
\de{Ameisen bilden ja oft Strassen,
die Nahrungsquellen mit dem Bau verbinden.}
%
\en{Only, the individual ant does not really know
that there is a trail, it simply follows the chemical scents,
and, if you observe it, often not in a straight line, as one might think,
but instead with a lot of going left and right,
and sometimes also with shortly turning back.}%
\de{Nur weiss die einzelne Ameise nicht wirklich,
dass es eine Strasse gibt, sie folgt einfach der chemischen Duftspur,
und, wenn man das beobachtet, im Detail gar nicht so geradlinig, wie man denken könnte,
sondern mit viel hin und her,
und manchmal auch mit kurz umkehren.}
%
\en{In the small brain of the ant there appears thus to be no concept of a “trail”,
but only that following the chemical scents is good and not following them is bad,
resp.\ probably that the ant typically feels more happy when it follows the scents than not,
hence that the scents makes the ant happy.}%
\de{Im kleinen Gehirn der Ameise scheint es also kein Konzept einer “Strasse” zu geben,
sondern nur, dass der Duftspur folgen gut ist und ihr nicht folgen schlecht,
bzw.\ wohl dass sie sich typischerweise glücklicher fühlt, wenn sie der Spur folgt,
also dass der Duft sie glücklich macht.}

\vspace{2mm}\hspace{5mm}
\noindent
\includegraphics[scale=0.12]{i-ant-trail.jpg}

\en{Of course it is questionable
whether such a compound of brains could really be more intelligent than individual humans,
since the connections between the brains could overall only be much less intensive
than inside a brain between nerve cells.}%
\de{Natürlich ist fraglich,
ob so ein Verbund vom Gehirnen wirklich intelligenter sein könnte als einzelne Menschen,
da ja die Verbindungen zwischen den Gehirnen insgesamt viel weniger intensiv sein könnten
als innerhalb des Gehirns zwischen den Nervenzellen.}
%
\en{But in any case such a collective brain would have a different perspective,
thus something similar to an “ant trail” would be more easily accessible
to the collective brain than to an individual brain,
if only because the “ant trail” is a collective concept.}%
\de{Aber in jedem Fall hätten kollektive Gehirne eine andere Perspektive,
also wäre etwas analoges zu einer “Ameisenstrasse” dem kollektiven Gehirn
viel eher zugänglich als einem einzelnen Gehirn,
nur schon weil die “Ameisenstrasse” eben ein kollektives Konzept ist.}

\en{The analogy with the ants might also mirror
how a collective superbrain might be able to “guide” individual people,
namely with something equivalent to a “scent trail” for the ants.}%
\de{Dieses Bild könnte auch spiegeln,
wie ein kollektives Superhirn die einzelnen Menschen “lenken” könnte,
nämlich mit so etwas wie einer “Duftspur”, wie bei den Ameisen.}
%
\en{More about this fundamental idea later.}%
\de{Zu dieser grundsätzlichen Idee später noch mehr.}

\en{In the immediate sense, the scent trail is created and refreshed by the ants themselves,
i.e.\ the physical environment is definitely also in play with regard to collective beings.}%
\de{Die Duftspur wird ja unmittelbar von den Ameisen selbst gemeinsam gelegt und aufgefrischt,
d.h.\ die physische Umwelt ist durchaus auch bei kollektiven Wesen mit im Spiel.}
%
\en{Already in a single human brain chemistry plays an important role,
and information is also stored outside the body,
in books, photos, films, or also in everyday objects, clothing and architecture,
simply in everything that is created and changed by human beings.}%
\de{Schon im einzelnen menschlichen Gehirn spielt Chemie eine grosse Rolle,
und Informationen werden auch ausserhalb vom Körper gespeichert,
in Büchern, Fotos, Filmen oder auch in Alltagsgegenständen, Kleidung und Architektur,
einfach in allem was von Menschen geschaffen und verändert wird.}
%
\en{That way a single thing, or one replicated into many copies,
can act on many people and help to form them.}%
\de{So kann oft ein einziges oder mehrfach kopiertes Ding
in der Aussenwelt auf viele Menschen gemeinsam wirkten
und sie mit prägen.}
%
\en{Thus collective brains would also be collective living beings with a “body”.}%
\de{Also wären kollektive Gehirne eben auch kollektive Lebewesen mit einem “Körper”.}

\en{Now to various cultural concepts that emerged over millenia and
which strongly resemble ideas of a collective being.}%
\de{Nun zu verschiedenen kulturellen Begriffen, die über die Jahrtausende entstanden sind
und die sehr stark Ideen von kollektiven Wesen ähneln.}

\en{\subsection{Religion}}%
\de{\subsection{Religion}}

\en{The idea of or the belief in higher beings, which are often immortal and invisible,
hence in goddesses and gods, probably exists in humanity already since primeval times.}%
\de{Die Idee,
bzw.\ den Glauben, an übergeordnete Wesen, die oft unsterblich und unsichtbar sind,
also an Göttinnen und Götter, gibt es wohl in der Menschheit schon seit Urzeiten.}
%
\en{A collective being formed by all believers
would probably also live much longer than individual people,
as long as believers keep having faithful offspring.}%
\de{Ein kollektives Wesen gebildet durch alle Gläubigen
wäre wohl auch viel langlebiger als einzelne Menschen,
solange die Gläubigen immerzu wieder gläubigen Nachwuchs haben.}
%
\en{Quite similarly in human brains
nerve cells are replaced with new ones during life,
but personality is still roughly maintained during life.}%
\de{Ganz analog werden im menschlichen Gehirn
Nervenzellen im Laufe des Lebens ersetzt durch neue,
und doch bleibt die Persönlichkeit meist das ganze Leben in etwa erhalten.}
%
\en{And such a collective being would also not be directly visible in the world,
resp.\ it would reflect in almost anything, which would also often fit with deities.}%
\de{Und unmittelbar erkennbar in der Welt wäre so ein kollektives Wesen auch nicht,
bzw.\ würde es sich fast überall spiegeln, was ja auch oft zu Gottheiten passen würde.}

\en{If previous argumentations were accurate,
would there now really be gods,
if in a certain way “only” created by the respective believers\,?}%
\de{Gäbe es nun, wenn die bisherigen Argumentationen tatsächlich zutreffen würden,
also wirklich Götter,
wenn auch in gewisser Weise “nur” durch die jeweiligen Gläubigen erschaffen\,?}
%
\en{The answer would essentially have to be yes.}%
\de{Die Antwort müsste im Wesentlichen ja lauten.}
%
\en{Because, if you admit that individual persons exist,
even if they “only” come to be from single, interconnected neutrons,
then there would also have to be goddesses and gods,
which would “only” come to be from individual, interconnected brains,
resp.\ the neurons in them.}%
\de{Denn wenn man zugesteht, dass es individuelle Personen gibt,
selbst wenn diese “nur” durch einzelne, untereinander vernetzte Neuronen gebildet werden,
dann müsste es auch Göttinnen und Götter, geben,
die “nur” durch individuelle, untereinander vernetzte Gehirne,
bzw.\ den Neuronen darin, gebildet würden.}

\en{Religions can be very helpful, can help believers to
experience life as deeper, more beautiful, richer, more meaningful than it is to non-believers,
not dissimilar to how lovers experience love;
and religions can also be quite generally useful for society and living together.}%
\de{Religionen können sehr hilfreich sein, können das Leben den Gläubigen
als tiefer, schöner, reichhaltiger, sinnvoller erlebbar machen als Ungläubigen,
nicht unähnlich dazu, wie Verliebte die Liebe erleben,
und auch ganz generell in der Gesellschaft und dem Zusammenleben nützlich sein.}
%
\en{Conversely, of course, also many wars and crimes
have come from religious backgrounds.}%
\de{Umgekehrt sind natürlich auch viele Kriege und Verbrechen
aus religiösem Ursprung hervorgegangen.}

\en{Would deities maybe all in all rather be
more like the ones in Greek mythology:
Not always without fail, but also with human traits,
plus maybe even some,
which might even surreally surpass humans, in good and in bad\,?}%
\de{Wären Göttinnen und Götter vielleicht eben doch insgesamt eher so
wie diejenigen in der Griechischen Mythologie:
nicht immer unfehlbar, sondern auch mit menschlichen Zügen,
plus vielleicht sogar noch welchen,
die diese im Guten wie im Schlechten noch surreal übersteigen könnten\,?}

\en{\subsection{Earth soul}}%
\de{\subsection{Erdseele}}

\en{Greek philosopher Plato coined the concept of a world soul
(lat.\,\textsl{anima mundi}, gr.\,\textsl{psyché tou pantós}),
and there are similar concepts
in different cultures.}%
\de{Der Griechische Philosoph Platon prägte den Begriff der Weltseele
(lat.\,\textsl{anima mundi}, gr.\,\textsl{psyché tou pantós}),
und es gibt ähnliche Vorstellungen
in verschiedenen Kulturen.}
%
\en{Behind that concept lies also the fundamental question
of whether the cosmos is overall alive or not.}%
\de{Dahinter liegt auch die grundsätzliche Frage, 
ob der Kosmos insgesamt lebendig ist oder nicht.}
%
\en{According to today’s science there are animals and plants,
plus some other lifeforms,
but a rock would be inanimate,
and also by far the largest part of the cosmos.}%
\de{Nach heutigem wissenschaftlichem Stand gibt es Tiere und Pflanzen,
plus noch einige andere Lebewesen,
aber ein Stein wäre unbelebt,
und auch der weitaus grösste Teil des Kosmos.}

\en{It could, of course, still be so that more things would be alive than assumed today.}%
\de{Trotzdem könnte es natürlich sein, dass mehr Dinge belebt wären, als heute angenommen.}
%
\en{As already mentioned, there are interactions
between living beings and inanimate matter.}%
\de{Wie schon erwähnt, besteht ja eine Wechselwirkung
zwischen Lebewesen und der unbelebten Materie.}
%
\en{Living beings consist apparently
of exactly the same building materials (atoms, etc.) as inanimate matter.}%
\de{Lebewesen bestehen anscheinend
aus genau den gleichen Baustoffen (Atomen, usw.) wie unbelebten Materie.}

\en{But all in all, such a world soul in the larger sense
would require assumptions that would go beyond the ones made so far,
so let me also come back to this toward the end of this text.}%
\de{Aber insgesamt würde eine solche Weltseele im grösseren Sinn
weitergehenden Annahmen erfordern als bisher gemacht,
also auch dazu erst wieder gegen Ende dieses Textes.}
%
\en{A world soul in the sense of a compound of all living beings on this earth
would, however, most likely exist under the assumptions made so far.}%
\de{Eine Weltseele im Sinne des Verbundes aller Lebewesen auf dieser Erde
müsste es aber unter den bisherigen Angaben wohl geben.}
%
\en{I will simply call it “earth soul” in the following.}%
\de{Ich nenne sie im Folgenden einfach “Erdseele”.}

\en{For this earth soul, the self would be earth
and the environment would be the “sky” resp.\ the cosmos around the earth,
with sun, moon, planets and stars.}%
\de{Für diese Erdseele wäre das Selbst die Erde
und die Umwelt wäre der “Himmel” bzw.\ der Kosmos darum herum
mit Sonne, Mond, Planeten und Sternen.}
%
\en{Would this earth soul now simply admire what it sees outside of itself,
and like to mirror itself in it, like in a lover,
or like in a mother or father, as creator\,?}%
\de{Würde nun diese Erdseele einfach das, was sie ausserhalb von sich sieht, bewundern
und sich gerne darin spiegeln, wie in einer Geliebten oder einem Geliebten,
oder wie in einer Mutter oder einem Vater, als Schöpferin oder Schöpfer\,?}

\vspace{2mm}\hspace{25mm}
\noindent
\includegraphics[scale=0.12]{i-earthly-world-soul.jpg}

\en{In any case, in such a rather lonely situation,
without any other inhabited planet in sight,
would there not be a very big wish
that what happens outside in heaven
would also mirror on earth,
if only to feel more connected, less lonely\,?}%
\de{Jedenfalls wäre aus so einer eher einsamen Lage,
ohne einen anderen bewohnten Planeten in Sicht,
wohl zumindest der Wusch sehr gross,
dass sich das, was im Himmel draussen passiert,
auch auf der Erde spiegeln würde,
und sei es nur, um sich verbundener, weniger alleine zu fühlen\,?}

\en{This reminds, of course, already strongly of the astrological mantra “as above so below”.}%
\de{Das erinnert natürlich schon stark an das astrologische Mantra “wie oben so unten”.}
%
\en{But first to another concept,
which is likely quite significant around astrology,
to Jung’s “collective unconscious”.}%
\de{Aber erst noch zu einem weiteren kollektiven Konzept,
das bezüglich Astrologie wohl sehr bedeutsam ist,
zu Jungs “kollektivem Unbewussten”.}

\en{\subsection{Collective unconscious}}%
\de{\subsection{Kollektives Unbewusstes}}

\en{Carl Gustav Jung went beyond Sigmund Freud by postulating
that unconscious processes in the psyche could also be of a collective nature,
probably based on the observation
that certain “archetypical” themes
keep surfacing very similarly again and again even to mutual strangers,
in dreams as well as more seldomly in real-life experiences.}%
\de{Carl Gustav Jung ging über Sigmund Freud hinaus, indem er postulierte,
dass unbewusste Vorgänge in der Psyche auch kollektiver Natur sein können,
wohl aufgrund der Beobachtung,
dass gewisse “archetypische” Themen auch bei einander fremden Menschen
sehr ähnlich immer und immer wieder auftreten,
sowohl in Träumen wie auch seltener in realen Erlebnissen.}

\en{Under the assumption of collective brains,
the collective unconscious would simply be that part of collective thoughts and feelings,
which is (at least most of the time) hidden from individual people,
hence is not conscious to them,
or even not directly stored in individual brains,
but would only indirectly come to be in the collective compound,
like the ant trail, which probably also does not exist in individual ants.}%
\de{Aus der Annahme von kollektiven Gehirnen heraus,
wäre das kollektive Unbewusste einfach der Teil der kollektiven Gedanken und Gefühle,
die den einzelnen Menschen (zumindest meist) verborgen sind,
also dem Individuum nicht bewusst sind
oder sogar gar nicht direkt in individuellen Gehirnen gespeichert,
sondern nur indirekt durch den kollektiven Verbund entstehen würden,
wie die Ameisenstrasse, die es wohl in einzelnen Ameisen auch nicht gibt.}

\en{All people, and also many animals, dream at night in their sleep.}%
\de{Es träumen ja alle Menschen, und auch viele Tiere, in der Nacht im Schlaf.}
%
\en{Could it now maybe even be so
that dreams would reflect collective thoughts more than individual ones\,?}%
\de{Könnte es nun vielleicht sogar sein,
dass Träume eher kollektive Gedanken wiederspiegeln würden als individuelle\,?}
%
\en{Or could it maybe even be
that a collective brain would sort of lay out its plans like a “scent trail” for dreamers,
such that the affected person, after waking up,
would more likely occupy him- or herself with certain themes, or would do things,
which would rather fit the plans of the collective brain\,?}%
\de{Oder könnte es sogar sein,
dass ein kollektives Wesen so seine Pläne quasi als “Duftspur” für die Träumenden auslegen würde,
damit sich der oder die Betroffene nach dem Aufwachen
mit einem gewissen Thema eher beschäftigen würde,
und vielleicht sogar daraufhin andere Dinge tun würde,
die eher den Plänen des kollektiven Wesens entsprechen würden\,?}
%
\en{And similarly with particularly impressive real-life experiences\,?}%
\de{Und ähnlich bei besonders eindrücklichen realen Erlebnissen\,?}

\en{In any case, such a collective unconscious,
or also generally a collective brain,
would often have the character of “fate” or “destiny”, roughly in the sense
in which Liz Greene cites Jung in her book “The Astrology of Fate” with
“Free will is the ability to do gladly that which I must do.”.}%
\de{Jedenfalls hätte so ein kollektives Unbewusstes,
oder auch generell ein kollektives Gehirn,
für die Menschen oft auch den Charakter von “Schicksal”, etwa in dem Sinn,
wie Liz Greene in ihrem Buch “Schicksal und Astrologie” Jung zitiert mit
“Free will is the ability to do gladly that which I must do.”,
also “Freier Wille ist die Fähigkeit, freudig das zu tun, was ich tun muss.”.}

\en{In other words, if you behave
according to the wishes of the collective brains at different scales,
thus family, country, religion, beekeeper club, etc.,
this would be honored by the surroundings with a feeling of happiness in exchange.}%
\de{Mit anderen Worten, wenn man sich so verhält,
wie es den Wünschen der kollektiven Gehirne in verschiedenen Grössenordnungen entspricht,
also Familie, Land, Religion, Bienenzüchterverein, usw.,
dann würde das vom Umfeld mit einem Glücksgefühl im Austausch belohnt.}
%
\en{You would thus be fundamentally free as an individual to do whatever you want,
but, as a social being, you would also respect your surroundings,
and there especially \textsl{not} only what
is \textsl{conscious} to individual people in your surroundings,
but also respect \textsl{unconscious collective wishes},
which could very well be diametrically opposite to immediate conscious surroundings,
for example, as the “black sheep” of a family or a village.}%
\de{Man wäre also daher grundsätzlich als Individuum frei, was man tut,
aber als soziales Wesen würde man auch auf das Umfeld Rücksicht nehmen,
und da insbesondere \textsl{nicht} nur auf das,
was den einzelnen Menschen im Umfeld \textsl{bewusst} ist,
sondern eben auch auf \textsl{unbewusste kollektive Wünsche},
die durchaus dem unmittelbaren bewussten Umfeld völlig entgegenstehend sein könnten,
zum Beispiel als “schwarzes Schaf”.}

\en{The collective unconscious would thus also have a “fated”, guiding side,
resp.\ collective beings would quite generally have a guiding influence
on individuals and also on smaller collective beings.}%
\de{Das kollektive Unbewusste hätte also durchaus auch eine “schicksalshafte”, lenkende Seite,
bzw.\ hätten wohl ganz allgemein kollektive Wesen lenkenden Einfluss
auf Individuen und auch kleinere kollektivere Wesen.}
%
\en{And, of course, collective thoughts
which reflect in dreams
could appear as precognition of the future or of remote events to individuals.}%
\de{Und natürlich könnten kollektive Gedanken,
die sich Träumen etc.\ spiegeln,
den Individuen als Vorhersehung der Zukunft oder von fernen Ereignissen erscheinen.}

\en{All in all,
it is difficult to distinguish between collective “brains”, “beings”, “souls” and “unconsciouses”
without more precise assumptions.}%
\de{Insgesamt ist es schwierig,
ohne präzisere Annahmen
zwischen kollektiven “Gehirnen”, “Lebewesen”, “Seelen” und “Unbewussten” zu unterscheiden.}

\en{\subsection{Astrology}}%
\de{\subsection{Astrologie}}

\en{My ansatz how astrology would work is the following:}%
\de{Mein Ansatz, wie Astrologie funktionieren würde, wäre folgender:}

\begin{quote}
\textsl{\color{xphi}%
\en{Unconsciously all people “believe” in astrology, resp.\ are part of a collective brain
that believes in astrology, resp.\ at least considers it useful and precious.}}%
\de{Unbewusst “glauben”\,alle Menschen an die Astrologie, bzw.\ sind Teil eines kollektiven Gehirns,
welches an die Astrologie glaubt, bzw.\ welches sie zumindest als nützlich und wertvoll erachtet.}
\end{quote}

\noindent
\en{Astrology, resp.\ its different forms in different cultures,
would thus be a view that the earth soul, resp.\ its smaller collectives,
would have of the world, and which they would let influence individuals.}%
\de{Astrologie, bzw.\ seine verschiedenen Ausprägungen je nach Kulturkreis,
wären also eine Vorstellung, die die Erdseele, bzw.\ ihre kleineren Kollektive,
von der Welt hätten, und sie auf ihre Individuen wirken liessen.}
%
\en{Immediately the strongest influence would thus come from the astrology of one’s own culture,
from other astrologies rather less, while, of course, nowadays cultures often also mix.}%
\de{Unmittelbar den stärksten Einfluss hätte also die Astrologie des eigenen Kulturkreises,
andere Astrologien eher weniger, wobei sich natürlich heutzutage Kulturen oft auch vermischen.}

\en{Many modern people will now probably ask:
Why would such an archaic belief
have persisted also in all the many people
who consciously think so little of astrology
and often know almost nothing about it in detail\,?}%
\de{Viele moderne Menschen werden jetzt wohl fragen:
Wieso hätte sich ein solch archaischer Glaube
auch in all den vielen Leuten erhalten,
die bewusst gar nichts von Astrologie halten,
und im Detail oft fast gar nichts darüber wissen\,?}
%
\en{What exactly would be useful or meaningful in that\,?}%
\de{Was genau wäre daran nützlich oder sinnvoll\,?}

\en{Maybe primarily this:}%
\de{Vielleicht vor allem dies:}
%
\en{Thanks to astrology it would be achieved
that also in small groups of people there would be different characters,
with different ways of approaching the tasks
that live poses every day.}%
\de{Dank der Astrologie würde man erreichen,
dass sich auch in kleinen Gruppen von Menschen verschiedene Charaktere befänden,
mit verschiedenen Herangehensweisen an die Aufgaben,
die die Welt tagtäglich stellt.}
%
\en{Since then different approaches would be tried,
on average presumably a solution would be found more quickly
than if people would be considerably more similar to each other.}%
\de{Indem nun verschiedene Herangehensweisen versucht würden,
fände sich vermutlich im Schnitt jeweils schneller eine Lösung,
als wenn die Menschen einander deutlich ähnlicher wären.}
%
\en{Astrology would thus have an evolutionary advantage in the sense of Darwin;
this is why it would also even have survived Enlightenment almost unperturbedly,
as far as it concerns the collective, unconscious part.}%
\de{Astrologie hätte also einen evolutionären Vorteil im Sinn von Darwin;
deshalb wohl hätte sie auch selbst die Aufklärung fast unbeirrt überdauert,
soweit es den kollektiven, unbewussten Teil betrifft.}

\en{In addition, during the normal course of a year,
for each month the assigned star sign and its attitude towards life
would fit well with activities
in a primarily traditional agricultural environment,
which dominated e.g.\ in Europe during centuries.}%
\de{Zudem würde im Jahresverlauf jeden Monat das
astrologisch zugeordnete Sternzeichen und dessen Lebensgefühl
gut zu Tätigkeiten
in einem vorwiegend traditionellen landwirtschaftlichen Umfeld passen,
welches ja z.B.\ in Europa viele Jahrhunderte über vorherrschte.}
%
\en{For example, towards the end of summer (Virgo)
people would like to work carefully and precisely, and sort things,
as in the past often useful for bringing in the harvest,
and then, at the beginning of autumn (Libra),
they would rather like to exchange parts of the harvest with others in trade,
in order to obtain balanced stocks of goods for the winter.}%
\de{Zum Beispiel würden gegen Ende Sommer (Jungfrau),
die Leute gerne sorgfältig und genau arbeiten, und sortieren,
wie in der Vergangenheit oft sinnvoll für das Einbringen der Ernte,
und dann Anfang Herbst (Waage)
eher lieber etwas spekulativer die Ernte mit anderen Menschen im Handel austauschen,
um so zu ausgeglichenen Vorräten für den Winter zu kommen.}
%
\en{Until recently, this would thus have been an additional evolutionary advantage,
at least compared to other collective views
that would mirror nature less directly.}%
\de{Das wäre also bis vor kurzem ein weiterer evolutionärer Vorteil gewesen,
wenigstens gegenüber anderen kollektiven Vorstellungen,
die die Natur weniger direkt spiegeln würden.}
%
\en{Of course, this is only true on the northern hemisphere and with Western Astrology,
not e.g.\ with sidereal Indian astrology.}%
\de{Das stimmt natürlich nur auf der Nordhalbkugel und mit Westlicher Astrologie,
nicht z.B.\ mit siderischer Indischer Astrologie.}

\en{With fate it would be more or less
like quoted from Jung above:
People would be fundamentally free as individuals,
but, as social beings,
they would be driven by often unconscious collective thoughts and wishes,
so that, wanting to feel happy and fulfilled,
they would still most often find their way in life
on the paths laid out by astrology,
almost like the ants on their ant trail,
with often as much back and forth,
and sometimes even going the opposite way.}%
\de{Mit dem Schicksal wäre es mehr oder weniger so,
wie von Jung weiter oben zitiert:
Die Menschen wären zwar grundsätzlich frei als Individuen,
aber als soziale Lebewesen wären sie getrieben
von oft unbewussten kollektiven Gedanken und Wünschen,
so dass, da sie sich glücklich und erfüllt fühlen möchten,
dann doch meist auf von der Astrologie vorgezeichneten Pfaden
ihren Lebensweg finden würden,
fast wie die Ameisen auf der Ameisenstrasse,
mit ebenso viel hin und her,
und sogar manchmal in die umgekehrte Richtung laufen.}

\en{But where would the stars be in that\,?}%
\de{Aber wo blieben da die Sterne\,?}
%
\en{Well, in this picture they would in the immediate sense actually have no influence,
instead “only” collective views about them,
which do not always mirror the sky accurately.}%
\de{Nun, die hätten in diesem Bild hier unmittelbar tatsächlich keine Wirkung,
sondern “nur” kollektive Vorstellungen davon,
die nicht immer den Himmel getreu spiegeln.}
%
\en{From a collective perspective, the earth soul, or parts of it,
would very well have its views about cosmos and reflecting it,
but it could also be wrong at times.}%
\de{Aus kollektiver Sicht, hätte die Erdseele, oder Teile davon,
wohl durchaus ihre Vorstellungen zum Kosmos und einer Spiegelung davon,
könnte sich aber auch täuschen.}

\en{The prime example for this is planet Pluto,
which had only been considered a planet during a certain time, from 1930 until 2006,
when it has been, scientifically consistently, reclassified,
to a so-called “dwarf planet”.}%
\de{Das Paradebeispiel dazu ist der Planet Pluto,
der nur eine gewisse Zeit lang als Planet erachtet wurde, von 1930 bis 2006,
als er, wissenschaftlich schlüssig, umklassiert wurde,
zu einem sogenannten “Zwergplaneten”.}

\en{Now, in the view of astrology,
Pluto would have a strong influence on human fates,
and also on many collective events,
including world politics, and so on.}%
\de{Nun hätte aber in der Sicht der Astrologie
Pluto durchaus einen sehr starken Einfluss auf menschliche Schicksale,
und auch auf viele kollektive Ereignisse,
inklusive Weltpolitik und so weiter.}
%
\en{In my view that was also actually the case in the 20th century,
thus these forces were effectively acting on people,
and probably still continue to do so now, to a somewhat reduced degree.}%
\de{In meiner Sicht war das auch tatsächlich so im 20.\ Jahrhundert,
also diese Kräfte wirkten effektiv auf die Menschen,
und tun es wohl auch zu einem etwas reduzierten Grad weiterhin.}

\en{Pluto was also the first planet discovered in the USA.}%
\de{Pluto war auch der erste Planet, der in den USA entdeckt wurde.}
%
\en{Uranus and Neptune were still discovered in the old world, in Europe.}%
\de{Uranus und Neptun wurden noch in der alten Welt, in Europa, entdeckt.}
%
\en{Hence behind Pluto there is also a lot of the collective
that the USA forms consciously and unconsciously,
which, of course,
also includes many people world-wide beyond the USA.}%
\de{Daher steckt hinter Pluto auch viel des Kollektivs,
das die USA bewusst und unbewusst bildet,
und welches natürlich auch darüber hinaus
viele andere Menschen weltweit mit einschliesst.}
%
\en{Hence it is not astonishing
that exactly scientists from the USA
and other English speaking regions
initially objected most to the idea
that Pluto would now suddenly no longer be a planet.}%
\de{Daher ist nicht erstaunlich,
dass sich gerade Wissenschaftler aus den USA
und anderen Englisch sprechenden Kulturkreisen
zuerst besonders dagegen sträubten,
dass Pluto nun plötzlich kein Planet mehr wäre.}

\en{But I do not want to talk about politics here; instead I just wanted to illustrate
that astrology really has an effect in daily life, at large and small scale,
but also certainly deviates far enough from the reality in cosmos outside of the earth
that an immediate symmetry can rather be excluded.}%
\de{Aber ich will hier nicht über Politik sprechen, sondern wollte hiermit nur illustrieren,
dass Astrologie zwar wirkt im täglichen Leben, auf kleiner und grosser Skala,
aber durchaus so weit von der Realität im Kosmos draussen abweicht,
dass eine unmittelbare Symmetrie eher ausgeschlossen werden kann.}

\en{Birds of pray can, by the way, see planet Uranus in the sky with the naked eye,
and possibly also Neptune or the asteroid Ceres.}%
\de{Raubvögel könnten übrigens von blossem Auge den Planeten Uranus im Himmel sehen,
und möglicherweise auch Neptun oder den Asteroiden Ceres.}
%
\en{Had the earth soul maybe already been conscious of these celestial bodies,
only Pluto came as a surprise\,?}%
\de{War sich die Erdseele also dieser Himmelskörper schon lange bewusst,
nur Pluto kam überraschend\,?}
%
\en{But even then, for Pluto there would still have been the freedom
to steer, which name the new planet gets, and thus a meaning
that could still fit with some events
that happened \textsl{before} its discovery\,?}%
\de{Aber auch dann wäre bei Pluto noch die Freiheit geblieben,
zu steuern, welchen Namen der neue Planet erhält, und damit eine Bedeutung,
die so trotzdem noch zu einigen Ereignissen
\textsl{vor} seiner Entdeckung passen würde\,?}

\en{And would the three pyramids at Giza maybe,
as suspected by Robert Bauval,
intentionally mirror the three stars of the Belt of Orion (Osiris),
only that this had not been conscious to the ancient Egyptians,
but “only” unconsciously collectively to all Egyptians,
and hence the sky is not perfectly mirrored\,?}%
\de{Und sollten die drei Pyramiden in Gizeh vielleicht,
wie von Robert Bauval vermutet,
tatsächlich die drei Sterne des Gürtels des Orion (Osiris) spiegeln,
nur dass dies den alten Ägyptern gar nicht bewusst war,
sondern “nur” unbewusst kollektiv allen Ägyptern,
und daher der Himmel nicht perfekt gespiegelt ist\,?}

\vspace{2mm}\hspace{6mm}
\noindent
\en{\includegraphics[scale=0.10]{i-orion-en.jpg}}%
\de{\includegraphics[scale=0.10]{i-orion-de.jpg}}

\en{There are more examples,
where astrology does not truly mirror the sky,
like that the moon is typically drawn geocentrically on horoscope charts,
hence where it would be seen from the center of earth,
not from the respective point on the surface of earth.}%
\de{Es gibt noch weitere Beispiele,
wo die Astrologie den Himmel nicht getreu spiegelt,
wie dass der Mond typischerweise auf Horoskopen geozentrisch eingezeichnet wird,
also dort, wo er vom Erdmittelpunkt her gesehen wäre,
nicht vom jeweiligen Punkt auf der Erdoberfläche aus.}
%
\en{And, of course, the division of the zodiac into 12 segments of equal size,
by now completely separated from constellations due to precession,
is not something that mirrors directly in the sky,
and a division into 12 segments
seems also rather to reflect
the somewhat more than 12 lunar months in a solar year,
than that it would immediately have natural causes.}%
\de{Und natürlich ist die Einteilung des Tierkreises
in 12 gleich grosse Segmente,
mittlerweile durch Präzession von den Sternbildern ganz getrennt,
nicht etwas, das sich direkt im Himmel spiegelt,
und eine Teilung in 12 Abschnitte scheint auch eher
von den etwas mehr als 12 Mondmonaten
in einem Sonnenjahr zu kommen,
als unmittelbar natürliche Ursachen zu haben.}
%
\en{In China there are quite different constellations,
for example, a division into 28 “mansions” on the ecliptic,
where the moon would be visiting a mansion each day of a lunar month.}%
\de{In China gibt es ganz andere Sternbilder,
zum Beispiel eine Einteilung in 28 “Landsitze” auf der Ekliptik,
wobei der Mond jeweils pro Tag im Monat einen der Landsitze besucht.}

\en{Astrologers within a cultural circle usually share many methods and views,
but besides that often also very often use further, quite diverse methods.}%
\de{Astrologen teilen in ihrem Kulturkreis meist viele Methoden und Ansichten,
verwenden daneben aber sehr oft auch weitere, ganz verschiedene Methoden.}
%
\en{Think only of the many different house systems,
or orbs for aspects.}%
\de{Man denke da nur schon an die vielen verschiedenen Häusersysteme,
oder Orben bei Aspekten.}
%
\en{How could such diversity ever mirror people\,?}%
\de{Wie könnte so eine Vielfalt jemals die Menschen spiegeln\,?}

\en{In order that a client goes to a particular astrologer,
he or she would probably have to somehow feel mirrored,
maybe less in the astrologer,
but rather that the astrologer would resemble a desired solution\,?}%
\de{Damit ein Klient zu einem bestimmten Astrologen geht,
müsste er oder sie sich wohl irgendwie gespiegelt fühlen,
vielleicht weniger im Astrologen selbst,
als dass jener einer erwünschten Lösung ähnlich wäre\,?}
%
\en{A client might come from an environment
where mainly the Koch house system would be used.}%
\de{Ein Klient käme z.B.\ aus einem Umfeld,
wo vorwiegend das Koch Häusersystem verwendet wird.}
%
\en{Should an astrological counselor now rather keep using her or his favorite
house system, or in this case rather use Koch houses\,?}%
\de{Sollte nun ein astrologischer Berater stur sein Lieblingshäusersystem verwenden
oder in diesem Fall doch Koch Häuser\,?}
%
\en{Or both\,?}%
\de{Oder beide\,?}

\en{Koch houses would probably better fit the environment of the client,
would thus rather mirror where the collective brains
around his or her environment would want to move the client to.}%
\de{Koch Häuser würden wohl besser zum Umfeld des Klienten passen,
würden also schon eher spiegeln, wohin ihn die kollektiven Gehirne
um sein Umfeld herum bewegen wollten.}
%
\en{Conversely, the individual has likely still also a free will,
in order to at least be able to switch surroundings,
sort of like changing the “tribe”
into a cultural environment with a different house system,
where then maybe different collective brains
could make a rather more desired life possible.}%
\de{Umgekehrt hat das Individuum eben wohl doch auch einen freien Willen,
um zumindest das Umfeld wechseln zu können,
also quasi den “Stamm” zu wechseln,
in ein kulturelles Umfeld mit anderem Häusersystem,
wo dann vielleicht andere kollektive Gehirne
ein eher erwünschtes Leben ermöglichen könnten.}
%
\en{Hence also here client and astrologer would have fundamentally much room to move,
to “gladly do that which they must do”.}%
\de{Also auch hier hätten Klient und Astrologe grundsätzlich viel Freiraum,
“freudig das zu tun, was sie tun müssen”.}

\en{Certain methods and views in astrology would be quite generally valid,
other methods only in surroundings where they would have supporters.}%
\de{Gewisse Methoden und Ansichten in der Astrologie wären also ziemlich allgemeingültig,
andere Methoden eher nur in dem Umfeld, wo sie Anhänger hätten.}
%
\en{This would then often be quite similar
to going to a general psychologist of a certain school of thought
or to a priest of a certain religion.}%
\de{Das wäre dann oft ganz ähnlich,
wie ob man zu einem allgemeinen Psychologen einer bestimmten Denkrichtung geht
oder einem Priester einer bestimmten Religion.}
%
\en{Also there a lot would often only help if it “fits” the client.}%
\de{Auch da würde vieles oft nur helfen, wenn es zum Klienten “passt”.}

\en{\subsection{Summary}}%
\de{\subsection{Zusammenfassung}}

\en{This has thus far been quite a conservative approach to these things,
resp.\ it was conservative concerning the physical assumptions,
thus for example without natural communication channels
between brains at large distance,
short, entirely from the viewpoint of the current state of natural sciences.}%
\de{Dies war soweit ein recht konservativer Zugang zu diesen Dingen,
bzw.\ konservativ soweit es die physikalischen Annahmen betrifft,
also zum Beispiel ohne natürliche Kommunikationskanäle
zwischen Gehirnen auf grosse Distanz,
kurz, ganz aus der Sicht des heutigen Standes der Naturwissenschaft.}

\en{This resulted roughly in the following picture, which seems to be qualitatively plausible,
but, of course, so far quantitatively, and whether it is correct at all, remains formally unproven:
There would be collectives of two and more brains
with independent thoughts, wishes, dreams, feelings, etc.,
and these would influence the fates of people on earth.}%
\de{Daraus ergab sich grob folgendes Bild, das zwar soweit qualitativ einleuchtend scheint,
aber natürlich soweit quantitativ, und ob überhaupt zutreffend, noch formal unbewiesen ist:
Es gäbe Kollektive von zwei bis vielen Gehirnen
mit eigenständigen Gedanken, Wünschen, Träumen, Gefühlen, usw.,
und diese würden das Schicksal der Menschen auf der Erde beeinflussen.}
%
\en{A direct influence of planets and stars would however not immediately exist;
in particular, there are clear indications that the majority of causes of astrology
would be purely located in views down on earth,
but would often also be helpful in living together every day.}%
\de{Einen direkten Einfluss der Planeten und Sterne würde es aber unmittelbar nicht geben;
insbesondere gibt es sogar klare Indizien, dass der grosse Hauptteil der Ursachen der Astrologie
rein in Vorstellungen unten auf der Erde beheimatet wäre,
die allerdings dort oft auch sinnvoll wären im täglichen Zusammenleben.}

\en{\subsection{Tiny outlook}}%
\de{\subsection{Kleiner Ausblick}}

\en{Even if the earth soul, as defined further above,
could sometimes be completely wrong, like with Pluto as a planet,
it could still have mirrored certain laws of the cosmos,
even some which are not known or conscious to anybody,
similar to how a person who can ride a bicycle
has unconsciously mirrored physical laws into her or his brain.}%
\de{Auch wenn die Erdseele, wie weiter oben definiert,
auch mal ganz falsch liegen könnte, wie mit Pluto als Planeten,
könnte sie in sich eben doch auch gewisse Gesetze des Kosmos gespiegelt haben,
durchaus auch solche, die heute niemandem bekannt oder bewusst sind,
ähnlich wie ein Mensch, der Fahrrad fahren kann,
in seinem Hirn physikalische Gesetze unbewusst gespiegelt hat.}

\en{How it appears at the moment,
the milky way is not buzzing with planets with intelligent life on them,
which would emit radio signals, etc.}%
\de{Wie es momentan scheint,
wimmelt es ja in unserer Milchstrasse nicht von Planeten mit intelligentem Leben,
welche Radiosignale etc.\ aussenden würden.}
%
\en{Is thus the architecture of our own solar system so special
that in it also part of the secret of life reflects\,?}%
\de{Ist also die Architektur unseres eigenen Sonnensystems so speziell,
dass sich darin auch ein Teil des Geheimnisses des Lebens spiegelt\,?}
%
\en{Without a relatively large moon for such a relatively small planet as earth,
the earth’s axis would not be stable and life would presumably never have emerged.}%
\de{Ohne z.B.\ einen relativ grossen Mond für so einen relativ kleinen Planeten wie die Erde,
wäre die Erdachse nicht stabil und Leben wäre vermutlich nicht entstanden.}

\en{Quantum mechanics knows entanglement of quantum states even across great distances,
as, for example, in the well-known thought experiment of Einstein, Podolski and Rosen (short EPR).}%
\de{Die Quantenmechanik kennt ja Verschränkungen von Quantenzuständen auch über grosse Distanzen,
zum Beispiel im wohlbekannten Gedankenexperiment von Einstein, Podolski und Rosen (kurz EPR).}
%
\en{Especially in the “New Age” movement there are many approaches
in which the whole world would be interconnected that way,
without, however, getting fully specific.}%
\de{Gerade im “New Age” Umfeld gibt es viele Ansätze,
wo die ganze Welt derart miteinander verwoben wäre,
ohne allerdings wirklich konkret zu werden.}
%
\en{Or Jung, who at the beginning of the 1950s,
at that time often in close contact with physicist Wolfgang Pauli,
postulated the concept of an “acausal synchronicity”,
was probably also substantially influenced by thoughts about such quantum effects.}%
\de{Oder Jung, der Anfang der 1950er Jahre,
damals oft in engem Kontakt mit dem Physiker Wolfgang Pauli,
das Konzept einer “akausalen Synchronizität” postulierte,
war da wohl auch von Gedanken an solche Quanteneffekte wesentlich mit beeinflusst.}

\en{This is a wide field, where I could add quite few more things.}%
\de{Das ist ein weites Feld, wozu ich durchaus noch vieles sagen könnte.}
%
\en{Let it suffice here, that you could then also explain oracles more easily,
hence events where randomness appears to take part, as with Tarot cards
or with the coins or yarrow stalks of the Chinese I Ching.}%
\de{Hier soll mal reichen, dass man dann auch Orakel besser erklären könnte,
also Ereignisse wo der Zufall mitzuspielen scheint, wie bei den Karten beim Tarot
oder bei den Münzen oder Schafgarbenstengeln beim Chinesischen I Ging.}
%
\en{Because otherwise collective brains would “only” have a possibility to influence things
by focussing the people involved in the oracle
after the random outcome on certain aspects of the oracle text,
but there would then be now way to influence the result of the oracle itself.}%
\de{Denn sonst bliebe kollektiven Gehirnen “nur” noch die Möglichkeit Einfluss zu nehmen,
indem sie die beteiligten Menschen nach dem Zufallsresultat
ein wenig auf gewisse Aspekte des Orakelspruchs zu fokussieren würden,
es gäbe aber dann keine unmittelbare Einflussnahme auf den Orakelspruch selbst.}

\en{There would be still another, very simple fundamental explanation for things
which resemble each other in big and small sizes or at the same size at different places,
namely that the same laws of nature could bring forth similar structures
even without immediate connections.}%
\de{Es gäbe noch eine andere, sehr einfache grundsätzliche Erklärung für Dinge,
die sich im Grossen und Kleinen oder in gleicher Grösse an verschiedenen Orten ähneln,
nämlich, dass gleiche Naturgesetze auch ähnliche Strukturen hervorbringen könnten,
ohne dass unmittelbar eine Verbindung bestünde.}
%
\en{This concept is called self-similarity.}%
\de{Dieses Konzept wird Selbstähnlichkeit genannt.}
%
\en{For example in the “Mandelbrot set”,
a mathematical figure that results from a simple equation,
you can find the same structure not only in the large whole (left image),
but also many times in very similar, smaller form,
if you zoom in at the border (example to the right).}%
\de{Zum Beispiel im “Apfelmännchen” (technisch “Mandelbrotmenge”),
einer mathematischen Figur, die sich aus einer einfachen Gleichung heraus ergibt,
findet man das Apfelmännchen nicht nur im grossen Ganzen (Bild links),
sondern auch vielfach sehr ähnlich verkleinert wieder,
wenn man am Rand mehrfach hinein zoomt (Beispiel rechts).}

\en{\vspace{1.2mm}}%
\de{\vspace{2mm}}%
\hspace{5mm}
\noindent
\includegraphics[scale=0.17]{i-mandelbrot.jpg}

\en{\vspace{0.8mm}
A practical idea regarding how to deal scientifically with collective phenomena:
Instead of trying to want to understand them analytically, maybe just try and see
if they can maybe be mirrored in artificial neural networks\,?}%
\de{\vspace{1mm}
Noch eine praktische Idee zum wissenschaftlichen Umgang mit kollektiven Phänomenen:
Anstatt sie analytisch verstehen zu wollen, vielleicht einfach nur schauen,
ob sie sich in künstlichen neuronalen Netzwerken spiegeln liessen\,?}
%
\en{Hence, for example,
feed a neural network with data that emerged at known times at known places,
so that you can also derive astrological information.}%
\de{Also zum Beispiel ein neuronales Netzwerk füttern mit Daten,
die zu bekannten Zeitpunkten an bekannten Orten entstanden sind,
so dass sich daraus auch die astrologischen Informationen ableiten lassen.}
%
\en{If such a neural network would then become able
to derive the creation time of undated data,
or at least limit times significantly,
that would be a proof of astrology.}%
\de{Wenn nun so ein neuronales Netzwerk, fähig würde,
bei undatierten Daten daraus den Zeitpunkt herzuleiten,
oder zumindest wesentlich einzuschränken,
wäre das ein Beweis der Astrologie.}

\en{A key assumption in astrology, namely that the situation when something starts,
like a human life, an organisation, a country, etc.,
would shape its fate,
could apparently not be directly derived as a physical effect
in any of the explanations proposed here.}%
\de{Eine Schlüsselannahme in der Astrologie, nämlich, dass die Situation wenn etwas beginnt,
wie ein Menschenleben, eine Organisation, ein Land, usw.,
sein Schicksal formen würde, liesse sich anscheinend
aus keiner der hier vorgeschlagenen Erklärungen
direkt als physikalischer Effekt herleiten.}
%
\en{Could a key element maybe still be missing\,?}%
\de{Könnte ein Schlüssel\-element vielleicht doch noch fehlen\,?}
