\documentclass[letterpaper]{article}
\pagestyle{empty}
\paperheight=3700mm
\textheight=3700mm
%\textwidth set depending on pdflatex or (experimentally)lualatex
\topmargin=-20mm
\oddsidemargin=25mm

\usepackage[utf8]{inputenc}
\usepackage[spanish,italian,french,ngerman,english]{babel} % last is main
\usepackage{graphicx}
\usepackage{multirow}
\usepackage{xcolor}
\usepackage{contour}
\usepackage{pict2e}
\usepackage{relsize}
\usepackage{amsmath}

\usepackage{iftex}
\ifpdftex
  % stronger fonts

% Find modes.mf, e.g. /usr/local/texlive/2025/texmf-dist/fonts/source/public/modes/modes.mf
%
% $ sudo cp modes.mf modes.mf.orig
%
% Add the following at the start of modes:
%
% mode_def xphi =
%   mode_param (pixels_per_inch, 1200);
%   mode_param (blacker, 1.9); % only difference to 'lexmarkr' (2 there)
%   mode_param (fillin, 0);
%   mode_param (o_correction, 1);
%   mode_common_setup_;
% enddef;
%
% Finally:
%
% $ sudo fmtutil-sys --byfmt mf

\pdfpkresolution=1200
\pdfpkmode={xphi}
\pdfmapfile{}

  \newcommand{\coretextwidth}{85.5mm}
\fi

% experimental, not used to produce the live website...
\ifluatex
  % about same heaviness in pdfs when rasterized in photoshop,
  % but since, unlike the metafont mechanisms I use, fake bold, "bleeds" in all directions,
  % seems heavier at least in core web page images
  \newcommand{\fontbleed}{0.8}
  % paragraphs wider and font looks larger, tried to fix, but then other things change a bit,
  % especially for section headings would have to change back, are now more narrow...
  \newcommand{\fontscale}{0.985}
  \newcommand{\coretextwidth}{85.2mm}
  \usepackage{fontspec}
  % microtype does maybe help and not help, maybe if would allow wider spaces...
  \usepackage{microtype}
  \setsansfont{Latin Modern Sans}[Scale=\fontscale, FakeBold=\fontbleed]
  \setmonofont{Latin Modern Mono}[Scale=\fontscale, FakeBold=\fontbleed]
  % this would be for "new computer modern" (but has many limitations so far)
  %\usepackage[default]{fontsetup}
  %\renewcommand{\familydefault}{\sfdefault}
\fi

\renewcommand{\familydefault}{\sfdefault}

\setcounter{secnumdepth}{-1}

\newcommand{\en}[1]{\iflanguage{english}{#1}{}}
\newcommand{\de}[1]{\iflanguage{ngerman}{#1}{}}
\newcommand{\fr}[1]{\iflanguage{french}{#1}{}}

% a bit less than ~255/256
\definecolor{almostwhite}{gray}{0.996}
\definecolor{xphi}{rgb}{0.0,0.5,0.5}
\definecolor{avant}{rgb}{1,0.5,0.5}

\definecolor{frame}{gray}{0.9}
\definecolor{lightgray}{gray}{0.8}
\definecolor{gray}{gray}{0.5}
\definecolor{darkgray}{gray}{0.3}

\definecolor{darkred}{rgb}{0.8,0.0,0.0}
\definecolor{darkyellow}{rgb}{0.7,0.7,0.0}
\definecolor{darkgreen}{rgb}{0.0,0.55,0.0}
\definecolor{darkviolet}{rgb}{0.5,0,0.5}

\definecolor{darkblue}{rgb}{0,0,0.7}
\definecolor{odyssey}{rgb}{0,0,0.8}
\definecolor{indigo}{rgb}{0.29,0,0.51}
\definecolor{indigoblue}{rgb}{0.1,0,0.6}

\definecolor{saffronback}{rgb}{1.000,0.878,0.627}
\definecolor{saffronfront}{rgb}{0.376,0.125,0.000}

\DeclareRobustCommand{\cometartemisscale}[1]{\includegraphics[scale=#1]{\sourcepath/i-comet.jpg}\hspace{-0.028453em} artemis}
\newcommand\cometartemis{\cometartemisscale{0.018}}
\newcommand\cometartemissection{\cometartemisscale{0.0225}}

\DeclareRobustCommand{\moebius}[1]{\includegraphics[scale=#1]{\sourcepath/i-moebius.jpg}}
\newcommand{\yinyang}{\includegraphics[scale=0.135]{\sourcepath/i-yinyang.jpg}}

\newcommand{\rarr}{\,$\rightarrow$\,}
\newcommand{\lrarr}{\,$\leftrightarrow$\,}

% greek elements
\newcommand{\elfire}{
\begin{picture}(9,6)
  \thicklines
  \put(1,-0.5){\line(1,0){7}}
  \put(1,-0.5){\line(1,1.732){3.5}}
  \put(8,-0.5){\line(-1,1.732){3.5}}
\end{picture}}
%
\newcommand{\elair}{
\begin{picture}(9,6)
  \thicklines
  \put(1,-0.5){\line(1,0){7}}
  \put(1,-0.5){\line(1,1.732){3.5}}
  \put(8,-0.5){\line(-1,1.732){3.5}}
  \put(2.75,1.9){\line(1,0){3.5}}
\end{picture}}
%
\newcommand{\elwater}{
\begin{picture}(9,6)
  \thicklines
  \put(1,5){\line(1,0){7}}
  \put(1,5){\line(1,-1.732){3.5}}
  \put(8,5){\line(-1,-1.732){3.5}}
\end{picture}}
%
\newcommand{\elearth}{
\begin{picture}(9,6)
  \thicklines
  \put(1,5){\line(1,0){7}}
  \put(1,5){\line(1,-1.732){3.5}}
  \put(8,5){\line(-1,-1.732){3.5}}
  \put(2.75,2.5){\line(1,0){3.5}}
\end{picture}}
%
\newcommand{\elhex}{
\begin{picture}(9,6)
  \thicklines
  \put(1,0.5){\line(1,0){7}}
  \put(1,0.5){\line(1,1.732){3.5}}
  \put(8,0.5){\line(-1,1.732){3.5}}
  \put(1,5){\line(1,0){7}}
  \put(1,5){\line(1,-1.732){3.5}}
  \put(8,5){\line(-1,-1.732){3.5}}
\end{picture}}

% i ching trigrams
\newcommand{\trigram}[3]{
\begin{picture}(9,6)
  \linethickness{0.36mm}
  \put(0,5){\line(1,0){#1}}
  \put(5.5,5){\line(1,0){3.5}}
  \put(0,2.5){\line(1,0){#2}}
  \put(5.5,2.5){\line(1,0){3.5}}
  \put(0,0){\line(1,0){#3}}
  \put(5.5,0){\line(1,0){3.5}}
\end{picture}}
\newcommand{\triheaven}{\trigram{5.5}{5.5}{5.5}}
\newcommand{\triearth}{\trigram{3.5}{3.5}{3.5}}
\newcommand{\trithunder}{\trigram{3.5}{3.5}{5.5}}
\newcommand{\triwater}{\trigram{3.5}{5.5}{3.5}}
\newcommand{\trimountain}{\trigram{5.5}{3.5}{3.5}}
\newcommand{\triwind}{\trigram{5.5}{5.5}{3.5}}
\newcommand{\trifire}{\trigram{5.5}{3.5}{5.5}}
\newcommand{\trilake}{\trigram{3.5}{5.5}{5.5}}

% i ching hexagrams
% 1+2 trigrams, 3 rest of line (see e.g. dreams.tex)
\DeclareRobustCommand{\hexagram}[3]{\raisebox{-3pt}{$\overset{\text{${#1}$}}{#2}$\,}#3\vspace{3pt}}

% white-red-black etc.
\DeclareRobustCommand{\outline}[1]{\contour{black}{{\color{white}#1}}}
\DeclareRobustCommand{\white}[1]{\outline{\textbf{#1}}}
\DeclareRobustCommand{\red}[1]{{\color{darkred}\textbf{#1}}}
\DeclareRobustCommand{\black}[1]{\textbf{#1}}
\DeclareRobustCommand{\yellow}[1]{{\color{darkyellow}\textbf{#1}}}
\DeclareRobustCommand{\green}[1]{{\color{darkgreen}\textbf{#1}}}
\DeclareRobustCommand{\violet}[1]{{\color{darkviolet}\textbf{#1}}}
\DeclareRobustCommand{\indigoblue}[1]{{\color{indigoblue}\textbf{#1}}}
\DeclareRobustCommand{\indigo}[1]{{\color{indigo}\textbf{#1}}}

% ELEMENTAL
\newcommand{\ELEMENTAL}{%
\colorlet{contour}{.}\textbf{\color{white}%
\raisebox{+0.001em}{\contour{contour}{E}}%
\raisebox{+0.015em}{\contour{contour}{L}}%
\raisebox{+0.016em}{\contour{contour}{E}}%
\raisebox{+0.023em}{\contour{contour}{M}}%
\raisebox{+0.023em}{\contour{contour}{E}}%
\raisebox{+0.017em}{\contour{contour}{N}}%
\raisebox{-0.020em}{\contour{contour}{T}}%
\raisebox{-0.002em}{\contour{contour}{A}}%
\raisebox{+0.006em}{\contour{contour}{L}}%
}}

% artemis pdf+web icons
\newcommand{\ipdfen}{\includegraphics[scale=0.5]{i-pdf-en.png}}
\newcommand{\ipdfde}{\includegraphics[scale=0.5]{i-pdf-de.png}}
\newcommand{\ipdffr}{\includegraphics[scale=0.5]{i-pdf-fr.png}}
\newcommand{\iweb}{\includegraphics[scale=0.055]{i-web.png}}
\newcommand{\ipdfblueen}{\includegraphics[scale=0.5]{i-pdf-blue-en.png}}
\newcommand{\ipdfbluede}{\includegraphics[scale=0.5]{i-pdf-blue-de.png}}
\newcommand{\ipdfbluefr}{\includegraphics[scale=0.5]{i-pdf-blue-fr.png}}
\newcommand{\iwebblue}{\includegraphics[scale=0.055]{i-web-blue.png}}


\textwidth=\coretextwidth



\begin{document}

\avantgarde

\section{Birth time of Caesar Rodney}

For all that it appears,
Caesar Rodney was born Monday, 7 October 1728 (Julian Calendar),
shortly \textsl{before} midnight (“I believe just about midnight”)
according to the diary of his father who also acted as midwife.
%
A birth on the \textsl{same date} shortly \textsl{after} midnight (i.e.\ almost 24 hours earlier)
as indicated in at least one secondary source
can be excluded based on diary entries for the previous and the following day.
%
Scans and transcripts of the diary entries can be found towards the end of this document.

\subsection{Details}

I am aware of two secondary sources that quote the diary of this father,
Caesar Rodney Sr.,
for the birth time.
%
The first one is:

\begin{quote}
\small\color{darkgray}
\textsl{Caesar Rodney patriot\newline
Delaware’s hero for all times and all seasons\newline
by William P.\ Frank\newline
Delaware American Revolution Bicentennial Commission\newline
1975\newline
\href{http://archives.delaware.gov/eBooks/CeasarRodneyPatriot.pdf}{http://archives.delaware.gov/eBooks/CeasarRodneyPatriot.pdf}}
\end{quote}

\noindent
Frank quotes and interprets the father’s diary as follows (page 9):

\begin{quote}
\small\color{darkgray}
\textsl{Caesar,
our hero,
was born shortly before midnight,
October 7, 1728,
amid unusual circumstances.
%
His father kept a diary,
which indicates that the father acted as midwife.\newline
\mbox{\ \ \ \ }The diary entry reads:
\begin{quote}
“October 7 — Hung some tobacco.
%
Came in, got dinner and killed some squirrels. …
%
About eleven o’clock at night,
my wife awakened me for she was very bad.
%
I got up and sent for ye midwife and women.
%
But before any came, ye child was born and it was a SON.
%
There was no soul with her but myself, being I believe just about midnight.”
\end{quote}
\mbox{\ \ \ \ }Caesar was born on his father’s farm in East Dover Hundred, Kent County,
near the Delaware River,
an area that had always been known as St.~Jones Neck.}
\end{quote}

\newpage

\noindent
The second source is:

\begin{quote}
\small\color{darkgray}
\textsl{A Gentleman as Well as a Whig\newline
Caesar Rodney and the American Revolution\newline
by Jane Harrington Scott\newline
National Society of The Colonial Dames of America in the State of Delaware\newline
University of Delaware Press, 2000}
\end{quote}

\noindent
Harrington Scott quotes and interprets the father’s diary as follows (page 16):

\begin{quote}
\small\color{darkgray}
\textsl{\mbox{\ \ \ \ }Caesar Rodney,
the first child of Caesar Rodney, Sr.\ and Elizabeth Crawford Rodney
was born shortly after midnight, on October 7, 1728.
%
It is thought that his young parents were living in a small house
owned by Elizabeth’s father,
not far from the Rodney farm at Byfield.\newline
%
\mbox{\ \ \ \ }According to Caesar Rodney, Sr.’s diary,
the young father immediately sent for a midwife “and other women,”
but “Before aney came ye Child wass Born and it wass a SON.”
%
As “There was no sole with her but myself—being I believe just about midnight.”
he “ran away for Isabelah Hughes.”
%
Apparently all was well,
for his entry for the following day,
October 8,
tells us that he: 
“Past ye Day away with Eating and Drinking
and at Night I got super Went to Bed fair and Good helth—%
My wife and Child Continues Brave
and well thanks be to God.”}
\end{quote}

\noindent
So two conflicting secondary sources,
birth before resp.\ after midnight,
both on 7 October 1728.

I ordered a transcript of the diary entries for October 6, 7 and 8
from the Historical Society of Pennsylvania.
%
In addition,
I got scans of two pages of the diary including for those three days.
%
The researcher wrote:

\begin{quote}
\small\color{darkgray}
\textsl{[Caesar Rodney Sr.’s diary] is located in the Simon Gratz autograph collection
(collection \#250B) in box 237, folder 17 “Rodney, Caesar Father of Signer.”\newline
%
The journal is written on loose pages,
with just few lines per day.}
\end{quote}

\noindent
Here is a rough transcript (by me) of the diary entries for the three days around birth,
plus at the end of the document scans of the two diary pages at slightly reduced resolution:

\begin{quote}
\small\color{darkgray}
\textsl{————————\newline
Sunday 106 / 6\newline
a.~m. I went to [Daniles?] got brakt then to John\newline
[Harts?] staid tell ye evening then to [M$^\mathrm{\textsf{\footnotesize r}}$ C$^\mathrm{\textsf{\footnotesize d}}$?] my\newline
wife was there then shee and I came home together\newline
we [sate?] sum [vituls?] and at night went to bed.\newline
Fair weather and good helth. ———}
\end{quote}

\newpage

\begin{quote}
\small\color{darkgray}
\textsl{Monday 107 / 7\newline
a~m. I went to [M$^\mathrm{\textsf{\footnotesize r}}$ C$^\mathrm{\textsf{\footnotesize d}}$?] thence to [Danils?]\newline
got brakt then came home went an hung\newline
sum tobaco came in got diner went and kilt\newline
sum squerrells came in got super went to bed\newline
fair weather good helth ———\newline
About elevin oclock at night my wife awakened me\newline
for shee was very bad. I got up and sent for for ye mid-\newline
wife and women but before aney came ye child was born\newline
and it was a \textbf{son}. There was no sole with her but my\newline
self, being I believe just about midnight ———\newline
\textasciitilde\textasciitilde\textasciitilde\textasciitilde\textasciitilde\textasciitilde
\textasciitilde\textasciitilde\textasciitilde\textasciitilde\textasciitilde\textasciitilde
\textasciitilde\textasciitilde\textasciitilde\textasciitilde\textasciitilde\textasciitilde\newline
Tuesday 108 / 8\newline
a.~m. My child being born and woman to take\newline
care of my wife I ran away for Isabelah [Hughes?]\newline
and left her all alone till I came back. Shee laid\newline
 my wife to bed and drest ye child (then ye midwife came\newline
 being Elizabeth Nedham) So we continued tell day there we\newline
 got brakt and past ye day away with eating and drinking and at\newline
 night I got super and went to bed fair and good health ——\newline
 My wife and child continues brave \& well thanks be to God}\newline
 ————————
\end{quote}

\noindent
For all that it appears,
the diary shows that a birth on 7 October shortly after midnight,
as indicated by Harrington Scott,
can be excluded.

What can be doubted is whether the father wrote
the last sentences for 7 October still on Monday or already on Tuesday,
as he was very busy that night
and birth was “I believe just about midnight”,
i.e.\ nominally no time at all left on Monday for making notes in the diary.

The date format e.g.\ “107 / 7” appears to be
simply a numbering of journal entries (“107”) followed by day of month (“7”). 
The researcher at the Historical Society of Pennsylvania
was so kind to give the diary another quick look:

\begin{quote}
\small\color{darkgray}
\textsl{This diary begins on “May the 30: 1727”,
and the page is labeled (1) at the top.
%
The entries are labeled only with the dates
until p.\ 6, with “Saturday July 1 / 31.” […$\!$]
the previous entry was June 30,
and the entries on the subsequent page continue to count up from 31 […$\!$].}
\end{quote}

\noindent
Dates at that time were in Julian Calendar.
%
Today’s Gregorian Calendar was adopted in Britain in 1752,
including in the colonies that would later become the first states of the U.S.A.
%
This is consistent with the diary,
as 1 July 1727 was a Saturday in Julian Calendar,
but not in Gregorian Calendar.
%
And 7 October 1728 was a Monday,
as indicated in the diary,
and October was the only month in 1728
in which the 7th was a Monday.
%
The entry for “Tuesday 101 / 1” on the first scanned page of the diary
also shows the month, “October ye [1st?]”.

Since the father did apparently not have the opportunity
to look at a clock at birth
and it is not certain (at least to me)
how precise their clocks were
and how precisely they were in sync with local mean time,
I guess maybe a time window of 10-20 minutes
before midnight on Monday or after midnight on Tuesday (8 October)
for the actual time of birth would be realistic\,?

FYI: The main reason I made this research
was to eventually help find out when Caesar Rodney arrived 2 July 1776 in Philadelphia
just in time to cast his vote for independence (Lee Resolution),
an important event for the USA.

\newpage

\noindent
\vspace{-20mm}\hspace{-20mm}
\includegraphics{i-diary1.jpg}

\newpage

\noindent
\vspace{-20mm}\hspace{-10mm}
\includegraphics{i-diary2.jpg}

\end{document}
