\documentclass[letterpaper]{article}
\pagestyle{empty}
\paperheight=3700mm
\textheight=3700mm
%\textwidth set depending on pdflatex or (experimentally)lualatex
\topmargin=-20mm
\oddsidemargin=25mm

\usepackage[utf8]{inputenc}
\usepackage[spanish,italian,french,ngerman,english]{babel} % last is main
\usepackage{graphicx}
\usepackage{multirow}
\usepackage{xcolor}
\usepackage{contour}
\usepackage{pict2e}
\usepackage{relsize}
\usepackage{amsmath}

\usepackage{iftex}
\ifpdftex
  % stronger fonts

% Find modes.mf, e.g. /usr/local/texlive/2025/texmf-dist/fonts/source/public/modes/modes.mf
%
% $ sudo cp modes.mf modes.mf.orig
%
% Add the following at the start of modes:
%
% mode_def xphi =
%   mode_param (pixels_per_inch, 1200);
%   mode_param (blacker, 1.9); % only difference to 'lexmarkr' (2 there)
%   mode_param (fillin, 0);
%   mode_param (o_correction, 1);
%   mode_common_setup_;
% enddef;
%
% Finally:
%
% $ sudo fmtutil-sys --byfmt mf

\pdfpkresolution=1200
\pdfpkmode={xphi}
\pdfmapfile{}

  \newcommand{\coretextwidth}{85.5mm}
\fi

% experimental, not used to produce the live website...
\ifluatex
  % about same heaviness in pdfs when rasterized in photoshop,
  % but since, unlike the metafont mechanisms I use, fake bold, "bleeds" in all directions,
  % seems heavier at least in core web page images
  \newcommand{\fontbleed}{0.8}
  % paragraphs wider and font looks larger, tried to fix, but then other things change a bit,
  % especially for section headings would have to change back, are now more narrow...
  \newcommand{\fontscale}{0.985}
  \newcommand{\coretextwidth}{85.2mm}
  \usepackage{fontspec}
  % microtype does maybe help and not help, maybe if would allow wider spaces...
  \usepackage{microtype}
  \setsansfont{Latin Modern Sans}[Scale=\fontscale, FakeBold=\fontbleed]
  \setmonofont{Latin Modern Mono}[Scale=\fontscale, FakeBold=\fontbleed]
  % this would be for "new computer modern" (but has many limitations so far)
  %\usepackage[default]{fontsetup}
  %\renewcommand{\familydefault}{\sfdefault}
\fi

\renewcommand{\familydefault}{\sfdefault}

\setcounter{secnumdepth}{-1}

\newcommand{\en}[1]{\iflanguage{english}{#1}{}}
\newcommand{\de}[1]{\iflanguage{ngerman}{#1}{}}
\newcommand{\fr}[1]{\iflanguage{french}{#1}{}}

% a bit less than ~255/256
\definecolor{almostwhite}{gray}{0.996}
\definecolor{xphi}{rgb}{0.0,0.5,0.5}
\definecolor{avant}{rgb}{1,0.5,0.5}

\definecolor{frame}{gray}{0.9}
\definecolor{lightgray}{gray}{0.8}
\definecolor{gray}{gray}{0.5}
\definecolor{darkgray}{gray}{0.3}

\definecolor{darkred}{rgb}{0.8,0.0,0.0}
\definecolor{darkyellow}{rgb}{0.7,0.7,0.0}
\definecolor{darkgreen}{rgb}{0.0,0.55,0.0}
\definecolor{darkviolet}{rgb}{0.5,0,0.5}

\definecolor{darkblue}{rgb}{0,0,0.7}
\definecolor{odyssey}{rgb}{0,0,0.8}
\definecolor{indigo}{rgb}{0.29,0,0.51}
\definecolor{indigoblue}{rgb}{0.1,0,0.6}

\definecolor{saffronback}{rgb}{1.000,0.878,0.627}
\definecolor{saffronfront}{rgb}{0.376,0.125,0.000}

\DeclareRobustCommand{\cometartemisscale}[1]{\includegraphics[scale=#1]{\sourcepath/i-comet.jpg}\hspace{-0.028453em} artemis}
\newcommand\cometartemis{\cometartemisscale{0.018}}
\newcommand\cometartemissection{\cometartemisscale{0.0225}}

\DeclareRobustCommand{\moebius}[1]{\includegraphics[scale=#1]{\sourcepath/i-moebius.jpg}}
\newcommand{\yinyang}{\includegraphics[scale=0.135]{\sourcepath/i-yinyang.jpg}}

\newcommand{\rarr}{\,$\rightarrow$\,}
\newcommand{\lrarr}{\,$\leftrightarrow$\,}

% greek elements
\newcommand{\elfire}{
\begin{picture}(9,6)
  \thicklines
  \put(1,-0.5){\line(1,0){7}}
  \put(1,-0.5){\line(1,1.732){3.5}}
  \put(8,-0.5){\line(-1,1.732){3.5}}
\end{picture}}
%
\newcommand{\elair}{
\begin{picture}(9,6)
  \thicklines
  \put(1,-0.5){\line(1,0){7}}
  \put(1,-0.5){\line(1,1.732){3.5}}
  \put(8,-0.5){\line(-1,1.732){3.5}}
  \put(2.75,1.9){\line(1,0){3.5}}
\end{picture}}
%
\newcommand{\elwater}{
\begin{picture}(9,6)
  \thicklines
  \put(1,5){\line(1,0){7}}
  \put(1,5){\line(1,-1.732){3.5}}
  \put(8,5){\line(-1,-1.732){3.5}}
\end{picture}}
%
\newcommand{\elearth}{
\begin{picture}(9,6)
  \thicklines
  \put(1,5){\line(1,0){7}}
  \put(1,5){\line(1,-1.732){3.5}}
  \put(8,5){\line(-1,-1.732){3.5}}
  \put(2.75,2.5){\line(1,0){3.5}}
\end{picture}}
%
\newcommand{\elhex}{
\begin{picture}(9,6)
  \thicklines
  \put(1,0.5){\line(1,0){7}}
  \put(1,0.5){\line(1,1.732){3.5}}
  \put(8,0.5){\line(-1,1.732){3.5}}
  \put(1,5){\line(1,0){7}}
  \put(1,5){\line(1,-1.732){3.5}}
  \put(8,5){\line(-1,-1.732){3.5}}
\end{picture}}

% i ching trigrams
\newcommand{\trigram}[3]{
\begin{picture}(9,6)
  \linethickness{0.36mm}
  \put(0,5){\line(1,0){#1}}
  \put(5.5,5){\line(1,0){3.5}}
  \put(0,2.5){\line(1,0){#2}}
  \put(5.5,2.5){\line(1,0){3.5}}
  \put(0,0){\line(1,0){#3}}
  \put(5.5,0){\line(1,0){3.5}}
\end{picture}}
\newcommand{\triheaven}{\trigram{5.5}{5.5}{5.5}}
\newcommand{\triearth}{\trigram{3.5}{3.5}{3.5}}
\newcommand{\trithunder}{\trigram{3.5}{3.5}{5.5}}
\newcommand{\triwater}{\trigram{3.5}{5.5}{3.5}}
\newcommand{\trimountain}{\trigram{5.5}{3.5}{3.5}}
\newcommand{\triwind}{\trigram{5.5}{5.5}{3.5}}
\newcommand{\trifire}{\trigram{5.5}{3.5}{5.5}}
\newcommand{\trilake}{\trigram{3.5}{5.5}{5.5}}

% i ching hexagrams
% 1+2 trigrams, 3 rest of line (see e.g. dreams.tex)
\DeclareRobustCommand{\hexagram}[3]{\raisebox{-3pt}{$\overset{\text{${#1}$}}{#2}$\,}#3\vspace{3pt}}

% white-red-black etc.
\DeclareRobustCommand{\outline}[1]{\contour{black}{{\color{white}#1}}}
\DeclareRobustCommand{\white}[1]{\outline{\textbf{#1}}}
\DeclareRobustCommand{\red}[1]{{\color{darkred}\textbf{#1}}}
\DeclareRobustCommand{\black}[1]{\textbf{#1}}
\DeclareRobustCommand{\yellow}[1]{{\color{darkyellow}\textbf{#1}}}
\DeclareRobustCommand{\green}[1]{{\color{darkgreen}\textbf{#1}}}
\DeclareRobustCommand{\violet}[1]{{\color{darkviolet}\textbf{#1}}}
\DeclareRobustCommand{\indigoblue}[1]{{\color{indigoblue}\textbf{#1}}}
\DeclareRobustCommand{\indigo}[1]{{\color{indigo}\textbf{#1}}}

% ELEMENTAL
\newcommand{\ELEMENTAL}{%
\colorlet{contour}{.}\textbf{\color{white}%
\raisebox{+0.001em}{\contour{contour}{E}}%
\raisebox{+0.015em}{\contour{contour}{L}}%
\raisebox{+0.016em}{\contour{contour}{E}}%
\raisebox{+0.023em}{\contour{contour}{M}}%
\raisebox{+0.023em}{\contour{contour}{E}}%
\raisebox{+0.017em}{\contour{contour}{N}}%
\raisebox{-0.020em}{\contour{contour}{T}}%
\raisebox{-0.002em}{\contour{contour}{A}}%
\raisebox{+0.006em}{\contour{contour}{L}}%
}}

% artemis pdf+web icons
\newcommand{\ipdfen}{\includegraphics[scale=0.5]{i-pdf-en.png}}
\newcommand{\ipdfde}{\includegraphics[scale=0.5]{i-pdf-de.png}}
\newcommand{\ipdffr}{\includegraphics[scale=0.5]{i-pdf-fr.png}}
\newcommand{\iweb}{\includegraphics[scale=0.055]{i-web.png}}
\newcommand{\ipdfblueen}{\includegraphics[scale=0.5]{i-pdf-blue-en.png}}
\newcommand{\ipdfbluede}{\includegraphics[scale=0.5]{i-pdf-blue-de.png}}
\newcommand{\ipdfbluefr}{\includegraphics[scale=0.5]{i-pdf-blue-fr.png}}
\newcommand{\iwebblue}{\includegraphics[scale=0.055]{i-web-blue.png}}


\textwidth=\coretextwidth



% for the marriage symbol
\usepackage{genealogytree}

% for the enumeration (a) (b)
\usepackage[shortlabels]{enumitem}

\begin{document}

\avantgarde

\section{The roots of the four elements in Empedocles’ poem,
and similarly veiled in the Hippocratic Oath\,?}

The following novel insights are presented:

\begin{enumerate}[(a)]

\item
An interpretation of “roots” in Empedocles fragment about the four elements
as literally as the root, origin or creator of an element
and thus an attribution
Zeus-Fire, Hera-Earth, Hades-Air and Nestis-Water,
i.e.\ the same as by Aetius according to the majority of surviving sources.

\item
An interpretation of the gods and goddesses in the Hippocratic Oath
in its perhaps oldest surviving form
similarly as also the four elements in veiled form:
Apollon-Fire, Hippocrates-Air, Hygieia-Water and Panacea-Earth.

\end{enumerate}

\subsection{Sources and interpretations around Empedocles to date}

In \textsl{Metaphysics} (book 1, chapter 3)
Aristotle mentions that Empedocles would have been the first philosopher
to speak of four elements (transl.\ W.~Ross):

\begin{quote}
\textsl{\color{xphi}%
Anaximenes and Diogenes make air prior to water,
and the most primary of the simple bodies,
while Hippasus of Metapontum and Heraclitus of Ephesus say this of fire,
and Empedocles says it of the four elements
(adding a fourth—earth—to those which have been named);
for these, he says, always remain and do not come to be,
except that they come to be more or fewer,
being aggregated into one and segregated out of one.}
\end{quote}

\noindent
Aristotle lived between 384 and 322 BCE,
Empedocles roughly between 490 and 430 BCE.
%
In a work that survived only indirectly,
the philosopher Aetius,
who lived in the 1st or 2nd century CE,
relates Empedocles’ mention of the four elements to a fragment (DK31B6)
that is usually considered part of a poem by Empedocles titled \textsl{On Nature}.
%
Here the fragment,
first in the original Greek,
then in the translation of William Leonard
from \textsl{The Fragments of Empedocles} (1908),
but with original Greek names for deities
instead of the Roman equivalents he used in his translation:

\vspace{3mm}
\hspace{2.5mm}
\includegraphics[scale=0.17]{i-empedocles.jpg}
\vspace{-1mm}

\begin{quote}
\textsl{\color{xphi}%
And first the fourtold root of all things hear!—\newline
White gleaming Zeus, life-bringing Hera, Aidoneus\newline
And Nestis whose tears bedew mortality.}
\end{quote}

\noindent
Aetius works only survived in several works attributed to different authors.
%
In the majority of them,
Aetius would have attributed
Zeus to Fire, Hera to Earth, Aidoneus (Hades) to Air and Nestis to Water,
in the minority Earth and Air would be flipped between Hera and Hades.
%
Sources in detail from \textsl{Die Vorsokratiker}, J.\ Mansfeld and O.\ Primavesi, Reclam, 2012:
(majority) Stobaios I 10,11b; p.\ 121,16-20 W.\ and Qusta ibn Luqa I 3,20;
(minority) Ps.-Plutarch, \textsl{Placita} I 3,20 (Hss.) and Euseb., \textsl{Praep.\ ev.} XIV 14,6.

Aetius argues as follows:
Zeus as boiling and [fiery] aether,
live-giving Hera as Earth,
Aidoneus [i.e.\ the invisible] as Air,
which has no own light but would be shone upon by sun, moon and stars,
Nestis as semen and water.

In \textsl{Ancient Philosophy, Mystery and Magic:
Empedocles and Pythagorean Tradition} (1995),
Peter Kingsley attributes Zeus to Air, Hera to Earth, Hades to Fire and Nestis,
who he interprets as Persephone, to Water.
%
He changes the attribution of Zeus
due to an apparent change of meaning for aether
between Empedocles’s time as mainly Air
to later on when it would rather mean Fire.

Aristotle used aether as the name of the fifth element,
which exists primarily in space and goes in circles.
%
In space you have “Air” as the void
and “Fire” as the lights that move periodically around up there,
namely sun, moon, planets and stars,
which is likely why aether had ambivalent associations,
including until at least the times of the Stoics.

Johann Leonhard Hug already suggested in 1812 in
\textsl{Mythos der berühmten Völker der alten Welt vorzüglich der Griechen}
that Nestis would have been a variation of the name of the ancient Egyptian goddess Nephthys
and that she would thus correspond to the Greek goddess Persephone.

So far, the sources and some interpretations I know of, now to my take.

\subsection{Should “roots” in Empedocles fragment be taken literally\,?}

My take on Empedocles’ fragment
is to interpret “root” in the sense of creator, origin,
as the \textsl{source} of the elements
rather than as the elements themselves,
and to assign gods and goddesses
via their explicit or implicit attributes.

Zeus is described as “white gleaming” or “flashing” or “shining”,
which I would interpret as Fire,
especially since Zeus is very prominently known for throwing bolts of lightning,
so he creates Fire that way.

His wife Hera
is described as “life-bringing” or “life-bearing”,
which I would interpret as pregnant and thus as creating Earth,
as creating new living matter in form of a newborn child.

Aidoneus is simply a well-known variant of Aides, Hades,
and has no attributes in the poem,
so let me skip Hades for a second.

Nestis is a goddess about which close to nothing seems to be known,
but her attributes “tears” and “dew”
leave almost no choice but to associate her with Water,
a goddess who creates Water,
in the form of dew or tears (rain?).

Hades has no attributes,
but maybe his name is the attribute\,?
%
His name means “invisible” or “unseen”,
while in Plato’s dialogue \textsl{Cratylus}
Socrates proposes “knowledge of all noble things”.
%
Let me simply assume that in this case,
since there are no attributes,
the name is the attribute,
which would fit well with invisible Air
and also with the fact that in astrology Air is related to thinking.

All in all,
this would lead to exactly the same attributions as the ones of Aetius,
as reported by the majority of variants in which his work survived.

Hug/Kingsley suggest that Nestis would be Persephone,
so that Empedocles would have listed two divine couples:
Zeus \gtrsymMarried\ Hera high up on Mount Olympus,
Hades \gtrsymMarried\ Persephone deep down in the underworld.

Kingsley also suggests
that Empedocles would not have been a philosopher in today’s usual meaning,
but would have had a background in more “magical”
and especially also “medical” practices.
%
This would also reflect in the oaths that presocratic philosophers of some schools would apparently take,
which would apparently have included vows to keep some knowledge secret.

\subsection{The four elements in the Hippocratic Oath\,?}

In that sense
let me also look at the beginning of the Hippocratic Oath,
the oath still sworn by doctors in modified form today,
in perhaps its oldest surviving form
(as found in Oxyrhynchus Papyrus 2547, around 275 CE):

\begin{quote}
\textsl{\color{xphi}%
I swear by Apollo the healer, by Asclepius, by Hygieia, by Panacea,\newline
and by all the gods and goddesses, […$\!$]}
\end{quote}

\noindent
My take is that the sun god Apollon would
be most strongly associated with Fire.
%
His son Asclepius,
as a wise doctor, would fit well with Air.
%
Asclepius’ daughter Hygieia would fit well with Water,
as she is often shown with a snake
that drinks from a bowl in her hands
(and with hygiene, of course, which often involves liquids for disinfection).
%
Panacea, another of Asclepius’ five daughters,
would most likely be Earth, as she used to heal with plants.

So,
did doctors implicitly take an oath on the four elements,
more so than on the explicitly named gods or saints\,?
Did Empedocles essentially do the same in an older form,
maybe in both cases in order to “blend in” with society
by superficially alluding to mainstream divinities at the time,
while secretly only feeling bound to the four elements,
or in some sense the laws of nature\,?
%
Would in both cases,
as, I guess, Kingsley also suggests,
secret traditions be involved,
where knowledge was maybe passed on
only orally from master to pupil\,?

\subsection{Visualizations}

In order to maybe approach an answer,
let me illustrate the proposed assignments to elements, first for Empedocles’ fragment.

Most surprising is that the female goddesses
would be associated exactly with the elements
that are now considered female in astrology, Water and Earth,
and the male gods with the ones that are now considered male, Fire and Air.
%
This is so surprising
because this attribution appears usually to be dated
to roughly the 2nd century CE in astrology
(Vettius Valens’s \textsl{Anthologia},
and hints in earlier texts by Dorotheus of Sidon and Marcus Manilius),
with precursors that attribute passive/active to the same pairs of elements
going apparently back to the Stoics,
but Zeno founded Stoicism only in 301 BCE,
more than a century after Empedocles lived.
%
(Or am I maybe missing something here\,?)

\vspace{-0.5mm}
\hspace{6mm}
\includegraphics[scale=0.15]{i-elements-empedocles.jpg}

\noindent
Around 350 BCE,
Aristotle categorized Water and Earth as cold, Fire and Air as hot.
%
He also categorized Fire and Earth as dry,
which would here be the couple Zeus \gtrsymMarried\ Hera above ground,
and Air and Water as wet,
which would here be the couple Hades \gtrsymMarried\ Nestis below ground
(if you follow Hug/Kingsley’s suggestion that Nestis would be Persephone).

Now let me take a similar look at the elements in the Hippocratic Oath:

\vspace{0.0mm}
\hspace{15mm}
\includegraphics[scale=0.15]{i-elements-oath.jpg}
\vspace{0.0mm}

\noindent
Again,
the male elements would be the male gods,
the female elements the female goddesses,
which is no longer surprising in the 3rd century CE.

Elements are listed in the order from light to heavy,
in exactly the way Aristotle and others sorted them.
%
That both men are listed first
and that both women are daughters of Asclepius,
who, in turn, is a son of Apollon,
gives this list a more patriarchal touch
compared to the pair of couples in Empedocles’ list,
even though Empedocles lists men first in each couple.

\subsection{Quick wrap-up and outlook}

Were the four elements
something that some people had known about in closed circles
for maybe many generations before this came out publicly\,?
%
Empedocles would have been very close
to what became mainstream in astrology several hundred years later
in my attribution to elements.
%
Maybe even psychological associations would not be too far fetched
for Empedocles’ fragment,
with Nestis and tears close to feelings, like Water
in astrology\,?
%
Even the couples would be between elements that are usually
considered to go well together in astrology.
%
The two couples Zeus \gtrsymMarried\ Hera and
Hades \gtrsymMarried\ Persephone
remind a bit of Isis \gtrsymMarried\ Osiris and Seth \gtrsymMarried\ Nephthys
from the ancient Egyptian Heliopolis creation myth,
especially since “Nephthys” reminds of “Nestis”,
as already mentioned further above.
%
Quite generally,
creation myths world-wide
practically always involve the elements in some form.

How far do things really go back, what was just made up later\,?
%
Is it certain that Empedocles’ fragment is genuinely from him in this form\,?

\subsection{Postscript}

Looking at Empedocles’ poem from the point of view
of the three colors \white{white}-\red{red}-\black{black} of a triple moon goddess,
as first proposed by Robert Graves in \textsl{The White Goddess},
in the chapter of the same name:
Might Empedocles have listed
first Zeus as white and bright as the white aspect of the goddess,
then Hera as pregnant and life bringing as the red aspect of the goddess,
and then Hades as the black aspect of the goddess\,?
%
Might Nestis, on her own line in the poem,
be the triple goddess herself\,?
%
See the link below,
as well as the section “origins” of the main content of this web site exactphilosophy.net.

\begin{list}{$\bullet$}{\setlength{\leftmargin}{10pt}}

\item \href{https://www.exactphilosophy.net/white-red-black-and-the-green-goddess.pdf}%
{\white{White}-\red{red}-\black{black} and the\,\green{“green”}\,goddess\newline
\color{xphi}exactphilosophy.net/white-red-black-and-the-green-goddess.pdf}

\end{list}

\noindent
Or,
since Nepthys was at the spinning house in Sais,
maybe in essence basically
the ancient creator goddess Neith of Sais in the Nile Delta,
with the Nile for water (as later Isis),
the delta for the female sex and a trinity,
sort of reversely spinning the Nile
from the strands in the delta,
hence also a creatrix\,?

\end{document}
