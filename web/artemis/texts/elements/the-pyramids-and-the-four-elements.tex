\documentclass[letterpaper]{article}
\pagestyle{empty}
\paperheight=3700mm
\textheight=3700mm
%\textwidth set depending on pdflatex or (experimentally)lualatex
\topmargin=-20mm
\oddsidemargin=25mm

\usepackage[utf8]{inputenc}
\usepackage[spanish,italian,french,ngerman,english]{babel} % last is main
\usepackage{graphicx}
\usepackage{multirow}
\usepackage{xcolor}
\usepackage{contour}
\usepackage{pict2e}
\usepackage{relsize}
\usepackage{amsmath}

\usepackage{iftex}
\ifpdftex
  % stronger fonts

% Find modes.mf, e.g. /usr/local/texlive/2025/texmf-dist/fonts/source/public/modes/modes.mf
%
% $ sudo cp modes.mf modes.mf.orig
%
% Add the following at the start of modes:
%
% mode_def xphi =
%   mode_param (pixels_per_inch, 1200);
%   mode_param (blacker, 1.9); % only difference to 'lexmarkr' (2 there)
%   mode_param (fillin, 0);
%   mode_param (o_correction, 1);
%   mode_common_setup_;
% enddef;
%
% Finally:
%
% $ sudo fmtutil-sys --byfmt mf

\pdfpkresolution=1200
\pdfpkmode={xphi}
\pdfmapfile{}

  \newcommand{\coretextwidth}{85.5mm}
\fi

% experimental, not used to produce the live website...
\ifluatex
  % about same heaviness in pdfs when rasterized in photoshop,
  % but since, unlike the metafont mechanisms I use, fake bold, "bleeds" in all directions,
  % seems heavier at least in core web page images
  \newcommand{\fontbleed}{0.8}
  % paragraphs wider and font looks larger, tried to fix, but then other things change a bit,
  % especially for section headings would have to change back, are now more narrow...
  \newcommand{\fontscale}{0.985}
  \newcommand{\coretextwidth}{85.2mm}
  \usepackage{fontspec}
  % microtype does maybe help and not help, maybe if would allow wider spaces...
  \usepackage{microtype}
  \setsansfont{Latin Modern Sans}[Scale=\fontscale, FakeBold=\fontbleed]
  \setmonofont{Latin Modern Mono}[Scale=\fontscale, FakeBold=\fontbleed]
  % this would be for "new computer modern" (but has many limitations so far)
  %\usepackage[default]{fontsetup}
  %\renewcommand{\familydefault}{\sfdefault}
\fi

\renewcommand{\familydefault}{\sfdefault}

\setcounter{secnumdepth}{-1}

\newcommand{\en}[1]{\iflanguage{english}{#1}{}}
\newcommand{\de}[1]{\iflanguage{ngerman}{#1}{}}
\newcommand{\fr}[1]{\iflanguage{french}{#1}{}}

% a bit less than ~255/256
\definecolor{almostwhite}{gray}{0.996}
\definecolor{xphi}{rgb}{0.0,0.5,0.5}
\definecolor{avant}{rgb}{1,0.5,0.5}

\definecolor{frame}{gray}{0.9}
\definecolor{lightgray}{gray}{0.8}
\definecolor{gray}{gray}{0.5}
\definecolor{darkgray}{gray}{0.3}

\definecolor{darkred}{rgb}{0.8,0.0,0.0}
\definecolor{darkyellow}{rgb}{0.7,0.7,0.0}
\definecolor{darkgreen}{rgb}{0.0,0.55,0.0}
\definecolor{darkviolet}{rgb}{0.5,0,0.5}

\definecolor{darkblue}{rgb}{0,0,0.7}
\definecolor{odyssey}{rgb}{0,0,0.8}
\definecolor{indigo}{rgb}{0.29,0,0.51}
\definecolor{indigoblue}{rgb}{0.1,0,0.6}

\definecolor{saffronback}{rgb}{1.000,0.878,0.627}
\definecolor{saffronfront}{rgb}{0.376,0.125,0.000}

\DeclareRobustCommand{\cometartemisscale}[1]{\includegraphics[scale=#1]{\sourcepath/i-comet.jpg}\hspace{-0.028453em} artemis}
\newcommand\cometartemis{\cometartemisscale{0.018}}
\newcommand\cometartemissection{\cometartemisscale{0.0225}}

\DeclareRobustCommand{\moebius}[1]{\includegraphics[scale=#1]{\sourcepath/i-moebius.jpg}}
\newcommand{\yinyang}{\includegraphics[scale=0.135]{\sourcepath/i-yinyang.jpg}}

\newcommand{\rarr}{\,$\rightarrow$\,}
\newcommand{\lrarr}{\,$\leftrightarrow$\,}

% greek elements
\newcommand{\elfire}{
\begin{picture}(9,6)
  \thicklines
  \put(1,-0.5){\line(1,0){7}}
  \put(1,-0.5){\line(1,1.732){3.5}}
  \put(8,-0.5){\line(-1,1.732){3.5}}
\end{picture}}
%
\newcommand{\elair}{
\begin{picture}(9,6)
  \thicklines
  \put(1,-0.5){\line(1,0){7}}
  \put(1,-0.5){\line(1,1.732){3.5}}
  \put(8,-0.5){\line(-1,1.732){3.5}}
  \put(2.75,1.9){\line(1,0){3.5}}
\end{picture}}
%
\newcommand{\elwater}{
\begin{picture}(9,6)
  \thicklines
  \put(1,5){\line(1,0){7}}
  \put(1,5){\line(1,-1.732){3.5}}
  \put(8,5){\line(-1,-1.732){3.5}}
\end{picture}}
%
\newcommand{\elearth}{
\begin{picture}(9,6)
  \thicklines
  \put(1,5){\line(1,0){7}}
  \put(1,5){\line(1,-1.732){3.5}}
  \put(8,5){\line(-1,-1.732){3.5}}
  \put(2.75,2.5){\line(1,0){3.5}}
\end{picture}}
%
\newcommand{\elhex}{
\begin{picture}(9,6)
  \thicklines
  \put(1,0.5){\line(1,0){7}}
  \put(1,0.5){\line(1,1.732){3.5}}
  \put(8,0.5){\line(-1,1.732){3.5}}
  \put(1,5){\line(1,0){7}}
  \put(1,5){\line(1,-1.732){3.5}}
  \put(8,5){\line(-1,-1.732){3.5}}
\end{picture}}

% i ching trigrams
\newcommand{\trigram}[3]{
\begin{picture}(9,6)
  \linethickness{0.36mm}
  \put(0,5){\line(1,0){#1}}
  \put(5.5,5){\line(1,0){3.5}}
  \put(0,2.5){\line(1,0){#2}}
  \put(5.5,2.5){\line(1,0){3.5}}
  \put(0,0){\line(1,0){#3}}
  \put(5.5,0){\line(1,0){3.5}}
\end{picture}}
\newcommand{\triheaven}{\trigram{5.5}{5.5}{5.5}}
\newcommand{\triearth}{\trigram{3.5}{3.5}{3.5}}
\newcommand{\trithunder}{\trigram{3.5}{3.5}{5.5}}
\newcommand{\triwater}{\trigram{3.5}{5.5}{3.5}}
\newcommand{\trimountain}{\trigram{5.5}{3.5}{3.5}}
\newcommand{\triwind}{\trigram{5.5}{5.5}{3.5}}
\newcommand{\trifire}{\trigram{5.5}{3.5}{5.5}}
\newcommand{\trilake}{\trigram{3.5}{5.5}{5.5}}

% i ching hexagrams
% 1+2 trigrams, 3 rest of line (see e.g. dreams.tex)
\DeclareRobustCommand{\hexagram}[3]{\raisebox{-3pt}{$\overset{\text{${#1}$}}{#2}$\,}#3\vspace{3pt}}

% white-red-black etc.
\DeclareRobustCommand{\outline}[1]{\contour{black}{{\color{white}#1}}}
\DeclareRobustCommand{\white}[1]{\outline{\textbf{#1}}}
\DeclareRobustCommand{\red}[1]{{\color{darkred}\textbf{#1}}}
\DeclareRobustCommand{\black}[1]{\textbf{#1}}
\DeclareRobustCommand{\yellow}[1]{{\color{darkyellow}\textbf{#1}}}
\DeclareRobustCommand{\green}[1]{{\color{darkgreen}\textbf{#1}}}
\DeclareRobustCommand{\violet}[1]{{\color{darkviolet}\textbf{#1}}}
\DeclareRobustCommand{\indigoblue}[1]{{\color{indigoblue}\textbf{#1}}}
\DeclareRobustCommand{\indigo}[1]{{\color{indigo}\textbf{#1}}}

% ELEMENTAL
\newcommand{\ELEMENTAL}{%
\colorlet{contour}{.}\textbf{\color{white}%
\raisebox{+0.001em}{\contour{contour}{E}}%
\raisebox{+0.015em}{\contour{contour}{L}}%
\raisebox{+0.016em}{\contour{contour}{E}}%
\raisebox{+0.023em}{\contour{contour}{M}}%
\raisebox{+0.023em}{\contour{contour}{E}}%
\raisebox{+0.017em}{\contour{contour}{N}}%
\raisebox{-0.020em}{\contour{contour}{T}}%
\raisebox{-0.002em}{\contour{contour}{A}}%
\raisebox{+0.006em}{\contour{contour}{L}}%
}}

% artemis pdf+web icons
\newcommand{\ipdfen}{\includegraphics[scale=0.5]{i-pdf-en.png}}
\newcommand{\ipdfde}{\includegraphics[scale=0.5]{i-pdf-de.png}}
\newcommand{\ipdffr}{\includegraphics[scale=0.5]{i-pdf-fr.png}}
\newcommand{\iweb}{\includegraphics[scale=0.055]{i-web.png}}
\newcommand{\ipdfblueen}{\includegraphics[scale=0.5]{i-pdf-blue-en.png}}
\newcommand{\ipdfbluede}{\includegraphics[scale=0.5]{i-pdf-blue-de.png}}
\newcommand{\ipdfbluefr}{\includegraphics[scale=0.5]{i-pdf-blue-fr.png}}
\newcommand{\iwebblue}{\includegraphics[scale=0.055]{i-web-blue.png}}


\textwidth=\coretextwidth



% for glyphs of the elements
\usepackage{starfont}

\begin{document}

\avantgarde

\section{The pyramids and the four elements\,?}

I had the idea$^*$ to relate dry/wet and hot/cold
to the heating and drying daily course of the sun
around one of the pyramids at Giza,
just as in one of the original images for yin-yang in China
as the shady and sunny sides of a hill [1]:

\vspace{3mm}
\hspace{26mm}
\includegraphics[scale=0.17]{i-pyramid-elements.jpg}
\vspace{3mm}

\noindent
This would fit well with the fascination of the ancient Greeks with ancient Egypt,
from the first philosopher Thales
who reportedly measured the height of the pyramids in Giza
by comparing the length of their shadows with his own shadow,
via the tetractys of the Pythagoreans,
up to Aristotle’s description of the four elements
in terms of tangible properties dry/wet and hot/cold,
with roots back to the primeval mound emerging from water,
or the mound of ashes around glowing charcoal,
\white{white} around \red{red} around \black{black},
colors of a ripening mulberry,
signatures of the universal “white” fire/moon creatress/goddess.

\vspace{-3mm}
\begin{center}
\includegraphics[scale=0.012]{i-hestia.jpg}
\hspace{0.5mm}
\includegraphics[scale=0.0125]{i-tetrahedron.jpg}
\end{center}
\vspace{-2mm}

\vspace{1mm}
\noindent
[1] Richard Wilhelm in the introduction of \textsl{I Ching or Book of Changes}.

\vspace{4mm}
\noindent
\hspace{8mm}
\includegraphics[scale=0.26]{i-pyramids.jpg}

\vspace{4mm}
\noindent
{\footnotesize $^*$ Saturday, 15 December 2018, around 8 AM.}

\newpage

\subsection{Sources and considerations}

Richard Wilhelm [1]:
“In its primary meaning yin is ‘the cloudy’, ‘the overcast’
and yang means actually ‘banners waving in the sun’\hspace{0.1mm}$^{\mathsf{15}}$,
that is,
something ‘shone upon’,
or bright.
%
By transference the two concepts were applied
to the light and dark sides of a mountain or of a river.
%
In the case of a mountain
the southern is the bright side and the northern the dark side,
while in the case of a river seen from above,
it is the northern side that is bright (yang),
because it reflects the light,
and the southern side that is in shadow (yin).
%
[…$\!$]\, 15.\ Cf.\ the noteworthy discussions of Liang Ch’i-ch’ao
in the Chinese journal \textsl{The Endeavor}, July 15 and 22, 1923,
also the English Essay by B.\ Schindler,
‘The Development of the Chinese Conceptions of Supreme Beings’,
\textsl{Asia Major}, Hirth Anniversary Volume (London: Probsthain, n.d.), pp.\ 298-366.”

The attribution of elements to points of the compass
would also mirror later attributions of seasons (winter-north-Water, etc.) to elements.
%
Would flipping dry-hot and cold-wet at the corners also make sense\,?
%
Or assigning elements to corners instead of edges or faces\,?
%
However, the today usual symbols for the elements are triangles,
just like the faces of a pyramid.
%
It would even be so that the two faces that see most resp.\ least of the sun during the day
would have a triangle without intersection (Fire \Fire\ and Water \Water),
while the faces that would only see the sun about half of daytime
would have an intersected triangle (Air \Air\ and Earth \Earth),
reminding also a bit of Chephren’s pyramid.

Could it maybe even be so
that Aristotle in \textsl{On Generation and Corruption}
would not have been able to argue as freely as he may have wanted,
since he was either restraining himself in face of contemporary conventions in society,
or could he have been bound by something like a secret pythagorean oath\,?
%
C.\,G.\ Jung in his book \textit{Psychological Types} of 1921
(in the original German)
relates implicitly to older traditions
that attribute four temperaments to the classical elements
(Fire-choleric, Air-sanguine, Water-phlegmatic, Earth-melancholic),
but does not mention this with any word,
even though he considers earlier works over several hundred pages
and knew astrology since at least 1911 (letter to Freud).

Pity that knowledge that had been kept secret can exactly for that
reason hardly be distinguished from pure fiction:
In both cases, at much later times usually no artifacts remain
from the time something was supposedly already known.
%
The only other thing to do would be to argue indirectly via symbolism,
but also that is difficult with regard to the elements,
since in that case,
because all is so “elementary”$\!$,
so minimal,
there are not many essentially natural ways
how to attribute things to each other.
%
And, yet,
even today pyramids are such impressive buildings that one keeps wondering:
Why exactly pyramids\,?

But how pyramids evolved from single “floor” mastabas via stepped pyramids
to their final form is well researched.
%
Especially how Sneferu had the first three pyramids without steps built
and the first two attempts failed,
does not suggest that a lot of symbolism
was in the conscious minds of ancient Egyptians at the time,
the issues at hand were much more basic,
even though the idea of a mound emerging from water (“north face”)
as the beginning of the world
was not unlikely already then part of ancient Egyptian mythology.

Also, the triangular glyphs for the elements
seem to be relatively recent attributions from alchemy
without known roots in antiquity.
%
In other words,
the association of elements with the pyramids
fits symbolically very nicely,
but historically there seem to be no direct conscious traces from antiquity.

\subsection{Epilogue with some introductions}

I guess it is difficult to grasp all explicit or implicit allusions so far
to someone not familiar with the context.
%
Not wanting to destroy the beauty it may have to readers who are,
here some additional clues,
initially written in May 2023.

The pairs of opposites dry/wet and hot/cold
are the ones that Aristotle used in \textsl{On Generation and Corruption}
to define elements in terms of things that can be felt by touching.
%
According to him,
fire is hot and dry, air wet and hot, water cold and wet, earth dry and cold.
%
And they mostly transform into each other by switching just one of the attributes.
%
Fire gets wet thus becoming air, which gets cold thus becoming water,
which gets dry thus becoming earth,
which gets hot thus becoming fire again,
and also the other way round.
%
Note that the arragement of elements
in the first illustration on the first page of this article
is exactly according to Aristotle's cycle of elemental transitions.

The sunny side of a hill tends to dry and heat up due to the energy of the sunlight,
hence the sunny side of a hill
(male yang in the Chinese picture for yin-yang)
would in terms of Aristotle’s definition of elements be fire,
while the shadowy side of the hill would be wet and cold,
hence water (female yin).
%
The pyramids in Giza are all aligned with the points of the compass.
%
So there is a face that faces south
and hence gets most of the light of the sun
during the day,
hence would be related to fire,
again in Aristotle’s definition.
%
And accordingly the north face would be wet and cold,
water.

With earth and air,
things are less well-defined.
%
In a way,
the passage of the sun is symmetric regarding east and west,
just different times of day,
different direction of evolvement.
%
Hence,
there seems to be no 100\% natural way
of attributing air and earth to faces of the pyramids.
%
I chose the same attribution
as the one that became sort of canonical in late antiquity,
apparently first mentioned by Antiochus of Athens,
who attributed east/spring to air and west/autumn to earth
(and fire to south/summer, water to north/winter).
%
Physically,
one could argue that things first dry up and only then heat up quickly,
as evaporating water takes energy,
so that in the morning first wet and cold water would become dry and cold,
hence earth,
and only then heat up to fire around noon,
while in the afternoon matter would first cool down
and only later water condensate and make it wet,
in other words hot and dry fire would first get cold,
hence earth again,
and then wet,
water.
%
In other words,
two times earth in between,
again the crux of the symmetry of the daily cycle.

To the ancient Greeks,
the ancient Egyptians seem to have been something like the ancient Greeks to us,
an ancient culture that is admired for its findings
and also for the beauty of its art and other parts of its culture.
%
There are often legends of Greek philosophers having been in Egypt,
and it seems often not to be sure which ones are true.
%
Besides Thales who was mentioned,
it seems that an ancestor of Plato had been in Egypt,
more precisely at Sais in the Nile Delta,
another triangle,
by the way,
and brought back also the legend of Atlantis.

Atlantis is also a good cue to continue.
%
There are two major creation myths in ancient Egypt,
one from Heliopolis and one from Hermopolis.
%
In both of them at the beginning a mound of land emerges from water,
which is also, in reverse order, what happened to Atlantis in the legend.
%
A hill that emerges from water,
like a pyramid from water or north (cold and wet)\,?
%
At least symbolically,
there is a similarity,
although not necessarily historically.
%
It seems that from some point on pyramids
were associated with the emerging mound at creation,
but originally pyramids emerged from single floor mastabas
via step pyramids to ones with smooth faces.
%
At Giza,
the smooth faces were “stolen” over time,
except for the “hat” on Chephren’s pyramid.

It is also the top of the pyramid of Chephren
that reminds of the horizontal line in the symbols of air \Air\ and earth \Earth.
%
Interestingly,
those are also the ones in between
if attributing faces of the pyramids to elements,
while the clear cases fire \Fire\ and water \Water\
have no such vertical line.
%
I am not aware how exactly these symbols came to be,
but they seem to only have appeared a few hundred years ago.
%
Obviously,
they would mirror four triangular faces of a pyramid. 

In other works,
Aristotle postulated a fifth element ether,
which would only exist in the sky and go in circles
(while the earthly four elements would move in straight lines),
and the tip of the pyramid,
high up and sort of pointing to the sky,
would again symbolically mirror that.

In the introduction of \textsl{The Greek Myths},
Robert Graves mentions the mound of ashes
around the idea of a great (white) goddess in prehistoric times.
%
White ashes around red glowing embers around black coal.
%
These three colors may well have been the precursors of the four elements,
since in late antiquity the canonical colors became
fire-yellow, air-red, water-white and earth-black,
with indications of at some point a primeval color “red”
that would have spanned yellow-orange-red
split into red and yellow,
but see the core content and other articles on this site
for more clues around that;
it would seem too much to try to  cram this all in here.
%
In any case,
the invention of handling and creating fire
would have made a huge impression on humanity,
and,
symbolically,
you need to make things hot and dry to create fire,
which would again mirror the sunny side of the pyramid,
with a sea of cold and wet water at the shadowy side.

Maybe one day I would find the leisure and/or drive
to describe all of this in coherent ways that are accessible to almost all,
but I am presently not sure if that would be good,
if that would be desired by the fates,
as it would have the potential to change quite a few things in the world.
%
So let me turn things around:
If you are interested in a particular topic to be explained in more detail,
tell me,
and I will see what I can do.
%
Maybe explain it interactively,
and then maybe someone (me, you, etc.) would become able to better communicate it,
and it would live on by its own,
be preserved by many.

Back to pyramids and elements:
All in all,
while it seems that people at different times were not consciously aware
of all the apparent symbolic coherence,
they seem to sort of have mirrored or evolved the same things,
which only seems to show so clearly since relatively recently in human history,
and whether that has already been the end of the journey,
is hard to tell.

I hope that this “epilogue introduction”
helped to clarify some thoughts and considerations of mine
and maybe also to grasp some of the a priori rather unexpected simple symbolic beauty,
spanning different cultures in different epochs,
which is exactly what had flashed me that morning in late 2018.

Please take a look at the rest of this site.
%
There is so much to be found here.
%
Often not explained step-by-step,
but with a little good will,
many visitors should be able to find interesting new things.

I just noticed that this article now has four pages,
just like the four elements.
%
The vision (“light/sun”) on page 1 reminds of fire,
page 2 with its references (“mind”) of air,
page 3 with compassion (“feelings”) with readers of water,
and this page of earth (“reality”)\,?
%
So let me add a 5th, a Venus hill…

Venus is associated with the number 5
because during the course of 8 solar years
she stations (appears to stand still in the sky) 2\,$\times$\,5 times,
drawing an almost perfect pentagram in the sky.
%
Pentagrams (or pentagons) also contain the golden ratio,
which again relates to beauty and love,
as the goddess of many names,
like Aphrodite, Isis, Ishtar, and so on.
%
In Mesopotamia,
she was often depicted as an 8-pointed star,
along with sun and moon, since 8 solar years are also quite close to 99 lunar months,
which is also why the number 50 occurs often in that context:
The Olymics in ancient Greece were alternatively every 50 and 49 lunar months
(each about 4 solar years),
and in China in the yarrow stalk method of casting an I Ching oracle
you start with 50 stalks and put one away, 
for no practical reason,
except that 50+49 is 99.

But 5 is also related somewhat to Mercury,
since he moves apparently backwards in the sky almost exactly 1/5th of the time.
%
The fifth element seems to be related to transforming things,
also to create new things
(as the 5th house in astrology is related besides creativity in general also to children).
%
And in that respect,
the trickster Mercury is certainly helpful,
and his Greek name Hermes (“pillar”) may appeal to Venus,
who would from a later astrological perspective have ruled the Age of Taurus,
when the pyramids of Giza were built.

Was Jung aware of the four temperaments via the four elements in astrology
in 1921 when he published \textsl{Psychological Types}
(in German then, the Englisch translation followed in 2023)\,?
%
He writes the following in chapter 11 for the term ‘function’:
“I can give no \textsl{a priori} reason for selecting these four as basic functions,
and can only point out
that this conception has shaped itself
out of many years’ experience.”,
where ‘a priori’ might be his loophole,
the excuse for appearing to have come to four types
purely from personal observation of people,
while other possible influences are at least not mentioned.
%
But it cannot be excluded that ‘Venus’ lured him
into not making some conscious and in retrospect pretty obvious mental steps,
so that he maybe really “discovered” four types on his own
before ever getting in contact with astrology.
%
But even then,
after at least 10 years of experience with astrology,
expecting him to not have known anything
about the four elements and associated temperaments
seems to be quite a stretch.
%
It should also be mentioned that around 1921
he first got into contact with Richard Wilhelm,
who first translated the I Ching (and later other texts) to German.
%
In the I Ching there are 8 trigrams
and in \textsl{Psychological Types} there are actually 8 types,
if you count each as introverted and extroverted,
as he does in the book,
with no mention of the I Ching.

But what about Aristotle\,?
%
Did he also purposely not mention some older traditions
that were maybe viewed as outdated or,
as I speculated,
were protected by an oath,
as Pythagoreans had to take\,?
%
This might very well be in a way,
see also my article with the short, catchy title
\textsl{The roots of the four elements in Empedocles’ poem,
and similarly veiled in the Hippocratic Oath\,?}
for some other examples
that would suggest that also in antiquity
not always everything was spelled out.
%
Plato’s \textsl{Timaios}
(where he talks about Egypt, Atlantis
and \textsl{first time} about five elements,
made of triangles)
starts with an allusion to the pythagorean tetractys,
which is a triangle made of 10 dots, 4+3+2+1.

\vspace{-2mm}
\begin{center}
\includegraphics[scale=0.15]{i-tetractys.jpg}
\end{center}
\vspace{-2.7mm}

\noindent
Let me leave it at that.
%
As always there would be more that could be said.

\end{document}
