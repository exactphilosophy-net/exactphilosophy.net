\documentclass[letterpaper]{article}
\pagestyle{empty}
\paperheight=3700mm
\textheight=3700mm
%\textwidth set depending on pdflatex or (experimentally)lualatex
\topmargin=-20mm
\oddsidemargin=25mm

\usepackage[utf8]{inputenc}
\usepackage[spanish,italian,french,ngerman,english]{babel} % last is main
\usepackage{graphicx}
\usepackage{multirow}
\usepackage{xcolor}
\usepackage{contour}
\usepackage{pict2e}
\usepackage{relsize}
\usepackage{amsmath}

\usepackage{iftex}
\ifpdftex
  % stronger fonts

% Find modes.mf, e.g. /usr/local/texlive/2025/texmf-dist/fonts/source/public/modes/modes.mf
%
% $ sudo cp modes.mf modes.mf.orig
%
% Add the following at the start of modes:
%
% mode_def xphi =
%   mode_param (pixels_per_inch, 1200);
%   mode_param (blacker, 1.9); % only difference to 'lexmarkr' (2 there)
%   mode_param (fillin, 0);
%   mode_param (o_correction, 1);
%   mode_common_setup_;
% enddef;
%
% Finally:
%
% $ sudo fmtutil-sys --byfmt mf

\pdfpkresolution=1200
\pdfpkmode={xphi}
\pdfmapfile{}

  \newcommand{\coretextwidth}{85.5mm}
\fi

% experimental, not used to produce the live website...
\ifluatex
  % about same heaviness in pdfs when rasterized in photoshop,
  % but since, unlike the metafont mechanisms I use, fake bold, "bleeds" in all directions,
  % seems heavier at least in core web page images
  \newcommand{\fontbleed}{0.8}
  % paragraphs wider and font looks larger, tried to fix, but then other things change a bit,
  % especially for section headings would have to change back, are now more narrow...
  \newcommand{\fontscale}{0.985}
  \newcommand{\coretextwidth}{85.2mm}
  \usepackage{fontspec}
  % microtype does maybe help and not help, maybe if would allow wider spaces...
  \usepackage{microtype}
  \setsansfont{Latin Modern Sans}[Scale=\fontscale, FakeBold=\fontbleed]
  \setmonofont{Latin Modern Mono}[Scale=\fontscale, FakeBold=\fontbleed]
  % this would be for "new computer modern" (but has many limitations so far)
  %\usepackage[default]{fontsetup}
  %\renewcommand{\familydefault}{\sfdefault}
\fi

\renewcommand{\familydefault}{\sfdefault}

\setcounter{secnumdepth}{-1}

\newcommand{\en}[1]{\iflanguage{english}{#1}{}}
\newcommand{\de}[1]{\iflanguage{ngerman}{#1}{}}
\newcommand{\fr}[1]{\iflanguage{french}{#1}{}}

% a bit less than ~255/256
\definecolor{almostwhite}{gray}{0.996}
\definecolor{xphi}{rgb}{0.0,0.5,0.5}
\definecolor{avant}{rgb}{1,0.5,0.5}

\definecolor{frame}{gray}{0.9}
\definecolor{lightgray}{gray}{0.8}
\definecolor{gray}{gray}{0.5}
\definecolor{darkgray}{gray}{0.3}

\definecolor{darkred}{rgb}{0.8,0.0,0.0}
\definecolor{darkyellow}{rgb}{0.7,0.7,0.0}
\definecolor{darkgreen}{rgb}{0.0,0.55,0.0}
\definecolor{darkviolet}{rgb}{0.5,0,0.5}

\definecolor{darkblue}{rgb}{0,0,0.7}
\definecolor{odyssey}{rgb}{0,0,0.8}
\definecolor{indigo}{rgb}{0.29,0,0.51}
\definecolor{indigoblue}{rgb}{0.1,0,0.6}

\definecolor{saffronback}{rgb}{1.000,0.878,0.627}
\definecolor{saffronfront}{rgb}{0.376,0.125,0.000}

\DeclareRobustCommand{\cometartemisscale}[1]{\includegraphics[scale=#1]{\sourcepath/i-comet.jpg}\hspace{-0.028453em} artemis}
\newcommand\cometartemis{\cometartemisscale{0.018}}
\newcommand\cometartemissection{\cometartemisscale{0.0225}}

\DeclareRobustCommand{\moebius}[1]{\includegraphics[scale=#1]{\sourcepath/i-moebius.jpg}}
\newcommand{\yinyang}{\includegraphics[scale=0.135]{\sourcepath/i-yinyang.jpg}}

\newcommand{\rarr}{\,$\rightarrow$\,}
\newcommand{\lrarr}{\,$\leftrightarrow$\,}

% greek elements
\newcommand{\elfire}{
\begin{picture}(9,6)
  \thicklines
  \put(1,-0.5){\line(1,0){7}}
  \put(1,-0.5){\line(1,1.732){3.5}}
  \put(8,-0.5){\line(-1,1.732){3.5}}
\end{picture}}
%
\newcommand{\elair}{
\begin{picture}(9,6)
  \thicklines
  \put(1,-0.5){\line(1,0){7}}
  \put(1,-0.5){\line(1,1.732){3.5}}
  \put(8,-0.5){\line(-1,1.732){3.5}}
  \put(2.75,1.9){\line(1,0){3.5}}
\end{picture}}
%
\newcommand{\elwater}{
\begin{picture}(9,6)
  \thicklines
  \put(1,5){\line(1,0){7}}
  \put(1,5){\line(1,-1.732){3.5}}
  \put(8,5){\line(-1,-1.732){3.5}}
\end{picture}}
%
\newcommand{\elearth}{
\begin{picture}(9,6)
  \thicklines
  \put(1,5){\line(1,0){7}}
  \put(1,5){\line(1,-1.732){3.5}}
  \put(8,5){\line(-1,-1.732){3.5}}
  \put(2.75,2.5){\line(1,0){3.5}}
\end{picture}}
%
\newcommand{\elhex}{
\begin{picture}(9,6)
  \thicklines
  \put(1,0.5){\line(1,0){7}}
  \put(1,0.5){\line(1,1.732){3.5}}
  \put(8,0.5){\line(-1,1.732){3.5}}
  \put(1,5){\line(1,0){7}}
  \put(1,5){\line(1,-1.732){3.5}}
  \put(8,5){\line(-1,-1.732){3.5}}
\end{picture}}

% i ching trigrams
\newcommand{\trigram}[3]{
\begin{picture}(9,6)
  \linethickness{0.36mm}
  \put(0,5){\line(1,0){#1}}
  \put(5.5,5){\line(1,0){3.5}}
  \put(0,2.5){\line(1,0){#2}}
  \put(5.5,2.5){\line(1,0){3.5}}
  \put(0,0){\line(1,0){#3}}
  \put(5.5,0){\line(1,0){3.5}}
\end{picture}}
\newcommand{\triheaven}{\trigram{5.5}{5.5}{5.5}}
\newcommand{\triearth}{\trigram{3.5}{3.5}{3.5}}
\newcommand{\trithunder}{\trigram{3.5}{3.5}{5.5}}
\newcommand{\triwater}{\trigram{3.5}{5.5}{3.5}}
\newcommand{\trimountain}{\trigram{5.5}{3.5}{3.5}}
\newcommand{\triwind}{\trigram{5.5}{5.5}{3.5}}
\newcommand{\trifire}{\trigram{5.5}{3.5}{5.5}}
\newcommand{\trilake}{\trigram{3.5}{5.5}{5.5}}

% i ching hexagrams
% 1+2 trigrams, 3 rest of line (see e.g. dreams.tex)
\DeclareRobustCommand{\hexagram}[3]{\raisebox{-3pt}{$\overset{\text{${#1}$}}{#2}$\,}#3\vspace{3pt}}

% white-red-black etc.
\DeclareRobustCommand{\outline}[1]{\contour{black}{{\color{white}#1}}}
\DeclareRobustCommand{\white}[1]{\outline{\textbf{#1}}}
\DeclareRobustCommand{\red}[1]{{\color{darkred}\textbf{#1}}}
\DeclareRobustCommand{\black}[1]{\textbf{#1}}
\DeclareRobustCommand{\yellow}[1]{{\color{darkyellow}\textbf{#1}}}
\DeclareRobustCommand{\green}[1]{{\color{darkgreen}\textbf{#1}}}
\DeclareRobustCommand{\violet}[1]{{\color{darkviolet}\textbf{#1}}}
\DeclareRobustCommand{\indigoblue}[1]{{\color{indigoblue}\textbf{#1}}}
\DeclareRobustCommand{\indigo}[1]{{\color{indigo}\textbf{#1}}}

% ELEMENTAL
\newcommand{\ELEMENTAL}{%
\colorlet{contour}{.}\textbf{\color{white}%
\raisebox{+0.001em}{\contour{contour}{E}}%
\raisebox{+0.015em}{\contour{contour}{L}}%
\raisebox{+0.016em}{\contour{contour}{E}}%
\raisebox{+0.023em}{\contour{contour}{M}}%
\raisebox{+0.023em}{\contour{contour}{E}}%
\raisebox{+0.017em}{\contour{contour}{N}}%
\raisebox{-0.020em}{\contour{contour}{T}}%
\raisebox{-0.002em}{\contour{contour}{A}}%
\raisebox{+0.006em}{\contour{contour}{L}}%
}}

% artemis pdf+web icons
\newcommand{\ipdfen}{\includegraphics[scale=0.5]{i-pdf-en.png}}
\newcommand{\ipdfde}{\includegraphics[scale=0.5]{i-pdf-de.png}}
\newcommand{\ipdffr}{\includegraphics[scale=0.5]{i-pdf-fr.png}}
\newcommand{\iweb}{\includegraphics[scale=0.055]{i-web.png}}
\newcommand{\ipdfblueen}{\includegraphics[scale=0.5]{i-pdf-blue-en.png}}
\newcommand{\ipdfbluede}{\includegraphics[scale=0.5]{i-pdf-blue-de.png}}
\newcommand{\ipdfbluefr}{\includegraphics[scale=0.5]{i-pdf-blue-fr.png}}
\newcommand{\iwebblue}{\includegraphics[scale=0.055]{i-web-blue.png}}


\textwidth=\coretextwidth



\begin{document}
\selectlanguage{ngerman}

\avantgarde

\section{Die neugierige Statue}

Wie lange es sie schon gab,
wusste sie nicht mehr,
aber sonst hatte sie dank ihrer Neugier fast nichts vergessen.
%
Manchmal stellte sie sich vor,
sie hätte Schubladen in sich,
etwa so wie eine Skulptur von Dalí,
und plötzlich würde sich eine öffnen,
und darin wäre dann eine glänzende Scheibe,
vielleicht ja chrysolithfarben,
je nachdem wie ihr gerade zumute wäre,
und darauf wäre dann alles gespeichert,
was ihr über die Jahrtausende so vor der Nase herumlief.
%
Da sie so neugierig war,
hätte sie natürlich auch brennend interessiert,
wie die Menschen heute auf ihre früheren Zeitgenossen reagiert hätten.

Natürlich kann ich als Erzähler nun,
solange das nicht wirklich passiert ist,
nicht wirklich erzählen,
wie sich die Menschen,
oder auch die Tiere, usw.,
vor der Statue benommen hatten,
da die Statue darauf das Copyright hat,
aber ich kann vieles so erzählen,
dass es der Spur nach plausibel erscheinen mag,
und, wer weiss,
vielleicht stimmt ja doch oft viel mehr als ich zugeben durfte?

Was sie ja ab und zu recht fuchste,
war,
dass sie sich selbst so selten sah.
%
Irgendwann mal kamen ein paar Leute
und setzten eine metallisch verspiegelte grüne Kugel neben sie
und machten Fotos.
%
Vielleicht war die Kugel auch blau,
so genau hatte sie es nicht mit den Farben.
%
Das hatte sich ja auch immer wieder gewandelt,
manche Dialekte hatten gar nicht verschiedene Namen für grün und blau,
oder für rot und gelb.
%
Manchmal kamen auch Besucher,
die voller Imbrunst erklärten,
dass weiss und schwarz gar keine Farben wären.
%
Was denn sonst,
fragte sie sich manchmal,
auch wenn sie natürlich in Quantenfeldtheorien bis hin zu Strings und Membranen
und noch viel gewagteren Spekulationen
seit der ersten Stunde sehr bewandert war.

Wann genau sie den linken Arm bis fast unter ihre Schulter verloren hatte,
konnte sie daher nicht mehr so genau sagen.
%
Vielleicht war es ja damals als die Tempelanlage um sie …\
war das vielleicht sogar Delphi,
oder Ephesus,
oder doch irgendwo in der Tundra …\
jedenfalls war da ein rechtes Chaos und viel ging kaputt;
vielleicht war sie ja danach ein paar Jahrhunderte oder so
in einer recht kleinen Kammer mit meist nur Ratten als Besuchern.
%
Aber zurück zu ihrem Arm.
%
Erst war sie enttäuscht, dass dadurch ihre Symmetrie gebrochen wurde,
wohl noch lange vor Emmy Noether,
aber dann bemerkte sie,
dass gerade diese Asymmetrie sie geheimnisvoller machte,
wenn sich die Betrachter im Geist den Arm doch wieder vorstellten,
wenn auch oft nicht bewusst.
%
Aber ganz so einfach gebaut war sie natürlich nicht,
vielleicht hatte sie ja sogar damals das mit dem Arm genauso arrangiert,
ganz im Wissen um die Wirkung.
%
Schade allerdings,
dass bei ihr nie etwas nachwuchs;
so gerne hätte sie mit anderen Varianten
und deren Wirkung auf die Betrachter experimentiert.
%
Aber wer weiss, die Akropolis wird ja mittlerweile anscheinend auch wieder ganz ergänzt…

Am schlimmsten fand sie oft die Eltern
oder auch die meist männlichen Liebhaber gegenüber ihren Geliebten,
wenn sie ihren Begleiterinnen und Begleitern unbeholfen und unbedarft,
aber sehr überzeugt scheinend,
erklärten,
was sie gerade zu sehen hätten.
%
Dabei waren doch Kinder,
ganz alte Leute,
und die unscheinbareren Teile der Liebespaare,
also oft die Frauen,
gleich von Beginn weg viel näher an der Statue.
%
Und das meine ich gar nicht mal so vegetativ als noch oder schon näher beim Tod,
wie die kalte, unbewegliche Statue,
die nur noch verfällt.
%
Aber wieso doch nicht,
das kann ich hier nun wirklich nicht verraten,
beziehungsweise ist das doch eben so offensichtlich…
%
Die wirklich wichtigen Dinge erklärt einem nie jemand im Leben.

Die so hochgelobte Renaissance!
%
Ja,
es stimmt schon,
die Betrachter ähnelten da schon eher denjenigen in der Antike,
und es gab dann auch genauso oft verirrte Gestalten,
die sich an ihr zu vergehen versuchten,
wo doch von Anfang an klar war,
dass daraus nichts Ernsteres werden könnte.
%
Wie stellten sich das die jungen Schnösel vor mit einer Beziehung mit ihr?
%
Dass sie zusammen im Museum wohnen würden,
er vielleicht als Museumswärter?
%
Und ihre Kinder, wären das eine Art Roboter,
die aus ihren Schubladen quellen würden,
und dann was?
%
Sie geheim halten wäre schwierig,
denn so konsequent stillhalten wie ihre Mutter könnten sie ja doch nicht,
und wenn bekannt würde,
was sie wären,
dann wäre wohl jede Chance auf ein gemeinsames Familienleben vertan.
%
Da wären ja die altbekannten Geschichten mit den Melusinen,
oder hiessen sie Meerjungfrauen,
und wieso eigentlich Jungfrauen,
noch ganz harmonisch im Vergleich.
%
Aber sie fühlte sich natürlich auch bei den hässlichsten Kandidaten
immer doch noch insgeheim noch ein wenig geehrt,
denn,
wie gesagt,
ihr wirklich was antun konnte sowieso niemand,
und daher hätte sie wohl auch niemals Nein gesagt,
wenn sie da Stimmbänder gehabt hätte,
nicht zuletzt da sie eben so neugierig war.
%
Oder hätte sich doch ab und zu auch mal Nein gesagt,
einfach um dann die Reaktionen zu sehen,
oder vielleicht eher weil es ihr mit der Zeit langweilig geworden wäre?

Männer schauen ja meist viel weniger in den Spiegel als Frauen.
%
Daher kann es schon sein, dass die Statue über die Epochen
auch gewisse männliche Züge entwickelte.
%
So konnte sie auch gut verstehen,
dass z.B. für einen Harley Wiesenstein,
wenn auf seiner Netzhaut die Bilder einer sehr schönen Frau erschienen,
er nie bedachte,
dass es von der Frau aus gesehen einen grossen Unterschied machte,
ob er nun 20 oder 60 Jahre alt wäre,
von anderen wohl kriminell arrangierten Umständen mal ganz abgesehen.
%
Aber bei allem Verständnis,
war ihre Verachtung und Solidarität mit ihren vergleichsweise so kurzlebigen Schwestern
immer auch fast ohne Grenzen.
%
Sie hätte jede Petition und insbesondere jedes Kunstprojekt gefördert,
das dem Einhalt gebietet.
%
Böse Stimmen mögen nun einwenden,
dass sie das genau dadurch wieder zum Thema macht.
%
Ja,
genau,
aber gespielt,
mit voller Intensität,
und daher ideal geheilt und umgangen.
%
Oder glaubt ihr tatsächlich,
dass ihr die Männer ändern könnt,
statt sie nur immer und immer wieder geschickt abzulenken?
%
Daraus ist doch nicht zuletzt nach Freud die ganze menschliche Kultur entstanden.
%
Aber vielleicht spricht daraus ja doch nicht mehr die Statue,
sondern nur ein sogar vielleicht aus Nachlässigkeit nicht ganz so genau gelenkter Erzähler?
%
Oder ist auch das wieder ganz absichtlich von der Statue?
%
Tja,
die wirklich wichtigen Dinge erklärt einem eben nie jemand im Leben…

\newpage

Ich kann es übrigens kaum glauben,
dass das heute einfach so entsteht,
nachdem ich die paar vorherigen Tage schon einiges geschaffen hatte,
fast wie ein Gordischer Knoten,
der sich endlich ein wenig lösen würde.
%
Und Nein,
die Idee mit der Schublade ist nicht erst vor zwei Wochen entstanden
nach dem Publicity Stunt von Banksy mit dem sich selbst schreddernden Gemälde,
sondern ist bereits einiges älter,
vielleicht so um 2010/11 entstanden,
aber so ganz genau weiss ich das nicht mehr,
aber sicher einiges vor Mitte 2015.
%
Also die Idee natürlich nicht weitersagen,
einfach machen…

Na ja,
jedenfalls war das Mittelalter schon auch recht interessant für die Statue;
auch wenn dann das Interesse eher gering war,
waren die wenigen Begegnungen damals oft recht vielfältig und interessant.
%
Ganz erstaunlich waren immer wieder die treuen Elstern.
%
Aber damit sollte ich nun wirklich schliessen,
sonst verlaufe ich mich vielleicht noch ganz.

\vspace{7mm}
\noindent
\includegraphics[scale=0.1865]{i-feather.jpg}

\end{document}
