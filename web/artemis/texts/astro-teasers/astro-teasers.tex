\documentclass[letterpaper]{article}
\pagestyle{empty}
\paperheight=3700mm
\textheight=3700mm
%\textwidth set depending on pdflatex or (experimentally)lualatex
\topmargin=-20mm
\oddsidemargin=25mm

\usepackage[utf8]{inputenc}
\usepackage[spanish,italian,french,ngerman,english]{babel} % last is main
\usepackage{graphicx}
\usepackage{multirow}
\usepackage{xcolor}
\usepackage{contour}
\usepackage{pict2e}
\usepackage{relsize}
\usepackage{amsmath}

\usepackage{iftex}
\ifpdftex
  % stronger fonts

% Find modes.mf, e.g. /usr/local/texlive/2025/texmf-dist/fonts/source/public/modes/modes.mf
%
% $ sudo cp modes.mf modes.mf.orig
%
% Add the following at the start of modes:
%
% mode_def xphi =
%   mode_param (pixels_per_inch, 1200);
%   mode_param (blacker, 1.9); % only difference to 'lexmarkr' (2 there)
%   mode_param (fillin, 0);
%   mode_param (o_correction, 1);
%   mode_common_setup_;
% enddef;
%
% Finally:
%
% $ sudo fmtutil-sys --byfmt mf

\pdfpkresolution=1200
\pdfpkmode={xphi}
\pdfmapfile{}

  \newcommand{\coretextwidth}{85.5mm}
\fi

% experimental, not used to produce the live website...
\ifluatex
  % about same heaviness in pdfs when rasterized in photoshop,
  % but since, unlike the metafont mechanisms I use, fake bold, "bleeds" in all directions,
  % seems heavier at least in core web page images
  \newcommand{\fontbleed}{0.8}
  % paragraphs wider and font looks larger, tried to fix, but then other things change a bit,
  % especially for section headings would have to change back, are now more narrow...
  \newcommand{\fontscale}{0.985}
  \newcommand{\coretextwidth}{85.2mm}
  \usepackage{fontspec}
  % microtype does maybe help and not help, maybe if would allow wider spaces...
  \usepackage{microtype}
  \setsansfont{Latin Modern Sans}[Scale=\fontscale, FakeBold=\fontbleed]
  \setmonofont{Latin Modern Mono}[Scale=\fontscale, FakeBold=\fontbleed]
  % this would be for "new computer modern" (but has many limitations so far)
  %\usepackage[default]{fontsetup}
  %\renewcommand{\familydefault}{\sfdefault}
\fi

\renewcommand{\familydefault}{\sfdefault}

\setcounter{secnumdepth}{-1}

\newcommand{\en}[1]{\iflanguage{english}{#1}{}}
\newcommand{\de}[1]{\iflanguage{ngerman}{#1}{}}
\newcommand{\fr}[1]{\iflanguage{french}{#1}{}}

% a bit less than ~255/256
\definecolor{almostwhite}{gray}{0.996}
\definecolor{xphi}{rgb}{0.0,0.5,0.5}
\definecolor{avant}{rgb}{1,0.5,0.5}

\definecolor{frame}{gray}{0.9}
\definecolor{lightgray}{gray}{0.8}
\definecolor{gray}{gray}{0.5}
\definecolor{darkgray}{gray}{0.3}

\definecolor{darkred}{rgb}{0.8,0.0,0.0}
\definecolor{darkyellow}{rgb}{0.7,0.7,0.0}
\definecolor{darkgreen}{rgb}{0.0,0.55,0.0}
\definecolor{darkviolet}{rgb}{0.5,0,0.5}

\definecolor{darkblue}{rgb}{0,0,0.7}
\definecolor{odyssey}{rgb}{0,0,0.8}
\definecolor{indigo}{rgb}{0.29,0,0.51}
\definecolor{indigoblue}{rgb}{0.1,0,0.6}

\definecolor{saffronback}{rgb}{1.000,0.878,0.627}
\definecolor{saffronfront}{rgb}{0.376,0.125,0.000}

\DeclareRobustCommand{\cometartemisscale}[1]{\includegraphics[scale=#1]{\sourcepath/i-comet.jpg}\hspace{-0.028453em} artemis}
\newcommand\cometartemis{\cometartemisscale{0.018}}
\newcommand\cometartemissection{\cometartemisscale{0.0225}}

\DeclareRobustCommand{\moebius}[1]{\includegraphics[scale=#1]{\sourcepath/i-moebius.jpg}}
\newcommand{\yinyang}{\includegraphics[scale=0.135]{\sourcepath/i-yinyang.jpg}}

\newcommand{\rarr}{\,$\rightarrow$\,}
\newcommand{\lrarr}{\,$\leftrightarrow$\,}

% greek elements
\newcommand{\elfire}{
\begin{picture}(9,6)
  \thicklines
  \put(1,-0.5){\line(1,0){7}}
  \put(1,-0.5){\line(1,1.732){3.5}}
  \put(8,-0.5){\line(-1,1.732){3.5}}
\end{picture}}
%
\newcommand{\elair}{
\begin{picture}(9,6)
  \thicklines
  \put(1,-0.5){\line(1,0){7}}
  \put(1,-0.5){\line(1,1.732){3.5}}
  \put(8,-0.5){\line(-1,1.732){3.5}}
  \put(2.75,1.9){\line(1,0){3.5}}
\end{picture}}
%
\newcommand{\elwater}{
\begin{picture}(9,6)
  \thicklines
  \put(1,5){\line(1,0){7}}
  \put(1,5){\line(1,-1.732){3.5}}
  \put(8,5){\line(-1,-1.732){3.5}}
\end{picture}}
%
\newcommand{\elearth}{
\begin{picture}(9,6)
  \thicklines
  \put(1,5){\line(1,0){7}}
  \put(1,5){\line(1,-1.732){3.5}}
  \put(8,5){\line(-1,-1.732){3.5}}
  \put(2.75,2.5){\line(1,0){3.5}}
\end{picture}}
%
\newcommand{\elhex}{
\begin{picture}(9,6)
  \thicklines
  \put(1,0.5){\line(1,0){7}}
  \put(1,0.5){\line(1,1.732){3.5}}
  \put(8,0.5){\line(-1,1.732){3.5}}
  \put(1,5){\line(1,0){7}}
  \put(1,5){\line(1,-1.732){3.5}}
  \put(8,5){\line(-1,-1.732){3.5}}
\end{picture}}

% i ching trigrams
\newcommand{\trigram}[3]{
\begin{picture}(9,6)
  \linethickness{0.36mm}
  \put(0,5){\line(1,0){#1}}
  \put(5.5,5){\line(1,0){3.5}}
  \put(0,2.5){\line(1,0){#2}}
  \put(5.5,2.5){\line(1,0){3.5}}
  \put(0,0){\line(1,0){#3}}
  \put(5.5,0){\line(1,0){3.5}}
\end{picture}}
\newcommand{\triheaven}{\trigram{5.5}{5.5}{5.5}}
\newcommand{\triearth}{\trigram{3.5}{3.5}{3.5}}
\newcommand{\trithunder}{\trigram{3.5}{3.5}{5.5}}
\newcommand{\triwater}{\trigram{3.5}{5.5}{3.5}}
\newcommand{\trimountain}{\trigram{5.5}{3.5}{3.5}}
\newcommand{\triwind}{\trigram{5.5}{5.5}{3.5}}
\newcommand{\trifire}{\trigram{5.5}{3.5}{5.5}}
\newcommand{\trilake}{\trigram{3.5}{5.5}{5.5}}

% i ching hexagrams
% 1+2 trigrams, 3 rest of line (see e.g. dreams.tex)
\DeclareRobustCommand{\hexagram}[3]{\raisebox{-3pt}{$\overset{\text{${#1}$}}{#2}$\,}#3\vspace{3pt}}

% white-red-black etc.
\DeclareRobustCommand{\outline}[1]{\contour{black}{{\color{white}#1}}}
\DeclareRobustCommand{\white}[1]{\outline{\textbf{#1}}}
\DeclareRobustCommand{\red}[1]{{\color{darkred}\textbf{#1}}}
\DeclareRobustCommand{\black}[1]{\textbf{#1}}
\DeclareRobustCommand{\yellow}[1]{{\color{darkyellow}\textbf{#1}}}
\DeclareRobustCommand{\green}[1]{{\color{darkgreen}\textbf{#1}}}
\DeclareRobustCommand{\violet}[1]{{\color{darkviolet}\textbf{#1}}}
\DeclareRobustCommand{\indigoblue}[1]{{\color{indigoblue}\textbf{#1}}}
\DeclareRobustCommand{\indigo}[1]{{\color{indigo}\textbf{#1}}}

% ELEMENTAL
\newcommand{\ELEMENTAL}{%
\colorlet{contour}{.}\textbf{\color{white}%
\raisebox{+0.001em}{\contour{contour}{E}}%
\raisebox{+0.015em}{\contour{contour}{L}}%
\raisebox{+0.016em}{\contour{contour}{E}}%
\raisebox{+0.023em}{\contour{contour}{M}}%
\raisebox{+0.023em}{\contour{contour}{E}}%
\raisebox{+0.017em}{\contour{contour}{N}}%
\raisebox{-0.020em}{\contour{contour}{T}}%
\raisebox{-0.002em}{\contour{contour}{A}}%
\raisebox{+0.006em}{\contour{contour}{L}}%
}}

% artemis pdf+web icons
\newcommand{\ipdfen}{\includegraphics[scale=0.5]{i-pdf-en.png}}
\newcommand{\ipdfde}{\includegraphics[scale=0.5]{i-pdf-de.png}}
\newcommand{\ipdffr}{\includegraphics[scale=0.5]{i-pdf-fr.png}}
\newcommand{\iweb}{\includegraphics[scale=0.055]{i-web.png}}
\newcommand{\ipdfblueen}{\includegraphics[scale=0.5]{i-pdf-blue-en.png}}
\newcommand{\ipdfbluede}{\includegraphics[scale=0.5]{i-pdf-blue-de.png}}
\newcommand{\ipdfbluefr}{\includegraphics[scale=0.5]{i-pdf-blue-fr.png}}
\newcommand{\iwebblue}{\includegraphics[scale=0.055]{i-web-blue.png}}


\textwidth=\coretextwidth



\begin{document}

\avantgarde

\section{Astro teasers}

Some ideas around astrology that felt original and interesting enough to share,
but for many different reasons did not make it into dedicated articles or books,
including that in much of astrology I am not so sure how close I might be…

\subsection{Deep Learning and astrology}

Big Tech companies have access to trillions of images on the Internet
(part of what is called \textsl{Big Data}),
often with user comments like “this is my car in front of my house”.
%
What those companies then do,
is feed computers with this information (this is called \textsl{Deep Learning}),
creating basically an electronic copy
of how brain cells in a human brain are interconnected,
a so-called \textsl{neural network}.
%
After that,
new images can be presented to the neural network
and it can often tell what is on them.
%
Details are certainly hairy,
but it turns out that already in 2019 the computer could often beat humans at recognizing
what is on an image\,!

To make things more tangible,
here an example of what you might typically feed a neural network for a single image:

\vspace{2mm}
\includegraphics[scale=0.2]{i-beetle.jpg}

\textbf{\color{odyssey}Car / House / VW Beetle / …}

\vspace{2mm}
\noindent
Now,
suppose you knew when the photo was taken and where.
%
Note that most electronic cameras today,
including the ones in cell phones,
usually provide this information for free,
via GPS and a built-in or Internet clock,
and save this information automatically with each image.

That way you could derive quite similar \textsl{astrological} tags from the image:

\vspace{2mm}
\hspace{4mm}\includegraphics[scale=0.2]{i-beetle.jpg}

\hspace{4mm}\textbf{\color{odyssey}Ascendent Taurus / Venus in Leo / Saturn in Pisces / …}

\vspace{5mm}\noindent
Now you could simply first feed the neural network with images
plus their astrological tags
and afterwards present it with new images.
%
If there was something to astrology,
the neural network would likely be able to say something like
“Ascendent Taurus” or “Venus in Leo”.

\subsection{Funky Venus-Saturn}

Funky musicians seem to often have Venus-Saturn aspects in their birth charts.

Even though such aspects are far from being rare,
their prominence in well-known funky musicians
looks quite astounding.
%
See the posts in Yvette’s thread
“Welche Musik bringt Eure Sonne auf Trab\,?”
at the Astrodienst (astro.com) astrology forum
starting February 2021 for a few months with many examples.

The basic idea is
that Saturn is “rhythm”
and Venus “melody”,
and that it would be “funky”
when the two things vary in relation,
are not perfectly in tune,
when there are surprises,
or maybe, also as Prince sings,
“a little bit behind the beat, just enough to turn you on”,
since Venus is also the goddess of love.

I am quite confident that by analyzing recordings
for how rhythm and melody interplay,
it would be possible to identify “funky”
and correlate it with birth charts,
i.e.\ this might be one of the most viable and direct ways
to prove something in astrology,
and it would be fun to listen to all that funky music.

The whole thing culminated for me in some recordings by The Doors,
and Ray Manzarek in several interviews.
%
All four members of The Doors have/had Venus-Saturn aspects,
and there were aspects from Venus and Saturn in transit
to Saturn resp.\ Venus at birth
when Jim Morrison died in Paris in 1971.

I am aware that many people can hardly tell
whether something is funky or not.
%
But the correlation between the music down on earth (rhythm and melody)
and the music of the spheres up in the sky (Venus-Saturn aspects)
should be sufficient to identify the interrelation.
%
Interestingly Venice (California)—%
where,
according to Ray Manzarek,
The Doors emerged one evening in mid-July 1965 at sunset on the beach
when Jim and Ray met there by chance—%
was characterized by Ray to have been
at the time “dark and funky”.

\begin{center}
\includegraphics[scale=0.23]{i-jim-morrison.jpg}
\end{center}
\vspace{-3mm}

\subsection{Uranus and Democracy}

I had always wondered how Uranus,
who was rather an archaic tyrant in mythology,
could be related to democracy.
%
In a nutshell,
it is because Uranus as the leader of the herd or tribe
has no army or other institutions to protect him,
he needs the support of many,
better a clear majority,
to stay relatively safe as the leader.
%
This is actually also the key to understanding quite a few things about the crazy times
we have been living in in the past 2.5 centuries since the discovery of Uranus in 1781,
right between the American and French Revolutions.
 
In a herd there is often a leading male
that gets to fertilize practically all females.
%
Other males try to replace the leader again and again,
and eventually one will succeed,
once the current leader gets older.
%
The takeover of Cronos/Saturn by castrating Uranus
with sort of a “moon crescent” reflects this in mythology.
%
It is also often considered the transition from hunters and gatherers to agriculture,
the replacement of often ad hoc moving elsewhere,
following the food,
to a more regular structure of the year,
with given times for seed and harvest.

This is obviously progress in many respects,
but with respect to democracy rather not.
%
Imagine you are a fellah (farmer) in ancient Egypt
and unhappy with how the pharaoh is running the country.
%
But the pharaoh is out of reach,
living in his palace far away
and guarded by walls and guards.
%
Now imagine the situation in a tribe that is constantly moving.
%
Probably everybody is lying close together at night,
maybe already around a fireplace.
%
The leader is much less protected that way;
better have many supporters around so that he would be warned,
making the risk for others to attack him in his sleep too big.

That way, both a crude form of democracy
and constant danger are imminent.
%
Also,
the leader can (and often has to) change path for the whole tribe
much more quickly
and unexpectedly than a king in well-organized society.
%
A leader who is good at this is important.
%
This makes things also easier for new inventions to be applied,
however \textsl{only} if they can immediately be applied,
i.e.\ if they are ripe and ready for \textsl{mainstream},
are useful for where the tribe is going.

If all of this now reminds you of the USA,
where there was the first modern revolution,
only 5 years before the conscious discovery of Uranus,
then I can only agree.
%
Take Trump, and some of his encouraged supporters entering the Capitol in January 2021.
%
Or his style of making decisions, hire and fire, seemingly unpredictable.
%
And always on the lookout for new solutions to the problems at hand.
%
All of this is not “bad” in principle,
just “Uranian”,
with all advantages and disadvantages that come with it.
%
The USA overall are much like this,
including both major political parties, etc.

The god Uranus is also the whole sky/heaven;
in mythology he was created by Gaia, the earth.
%
Now,
astronomically in the sky there have been more and more discoveries after Uranus.
%
This is mirrored in astrology.
%
Most newly discovered planets and similar objects
have aquarian themes,
as one would expect for a gradually emerging Age of Aquarius.
%
And each of them was once bringing a seemingly radical change,
that was supposed to make everything better.

Uranus was emphasis on consciousness and light (Enlightenment),
the steam engine,
and more,
including,
by the way,
undecorated sans serif fonts.
%
Aquarius has to do with water by its name,
and Neptune/Poseidon is the god of the sea,
who brought in some way the unconscious back,
the unknown world below,
and the fluid oil,
and all the machinery that used it.
%
Then came Pluto,
actually as it turned out since then,
not a planet,
more the temporary spark of the leader of the herd,
with the promise to bring more depth,
related to Februs,
the god who rules February,
which contains 2/3 of Aquarius,
and Februs was also an underworld god like Pluto/Hades,
which then also leads to themes of sexual domination,
at least of the leader of the herd over the females,
and exclusion of the other males.
%
But then came new objects,
like Chiron, Eris, Sedna, Chariklo, etc.,
and the overall picture is shifting again,
to other aquarian themes
like equality between sexes and (new word) genders,
and tolerance between all.

Let me stop with examples here
and just say that a million things in history since then
seem to quite obviously and stunningly reflect in this picture.

In astrology not only new objects have been found,
but also many new methods,
to the degree that in a way all of them are individually losing power.
%
My hope is that a lot of the boundlessness of change today,
that sort of explosion of new things,
will somehow settle in the coming years,
maybe by 2044,
since Aquarius is an air sign,
so reason should rule,
and it is a fixed sign,
so change would be constant in a way,
most of the time not really radical.

If you divide the great year of precession
into 12 equal sections,
in analogy to what happened when tropical astrology
with 12 equally sized star signs was introduced about 2000 years ago,
and start with fixed star Alrisha (“cord”) at the spring point,
the star that connects horizontal mother fish and vertical son fish
in the constellation Pisces as a “knot” in their umbilical cord,
symbolizing “birth”,
then the Age of Aquarius would hypothetically start in 2044.
%
See the appendix
for charts when Alrisha nominally switches tropical signs,
but the exact moment depends on challengeable definitions,
and Alrisha is a binary star.

\subsection{Sedna times\,?}

In the view of many astrologers,
Pluto has strongly influenced the 20th century,
both in terms of global events and personal fates,
since at least its discovery in 1930,
even though it is now clear that Pluto is just one of several objects of similar size in a similar orbit.

Just like Pluto is the first discovered object beyond the orbit of Neptune,
Sedna is the first discovered object from a region quite a bit further out.
%
Since births are so crucial in astrology,
could being discovered first be a crucial factor regarding which objects get how much weight\,?

Would Sedna thus maybe be a major factor in current astrology\,?
%
Would Sedna rule Libra, and would currently be well placed in Taurus\,?

In the following just a few teasers, feel free to investigate more.

The world is moving away from nuclear power;
countries that want to build nuclear weapons are tamed with economic sanctions.
%
In 2009 President Obama held a speech for a nuclear-free world in Prague,
the same city in which in 2006 it was decided that Pluto is no longer scientifically classified as a planet.

In 2004,
the year Sedna was discovered in photos from autumn 2003,
the video “1 Night in Paris” came out,
which arguably launched a new kind of celebrities who are famous even without conventional skills for stardom
like, for example, singing or acting,
but rather by just being desirable.

Since then cameras have essentially turned away from objects towards the persons taking them,
selfies,
just like Venus looking at herself in the mirror (see the glyph for planet Venus).
%
The whole internet has become even more focussed on money,
plus beauty, community, and so on.

It might even be so that male state leaders and other (plutonic) men in powerful positions are no longer in absolute power,
but rather at most tolerated;
the {\small\#}MeToo revelations appear not unlikely to be part of this, too.

It may seen strange that the apparently helpless young woman in the Inuit myth would have such an effect,
since she appears to be so much a victim there,
but is it exactly that experience which makes her stronger than Pluto\,?

Remember that Natascha Kampusch managed to escape her kidnapper in Vienna after eight years of captivity
just one day before the deciding vote on Pluto in Prague on 24 August 2006\,?

\subsection{Vesta as elected representative of asteroids and Virgo\,?}

Vesta was discovered as the fourth asteroid,
after Ceres, Pallas and Juno,
and before Astraea,
who marked the beginning of the discovery of many asteroids
around the time Neptune was discovered in 1846.
%
These five asteroids,
as well as many that followed shortly,
would seem to be often related to Virgo themes,
all in all more than to other star signs,
plus asteroids overall remind of the attention to detail of Virgo.
%
Small animals are related to Virgo (via the 6th house)
and in German there is the saying “Kleinvieh macht auch Mist”,
small animals produce manure, too.
%
Direct democracy is most prominent in Switzerland and California,
two regions with the sun in Virgo,
and it came up in antiquity most prominently and distinctively
in Athens in Greece
(in Jean Richer’s zodiac centered in Delphi late in the Virgo sector),
which leads via Pallas Athena also to asteroids.

Vesta is not the largest asteroid,
but the \yellow{brightest} one,
the only one that can apparently be spotted without a telescope
when ideally placed, under ideal conditions with good eyes,
a property otherwise apparently only shared with planet Uranus.
%
This might make Vesta,
first goddess,
related to the discovery of fire in prehistory
and to the central fire or hearth in caves and homes,
a good candidate to maybe democratically represent the other asteroids,
like Uranus close to having been already known in antiquity.

Vesta’s virginity might make her a lawyer that is impartial,
would always truly represent the other asteroids,
maybe depending on which ones are in aspect to her or to the other planets.
%
In any case,
Vesta might be the object that would be most naturally given Virgo as her domicile,
would have her dominion there,
not as a sole ruler,
as maybe the case with rulers of other signs,
but more as something along an elected representative
or even some kind of messenger for causes of other asteroids,
hence also a theme that rings with the current official ruler,
Mercury (Hermes),
messenger of Jupiter (Zeus) and other gods.

\vspace{-2mm}
\begin{center}
\includegraphics[scale=0.23]{i-harvest.jpg}
\end{center}
\vspace{-2mm}

\noindent
The works of three people
indirectly helped me to assemble these themes around Vesta
into a “case” for her domicile,
more as a messenger than anything else,
and note that I do not generally embrace their works:
%
Günther Pingitzer from Vienna,
who had been promoting Vesta as ruler of Virgo
for many years at the Astrodienst (astro.com) forum
as Quadrix or Moonie23 or Moonman23,
and maybe other screen names before I laid my eyes there;
Richard Vetter
with his idea of Vesta as “focus and amplifier”
of attributes of other objects when connected to Vesta
and the information about her role in ancient Rome
as protectress of the state beyond individual homes
(see German AstroWiki about Vesta for both and more);
Emil Lips with his postulated planet Iustitia for Virgo.
%
And thanks to Robert Graves
for associating Hestia/Vesta
with the ‘White Goddess’,
especially in the introduction of \textsl{The Greek Myths}.

The very first visit of the harvester of the Swiss National Library
to archive exactphilosophy.net, as announced a few weeks earlier,
was 10 December 2021,
just a few hours after I had published the first version of this section.
%
On the previous page the chart for the download of this article then,
from Switzerland according to the IP,
but not sure if precisely in Bern. Vesta near the MC.

The harvested version of the text about Vesta
also contained two speculative sentences about Vesta (and Uranus)
maybe having been known at least in some circles in prehistory.
%
I removed them the same afternoon
because I deemed them too speculative then,
but maybe in a time of hunters the average eye sight was really quite a bit better than today,
or maybe nature just used the eyes of eagles and the like,
so that maybe there could have even been some influence
on myths like the Heliopolis ennead (9 original deities) in ancient Egypt\,?

\subsection{Chains of dispositors in synastry}

Chains of dispositors are often very helpful for understanding how someone ticks,
especially if no birth time is known,
and in synastry it is often also very helpful,
if not more so.
%
Below a small example with Franz Kafka and Milena Jesenská,
with uncertain moon signs for both, since birth times are not known:
Leo-Virgo for Milena (before new moon), Gemini-Cancer for him (after new moon),
which via new moons further emphasize moon/Artemis themes.

\vspace{4mm}
\noindent
\includegraphics[scale=0.347]{i-milena-frank-chains-of-dispositors.jpg}

\subsection{The Cronian$\!$ Twenties}

It is not so common to cast birth charts for calendar events,
even though quite famous astrologers
like Dane Rudhyar did it in the past.
%
In my experience,
such charts mirror more than you would expect.
%
A bit more specifically,
for the 2020s for each country I take 1 January 2020 at midnight for each capital,
since calendars are cultural creations,
regulated usually by the respective states.
%
Chart for Berlin on the next page;
in most countries the sun is near the IC.

The nodal axis with the southern node near the IC
implies rather hard times for “roots”,
including old people, families and ancient traditions.
%
Moon/Neptune in Pisces could imply inundations
in the literal sense or also in a medical sense,
which would affect daily life since in the sixth house.
%
Uranus is in the eight house (“death/sex”) in the 3rd degree of Taurus,
where traditionally the moon is exalted;
in the second house (“body”) Mars is conjunct asteroid 100 Hekate in Scorpio,
which all can be taken as signs that death might often be nearer than usual in the 2020s.
%
Saturn/Pluto in Capricorn,
of course;
and note that Chariklo (“grace”),
the largest centaur and wife of Chiron in mythology,
is leading them.
%
Sedna conjunct Algol suggests “no fingers” to do much against death/sex/fate.

\begin{center}
\includegraphics[scale=0.23]{i-berlin-2020s.jpg}
\end{center}

\noindent
Where is hope in this\,?
%
I guess in the new.
%
The northern node at the MC
and Venus in Aquarius in the fifth house of creations and children.
%
I guess in that sense hope is in close relations,
as well as in progress,
also in technological ways,
but such that those ways immediately help people.

In the chain of dispositors below,
Saturn has the central role,
especially if you include traditional rulerships.
%
Corona is etymologically related to Cronos,
the Greek name for the planet Saturn.

\begin{center}
\hspace{7.9mm}\includegraphics[scale=0.2]{i-dispositors-2020s.jpg}
\end{center}

In other words,
Saturn would be clearly the dominant theme of the 2020s,
suggesting to maybe call the 2020s “The Cronian$\!$ Twenties” or similar,
in analogy to “The Roaring$\!$ Twenties” for the 1920s in the 20th century.

Pan as in pandemic is usually associated with Capricorn and Saturn, 
hence a Corona pandemic would fit a time generally ruled by Saturn very well.
%
Asteroid 4450 Pan was near the north node,
thus in many places also at the MC.
%
Thus “The Panning$\!$ Twenties” might maybe make an even catchier title\,?

\subsection{The “Caesar Rodney” birth chart of the USA}

This idea had originally been essentially written down in August 2017 by me,
at a time where I mistakenly thought the Sibly chart
would be for 4:50 PM (Neptune in Virgo at the MC)
instead of for 5:10 PM (early Libra MC).
%
Since I realized the mistake,
the Sibly chart appears also to me to be generally the first one to consult.
%
But during many investigations,
including quite some at the astro.com forum
around the autumn 2020 US presidential elections,
I also noticed that the chart I had proposed below would match
AC and MC of Jack Kerouac’s birth chart quite precisely,
with his moon at the AC,
so maybe a chart that could mirror
some subliminal currents in the US better than Sibly\,?

\vspace{2mm}
\noindent
\includegraphics[scale=0.1685]{i-usa-theaterversion.jpg}
\vspace{1mm}

\noindent
{\small \textsl{“I tentatively propose a birth around 9:45 AM on 2 July 1776 in Philadelphia PA.
%
Historically,
this could be a plausible time for when the Continental Congress passed the last part of the Lee Resolution,
the part declaring independence,
shortly after the arrival of Caesar Rodney
who had travelled through the night from Dover to Philadelphia.
%
The date is certain,
the time speculative,
but still somewhat in a plausible range according to the very few primary sources for that day.
%
The proposed birth chart features Uranus at the MC in Gemini,
Virgo rising with Neptune (and Lilith) in the first house,
and the moon in Capricorn closely behind Pluto, both in the fifth house.
%
For example,
the fact that the USA tend to be “split” on many issues
and into two parties or into Pepsi and Coke,
and so on,
could be considered to be largely due to the moon in Capricorn (“Janus head”)
and the MC with Uranus in Gemini (twins).”}\par}

\subsection{Saturn returns charts\,…}

\vspace{2mm}
\begin{center}
\includegraphics[scale=0.18]{i-saturnars.png}
\end{center}
\vspace{-2.5mm}

\noindent
…may be more important than maybe assumed,
as I noticed in January 2022.
%
At birth I have Saturn in the 9th house at the end of Pisces, 29°06’, slowly retrograde,
almost where the combined Saturn of my parents was, 29°23’, 9th,
so I was born close to their first Saturn return.
%
My 0th Saturn return was when I was 6 months old and it had an Aquarius MC with Saturn in the 11th,
and maybe accordingly until the first return I was only publishing scientific papers in physics
and otherwise not doing anything in public.
%
That changed shortly after my first Saturn return 30 March 1996,
when I was living in Montréal and almost certainly there at the return.
%
In June 1996, I published ‘trueColor’ for HyperCard,
in July 1997, back in Switzerland since May, I started to work as a software developer,
in 1998 I first grasped astrology,
in spring 2000 started to work 80\% and published the astrology program Delphi for PalmOS,
while doing research on the side
out of which exactphilosophy.net emerged in 2001/2002, and so on.
%
It even makes sense that a ‘Society for Exact Philosophy’,
which I am not associated with in any ways,
was founded in Montréal in 1970/1971
(and its maybe most prominent founder,
Mario Bunge,
had moved to Montréal in 1966
when I was born
and he then became a professor at the same university,
McGill (symbolically related to Pisces),
I worked at 30 years later).
%
Did I have a choice,
say by not moving to Canada for 1.5 years between November 1995 and early May 1996;
how would things have turned out then\,?
%
The first return has Saturn in the 5th house already conjunct the 6th house,
at a close trine to the MC in late Cancer and also trine Pluto in early Sagittarius in the 2nd.
%
This would seemingly mirror a lot very closely,
creative research with many new findings (5th) and also involving lots of work (6th),
via a fated trine related to the public and made public (MC),
but by the public mostly perceived as just fun ideas (5th) without much relevance to the public (MC),
and with Pluto also part of the trine even more fated
and likely also explaining why I also brought more personal things (5th)
into the public than usual.

It seems that each return influences exactly the time until the next one.
%
Hence for my second cycle of Saturn returns,
the third of three would be the one that would shape the 29 years after that.
%
It has Saturn in the 9th house,
as at my birth,
with Saturn conjunct Neptune at 0°16' and Artemis at 0°47' Aries,
at the head of a dragon aspect figure
with sextiles to Pluto/Diana in early Aquarius at the DC in the 7th house
and to Sedna/Uranus in early Gemini and late Taurus (near Algol) early in the 11th house,
and with an opposition to the moon at 3°03' Libra in the 3rd house.
%
To me this says Hollywood,
maybe mostly due to Neptune in early Aries.
%
And there are already signs pointing to that,
besides “Sleeping Beauty Dreaming” on my website since late spring 2021
also the idea of a pocket book titled \ELEMENTAL\ since early January 2019.

In any case,
I think it might be worthwhile for everyone
to be aware of their own Saturn return charts
and give them a look, just in case.
%
Had I known about this earlier,
I would have likely been a lot more reluctant to share personal things with the public,
but I also see that having provided that gives many things more depth (trines/Pluto),
and what defines the word ‘fate’ is in the end that it is fated, that there is no choice.
%
Well, maybe there is still some.
%
In the end I think I would not want to become widely known,
at least not much and rather only my ideas.
%
The website exactphilosophy.net and all that emerged around it are very beautiful,
a treasure that has already been widely preserved
and can be discovered at any time.
%
And,
was/is there a virtual relation,
as often alluded to\,?
%
What is real if not love (and life),
the pleasure of the fish,
eppur si muove…\,\includegraphics[scale=0.025]{../i-comet.jpg}

Looks like the glass ceiling/floor
between the public and my discoveries
might in the end be more “fated” than “real”\,?
%
(Maybe also my natal moon at the s.node of the 1996 Saturn return contributed to that?)
%
Reminds me also of something I once read
on a whiteboard on a photo posted on Facebook:

\vspace{1.8mm}
\noindent
{\color{xphi}Everyone you meet is fighting a battle you know nothing about. Be kind. Always}

\subsection{Chiron returns charts\,…}

\vspace{1mm}
\begin{center}
\includegraphics[scale=0.18]{i-chironars.png}
\end{center}
\vspace{-2mm}

\noindent
…may also be important,
also because Chiron is related to Saturn (and Uranus)
and often returns already at lifetime,
after about 50 years.
%
Yes,
I know,
such a repetition does not really fit into this article
of loosely connected ideas around astrology,
nor did the previous one
with its heavier and more personal tone,
or maybe they do just because of that.
%
And,
yes,
you could also make return charts for other planets,
down to every month for the moon,
while the 50 of Chiron also has some relevance in that regard,
as follows.
%
Chiron evokes themes of the moon or triple goddess
as the 50 lunar months that are also part of the 49+50 lunar months
that are close to 8 solar years
and where Venus almost perfect completes her cycles,
which is also why the number 50 occurs so often in mythology,
or why in the I Ching you start with 50 yarrow stalks
and put one away to end up with 49,
for no practical reason whatsoever.

My Chiron at birth is at 25°36' Pisces in the 9th house,
retrograde,
so it returned with the sun 15 March 1967,
again in the 9th.
%
This likely mirrors my philosophical interests also quite well.
%
My first returns were 30 March and 11 October 2017 and 27 January 2018,
in the 2nd, 8th and 12th house.
%
At the beginning of spring 2017
I started to write \textsl{Elementare Sternzeichen},
finished it in early summer
and published it in January 2018,
and so far (Feb 2022) it sold 50 copies
(only 1 in 2021).
%
I wrote the English version
with the twice-misleading title \textsl{Elementary Star Signs}
in March 2019;
it sold even less (23).

My Chiron is at the degree with the following Sabian Symbol
(plus Jones' added keyword):
\textbf{\color{xphi}A new moon that divides its influences. Finesse.}
%
Reminds also of exactphilosophy.
%
In the end
with my Saturn and Chiron at the end of Pisces
in the 9th
and both relatively slowly retrograde,
the answer is likely often to simply do less,
or dream…

\subsection{There is way more where this came from\,…}

I have written a ‘million’ things about astrology since the turn of the millenium.
%
Should those genuinely interest you,
you can find most of them via this website
if you look carefully,
or maybe ask someone (not me) to find them for you.
%
Maybe some day I would still write some of them up into a book or several,
or maybe not.
%
Alas this does not have much priority
since astrologers never cared,
never considered my contributions anything special,
just arbitrary additions to the kaleidoscope
of their personally and collectively imagined astrological worlds.

Maybe better to just wait,
just dream like Sleeping Beauty of young knights (f/m/\textasteriskcentered) to come by,
see, understand, embrace and carry on,
even if that might take centuries;
might be worth the wait…

\vspace{1mm}
\begin{center}
\includegraphics[scale=0.133]{i-sparrows.jpg}
\end{center}

\subsection{The Thirteenth Fairy\,…}

Previous pages seem to beautifully mirror
the star signs of corresponding page numbers,
as well as the symbolism of the numbers in more general ways,
all without any conscious planning.
%
Below a tiger I painted with my left hand
on the evening of 20 January 2022
with Ceres, Sedna and n.node conjunct.

\begin{center}
\includegraphics[scale=0.191]{i-tiger.jpg}
\end{center}

\noindent
22 March 2022 in the evening I cast a Sabian Symbols oracle
about the situation of the world, and what I should do about it,
and the answer was simply:\newline
“Libra 17 – A retired sea captain. – Relaxation”…

To knights (f/m/\textasteriskcentered)
who might some day stumble upon Sleeping Beauty dreaming in this castle
\href{https://www.exactphilosophy.net}{\color{xphi}exactphilosophy.net}:
%
If you want to prove something in astrology,
a good place to start might\,(!) be
return charts of planets,
cast for the place of residence at the return,
viewed from the point of view of the returning planet,
i.e.\ its house and aspects,
plus the aspected objects with their signs and houses,
and with secondary importance their aspects,
and so on,
which seem to shape the time to the next return
for the themes of the returned planet quite clearly,
and
specifically for Saturn and Jupiter
with their traditional relation to the public
and with ideal periods for research of (up to) 29 and 12 years,
this could lead to something
that might be relatively easy to prove,
and maybe start with Saturn – many small, trackable steps.
%
Then again,
worked for me personally,
but maybe just coincidence,
less sure with known personalities I tried it with so far,
so might also turn out to be simply nothing in the end…

\begin{center}
\includegraphics[scale=0.75]{i-clock.jpg}
\end{center}

\newpage

\subsection{Appendix: Alrisha Ingresses}

Nominal tropical sign transitions of fixed star Alrisha
due to precession of the earth’s axis (c.f.\ page 4),
as yielded in charts drawn at astro.com in March 2022.
%
Calculations and definitions are almost certainly not precise enough,
but maybe interesting enough as teasers or oracles\,?
%
Say, Jupiter-moon, Mercury-s.node,
and Venus in early Aries for the Age of Pisces
might ring a few bells…\,?

Page 14 and oracles (Apollon-7, Delphi, Libra, center, 7+7, etc.).

\vspace{2mm}
\begin{center}
\includegraphics[scale=0.19]{i-alrisha-taurus.jpg}
\end{center}

% makes a difference...
\vspace{0mm}
\begin{center}
\includegraphics[scale=0.19]{i-alrisha-aries.jpg}
\end{center}

\end{document}
