\avantgarde

\en{\section{First mention of Lilith as second focal point\newline of the lunar orbit}}%
\de{\section{Erste Erwähnung von Lilith als zweitem Brennpunkt\newline der Mondbahn}}%
\fr{\section{Première mention de Lilith comme second foyer\newline de l’orbite lunaire}}

\en{For all that it appears, this was in 1937
in \textsl{Les présages à la lumière des lois de l’évolution}
(\textsl{The Predictions in the Light of the Laws of Evolution})
by D.\ Néroman, on page 199 and 200.}%
\de{Allem Anschein nach war das 1937,
in \textsl{Les présages à la lumière des lois de l’évolution}
(\textsl{Die Vorhersagen im Licht der Gesetze der Evolution})
von D.\ Néroman, auf Seite 199 und 200.}%
\fr{Selon toute apparence, c’était en 1937
dans \textsl{Les présages à la lumière des lois de l’évolution}
par D.\ Néroman, sur page 199 et 200.}
%
\en{He says there that this was the first time he published about it.}%
\de{Er sagt dort auch, dass es das erste Mal ist, dass er dazu etwas publiziert.}%
\fr{It y dit que c’est la première fois qu’il en publie.}
%
\en{Extracts from the book can be found a bit further below.}%
\de{Auszüge aus dem Buch befinden sich etwas weiter unten.}%
\fr{Des extraits du livre se trouvent plus en bas.}

\en{\subsection{Details}}%
\de{\subsection{Details}}%
\fr{\subsection{Détails}}

\en{\noindent
The idea of a \textsl{Black Moon} Lilith
as a \textsl{real object} in space is older,
but today we know that there is certainly no such kind of object.}%
\de{Die Idee von einem \textsl{Schwarzen Mond} Lilith
als \textsl{realem Objekt} im Weltall gab es schon früher,
aber heute weiss man, dass es kein solches Objekt gibt.}%
\fr{L’idée d’une \textsl{lune noire} Lilith
comme \textsl{object réel} en espace est plus vieille,
mais aujourd’hui on sait qu’il n’y a sûrement pas d’object réel de ce genre.}
%
\en{This here is about the idea to define Lilith
as the \textsl{second (empty) focal point of the lunar orbit},
as shown in the drawing below, which is not true to scale
(the orbit is much rounder
and Earth and Moon are smaller in comparison with the orbit):}%
\de{Hier geht es um die Idee,
Lilith stattdessen als \textsl{zweiten (und leeren) Brennpunkt der Mondbahn} zu definieren,
wie in folgender Zeichnung, welche nicht massstabsgetreu ist
(die Mondbahn ist viel runder
und Erde und Mond sind kleiner im Vergleich zur Mondbahn):}%
\fr{Ici il s’agit de l’idée de définir Lilith
comme \textsl{second foyer (foyer vide) de l’orbite lunaire},
comme dans ce dessin, qui n’est pas a l’échelle
(l’orbite est beaucoup plus rond
et la terre et la lune sont plus petites en comparaison avec l’orbite):}

\vspace{2mm}
\begin{center}
\en{\includegraphics[scale=0.12]{i-lilith-ellipse-en.jpg}}%
\de{\includegraphics[scale=0.12]{i-lilith-ellipse-de.jpg}}%
\fr{\includegraphics[scale=0.12]{i-lilith-ellipse-fr.jpg}}
\end{center}
\vspace{2mm}

\noindent
\en{If you imagined the lunar orbit to be reflective like a mirror,
rays of light emitted from Earth would again meet
at the second focal point of the orbit,
hence on Lilith.}%
\de{Wenn man sich vorstellen würde, die Mondbahn wäre wie ein Spiegel,
dann würden sich Lichtstrahlen, die von der Erde ausgehen,
bei Lilith wieder treffen.}%
\fr{Si l’on imaginait que l’orbite lunaire serait un miroir,
des rayons de lumière émis de la terre se rencontrèrent
au second foyer de l’orbite,
alors sur Lilith.}
%
\en{Lilith has the same longitude as the apogee of the lunar orbit
(the largest distance from Earth),
which is why in astrology the two are often treated in practically the same way.}%
\de{Lilith und das Apogäum der Mondbahn
(der Punkt der Mondbahn, der von der Erde am weitesten entfernt ist)
sind longitudinal in der gleichen Richtung,
daher werden sie astrologisch oft praktisch wie das Gleiche behandelt.}%
\fr{C’est la même longitude que l’apogée de l’orbite lunaire
(la plus grande distance de la terre),
alors pour l’astrologie c’est souvent pratiquement la même chose.}

\en{I looked at three books by D.\ Néroman,
one from 1933, \textsl{Planètes et destins} (\textsl{Planets and Destinies}),
published under the name Dom Nécroman,
the one from 1937, and one from 1943,
\textsl{Traité d’astrologie rationelle} (\textsl{Treatise of Rational Astrology}).}%
\de{Ich habe drei Bücher von D.\ Néroman angeschaut,
\textsl{Planètes et destins} (\textsl{Planeten und Schicksale}) von 1933,
publiziert unter dem Namen Dom Nécroman,
das von 1937 und eines von 1943,
\textsl{Traité d’astrologie rationelle} (\textsl{Lehrbuch der rationalen Astrologie}).}%
\fr{J’ai regardé trois livres de D.\ Néroman,
un de 1933, \textsl{Planètes et destins}, publié sous le nom de Dom Nécroman,
celui de 1937 et un de 1943,
\textsl{Traité d’astrologie rationelle}.}
%
\en{Apparently Néroman and Nécroman were pseudonyms
and his real name was Pierre Rougié,
according to Patrice Guinard in
\textsl{L’astrologie française au XXème siècle},
\href{http://cura.free.fr/docum/10astrof.html}{http://cura.free.fr/docum/10astrof.html}.}%
\de{Anscheinend waren Néroman und Nécroman Pseudonyme
und sein wirklicher Name war Pierre Rougié,
laut Patrice Guinard in
\textsl{L’astrologie française au XXème siècle},
\href{http://cura.free.fr/docum/10astrof.html}{http://cura.free.fr/docum/10astrof.html}.}%
\fr{Apparement Néroman et Nécroman était des noms de plume
et son vrai nom était Pierre Rougié,
selon Patrice Guinard dans
\textsl{L’astrologie française au XXème siècle},
\href{http://cura.free.fr/docum/10astrof.html}{http://cura.free.fr/docum/10astrof.html}.}

\newpage

\noindent
\en{No mention of Lilith in the book of 1933, only in the book of 1937:}%
\de{Im Buch von 1933 wird Lilith noch nicht erwähnt, erst im Buch von 1937:}%
\fr{Pas de mention de Lilith dans le livre de 1933, mais dans le livre de 1937:}

\begin{quote}
\small
\color{darkgray}
\begin{otherlanguage}{french}
\textsl{Les présages\newline
à la lumière des lois de l’évolution\newline
par D.\ Néroman\newline
Ingénieur civil des mines\newline
Collège astrologique de France\newline
Éditions \og sous-le-ciel\fg\newline
1937\newline
\newline
Achevé d’imprimer le 10 mars 1937,
sur les presses de l’Imprimerie E.~G.~I.,
Directeur: Charles Fischer,
107, avenue de France, Anvers.}
\end{otherlanguage}
\end{quote}

\en{\noindent
Translation:}%
\de{\noindent
Übersetzt:}

\en{\begin{quote}
\small
\color{xphi}
\textsl{The Predictions\newline
in the Light of the Laws of Evolution\newline
by D.\ Néroman\newline
Civil mining engineer\newline
Collège astrologique de France\newline
Éditions “sous-le-ciel”\newline
1937\newline
\newline
Printing completed 10 March 1937,
on the printing presses of E.~G.~I.,
Director: Charles Fischer,
107, avenue de France, Anvers [Belgium].}
\end{quote}}%
\de{\begin{quote}
\small
\color{xphi}
\textsl{Die Vorhersagen\newline
im Licht der Gesetze der Evolution\newline
von D.\ Néroman\newline
Ziviler Bergbauingenieur\newline
Collège astrologique de France\newline
Éditions “sous-le-ciel”\newline
1937\newline
\newline
Druck beendet am 10.\ März 1937,
auf den Pressen der Druckerei E.~G.~I.,
Direktor: Charles Fischer,
107, avenue de France, Anvers [Belgien].}
\end{quote}}

% vertical spaces are a bit unexpected from now on,
% presumably because the ‘quotes’ would be floating...
\vspace{2mm}
\noindent
\en{He first mentions Lilith on page 199 and 200:}%
\de{Lilith wird auf Seite 199 und 200 erstmals erwähnt:}%
\fr{Il en parle sur page 199 et 200:
\vspace{2mm}}

\begin{quote}
\small
\color{darkgray}
\begin{otherlanguage}{french}
\textsl{\textbf{99. — Lilith, le trouble sexuel.}\newline
 \newline
On connaît le mythe de Lilith,
\og lune noire\fg.
%
Puisqu’il s’agit d’un facteur astrologique,
la question qu’il pose a deux aspects:
l’aspect astronomique,
qui permet de situer Lilith sur le Zodiaque,
et l’aspect influentiel,
qui est scabreux et que je n’entends pas traiter ici.\newline
%
\mbox{\ \ \ \ }Je n’ai jamais rien publié des travaux auxquels je me suis livré sur ce facteur,
la nécessité de cette publication ne s’étant jamais imposée;
mais voici que,
dans cette étude d’une évolution générale,
garder le silence sur Lilith
serait introduire une lacune;
je dirai donc ce que j’ai établi
ou cru établir sur cet élément du thème astrologique,
et je me contenterai naturellement de résumer,
me réservant de publier assez prochainement,
sous le titre \og Les luminaires noirs\fg,
l’étude qui m’a donné ces résultats.\newline
%
\mbox{\ \ \ \ }Le facteur qu’on appelle Lilith,
c’est tout simplement,
d’après cette étude,
le foyer vide de l’orbite lunaire;
donc sur le thème il se confond avec l’apogée de la Lune.\newline
%
\mbox{\ \ \ \ }[…$\!$]\newline
%
\mbox{\ \ \ \ }Mais bien entendu ce n’est pas sur ce rapprochement
que j’ai établi l’identité de Lilith et du foyer vide,
et j’exposerai la question aussi com\-plè\-te\-ment que possible
dans \og Les luminaires noirs\fg.\newline
%
\mbox{\ \ \ \ }Pour l’instant,
admettons que ce foyer est générateur
des troubles et des exaspérations de la fonction sexuelle,
sur lesquels nous n’avons pas l’intention d’insister,
et que par conséquent nous pouvons l’appeler \textbf{Lilith}.}
\end{otherlanguage}
\end{quote}

\en{\newpage}%
\de{\newpage}

\en{\noindent
Approximate translation:}%
\de{\noindent
Näherungsweise Übersetzung (das zentrale Wort “trouble” kann zum Beispiel
Aufregung, Aufruhr, Trübung, Verstörtheit, Verwirrung, Wirrnis, Unruhe, Trouble, Störung,
Durcheinander, verbotene Eigenmacht und Drasch bedeuten):}

\en{\begin{quote}
\small
\color{xphi}
\textsl{\textbf{99. — Lilith, the sexual trouble.}\newline
 \newline
The myth of Lilith,
the “Black Moon”,
is well-known.
%
Since it is an astrological factor,
the resulting question has two aspects:
the astronomical aspect of situating Lilith in the zodiac,
and the influential aspect,
which is scabrous and which I do not intend to treat here.\newline
%
\mbox{\ \ \ \ }I have never published anything from the works
to which I devoted myself regarding to this factor,
the necessity to publish never imposed itself;
but here,
in this study of a general evolution,
keeping the silence on Lilith
would be to make an omission;
I will thus say what I established,
or believe to have established,
on this astrological element,
and I will naturally content myself to summarize,
reserving to publish quite soon the study that gave me these results
under the title “Les luminaires noirs” [“The Black Luminaries”].\newline
%
\mbox{\ \ \ \ }According to this study,
the factor called Lilith is simply the empty focal point of the lunar orbit,
hence topically it melts with the lunar apogee.\newline
\mbox{\ \ \ \ }[…$\!$]\newline
\mbox{\ \ \ \ }But of course it is not because of this approximation
that I have established the identity of Lilith with the empty focal point,
and I will expose the question as completely as possible
in “Les luminaires noirs”.\newline
\mbox{\ \ \ \ }For the moment,
let us admit that this focal point is a generator
of troubles and exasperations of the sexual function,
which we do not have the intention to insist on,
and consequently we can call it \textbf{Lilith}.}
\end{quote}}%
\de{\begin{quote}
\small
\color{xphi}
\textsl{\textbf{99. — Lilith, die sexuelle Störung.}\newline
 \newline
Der Mythos von Lilith,
dem “Schwarzen Mond”,
ist bekannt.
%
Da es sich um einen astrologischen Faktor handelt,
stellt sich die Frage aus zwei Blickwinkeln:
dem astronomischen Blickwinkel,
der es erlaubt Lilith im Tierkreis zu situieren,
und dem Blickwinkel des Einflusses,
welcher heikel ist und den ich hier nicht behandeln möchte.\newline
%
\mbox{\ \ \ \ }Ich habe bisher noch nie etwas zu den Arbeiten,
denen ich mich zu diesem Faktor gewidmet hatte,
publiziert,
die Notwendigkeit dazu hatte sich nie ergeben;
aber hier,
in dieser Studie der allgemeinen Evolution,
zu \mbox{Lilith} zu schweigen,
wäre eine Auslassung;
ich werde daher sagen,
was ich zu diesem astrologischen Element festgestellt habe,
oder glaube festgestellt zu haben,
und ich werde mich natürlich darauf beschränken,
zusammenzufassen,
und es mir vorbehalten,
dazu recht bald unter dem Titel “Les luminaires noirs” [“Die Schwarzen Leuchtkörper”]
die Studie zu publizieren,
die mir diese Resultate erbrachte.\newline
%
\mbox{\ \ \ \ }Der Faktor,
der Lilith genannt wird,
ist gemäss dieser Studie einfach der leere Brennpunkt der Mondbahn;
also verschmilzt er in der Hinsicht mit dem Apogäum des Mondes.\newline
%
\mbox{\ \ \ \ }[…$\!$]\newline
%
\mbox{\ \ \ \ }Aber selbstverständlich ist es nicht beruhend auf dieser Annäherung,
dass ich die Identität von Lilith und dem leeren Brennpunkt etabliert habe,
und ich werde diese Frage so vollständig wie möglich
in “Les luminaires noirs” ausführen.\newline
%
\mbox{\ \ \ \ }Gestehen wir für den Moment zu,
ohne darauf beharren zu wollen,
dass dieser Brennpunkt Erzeuger von Störungen und Verärgerungen der sexuellen Funktion ist,
und wir ihn daher \textbf{Lilith} nennen können.}
\end{quote}}

\en{\vspace{2mm}
\noindent
It seems that he would have never published “Les luminaires noirs”,
since in the book from 1943
this title does does not appear in the lists of works by the author,
but in the book itself he speaks about Lilith,
and also about the perigee,
which he names \textsl{Priapus},
and also about the similar points for the orbit of the Earth around the Sun,
namely the \textsl{Black Sun} (aphelion)
and implicitly also the perihelion,
which today is often called \textsl{Diamond} in astrology.}%
\de{\vspace{0mm} % seems that vspaces add up
\noindent
Es scheint, dass er “Les luminaires noirs” nie publiziert hätte,
denn im Buch von 1943
erscheint dieser Titel in keiner der Werklisten des Autors,
aber im Buch selber spricht er von Lilith,
und auch vom Perigäum,
das er \textsl{Priapus} benennt,
sowie von den analogen Punkten für die Bahn der Erde um die Sonne,
also von der \textsl{Schwarzen Sonne} (Aphelion)
und implizit auch vom Perihelion,
das man heutzutage in der Astrologie oft \textsl{Diamant} nennt.}%
\fr{\vspace{-2mm} % seems that vspaces add up
\noindent
Il semble qu’il n’aurait jamais publié “Les luminaires noirs”,
car dans le livre de 1943
ce titre n’apparaît pas dans les listes des ouvrages de l’auteur,
mais dans le livre même il parle de Lilith,
et aussi du perigée qu’il y nomme \textsl{Priape},
et aussi des points similaires pour l’orbite de la terre autour du soleil,
alors du \textsl{soleil noir} (aphelion)
et implicitement aussi du perihelion,
qu’on appelle aujourd’hui souvent \textsl{diamant} en astrologie.}

\en{What is also interesting in the book from 1937,
is how Néroman explains the influence
of \textsl{virtual} points like Lilith and the lunar nodes
by making a comparison with the center of gravity,
which can be empty,
like for example for a hollow sphere,
but still exert force.}%
\de{Auch noch interessant ist,
wie Néroman,
noch im Buch von 1937,
den Einfluss von \textsl{virtuellen} Punkten wie Lilith und den Mondknoten erklärt,
indem er den Vergleich mit dem Schwerpunkt macht,
der im leeren Raum sein kann,
zum Beispiel bei einer Hohlkugel,
und doch Kräfte ausüben kann.}%
\fr{Ce qui est aussi interessant dans le livre de 1937
est comment Néroman explique l’influence
de points \textsl{virtuels} comme Lilith et les nœuds lunaires
en faisant comparaison avec le centre de gravité,
qui peut être vide,
par example pour une sphere creuse,
mais quand même exerce de la force.}
%
\en{Similarly,
the empty focal point and the lunar nodes
are also strongly linked to gravitational forces.}%
\de{Ähnlich sind der leere Brennpunkt und die Mondknoten
auch stark mit den Gravitationskräften verbunden.}%
\fr{Similairement,
le foyer vide et les nœuds lunaires
sont aussi fortement liés aux forces de la gravitation.}
%
\en{More details in the book on page 200 and 201.}%
\de{Mehr Details im Buch, auf Seite 200 und 201.}%
\fr{Plus de détails dans le livre, sur page 200 et 201.}

\newpage

\noindent
\en{Here an impression of the book and the pages about Lilith:}%
\de{Hier ein Eindruck vom Buch und den Seiten zu Lilith:}%
\fr{Voici une impression du livre et des pages sur Lilith:}

\vspace{2mm}
\noindent
\includegraphics[scale=0.24]{i-livre.jpg}

\vspace{3mm}
\noindent
\includegraphics[scale=0.251]{i-page-198-199.jpg}

\vspace{3mm}
\noindent
\includegraphics[scale=0.251]{i-page-200-201.jpg}
