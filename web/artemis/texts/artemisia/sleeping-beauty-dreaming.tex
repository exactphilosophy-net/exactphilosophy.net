\documentclass[letterpaper]{article}
\pagestyle{empty}
\paperheight=3700mm
\textheight=3700mm
%\textwidth set depending on pdflatex or (experimentally)lualatex
\topmargin=-20mm
\oddsidemargin=25mm

\usepackage[utf8]{inputenc}
\usepackage[spanish,italian,french,ngerman,english]{babel} % last is main
\usepackage{graphicx}
\usepackage{multirow}
\usepackage{xcolor}
\usepackage{contour}
\usepackage{pict2e}
\usepackage{relsize}
\usepackage{amsmath}

\usepackage{iftex}
\ifpdftex
  % stronger fonts

% Find modes.mf, e.g. /usr/local/texlive/2025/texmf-dist/fonts/source/public/modes/modes.mf
%
% $ sudo cp modes.mf modes.mf.orig
%
% Add the following at the start of modes:
%
% mode_def xphi =
%   mode_param (pixels_per_inch, 1200);
%   mode_param (blacker, 1.9); % only difference to 'lexmarkr' (2 there)
%   mode_param (fillin, 0);
%   mode_param (o_correction, 1);
%   mode_common_setup_;
% enddef;
%
% Finally:
%
% $ sudo fmtutil-sys --byfmt mf

\pdfpkresolution=1200
\pdfpkmode={xphi}
\pdfmapfile{}

  \newcommand{\coretextwidth}{85.5mm}
\fi

% experimental, not used to produce the live website...
\ifluatex
  % about same heaviness in pdfs when rasterized in photoshop,
  % but since, unlike the metafont mechanisms I use, fake bold, "bleeds" in all directions,
  % seems heavier at least in core web page images
  \newcommand{\fontbleed}{0.8}
  % paragraphs wider and font looks larger, tried to fix, but then other things change a bit,
  % especially for section headings would have to change back, are now more narrow...
  \newcommand{\fontscale}{0.985}
  \newcommand{\coretextwidth}{85.2mm}
  \usepackage{fontspec}
  % microtype does maybe help and not help, maybe if would allow wider spaces...
  \usepackage{microtype}
  \setsansfont{Latin Modern Sans}[Scale=\fontscale, FakeBold=\fontbleed]
  \setmonofont{Latin Modern Mono}[Scale=\fontscale, FakeBold=\fontbleed]
  % this would be for "new computer modern" (but has many limitations so far)
  %\usepackage[default]{fontsetup}
  %\renewcommand{\familydefault}{\sfdefault}
\fi

\renewcommand{\familydefault}{\sfdefault}

\setcounter{secnumdepth}{-1}

\newcommand{\en}[1]{\iflanguage{english}{#1}{}}
\newcommand{\de}[1]{\iflanguage{ngerman}{#1}{}}
\newcommand{\fr}[1]{\iflanguage{french}{#1}{}}

% a bit less than ~255/256
\definecolor{almostwhite}{gray}{0.996}
\definecolor{xphi}{rgb}{0.0,0.5,0.5}
\definecolor{avant}{rgb}{1,0.5,0.5}

\definecolor{frame}{gray}{0.9}
\definecolor{lightgray}{gray}{0.8}
\definecolor{gray}{gray}{0.5}
\definecolor{darkgray}{gray}{0.3}

\definecolor{darkred}{rgb}{0.8,0.0,0.0}
\definecolor{darkyellow}{rgb}{0.7,0.7,0.0}
\definecolor{darkgreen}{rgb}{0.0,0.55,0.0}
\definecolor{darkviolet}{rgb}{0.5,0,0.5}

\definecolor{darkblue}{rgb}{0,0,0.7}
\definecolor{odyssey}{rgb}{0,0,0.8}
\definecolor{indigo}{rgb}{0.29,0,0.51}
\definecolor{indigoblue}{rgb}{0.1,0,0.6}

\definecolor{saffronback}{rgb}{1.000,0.878,0.627}
\definecolor{saffronfront}{rgb}{0.376,0.125,0.000}

\DeclareRobustCommand{\cometartemisscale}[1]{\includegraphics[scale=#1]{\sourcepath/i-comet.jpg}\hspace{-0.028453em} artemis}
\newcommand\cometartemis{\cometartemisscale{0.018}}
\newcommand\cometartemissection{\cometartemisscale{0.0225}}

\DeclareRobustCommand{\moebius}[1]{\includegraphics[scale=#1]{\sourcepath/i-moebius.jpg}}
\newcommand{\yinyang}{\includegraphics[scale=0.135]{\sourcepath/i-yinyang.jpg}}

\newcommand{\rarr}{\,$\rightarrow$\,}
\newcommand{\lrarr}{\,$\leftrightarrow$\,}

% greek elements
\newcommand{\elfire}{
\begin{picture}(9,6)
  \thicklines
  \put(1,-0.5){\line(1,0){7}}
  \put(1,-0.5){\line(1,1.732){3.5}}
  \put(8,-0.5){\line(-1,1.732){3.5}}
\end{picture}}
%
\newcommand{\elair}{
\begin{picture}(9,6)
  \thicklines
  \put(1,-0.5){\line(1,0){7}}
  \put(1,-0.5){\line(1,1.732){3.5}}
  \put(8,-0.5){\line(-1,1.732){3.5}}
  \put(2.75,1.9){\line(1,0){3.5}}
\end{picture}}
%
\newcommand{\elwater}{
\begin{picture}(9,6)
  \thicklines
  \put(1,5){\line(1,0){7}}
  \put(1,5){\line(1,-1.732){3.5}}
  \put(8,5){\line(-1,-1.732){3.5}}
\end{picture}}
%
\newcommand{\elearth}{
\begin{picture}(9,6)
  \thicklines
  \put(1,5){\line(1,0){7}}
  \put(1,5){\line(1,-1.732){3.5}}
  \put(8,5){\line(-1,-1.732){3.5}}
  \put(2.75,2.5){\line(1,0){3.5}}
\end{picture}}
%
\newcommand{\elhex}{
\begin{picture}(9,6)
  \thicklines
  \put(1,0.5){\line(1,0){7}}
  \put(1,0.5){\line(1,1.732){3.5}}
  \put(8,0.5){\line(-1,1.732){3.5}}
  \put(1,5){\line(1,0){7}}
  \put(1,5){\line(1,-1.732){3.5}}
  \put(8,5){\line(-1,-1.732){3.5}}
\end{picture}}

% i ching trigrams
\newcommand{\trigram}[3]{
\begin{picture}(9,6)
  \linethickness{0.36mm}
  \put(0,5){\line(1,0){#1}}
  \put(5.5,5){\line(1,0){3.5}}
  \put(0,2.5){\line(1,0){#2}}
  \put(5.5,2.5){\line(1,0){3.5}}
  \put(0,0){\line(1,0){#3}}
  \put(5.5,0){\line(1,0){3.5}}
\end{picture}}
\newcommand{\triheaven}{\trigram{5.5}{5.5}{5.5}}
\newcommand{\triearth}{\trigram{3.5}{3.5}{3.5}}
\newcommand{\trithunder}{\trigram{3.5}{3.5}{5.5}}
\newcommand{\triwater}{\trigram{3.5}{5.5}{3.5}}
\newcommand{\trimountain}{\trigram{5.5}{3.5}{3.5}}
\newcommand{\triwind}{\trigram{5.5}{5.5}{3.5}}
\newcommand{\trifire}{\trigram{5.5}{3.5}{5.5}}
\newcommand{\trilake}{\trigram{3.5}{5.5}{5.5}}

% i ching hexagrams
% 1+2 trigrams, 3 rest of line (see e.g. dreams.tex)
\DeclareRobustCommand{\hexagram}[3]{\raisebox{-3pt}{$\overset{\text{${#1}$}}{#2}$\,}#3\vspace{3pt}}

% white-red-black etc.
\DeclareRobustCommand{\outline}[1]{\contour{black}{{\color{white}#1}}}
\DeclareRobustCommand{\white}[1]{\outline{\textbf{#1}}}
\DeclareRobustCommand{\red}[1]{{\color{darkred}\textbf{#1}}}
\DeclareRobustCommand{\black}[1]{\textbf{#1}}
\DeclareRobustCommand{\yellow}[1]{{\color{darkyellow}\textbf{#1}}}
\DeclareRobustCommand{\green}[1]{{\color{darkgreen}\textbf{#1}}}
\DeclareRobustCommand{\violet}[1]{{\color{darkviolet}\textbf{#1}}}
\DeclareRobustCommand{\indigoblue}[1]{{\color{indigoblue}\textbf{#1}}}
\DeclareRobustCommand{\indigo}[1]{{\color{indigo}\textbf{#1}}}

% ELEMENTAL
\newcommand{\ELEMENTAL}{%
\colorlet{contour}{.}\textbf{\color{white}%
\raisebox{+0.001em}{\contour{contour}{E}}%
\raisebox{+0.015em}{\contour{contour}{L}}%
\raisebox{+0.016em}{\contour{contour}{E}}%
\raisebox{+0.023em}{\contour{contour}{M}}%
\raisebox{+0.023em}{\contour{contour}{E}}%
\raisebox{+0.017em}{\contour{contour}{N}}%
\raisebox{-0.020em}{\contour{contour}{T}}%
\raisebox{-0.002em}{\contour{contour}{A}}%
\raisebox{+0.006em}{\contour{contour}{L}}%
}}

% artemis pdf+web icons
\newcommand{\ipdfen}{\includegraphics[scale=0.5]{i-pdf-en.png}}
\newcommand{\ipdfde}{\includegraphics[scale=0.5]{i-pdf-de.png}}
\newcommand{\ipdffr}{\includegraphics[scale=0.5]{i-pdf-fr.png}}
\newcommand{\iweb}{\includegraphics[scale=0.055]{i-web.png}}
\newcommand{\ipdfblueen}{\includegraphics[scale=0.5]{i-pdf-blue-en.png}}
\newcommand{\ipdfbluede}{\includegraphics[scale=0.5]{i-pdf-blue-de.png}}
\newcommand{\ipdfbluefr}{\includegraphics[scale=0.5]{i-pdf-blue-fr.png}}
\newcommand{\iwebblue}{\includegraphics[scale=0.055]{i-web-blue.png}}


\textwidth=\coretextwidth



\begin{document}

\avantgarde

\section{\white{Sleeping}-\red{Beauty}-\black{Dreaming}}

Starting roughly with summer 2021
I started to show a sentence that his website would be sleeping like Sleeping Beauty,
at some point also adding that she would be dreaming,
which gives the whole thing also something \textsl{avantgarde},
something surreal,
as in the fairy tales the sleep of Sleeping Beauty
would usually rather be related to death,
as a dreamless sleep.

I do not know why exactly,
even though I have a few hints,
like that maybe my discoveries would have reached a state
where making them public
would become dangerous
as they might influence so much.
%
An oracle I cast into the text version
of my ‘A few new discoveries in physics’ document of 2002
in the way described in some of my Usenet posts gave this:

\vspace{-1.6mm}
\begin{center}
\includegraphics[scale=0.08]{i-secrets.png}
\end{center}
\vspace{-3.2mm}

\noindent
In a way this probably says it all.
%
This is not the time to continue,
not because of me,
but because the fates decided so,
for all.

For some ruminations around why and what see inside 
\href{https://www.exactphilosophy.net/web2021.zip}{\color{xphi}web2021.zip}
at several places.
%
Let me just add that since late spring
the asteroid 100 Hekate has been going back and forth across my MC and moon,
which will last into January 2022.
%
Maybe afterwards things would become easier again with xphi,
but in my feeling the only thing that will be easy for quite some time,
possibly for the rest of my life,
will be to just let xphi rest,
only make very minimal changes.

I guess external ‘Seeds’
as mentioned in the above zip file
and in the corresponding section on xphi
may be more likely possible in time,
but I guess only very leisurely,
in a way rather by waiting for them to grow,
as was often also the case with evolutions at xphi,
than actively pushing and promoting anything,
which would,
of course,
overall continue to mirror
the meaning of the word \textsl{exactphilosophy},
as I defined it,
quite closely and taoistically.

Let me add a few memories around xphi on the following pages,
and then most likely really let it rest,
maybe only for a few years,
maybe for longer,
maybe for the rest of my life.
%
I had often considered xphi also a statue,
and at some point the ‘chiseling’ of the xphi statue simply has to stop,
not in my interest,
but in the interest of all,
while spin-off ‘statues’ might still be possibilities.

\newpage

\subsection{Tiefenbrunnen}

I read most of Liz Greene’s ‘The Astrology of Fate’
and especially Robert Graves’ ‘The Greek Myths’
in summers at Badi Tiefenbrunnen,
starting from 1998.
%
That is also where I considered many aspects of my idea
to define elements in terms of immediate experience of space and time,
as in/out and rest/move,
plus their transformations,
while looking at the beautiful tree near the diving platform.

\vspace{-1.0mm}
\begin{center}
\includegraphics[scale=0.35]{i-tiefenbrunnen.jpg}
\end{center}
\vspace{-1.2mm}

\noindent
From there I could also often see where I live
across the lake on the less sunny side,
somewhere below the Felseneggturm:

\vspace{-1.0mm}
\begin{center}
\includegraphics[scale=0.138]{i-felseneggturm.jpg}
\end{center}
\vspace{-1.2mm}

\noindent
Actually,
the Felseneggturm of 1961 recently got a modern twin,
and the old one will itself be dismounted in early 2022,
maybe most fitting with Jupiter-Saturn cycles,
more precisely the conjunctions of 1962 and 2020.

Not far from Badi Tiefenbrunnen is actually the villa of Astrodienst (astro.com),
and after discovering elemental transitions in the star signs in 2001,
I thought that the combination of Liz Greene,
then living in Zürich and chief astrologer at astro.com,
and Alois Treindl,
founder and then CEO of Astrodienst,
as well as a physicist with a PhD from ETH,
would be a match made in heaven for my new ideas.
%
I honestly was so naive to think
that they would be happy to see such a development
of the ideas of combining astrology with mythology
that their generation had been evolving so beautifully,
that they would invite me to visit them at Astrodienst
to present and discuss my ideas.

Almost exactly 20 years after 9/11,
actually at Badi Tiefenbrunnen something strange happened.
%
Let me quote from a text I wrote the same evening
and which is also contained in web2021.zip.

\begin{quote}
\textsl{[…$\!$] I was driving to the lake
and was thinking about what makes the difference
between mainstream stars and avantgarde artists
and I think I understood.
%
Stars are at the mercy of the public
because they define themselves by how the public perceives them;
if the public adores them all is great,
if the public hates them or wants them to go through all kinds of excesses,
they often do not have enough force against it,
as the public is their “family”.
%
In contrast,
in the avantgarde what counts is how fellow artists
in the same small “circle” see it,
while the public is just made fun of,
but in such a way that the public
does usually not realize it
or there is at least plausible deniability
to keep a bourgeois facade.}

\textsl{The way I am made,
for all that it appears
I cannot build a link to the general public,
too diverse these worlds, expectations, etc.,
as far as it seems.
%
So,
the solution in my case—%
as I realized when already at the lake at the Tiefenbrunnen Badi […$\!$]
is to bind my well-being to that creature I mentioned many times,
and in some sense probably what is around her,
what is her world.}
\end{quote}

\noindent
But only if and whenever she feels like it in that remote ‘telepathic’ way.
%
In my feeling she also decided this in the immediate sense—%
to let xphi be and focus instead on essentially two specific seeds.

Some beautiful projects that were up only for a short time or,
at least so far,
never materialized.
%
First a website sabian-symbols-oracle.com (2014):

\begin{center}
\includegraphics[scale=0.31]{i-sabian-symbols-oracle-com.jpg}
\end{center}

\noindent
The tiny web server for the website
artem\i\hspace{-0.37em}{\color{red}¨}\hspace{-0.14em}s{\color{red}.}{\color{gray}com} (2015):

\begin{center}
\includegraphics[scale=0.324]{i-artemis-com.jpg}
\end{center}

\vspace{2mm}
\noindent
The book \textsl{Artemis – Die geheime surreale Welt des Mondes} (2016):

\begin{center}
\includegraphics[scale=0.318]{i-artemis-book-front.jpg}
\end{center}

\newpage

\begin{center}
\includegraphics[scale=0.32]{i-artemis-book-back.jpg}
\end{center}

\noindent
In 2021 I designed two fonts,
which I am so far using to write the book ‘Elemental’,
one of the seeds.
%
One font is Stoicheion,
after the word that Plato and Aristotle used for element,
also used for letter and objects in the sky,
a font with only uppercase letters plus space.
%
The other font is Jackwrite,
a proportional typewriter font
with special features inherited from Fredrick Brennan’s hyperrealistic TT2020 font
with 9 slightly different glyphs for each character.

\vspace{4mm}
\begin{center}
\includegraphics[scale=0.12]{i-the-world-in-a-nutshell.png}
\end{center}

\newpage

\noindent
Stoicheion was also inspired by the beauty of ancient Greek Stoichedon style,
which was,
however monospaced.
%
And first I wanted to call the font Sihlmatte,
after the housing estate where I grew up with my parents.
%
In any case,
I think Stoicheion would make a beautiful font for art exhibitions,
while Jackwrite would be very useful for the main text in novels and the like.

\begin{center}
\includegraphics[scale=0.16]{i-stoicheion-pride-and-prejudice.png}
\end{center}

\noindent
Also interesting to see how sans-serif fonts came up
around the time of the discovery of Uranus,
via the architect John Soane
inspired by Roman inscriptions
and eventually removing the slight serifs carved there,
but that is another story.

I am not sure when I first had the idea,
maybe as early as the 1970s,
maybe only in the 1990s after contact with LaTeX,
but I have been dreaming about a pocket book
‘The World in a Nutshell’ with lots of small illustrations,
often only one to three lines high,
right as part of the text,
since a long time.
%
In a way the pocket book ‘Space, Time, Elements at exactphilosophy.net’,
resp.\ the core content of xphi,
come somewhat close to that,
and likely the book ‘Elemental’
I am leisurely letting grow
might also be very similar,
maybe even closer than it appears now,
despite LaTeX not being made for small graphics in paragraphs.

Two ruminations for book covers of the middle of spring 2021,
as I do not want to show the cover of ‘Elemental’,
yet,
even though it would be beautiful.

\begin{center}
{\color{frame}\frame{\includegraphics[scale=0.15]{i-cover-rumination-stoicheion.jpg}}}\hspace{3mm}%
{\color{frame}\frame{\includegraphics[scale=0.15]{i-cover-rumination-space-time-elements.jpg}}}
\end{center}

\noindent
I hope this document, actually the third of some sort of fated triptychon,
will allow me to settle xphi gently.
As the third one it would be related to Atropos,
and do not ask why Baba Yaga has three pairs of helping hands…

\newpage

\subsection{Zen Garden in Kyoto}

I guess the previous pages sounded pretty sad overall, and that is true.
%
Then again, xphi was only growing so quickly
between about summer 2016 and early 2021,
so a state where ‘she’ almost exclusively rests by changing
is not that unusual.
%
Also,
with the ‘seeds’ she is a mother,
while her children are growing
she does not have to do a lot centered on herself.

Would be so beautiful
if some people would pick up some things from here,
but maybe that is more likely the less I keep touching it\,?
%
Might be.

Here is a photo my father took in Kyoto in 1964
when they were traveling around the world
Greece-Egypt-Hong\,Kong-Thailand-$\!$Toyko-Kyoto-Hollywood.

\begin{center}
\includegraphics[scale=0.2]{i-kyoto.jpg}
\end{center}

\noindent
I hope my ability to produce beautiful things
and also find some happiness in life
did not evaporate after they were gone,
I hope was just parallel circumstances,
also since that seems to have
actually only really happened at xphi
after I made the breakthroughs of late 2020 and early 2021,
which give my core findings in essence \textsl{\color{odyssey}critical mass}.
%
Maybe simply I myself unconsciously
do not want to continue at xphi,
but rather want to evolve some core themes further
with more focus
and carry them to people;
maybe ‘she’ is even not involved ‘telepathically’ at all,
maybe never really was
or no longer much,
who knows* in the end\,?

As far as I am concerned,
whenever I relax and lay back, 
 am perfectly happy with what I created so far in my life,
 and my parents were,
 too;
 the solution to anything that is stuck
 cannot be “more and more” all the time anyways,
 everything rests by changing,
 but if the little fox…

\vspace{2mm}
\noindent
\footnotesize
* You cannot know or prove love, only believe, and I guess I still do…

\begin{center}
\includegraphics[scale=0.05]{i-foxyfox.jpg}
\end{center}

\vspace{-8mm}
\begin{center}
\textbf{\scriptsize THE END}
\end{center}

\end{document}