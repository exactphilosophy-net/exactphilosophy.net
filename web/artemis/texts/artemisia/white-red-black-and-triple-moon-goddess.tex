\documentclass[letterpaper]{article}
\pagestyle{empty}
\paperheight=3700mm
\textheight=3700mm
%\textwidth set depending on pdflatex or (experimentally)lualatex
\topmargin=-20mm
\oddsidemargin=25mm

\usepackage[utf8]{inputenc}
\usepackage[spanish,italian,french,ngerman,english]{babel} % last is main
\usepackage{graphicx}
\usepackage{multirow}
\usepackage{xcolor}
\usepackage{contour}
\usepackage{pict2e}
\usepackage{relsize}
\usepackage{amsmath}

\usepackage{iftex}
\ifpdftex
  % stronger fonts

% Find modes.mf, e.g. /usr/local/texlive/2025/texmf-dist/fonts/source/public/modes/modes.mf
%
% $ sudo cp modes.mf modes.mf.orig
%
% Add the following at the start of modes:
%
% mode_def xphi =
%   mode_param (pixels_per_inch, 1200);
%   mode_param (blacker, 1.9); % only difference to 'lexmarkr' (2 there)
%   mode_param (fillin, 0);
%   mode_param (o_correction, 1);
%   mode_common_setup_;
% enddef;
%
% Finally:
%
% $ sudo fmtutil-sys --byfmt mf

\pdfpkresolution=1200
\pdfpkmode={xphi}
\pdfmapfile{}

  \newcommand{\coretextwidth}{85.5mm}
\fi

% experimental, not used to produce the live website...
\ifluatex
  % about same heaviness in pdfs when rasterized in photoshop,
  % but since, unlike the metafont mechanisms I use, fake bold, "bleeds" in all directions,
  % seems heavier at least in core web page images
  \newcommand{\fontbleed}{0.8}
  % paragraphs wider and font looks larger, tried to fix, but then other things change a bit,
  % especially for section headings would have to change back, are now more narrow...
  \newcommand{\fontscale}{0.985}
  \newcommand{\coretextwidth}{85.2mm}
  \usepackage{fontspec}
  % microtype does maybe help and not help, maybe if would allow wider spaces...
  \usepackage{microtype}
  \setsansfont{Latin Modern Sans}[Scale=\fontscale, FakeBold=\fontbleed]
  \setmonofont{Latin Modern Mono}[Scale=\fontscale, FakeBold=\fontbleed]
  % this would be for "new computer modern" (but has many limitations so far)
  %\usepackage[default]{fontsetup}
  %\renewcommand{\familydefault}{\sfdefault}
\fi

\renewcommand{\familydefault}{\sfdefault}

\setcounter{secnumdepth}{-1}

\newcommand{\en}[1]{\iflanguage{english}{#1}{}}
\newcommand{\de}[1]{\iflanguage{ngerman}{#1}{}}
\newcommand{\fr}[1]{\iflanguage{french}{#1}{}}

% a bit less than ~255/256
\definecolor{almostwhite}{gray}{0.996}
\definecolor{xphi}{rgb}{0.0,0.5,0.5}
\definecolor{avant}{rgb}{1,0.5,0.5}

\definecolor{frame}{gray}{0.9}
\definecolor{lightgray}{gray}{0.8}
\definecolor{gray}{gray}{0.5}
\definecolor{darkgray}{gray}{0.3}

\definecolor{darkred}{rgb}{0.8,0.0,0.0}
\definecolor{darkyellow}{rgb}{0.7,0.7,0.0}
\definecolor{darkgreen}{rgb}{0.0,0.55,0.0}
\definecolor{darkviolet}{rgb}{0.5,0,0.5}

\definecolor{darkblue}{rgb}{0,0,0.7}
\definecolor{odyssey}{rgb}{0,0,0.8}
\definecolor{indigo}{rgb}{0.29,0,0.51}
\definecolor{indigoblue}{rgb}{0.1,0,0.6}

\definecolor{saffronback}{rgb}{1.000,0.878,0.627}
\definecolor{saffronfront}{rgb}{0.376,0.125,0.000}

\DeclareRobustCommand{\cometartemisscale}[1]{\includegraphics[scale=#1]{\sourcepath/i-comet.jpg}\hspace{-0.028453em} artemis}
\newcommand\cometartemis{\cometartemisscale{0.018}}
\newcommand\cometartemissection{\cometartemisscale{0.0225}}

\DeclareRobustCommand{\moebius}[1]{\includegraphics[scale=#1]{\sourcepath/i-moebius.jpg}}
\newcommand{\yinyang}{\includegraphics[scale=0.135]{\sourcepath/i-yinyang.jpg}}

\newcommand{\rarr}{\,$\rightarrow$\,}
\newcommand{\lrarr}{\,$\leftrightarrow$\,}

% greek elements
\newcommand{\elfire}{
\begin{picture}(9,6)
  \thicklines
  \put(1,-0.5){\line(1,0){7}}
  \put(1,-0.5){\line(1,1.732){3.5}}
  \put(8,-0.5){\line(-1,1.732){3.5}}
\end{picture}}
%
\newcommand{\elair}{
\begin{picture}(9,6)
  \thicklines
  \put(1,-0.5){\line(1,0){7}}
  \put(1,-0.5){\line(1,1.732){3.5}}
  \put(8,-0.5){\line(-1,1.732){3.5}}
  \put(2.75,1.9){\line(1,0){3.5}}
\end{picture}}
%
\newcommand{\elwater}{
\begin{picture}(9,6)
  \thicklines
  \put(1,5){\line(1,0){7}}
  \put(1,5){\line(1,-1.732){3.5}}
  \put(8,5){\line(-1,-1.732){3.5}}
\end{picture}}
%
\newcommand{\elearth}{
\begin{picture}(9,6)
  \thicklines
  \put(1,5){\line(1,0){7}}
  \put(1,5){\line(1,-1.732){3.5}}
  \put(8,5){\line(-1,-1.732){3.5}}
  \put(2.75,2.5){\line(1,0){3.5}}
\end{picture}}
%
\newcommand{\elhex}{
\begin{picture}(9,6)
  \thicklines
  \put(1,0.5){\line(1,0){7}}
  \put(1,0.5){\line(1,1.732){3.5}}
  \put(8,0.5){\line(-1,1.732){3.5}}
  \put(1,5){\line(1,0){7}}
  \put(1,5){\line(1,-1.732){3.5}}
  \put(8,5){\line(-1,-1.732){3.5}}
\end{picture}}

% i ching trigrams
\newcommand{\trigram}[3]{
\begin{picture}(9,6)
  \linethickness{0.36mm}
  \put(0,5){\line(1,0){#1}}
  \put(5.5,5){\line(1,0){3.5}}
  \put(0,2.5){\line(1,0){#2}}
  \put(5.5,2.5){\line(1,0){3.5}}
  \put(0,0){\line(1,0){#3}}
  \put(5.5,0){\line(1,0){3.5}}
\end{picture}}
\newcommand{\triheaven}{\trigram{5.5}{5.5}{5.5}}
\newcommand{\triearth}{\trigram{3.5}{3.5}{3.5}}
\newcommand{\trithunder}{\trigram{3.5}{3.5}{5.5}}
\newcommand{\triwater}{\trigram{3.5}{5.5}{3.5}}
\newcommand{\trimountain}{\trigram{5.5}{3.5}{3.5}}
\newcommand{\triwind}{\trigram{5.5}{5.5}{3.5}}
\newcommand{\trifire}{\trigram{5.5}{3.5}{5.5}}
\newcommand{\trilake}{\trigram{3.5}{5.5}{5.5}}

% i ching hexagrams
% 1+2 trigrams, 3 rest of line (see e.g. dreams.tex)
\DeclareRobustCommand{\hexagram}[3]{\raisebox{-3pt}{$\overset{\text{${#1}$}}{#2}$\,}#3\vspace{3pt}}

% white-red-black etc.
\DeclareRobustCommand{\outline}[1]{\contour{black}{{\color{white}#1}}}
\DeclareRobustCommand{\white}[1]{\outline{\textbf{#1}}}
\DeclareRobustCommand{\red}[1]{{\color{darkred}\textbf{#1}}}
\DeclareRobustCommand{\black}[1]{\textbf{#1}}
\DeclareRobustCommand{\yellow}[1]{{\color{darkyellow}\textbf{#1}}}
\DeclareRobustCommand{\green}[1]{{\color{darkgreen}\textbf{#1}}}
\DeclareRobustCommand{\violet}[1]{{\color{darkviolet}\textbf{#1}}}
\DeclareRobustCommand{\indigoblue}[1]{{\color{indigoblue}\textbf{#1}}}
\DeclareRobustCommand{\indigo}[1]{{\color{indigo}\textbf{#1}}}

% ELEMENTAL
\newcommand{\ELEMENTAL}{%
\colorlet{contour}{.}\textbf{\color{white}%
\raisebox{+0.001em}{\contour{contour}{E}}%
\raisebox{+0.015em}{\contour{contour}{L}}%
\raisebox{+0.016em}{\contour{contour}{E}}%
\raisebox{+0.023em}{\contour{contour}{M}}%
\raisebox{+0.023em}{\contour{contour}{E}}%
\raisebox{+0.017em}{\contour{contour}{N}}%
\raisebox{-0.020em}{\contour{contour}{T}}%
\raisebox{-0.002em}{\contour{contour}{A}}%
\raisebox{+0.006em}{\contour{contour}{L}}%
}}

% artemis pdf+web icons
\newcommand{\ipdfen}{\includegraphics[scale=0.5]{i-pdf-en.png}}
\newcommand{\ipdfde}{\includegraphics[scale=0.5]{i-pdf-de.png}}
\newcommand{\ipdffr}{\includegraphics[scale=0.5]{i-pdf-fr.png}}
\newcommand{\iweb}{\includegraphics[scale=0.055]{i-web.png}}
\newcommand{\ipdfblueen}{\includegraphics[scale=0.5]{i-pdf-blue-en.png}}
\newcommand{\ipdfbluede}{\includegraphics[scale=0.5]{i-pdf-blue-de.png}}
\newcommand{\ipdfbluefr}{\includegraphics[scale=0.5]{i-pdf-blue-fr.png}}
\newcommand{\iwebblue}{\includegraphics[scale=0.055]{i-web-blue.png}}


\textwidth=\coretextwidth



\begin{document}

\avantgarde

\section{\white{White}-\red{red}-black and triple moon goddess\,?}

A dreamy little tour de force into a quote
by Robert Graves in \textsl{The White Goddess} (1948)
that relates \white{white}–\red{red}–\black{black} to a triple moon goddess:

\begin{quote}
\textsl{\color{xphi}
I write of her as the White Goddess
because white is her principal colour,
the colour of the first member of her moon-trinity,
but when Suidas the Byzantine records
that Io was a cow that changed her colour
from white to rose and then to black
he means that the New Moon is the white goddess of birth and growth;
the Full Moon, the red goddess of love and battle;
the Old Moon, the black goddess of death and divination.
Suidas’s myth is supported by Hyginus’s fable
of a heifer-calf born to Minos and Pasiphae
which changed its colours thrice daily in the same way.
In response to a challenge from an oracle
one Polyidus son of Coeranus
correctly compared it to a mulberry—%
a fruit sacred to the Triple Goddess.}
(Chapter 4)
\end{quote}

I will trace back some sources from the quote above,
then expand a little into quite some directions
and finally come to some conclusions,
which maybe carve out the goddess a bit more clearly than ever before,
or maybe just add to the mystery.%
%
\footnote{%
This text is immediately based
on some discussions at the Astrodienst (astro.com) forum in early spring 2019,
which involved quite a few posts by \textsl{*Momo*},
as well as two by \textsl{Sonnenkind} and one by \textsl{Quadrix}.
%
Further back,
this was also related to an earlier and very voluminous thread
colloquially called \textsl{Mondfaden} (moon thread, December 2014 - June 2016)
where further people contributed:
\textsl{anna.} and \textsl{*Momo*} throughout,
and initially also \textsl{Novalis}, \textsl{fünftes Element}, \textsl{Moonman}, and quite a few more\,(!).
%
Going even further back,
this relates to and originates effectively from personal encounters with a woman—who would have guessed\,?
%
Of course,
reading in Robert Graves’ \textsl{The Greek Myths}
and later \textsl{The White Goddess} helped, too.}
%
Just keep reading and get immersed without knowing.

“Suidas the Byzantine” refers to what is now usually known as the first encyclopedia ever,
written in Byzantium (Istanbul, Constantinople) in the 10th century CE.
%
There is no mention of a cow in that encyclopedia under \textsl{Io},
but the following under \textsl{Isis} (an ancient Egyptian goddess):

\begin{quote}
\textsl{\color{xphi}
She is [sc.\ also] called Io.
%
She was snatched by Zeus from Argos [Myth, Place]
and he,
fearing Hera,
changed her first into a white cow,
then into a black one,
and then into one that was violet-coloured.
%
After wandering around with her,
he came into Egypt.
%
The Egyptians, then, honour Isis,
and for this reason they carve the horns of a cow on the head of her statue,
alluding to the change from maiden to cow.}
%
(stoa.org, transl.\ Jennifer Benedict)
\end{quote}

\newpage

Well, at first sight no mention of the moon at all,
and a different order of colors:
\white{white}–\black{black}–\red{‘red’} instead of \white{white}–\red{red}–\black{black}.

To be sure,
I checked the words in the original Greek text (also at stoa.org).
%
It really speaks of a cow%
—as opposed to a bull—%
that was first white (color of light),
then black,
then “violet”.
%
Note that relating color words to actual colors is usually difficult.
%
A clear attribution is only possible
when text passages are known in which the described object has a well-defined color,
like e.g.\ a ripe blackberry,
or when the text is next to a painting
for which the original color could be reconstructed from chemical analysis,
or something like that…

Let me simply look at the other reference that Graves makes,
namely “Hyginus’s fable”.
%
It is not clear who exactly Hyginus was
and when exactly he lived,
but most probably the Latin text in its surviving form
is from roughly the 2nd century CE.
%
But even after that things did probably not settle,
as the following excerpt from “Hygini Fabuluae” (transl. Mauricius Schmidt, 1872) shows:

\begin{quote}
\textsl{\color{xphi}
[…$\!$] cui dixerunt natum esse vitulum,
qui ter in die colorem mutaret per quaternas horas,
primum album secundo rubeum deinde nigrum.}
(Page 115)
\end{quote}

It speaks of a calf that had been born,
which changed its color three times in a day,
every four hours,
first white, then red, then black.
%
But, as a footnote in the book shows,
“every four hours” (“per quaternas horas”)
had apparently been added by “Tollius”,
which probably refers to the 17th century Dutch classicist Jakob Tollius.
%
This time it’s Graves’s order,
\white{white}–\red{red}–\black{black}.

Let me look into things so far and where a%
—possibly premature—%
conclusion would lead

In ancient Egypt,
the sun had three manifestations during a day.
%
The sun in the morning was Kephri,
the scarab,
the sun at noon was Ra,
the primary sun god
and mythologically also the first pharaoh Egypt ever had,
the sun towards noon was Atum.
%
So,
something that changes three times a day,
like the color of the calf,
maybe even in equal timeframes,
which is close to four hours closer to the equator,
as in Egypt,
where the length of a day varies less than in Europe.

In ancient Egypt,
there was also the Apis bull,
already since the first Dynasty,
a bull usually painted with black and white skin,
plus red genitals
and maybe a red blanket over his back.
%
So,
an ancient solar bull cult instead of a female moon goddess\,?
%
Also a red sun at noon might fit,
as several Egyptian gods and goddesses
are shown carrying a red sun disk over their heads.

As it will turn out,
this interpretation is most likely correct,
but probably only confirms the power of the goddess…

The mulberry is mentioned just a bit further below in the fable:

\begin{quote}
\textsl{\color{xphi}
arbori moro similem esse;
nam primum album est,
deinde rubrum, 
cum permatur[a]uit nigrum}
(Page 115)
\end{quote}

The “arbori moro” would be the mulberry tree,
most likely the black mulberry tree,
but I guess the white mulberry tree cannot be excluded for sure
(the red mulberry tree can,
because it is originally from North America).
%
Independently of the exact species,
mulberries ripen from white via red to black,
so exactly the order Graves gives,
\white{white}–\red{red}–\black{black}.

Now,
and this is crucial,
the time it takes a mulberry to ripen
from white via red to black
is something like a moon cycle,
possibly more precisely
if you look at the whole tree
instead of a single berry,
possibly not.
%
This links two cycles,
the one of the sun during the day
and one for the phases of the moon.
%
Whichever god or goddess is related to that
in the end has the power
over \textsl{both} of these things,
over both sun and moon,
and possibly more.

Going back to some facts,
note that Minos,
the king of Crete,
was in mythology one of the three sons of Europe,
who had ridden to Europe on Zeus disguised as a bull.
%
Another son of hers was Midas,
who turned everything to gold he touched.
%
Gold is usually associated rather with the sun than the moon,
since at least around roughly 0 CE,
while the moon would rather be silver.
%
Also,
jewelry and money would immediately rather be Venus,
who traditionally rules the astrological sign of Taurus,
the bull.
%
Then again,
the moon is considered exalted (a good guest) in Taurus,
at 3$^\circ$ Taurus%
—a hidden hint at a triplicity of a moon goddess,
maybe\,?

In the fable,
the oracle had been cast after the son of Minos and Pasiphaë
had disappeared
and predicted that whoever could interpret it correctly
would save the child.
%
Polyidus actually finds the child,
but drowned in a large jar of honey,
while chasing a ball
(or maybe a mouse, as Graves writes).
%
Polyidus finds the child below ground,
I guess in the very labyrinth
in which later the Minotaur was kept.
%
Minos traps Polyidus down there,
as the oracle is not yet fulfilled.
%
Polyidus sees a snake,
kills it,
then a second snake comes
and reanimates the first one with some herbs,
which Polyidus uses,
in turn,
to reanimate the child.

This reminds again of ancient Egypt,
where Ra,
usually with the help of Seth,
defeats the Apophis snake each night below ground
and rises again every morning,
victorious as the morning sun.
%
It also reminds of the seer Tiresias in Greek mythology,
who observed two snakes at sex.
%
In ancient Egypt seeing such a thing
was believed to lead to the “female disease”,
(male) homosexuality,
which makes it clear what part of the male bodies
the snakes often stand for.
%
Apollon killed the snake Python at Delphi,
and Delphi means womb.

Well,
again rather an ancient Egyptian sun god
and some later derived myths of the Greeks,
but no moon goddess in sight\,?

Let me present a different story first,
and then come back to Egypt later.
%
The Slavic fairy tale of the beautiful Vasilisa
features an old “witch” called Baba Yaga.
%
Vasilisa already lost her mother,
and her step mother and two step sisters are just as mean
with her as in the case of Cinderella.
%
They send her to Baba Yaga to fetch something.
%
While Vasilisa walks to Baba Yaga’s house in the woods,
like Hänsel and Gretel,
she encounters three horsemen.
%
The first horseman is all \white{white},
horse and clothes,
and passes her by just at dawn,
before sunrise,
when the sky starts to get a silver color.
%
The second horseman,
all \red{red},
rides by just a little bit later,
at sunrise.
%
Then she walks all day
and just when the night falls,
the third horseman,
all \black{black}, rides by,
and disappears just at Baba Yaga’s house
into the ground.

Later on,
Baba Yaga answers Vasilisa three questions,
one about each horseman.
%
According to her answers,
they would all three be her servants,
the white one would be day,
the red one the sun,
the black one night.
%
Vasilisa is wise enough not to ask Baba Yaga
about three pairs of hands that help in the house.

Now, what kind of goddess could possibly have six hands\,?

Yes, a triple goddess.
%
The Greek Hekate,
both a goddess of death and a midwife,
like also the moon goddess Artemis,
is often shown as three women standing back to back
(or sometimes just with three heads or faces).
%
But let me ask how you would complete
a sequence of supposedly four things that starts with:
day, sun, night, …?
%
Yes, obviously with “moon”.

As soon as the black horseman disappeared,
the eyes of the skulls on Baba Yaga’s
fence of bones started to glow
so much that her house was lighted as brightly as at day,
which reminds immediately of a full moon.
%
She later kills the step mother and the two sisters by creating a fire.
%
This relates her also to fire,
also in its ability to kill by burning.

The four elements in antiquity were water, air, earth and fire.
%
Fire was a bit special in that group,
as it appeared to exist both on the earth and in the sky,
as sun, moon, planets and stars,
i.e.\ as the lights (fires) in the sky.
%
Aristotle introduced a fifth element
that would essentially only exist in space,
but not down on earth and move in circles.

Now,
life can only exist in water, air and earth,
but not in fire
(except in mythology the salamander).
%
So could Baba Yaga maybe stand for fire
and the three horsemen for the other three elements\,?
%
Seems a bit far fetched, right\,?
%
But wait:
Antiochus of Athens,
who lived roughly in the 2nd century CE,
attributed colors to the four elements,
as follows:
\white{water-white},
\red{air-red},
\black{earth-black}
and \yellow{fire-yellow}.
%
That would be exactly the colors of the three horsemen,
plus yellow for fire and Baba Yaga!

The colors of the four horsemen of the apocalypse are similar.
%
The first three are \white{white}–\red{red}–\black{black},
in that order.
%
The color of the fourth horseman,
“death”,
is usually translated as “pale”.
%
In the original Greek,
it was \textsl{khlōros},
which stands for roughly for a pale or yellowish green,
not unlikely referring here to the color of a dead person.
%
It is also the root of \textsl{chlorophyll},
which makes leaves green and allows them to do photosynthesis.
%
But back to that soon.

Remember the idea of a goddess or god
that would govern the cycles of sun and moon\,?
%
That would be Baba Yaga here:
She is the boss over all fires in the sky
and also over all fires down on earth.
%
And also the other three elements (her horsemen) serve her.
%
So she would be all fires
and in command of all that moves,
via the fire (energy) that makes them move.

And she would also be the moon, for the following simple reason:
The sequence of day, sunrise and night
is always stereotypically the same.
%
Nobody has ever seen the sun rise
before the sky started to light up
or after the sun went down.
%
Sometimes the sun is not visible at all,
when hidden behind clouds,
but I guess in places like ancient Egypt
this was quite rare.
%
But the \yellow{moon} can rise at any time,
before or after \white{day}, \red{sun} or \black{night}.
%
Thus the moon is the boss of these three things.
%
That the moon is in the end stronger than the sun
also shows during a total solar eclipse,
where it is the moon that darkens the sun,
not the sun that outshines the moon.

Baba Yaga’s house is often described as standing on chicken legs.
%
So it is mobile,
it can move,
figuratively in the sky,
like the moon.
%
An ancient historian describes fire sacrifices
of animals to Artemis at Ephesus (now in Turkey).
%
The animals which Artemis hunts, rules and protects,
are the zodiac,
and other constellations in the sky.
%
She,
in turn,
is reportedly,
just like her twin brother Apollon,
from Hyperborea,
a mythical country rather north
“beyond the North Wind”,
so maybe also hinting at the north pole in the sky
around which all else rotates.
%
The two constellations near the north pole are bears,
which fits with quite a few things in the mythology around Artemis.

But let me present the basic idea I have
of the three aspects of the moon goddess and her three colors.
%
They would \textsl{not} simply be the colors of the moon at night,
but rather the colors that make the moon change its color,
as the \textsl{energy (fire)}
that is driving all changes.

The first phase would be the \white{white} goddess making a new moon bright again,
towards a full moon.
%
The second phase would be the \red{red} goddess around full moon,
somewhat before and after full moon,
and what makes the moon pregnant.
%
The third phase would be the \black{black} goddess from sometime after full moon,
making the moon dark again,
towards new moon.

The reason the middle phase would be red
would be the cycle of menstruation.
%
At full moon the seed for a new child would have grown inside her womb,
ripened, like a baby in the full, round belly of pregnant woman.
%
If not getting pregnant,
the seed for the baby would come out as menstrual blood
(and the placenta)
at new moon.

The first phase would also be
a child or a maiden before menstruation,
the second phase a mature woman who can have children,
and the third phase an old woman,
who cannot have children any more.
%
Remember that the encyclopedia entry above saw Isis
as at the transition from maiden to cow\,?

So three phases:
Growth, ripening (or fruit) and withering.
%
The mulberry turning \white{white}–\red{red}–\black{black},
the elements water (the sea),
air (heaven)
and earth (also the underworld),
over which a trinity would have ruled,
in ancient times according to Graves a triple goddess with ever changing member goddesses,
and later the male gods Poseidon, Zeus and Hades, for example.

But where is the \yellow{fire},
the light green color\,?
%
Well,
the mulberry first forms \textsl{catkins},
which are apparently often also slightly bent,
like a reborn moon after having become invisible for a few days.
%
So that color is both death and rebirth,
which is why Artemis and Hecate were midwifes.
%
They have the power to create new life, or not, if they desire.
%
And cats.

Aristotle put the four elements into a circle,
which they follow when changing from one element to the other:
fire-air-water-earth-fire-…\
%
Yes,
this is not the order of the mulberry.
%
But at least his proposed cyclic nature of the fifth element would confirm this picture:
Fire in the sky,
the fifth element,
would be related to the cycle of four elements down on earth.

After all,
if you have a cycle of four elements that starts with fire,
the fifth element in the cycle is fire again!
%
This is maybe also why Dionysos was first born from fire,
when Zeus had to reveal himself to his mother in his true form,
as lightning and burnt the poor mother Semele to ashes,
who had been tricked by jealous Hera into this.
%
After being woven into Zeus’ thigh,
he was born a second time,
this time not from fire,
but from earth.
%
Then he was cut up into pieces
(similar to Osiris in Egypt)
and cooked in water,
the third element.
%
Finally he was also stricken with madness,
so,
I guess,
his mind reborn from air.
%
That he often wore a lion’s skin,
might relate him to the fifth star sign in the zodiac,
Leo,
the lion,
a fire sign,
just like the first one,
Aries,
the ram.

In the scientific article
\textsl{Flowering and fruiting of cv.\ Pakistan mulberry under saline soil conditions in Egypt}
by Ahmed A. El Obeidy (Fruits, vol.\ 60 (6), 2005),
experimental introduction of a special breed of black mulberry on saline soils is described.
%
The fruit of the black mulberry are apparently the best,
but that is not what caught my attention:
“Fruit ripening began in the second week of March
and extended to the third week of April”.

In the Czech tale around Libuše, 
the mythological foundress of Prague,
the queen Niva and her husband Krok had three daughters,
all with magical abilities,
while the prophetic Libuše was the best of them all.
%
Niva is the snow,
winter,
Krok the crocus flower,
the first flower to start growing near the beginning of spring.
%
The life of Niva was tied to an oak tree,
which was guarded by Krok,
so he was her servant.
%
According to the Celtic tree circles found on the Internet,
the oak tree would be the first day of spring
(spring equinox around 21 March).
%
Niva died when lightning (fire) hit the oak tree,
and her youngest daughter,
Libuše,
became queen.
%
That it apparently was the youngest daughter who would follow in reign,
as opposed to the oldest son in patriarchal traditions,
would show that she had absolute control
about how many children would be born,
making the youngest the most gifted.

Libuše used to give council and settle disputes sitting under a linden tree,
which would apparently in the Celtic tree circle
be responsible for two periods of time,
including ten days before the beginning of spring,
which is close to when the mulberries started to ripen in the paper cited above.

At some point in time,
people no longer wanted a female rulership
and asked Libuše to find a husband
who would then be king and she his supportive wife.
%
She said that they should look for a young farmer
with just one shoe in a certain region.
%
And,
yes,
they found such a farmer,
be it because she actually was prophetic
or because she thought that a poor, young farmer would make a good lover
and probably not be too smart
(else he would have had a second shoe and not be a farmer),
so that she could easily direct him as king.

The lost shoe refers also to the last star sign in the Zodiac,
Pisces,
the fish,
associated with the feet in the human body.
%
It also refers to the sandal that Perseus lost when he helped Hera,
disguised as an old woman,
to cross a river by carrying her on his back.
%
In ancient Egypt,
the dead were buried on the western side of the Nile,
where the sun sets.
%
So,
crossing the river would also be both death and rebirth
by grace of the great goddess.

As Robert Graves also essentially writes,
not long after the initial quote on top of this text,
the single best reference to the “white goddess” is
Apuleius’ \textsl{The Golden Ass},
a Latin text from roughly the 2nd century CE.
%
Before going into some of the content of the book,
let me simply quote how Isis describes herself to Lucius
the night before the beginning of spring
when she appears to him at full moon at the beach,
rising out of the sea,
like Venus in greek mythology
or arguably her “hill” as the first earth
out of the primeval sea Nun in ancient Egyptian creation myths.

A shining disk hovers above her head,
which Lucius interprets as the moon
and thus that she is a goddess of the moon.
%
Her clothes are described in great detail:
\white{white},
\yellow{crocus-yellow}
and \red{red},
plus a \black{black} mantle
on which there are the stars and a full moon,
and flowers and fruit on the hem.

She says this to Lucius:

\begin{quote}
\textsl{\color{xphi}
[…$\!$]
rerum naturae parens,
elementorum omnium domina,
saeculorum progenies initialis,
summa numinum,
regina manium,
prima caelitum,
deorum dearumque facies uniformis,
quae caeli luminosa culmina,
maris salubria flamina,
inferum deplorata silentia nutibus meis dispenso:
cuius numen unicum multiformi specie,
ritu vario,
nomine multiiugo totus veneratus orbis.}
\end{quote}

\newpage

Impressive,right\,?
%
Oh, you don’t speak Latin\,?
%
Neither do I,
but I read several translations,
including the one by Robert Graves,
and ended up with this translation to English:

\begin{quote}
\textsl{\color{xphi}
[…$\!$]
mother of nature,
all encompassing mistress of the elements,
first progeny of the times,
highest power/deity/queen,
first/best (sky) deity,
uniform face of gods and goddesses,
who dispenses over
heavenly, shining summits,
salty sea breezes
[and]
the dead down below in earth,
which are silently weeped.
A single/unique goddess in multiple shapes,
with changing rites,
many names,
worshipped all over the world.}
\end{quote}

This brings it all together,
the colors and rulership over air (heaven), water (sea) and earth (underworld),
and the moon, as well as the beginning of spring.
%
Why exactly spring,
here and around Libuše\,?
%
Well,
simply because that is again where the goddess lives,
at the point where the cycle both starts and ends,
beginning of the year, new moon and catkins.

Before describing the story of the golden ass a bit in more detail
and relating it to its times%
—in maybe quite surprising ways—%
let me dive a bit into lesser know waters,
quite speculatively,
or so it might appear at first…

Baba apparently simply means an old woman or grandmother,
but there was an ancient Egyptian god Babi or Baba
(the ancient Egyptians only wrote consonants, but not vowels),
a baboon god especially known for his large, red genitals,
which even served as a mast on the ship in the underworld.
%
Like the Apis bull,
Babi was already present in the first Dynasty (before 3000 BCE).
%
There is apparently an image that shows the pharaoh with a white crown
running first in front of a baboon,
then behind or besides the Apis bull.

Sure,
maybe just a coincidence.
%
But then I remembered the Greek Baubo
who showed her genitals to Demeter,
who was weeping about her lost daughter Persephone,
who Hades had abducted into the underworld.
%
This cheered Demeter up,
probably reminded her of her unique power to create life.
%
So,
again a reference to genitals and a similar name.
%
Almost the same story appears also with Hathor
showing her genitals to her father at some point,
where he was angry about the process between Seth and Horus
about who should be pharaoh that took 80 years to settle.
%
This is why daughters even today
still show their genitals to their fathers
when fathers are feeling blue…\
– just making sure you are still reading attentively,
this is not a serious suggestion from my side.

Now,
Hathor is often a cow goddess,
I guess even was before Isis got that role,
too,
but it is often very hard to tell things apart with certainty
regarding ancient Egyptian gods and goddesses,
maybe because the culture evolved across several millennia,
or also because it is said that they all could transform into each other or,
of course,
into all kinds of animals.

As far as I remember,
there was the notion in ancient Egypt at some time
that the whole sky was a cow,
which would also explain,
where the milky way came from.
%
So,
be it the path of the sun during the day
or the one of the moon at any time,
it would always cross the universal sky goddess as the whole sky.

But is it certain
that this was always or originally a cow and not a bull\,?
%
Apis is a bull,
and the sun god Re is male,
too.
%
For example in Theban tomb TT359,
Re is shown as some kind of cat
(but curiously also resembling a rabbit)
in \yellow{yellow},
slaying the Apophis snake in \white{white}-\red{red}-\black{black} with a \red{red} knife.

Well,
maybe the “Gretchenfrage” is a bit different.
%
Women are more cyclic than men
due to their menstrual cycles,
and they are physically rounder,
with their breasts,
and overall more curvy than men.
%
This would relate them more directly than men to things that go in circles,
like lunar phases, seasons, planets in the sky, and so on,
and also more to round objects in the sky like sun and moon.
%
So,
yes,
this would apparently mirror women much more than men.

Now,
this does not automatically mean that women would rule this.
%
In a patriarchal society,
men rule over women,
so why not also about anything female,
like sun and moon,
and all that motion in the sky\,?

Well,
this is maybe also not the question,
rather:
Should men do that,
can they do that\,?
%
I would say rather not,
but,
of course,
they can and should provide input to consider, ideas,
do stupid but loveable things.
%
But,
all in all,
I guess this remains a matter that also causes a lot of pain,
besides also often a lot of fun,
sometimes closely interlocked.
%
But actually I do not feel like I am a good person
to make judgements like these.

But back to \textsl{The Golden Ass}.
%
It was written in about the 2nd century CE,
as already mentioned.
%
Now,
since about 103 BCE,
the star that rises with the sun at the beginning of spring
is in Pisces,
when presumably the astrological age of Aries
was replaced by an age of Pisces.
%
The first star of Pisces is Alrisha,
the knot,
and probably also symbolizes birth,
as it is part of the (umbilical) cord
that connects the two fish that constitute the constellation of Pisces.
%
These two fish are usually seen as mother and son,
with old roots.
%
In ancient Greece,
they were Aphrodite (Venus) and her son Eros (Amor, Cupid).

It is also Aphrodite that gives the pregnant Psyche four tasks,
in the fairy tale in the middle of the book,
told by an old “Baba” at night in a cave.
%
The four tasks are clearly related to the four elements
and all have to be solved during a day.
%
See my book \textsl{Elementary Star Signs} for an%
—arguably often quite bourgeois—%
exploration of these tasks
and a model that would describe the 12 star signs of the zodiac
as transitions between the four elements.

Psyche is,
like later Snow White,
the most beautiful woman on earth.
%
Hey,
remember that in Grimms’ version of Snow White,
she is described as having snow \white{white} skin,
pitch \black{black} hair
and \red{red} lips,
related to her mother hurting her finger while sewing
and her blood dropping into the snow,
her window enclosed by a black wooden frame\,?
%
Frau Holle (Mother Hulda) was apparently
(as *Momo* noticed)
also described as travelling across the sky
in a chariot pulled by lady beetles,
which are,
yes,
colored \white{white}-\red{red}-\black{black}.

But,
hey,
why don’t you read \textsl{The Golden Ass} yourself\,?
Graves’ translation is very easy to read
and there is so much more in it than I could ever summarize,
from the theater prank they play on Lucius,
via his transformation to an ass by a woman with magical skills,
his dinner at the place of Artemis/Diana,
and so on,
and so on.
%
Let me just mention where he mainly lived: Egypt.

In any case,
the time frame also screams “beginning of spring”.
%
So,
is all of that maybe just self-confirming\,?
%
That was also a time in which Christian religion
arguably wanted to sort of unify the various pagan cults
into more or less a single deity,
in that case a male one.
%
This was also the time in which the Mithras cult
that included slaying of a bull
was having quite a few followers.
%
So,
is the “white goddess” Isis in \textsl{The Golden Ass},
who actually rather wears black plus other colors,
just a child of her time,
nothing that was before or even much after that\,?
%
Apuleius just one of several priests messiases back then\,?

Well,
as always around the great goddess,
or around women,
in general,
the answer is a bit of both,
but quite specifically.
%
She does embody all that is female and which had been there before,
in all kinds of cults,
but possibly none of them combined all of that so synthetically into all stereotypes at once\,?
%
Yeah, sure.
%
But it is true that Pisces are generally seen a sign that is good at synthesis,
as opposed to Virgo,
who is rather analysis,
seen as taking things apart in order to see how exactly they related to each other.
%
Both are considered female signs in astrology,
and there are four more female signs.
%
So,
all in all,
only the carpet of associations is what creates the complete image.

Let me just leave it at that,
just follow all the leads I showed
and copy my approaches to these themes,
which,
of course,
also resemble the ones of Robert Graves somewhat:
Partially careful and precise research,
partially poetic, synthetic intuition.
%
I would add,
that in the end it should all not be taken all too seriously,
because ruling over it all in the end often is.

\vspace{2mm}\noindent
Adliswil, 14 April 2019,
Alain Stalder

\subsection{Notes and Sources}

This text was written on the late afternoon and evening of 13 April
and the relatively early morning of 14 April 2019.
%
Similarly,
Graves wrote \textsl{The White Goddess} in a very short time,
starting sometime in early spring.
%
In 2019,
Mercury went retrograde around 5 March at the end of Pisces,
and is only in a few days going to return to that spot,
so this gives me some time to collect a few more facts and sources here. 

Robert Graves’ northern lunar node was in the middle of Pisces,
which may explain his approach.
%
This is also where Neptune is now,
or my natal Lilith
or maybe the sun of someone implicitly mentioned before.
%
My Saturn is in the last degree of Pisces,
where Mercury had gone retrograde,
as already mentioned.

Did you know that Robert Graves was briefly a professor in Cairo (Egypt)
between January and about June 1926\,?
%
Along with his wife at the time,
Nancy,
and their children,
the young poet Laura Riding came along.
%
In 1929 he separated from his wife,
wrote \textsl{Goodbye to all that} in a few weeks
and then the whole story with Laura Riding took really off,
which arguably was a main source for \textsl{The White Goddess},
and so on,
not to forget Beryl
(or Rosaleen).

Well,
for sources,
just ask Google.
%
Some information is,
of course,
from Wikipedia.
%
Did you know that Jimmy Wales,
one of its founders,
was born the same day
(in the same year)
as me\,?
%
The notion of offering all this for free and without ads
is something that I can relate well to.
%
But still,
of course,
it would also be nice to mention those sources explicitly.
%
Conversely,
I had more than once the impression
that some information in Wikipedia
had been obtained from original sources
that I had possibly mentioned first on the Internet and others had googled
and then worked into Wikipedia.

Hey,
this has nine pages,
an Ennead\,?
%
Uranus is in Taurus,
which might explain all the bull/cow symbolism,
\white{white}–\red{red}–\black{black}\,?

Thanks a lot for reading!
%
And see my website
(\href{https://www.exactphilosophy.net/}{\color{xphi}exactphilosophy.net})
and the Astrodienst
(\href{https://forum.astro.com/cgi/forum.cgi?lang=g}{\color{xphi}astro.com})
forum for more,
if you want.

Frozen except possibly tiniest formal fixes
Sun 14 April 2019 near noon.

\textsl{Egypt in Transition:
Social and Religious Development of Egypt in the First Millennium BCE},
eds.\ L.~Bareš, F.~Coppens and K.~Smoláriková, Prague, 2010.

\newpage

\subsection{Postscript 18 April 2019}

A few days later (17 April),
I researched when Robert Graves started to write \textsl{The White Goddess},
mainly in Richard Perceval Graves’ \textsl{Robert Graves and the White Goddess},
near the end of chapter 4.
%
According to that,
he would have started sometime after Easter 1944,
which was 9 April,
and after “making serious headway”
with his maps for \textsl{The Golden Fleece},
but “could not get out of his head some lines from Alun Lewis’s last letter”
(died 5 March).

What would be a better moment to start a book about the white goddess,
if not a new moon,
the beginning of a new cycle,
with the first part her own\,?
%
New moon was Saturday 22 April at 22:43:23 in Galmpton (Devon).
%
The new moon was at 2$^\circ$\,34’\,41” Taurus.
%
Now,
the 3rd degree of Taurus is where,
according to Ptolemy,
the moon is exalted.
%
Earlier that day (around 2 AM),
Mercury had gone retrograde for almost three weeks;
Robert Graves writes that he wrote the first version in about three weeks.
%
An archetypal female lunar cycle would be three weeks of activity,
followed by one week of reorientation.

That is why the three colors of the cow came up,
and,
Hermes (Mercury) stole 50 cattle from Apollon in the \textsl{Homeric Hymns}
by walking backwards,
just like Robert Graves in the book.
%
There would be more details,
but…wow!

\vspace{2mm}\hspace{-10mm}
\includegraphics[scale=0.25]{i-new-moon-the-white-goddess.jpg}

\noindent
Uranus is in the 3rd degree of Taurus
since last Saturday 13 April at 19:45.

\vspace{3mm}\noindent
\footnotesize{%
See \href{https://www.exactphilosophy.net/white-red-black-and-the-green-goddess.pdf}%
{\color{xphi}www.exactphilosophy.net/white-red-black-and-the-green-goddess.pdf}
for a longer article about the same themes,
titled
\href{https://www.exactphilosophy.net/white-red-black-and-green-goddess.pdf}%
{‘\white{White}-\red{red}-black and the\,\green{“green”}\,goddess’}.

\end{document}
