\documentclass[letterpaper]{article}
\pagestyle{empty}
\paperheight=3700mm
\textheight=3700mm
%\textwidth set depending on pdflatex or (experimentally)lualatex
\topmargin=-20mm
\oddsidemargin=25mm

\usepackage[utf8]{inputenc}
\usepackage[spanish,italian,french,ngerman,english]{babel} % last is main
\usepackage{graphicx}
\usepackage{multirow}
\usepackage{xcolor}
\usepackage{contour}
\usepackage{pict2e}
\usepackage{relsize}
\usepackage{amsmath}

\usepackage{iftex}
\ifpdftex
  % stronger fonts

% Find modes.mf, e.g. /usr/local/texlive/2025/texmf-dist/fonts/source/public/modes/modes.mf
%
% $ sudo cp modes.mf modes.mf.orig
%
% Add the following at the start of modes:
%
% mode_def xphi =
%   mode_param (pixels_per_inch, 1200);
%   mode_param (blacker, 1.9); % only difference to 'lexmarkr' (2 there)
%   mode_param (fillin, 0);
%   mode_param (o_correction, 1);
%   mode_common_setup_;
% enddef;
%
% Finally:
%
% $ sudo fmtutil-sys --byfmt mf

\pdfpkresolution=1200
\pdfpkmode={xphi}
\pdfmapfile{}

  \newcommand{\coretextwidth}{85.5mm}
\fi

% experimental, not used to produce the live website...
\ifluatex
  % about same heaviness in pdfs when rasterized in photoshop,
  % but since, unlike the metafont mechanisms I use, fake bold, "bleeds" in all directions,
  % seems heavier at least in core web page images
  \newcommand{\fontbleed}{0.8}
  % paragraphs wider and font looks larger, tried to fix, but then other things change a bit,
  % especially for section headings would have to change back, are now more narrow...
  \newcommand{\fontscale}{0.985}
  \newcommand{\coretextwidth}{85.2mm}
  \usepackage{fontspec}
  % microtype does maybe help and not help, maybe if would allow wider spaces...
  \usepackage{microtype}
  \setsansfont{Latin Modern Sans}[Scale=\fontscale, FakeBold=\fontbleed]
  \setmonofont{Latin Modern Mono}[Scale=\fontscale, FakeBold=\fontbleed]
  % this would be for "new computer modern" (but has many limitations so far)
  %\usepackage[default]{fontsetup}
  %\renewcommand{\familydefault}{\sfdefault}
\fi

\renewcommand{\familydefault}{\sfdefault}

\setcounter{secnumdepth}{-1}

\newcommand{\en}[1]{\iflanguage{english}{#1}{}}
\newcommand{\de}[1]{\iflanguage{ngerman}{#1}{}}
\newcommand{\fr}[1]{\iflanguage{french}{#1}{}}

% a bit less than ~255/256
\definecolor{almostwhite}{gray}{0.996}
\definecolor{xphi}{rgb}{0.0,0.5,0.5}
\definecolor{avant}{rgb}{1,0.5,0.5}

\definecolor{frame}{gray}{0.9}
\definecolor{lightgray}{gray}{0.8}
\definecolor{gray}{gray}{0.5}
\definecolor{darkgray}{gray}{0.3}

\definecolor{darkred}{rgb}{0.8,0.0,0.0}
\definecolor{darkyellow}{rgb}{0.7,0.7,0.0}
\definecolor{darkgreen}{rgb}{0.0,0.55,0.0}
\definecolor{darkviolet}{rgb}{0.5,0,0.5}

\definecolor{darkblue}{rgb}{0,0,0.7}
\definecolor{odyssey}{rgb}{0,0,0.8}
\definecolor{indigo}{rgb}{0.29,0,0.51}
\definecolor{indigoblue}{rgb}{0.1,0,0.6}

\definecolor{saffronback}{rgb}{1.000,0.878,0.627}
\definecolor{saffronfront}{rgb}{0.376,0.125,0.000}

\DeclareRobustCommand{\cometartemisscale}[1]{\includegraphics[scale=#1]{\sourcepath/i-comet.jpg}\hspace{-0.028453em} artemis}
\newcommand\cometartemis{\cometartemisscale{0.018}}
\newcommand\cometartemissection{\cometartemisscale{0.0225}}

\DeclareRobustCommand{\moebius}[1]{\includegraphics[scale=#1]{\sourcepath/i-moebius.jpg}}
\newcommand{\yinyang}{\includegraphics[scale=0.135]{\sourcepath/i-yinyang.jpg}}

\newcommand{\rarr}{\,$\rightarrow$\,}
\newcommand{\lrarr}{\,$\leftrightarrow$\,}

% greek elements
\newcommand{\elfire}{
\begin{picture}(9,6)
  \thicklines
  \put(1,-0.5){\line(1,0){7}}
  \put(1,-0.5){\line(1,1.732){3.5}}
  \put(8,-0.5){\line(-1,1.732){3.5}}
\end{picture}}
%
\newcommand{\elair}{
\begin{picture}(9,6)
  \thicklines
  \put(1,-0.5){\line(1,0){7}}
  \put(1,-0.5){\line(1,1.732){3.5}}
  \put(8,-0.5){\line(-1,1.732){3.5}}
  \put(2.75,1.9){\line(1,0){3.5}}
\end{picture}}
%
\newcommand{\elwater}{
\begin{picture}(9,6)
  \thicklines
  \put(1,5){\line(1,0){7}}
  \put(1,5){\line(1,-1.732){3.5}}
  \put(8,5){\line(-1,-1.732){3.5}}
\end{picture}}
%
\newcommand{\elearth}{
\begin{picture}(9,6)
  \thicklines
  \put(1,5){\line(1,0){7}}
  \put(1,5){\line(1,-1.732){3.5}}
  \put(8,5){\line(-1,-1.732){3.5}}
  \put(2.75,2.5){\line(1,0){3.5}}
\end{picture}}
%
\newcommand{\elhex}{
\begin{picture}(9,6)
  \thicklines
  \put(1,0.5){\line(1,0){7}}
  \put(1,0.5){\line(1,1.732){3.5}}
  \put(8,0.5){\line(-1,1.732){3.5}}
  \put(1,5){\line(1,0){7}}
  \put(1,5){\line(1,-1.732){3.5}}
  \put(8,5){\line(-1,-1.732){3.5}}
\end{picture}}

% i ching trigrams
\newcommand{\trigram}[3]{
\begin{picture}(9,6)
  \linethickness{0.36mm}
  \put(0,5){\line(1,0){#1}}
  \put(5.5,5){\line(1,0){3.5}}
  \put(0,2.5){\line(1,0){#2}}
  \put(5.5,2.5){\line(1,0){3.5}}
  \put(0,0){\line(1,0){#3}}
  \put(5.5,0){\line(1,0){3.5}}
\end{picture}}
\newcommand{\triheaven}{\trigram{5.5}{5.5}{5.5}}
\newcommand{\triearth}{\trigram{3.5}{3.5}{3.5}}
\newcommand{\trithunder}{\trigram{3.5}{3.5}{5.5}}
\newcommand{\triwater}{\trigram{3.5}{5.5}{3.5}}
\newcommand{\trimountain}{\trigram{5.5}{3.5}{3.5}}
\newcommand{\triwind}{\trigram{5.5}{5.5}{3.5}}
\newcommand{\trifire}{\trigram{5.5}{3.5}{5.5}}
\newcommand{\trilake}{\trigram{3.5}{5.5}{5.5}}

% i ching hexagrams
% 1+2 trigrams, 3 rest of line (see e.g. dreams.tex)
\DeclareRobustCommand{\hexagram}[3]{\raisebox{-3pt}{$\overset{\text{${#1}$}}{#2}$\,}#3\vspace{3pt}}

% white-red-black etc.
\DeclareRobustCommand{\outline}[1]{\contour{black}{{\color{white}#1}}}
\DeclareRobustCommand{\white}[1]{\outline{\textbf{#1}}}
\DeclareRobustCommand{\red}[1]{{\color{darkred}\textbf{#1}}}
\DeclareRobustCommand{\black}[1]{\textbf{#1}}
\DeclareRobustCommand{\yellow}[1]{{\color{darkyellow}\textbf{#1}}}
\DeclareRobustCommand{\green}[1]{{\color{darkgreen}\textbf{#1}}}
\DeclareRobustCommand{\violet}[1]{{\color{darkviolet}\textbf{#1}}}
\DeclareRobustCommand{\indigoblue}[1]{{\color{indigoblue}\textbf{#1}}}
\DeclareRobustCommand{\indigo}[1]{{\color{indigo}\textbf{#1}}}

% ELEMENTAL
\newcommand{\ELEMENTAL}{%
\colorlet{contour}{.}\textbf{\color{white}%
\raisebox{+0.001em}{\contour{contour}{E}}%
\raisebox{+0.015em}{\contour{contour}{L}}%
\raisebox{+0.016em}{\contour{contour}{E}}%
\raisebox{+0.023em}{\contour{contour}{M}}%
\raisebox{+0.023em}{\contour{contour}{E}}%
\raisebox{+0.017em}{\contour{contour}{N}}%
\raisebox{-0.020em}{\contour{contour}{T}}%
\raisebox{-0.002em}{\contour{contour}{A}}%
\raisebox{+0.006em}{\contour{contour}{L}}%
}}

% artemis pdf+web icons
\newcommand{\ipdfen}{\includegraphics[scale=0.5]{i-pdf-en.png}}
\newcommand{\ipdfde}{\includegraphics[scale=0.5]{i-pdf-de.png}}
\newcommand{\ipdffr}{\includegraphics[scale=0.5]{i-pdf-fr.png}}
\newcommand{\iweb}{\includegraphics[scale=0.055]{i-web.png}}
\newcommand{\ipdfblueen}{\includegraphics[scale=0.5]{i-pdf-blue-en.png}}
\newcommand{\ipdfbluede}{\includegraphics[scale=0.5]{i-pdf-blue-de.png}}
\newcommand{\ipdfbluefr}{\includegraphics[scale=0.5]{i-pdf-blue-fr.png}}
\newcommand{\iwebblue}{\includegraphics[scale=0.055]{i-web-blue.png}}


\textwidth=\coretextwidth



\newcommand{\easy}{\textbf{\color{odyssey}easy\hspace{0.02em}.\hspace{0.02em}.\hspace{0.02em}.}}

\begin{document}

\avantgarde

\section{Mysteries of life}

How life sometimes takes its decisions,
this time in the morning of 27 May 2023 ca.\ 07:36,
with the (mean) northern lunar node smack dab in the middle of the third degree of Taurus, where the moon is traditionally exalted*:

\small
\begin{itemize}

\item[\color{white}$\bullet$]
Since relatively recently it has become quite calm at my website,
almost like at Shangri-La in James Hilton’s novel \textsl{Lost Horizon},
which was the very book that started mass-market paperbacks in 1939,
relating in many, many ways to both of my parents and also to me relatively directly.
%
Anyways,
since there is in recent years only exactly one project
out of a great many related to the ideas of this site
that wants to at least grow very slowly,
I guess that will likely be it for the next few years,
the world in a nutshell in a mass-market-style pocket book:

Dreaming of {\smaller\ELEMENTAL}…

\end{itemize}
\normalsize

\noindent
Where else did that come from,
after all that back and forth the past few years,
with all kinds of projects to promote my ideas in my mind,
and after so many tries that then just did not want to grow,
even though rationally/formally they seemed quite fine\,?
%
Well, to me still mainly due to $\pi$, but feel free…

Here is how it looked like in the timeline just before the above reduction,
written only a few days earlier:

\small
\begin{itemize}

\item[\color{white}$\bullet$]
To me my contributions on this website and in forums
during the past 20+ years
are good enough to remain as they are,
no obligation on my side to continue anything.
%
Anything more just if I really, really like to do it.
%
And so far,
only the project for the pocket book {\smaller\ELEMENTAL}
seems to fit that description
(while I have tried to start many, many things during the past few years).

Since about 1.5 years,
I have a used copy of Jack Kerouac’s \textsl{On the Road},
which I did not read, yet,
and might never read.
It smells very special,
not sure of what,
something between perfume and aftershave,
which would have been spilled over it,
and it is a smell I think I remember from childhood,
and inside the book is the calling card of a hotel,
the \textsl{Sule Shangri-La} in Yangon (Rangoon),
which got me to James Hilton,
who invented the term \textsl{Shangri-La}
in his novel \textsl{Lost Horizon},
which I read recently,
and liked,
not being aware that it was this very book
that started mass-market paperbacks in 1939,
the year my mother was born,
who used to read such pocket books,
often Agatha Christie (in French) and Simenon,
and my moon is at the MC,
and do you remember that the mountain in the novel
had a name that translates to ‘blue moon’,
and the very philosophical side in the book
likely also mirrors my father,
or maybe all in all also me
via what Conway did in the end
despite all philosophical reasoning against it,
or rather had no choice
because a woman
(Lo\hspace{0.1em}-\hspace{-0.1em}Tsen,
reminding of Lao\hspace{0.1em}-\hspace{-0.1em}Tse (German variant of Lao Tzu)
or Lotse (German for a nautical pilot)
[added later: or Lohtse, the ‘South Peak’ of the Everest massif],
and, of course, $\pi$)
had made a choice,
where I should add
that my father liked Lao\hspace{0.1em}-\hspace{-0.1em}Tse’s \textsl{Tao Te Ching} a lot,
hence no imbalance in terms of heritage in the end,
and I like Taoist writings a lot, too,
not least Zhuangzi.
%
Why fight/promote anything…

This is not really what I dreamed of
when I started to publish my ideas in 2002.
%
I was hoping for some folks to perceive the beauty of it,
so that by now it would mainly be growing by itself.
%
For eventual readers of coming generations,
I would say,
I already did a lot,
and maybe the sort of emerging pocket book
would also in that sense be ideal,
have the best chance to maybe get some attention sometime.
%
It is just not possible to invest so much time into all of this for me any more.
%
I will just see what comes –
much less may be much more…

Hm,
I did dream,
already when younger,
of writing a pocket book (in classic small size)
titled \textsl{The World in a Nutshell},
which is,
I guess,
arguably quite close to the current project,
as these four words appear
both in the text on the back cover of the book
and as the title of its first chapter.
%
So maybe it would all at least just turn out
into what it was meant to be all the time.
%
\hspace{-0.1em}{\smaller+}$\pi$.
%
May.

\vspace{1mm}
Dreaming of {\smaller\ELEMENTAL}…

\end{itemize}
\normalsize

\noindent
And here is how the section ‘Seeds’ looked like 14 March 2023, more precisely the ‘dreams’ subsection there that had grown in about a month, mimicking the ‘leads’ subsections in contextually earlier sections on this site:

\vspace{4mm}
\hspace{5mm}\begin{minipage}[t]{85.5mm}
\subsection{dreams}

\small
Finding out what to do next is often not so easy for me.
Below a tiny glimpse, similar to the leads in earlier sections.
Cast in a way “at the end”, a bit like Hollywood, where all
movement west morphed into tangible virtual dreams, like
sprayed into Sedna’s digitally detached fingers…

\footnotesize
\begin{list}{$\bullet$}{\setlength{\leftmargin}{10pt}}

\item
Since the turn of the millenium, many beautiful findings have
been made and published at exactphilosophy.net. But nobody
knows about them, or almost nobody.

There would be so many interesting and useful things to
discover here – for so many people. I am still hoping for people
to just come, see, and fall in love with some things here and
then carry them on; that would be best for these things.

\item
Maybe a website or section with title \textbf{{\color{darkblue}easy} exactphilosophy},
something more accessible than this site so far, something
where I could add things here and there and relate to each
other\,? Maybe \textbf{\color{darkblue}easy}.exactphilosophy.net\,?

\item
I have no clue what to do next—now that I am apparently
done since early 2021 with finding out the major new things.
Promotion, or even marketing, is simply not my thing, for all
that it seems. Nor so far apparently even to write things down
in more detail than necessary to be accessible to someone who
has genuine interest in understanding nature.

This might be why{\smaller\&}how {\color{xphi}Sleeping Beauty Dreaming} emerged
as a theme in summer 2021\,…\,do nothing, just lay back and
hope for either some inspiration for something to do that I
would really love to do to somehow magically emerge in time,
or else maybe dream of some people who would turn up and
grasp some of it, pass it on, evolve it, let it grow.

Honestly, in my view, the descriptions in the sections here
are beautifully to the point. Twenty years of work have gone
into them and they describe things naturally in simple, direct
terms. Someone just has to read it…\,just has to read it.

\end{list}

\end{minipage}

\newpage

\hspace{5mm}\begin{minipage}[t]{85.5mm}

\footnotesize
\begin{list}{$\bullet$}{\setlength{\leftmargin}{10pt}}

\item
It seems the ancient Greeks used to say that the beginning
is half of the whole. Hence if public reception of my findings
only started late, that might not be a bad sign, and I am still
dreaming of just leaning back and watching things grow…

\item
The fox icon of this website is inspired by the first answer by
the I Ching to a question from me (“What about Prague?”),
and it gave hexagram 64: “Before Completion. Success. But
if the little fox, after nearly completing the crossing, gets his
tail in the water, there is nothing that would further.”

\item
Maybe {\smaller\ELEMENTAL} could reach out to all, focussed on the
most important new idea, like the main content on this site,
and hopefully streamlined Hollywood-style for almost all\,?

\item
Almost always when people read something I wrote in public
(website, post, book) it felt stressful; so never get famous,
while Jack Daw’s {\smaller\ELEMENTAL} might be detached enough\,?

5 March 2023 just after midnight: I just finished reading \textsl{Light
My Fire} by Ray Manzarek, about the Doors (of perception).
I guess Ray-air, Robby-fire, John-earth, Jim-water, and I
understand why being famous was not fun for Jim (yet for Ray),
because he could feel it all, so I guess I am very lucky not to
have become famous, the occasional visitors to my website
are already stressfull enough to feel. So, just wait happily…

\item
What I created is already very beautiful as it is and the findings
are absolutely exceptional, so should I really feel bad if I
essentially lacked the skills to successfully promote it, would
not just preserving it already be enough for the fates…$\!$?

In these postmodern times, is it possible at all to be special
in a flood of individual thoughts that are all deemed equal\,?

Or only with something near a lowest common denominator,
like maybe they way I arranged the trigrams of the I Ching in
a circle or Möbius strip between the ancient elements\,?

\item
I guess at my core I am an avantgarde researcher; so what I
create would by definition only become visible after me, since
otherwise it would not really have been avantgarde\,?

What would the avantgarde answer be\,? Just continue to do
avantgarde\,? And preserve its core\,? Or not even that\,?

I guess the public and future generations have in a way some
rights to know a bit of something around how things emerged,
but my core gift is new discoveries\,/\,ideas, not all of myself.

\item
Maybe still the idea of 15 Feb 2023, a book that would sort
of fuse two book ideas with an already existing book; maybe
just do that and not care about reception or even promotion\,?

Seems at least like I could write it and that it might well turn
out to be at least something beautiful in itself\,?

\item
Just \textsl{do} nothing, just imagine and dream for now…

{\color{xphi}Sleeping Beauty Dreaming}\,…

\item
{\smaller\ELEMENTAL} is slowly emerging. Not only is it about the core
themes of this site, it is even evolving them. While getting it
right is not easy, takes time, ongoing since early 2019…

\item
A beautiful 3D Möbius strip of elements would be cool.

\item
relax alain.\ elemental eventually.\ and relax.\ take your time.\ {\scriptsize $\pi$}

\end{list}

\end{minipage}

\noindent
Let me just say that despite many other considerations,
some of which have been preserved around this website and others that have not,
like the dream I had tonight,
also astrological ones like recent Mars and Venus returns plus a Venus-Uranus sextile,
I still feel that it was essentially the wish of $\pi$ in the end$^*$,
and I guess also in a broader sense of the fates,
the white goddess expressing her wish via the valve of the north node at her home.
%
And me, I guess I am just going to go with it,
even though the decisions came unexpected to me,
maybe as unexpected as to Conway in \textsl{Lost Horizon}.
%
{\footnotesize $^*$\,07:36 Sat\makeatletter @{\smaller MC} near her Ven/Sat}

And maybe they would at some later point also allow me,
for example, to write the little book shown below
that would also not be much larger than a classic mass-market pocket book,
and maybe do a 3D print of the Möbius strip of elements and trigrams of the I Ching,
and maybe other things, like animations, maybe other books, and so on …

\begin{center}
{\color{frame}\frame{\includegraphics[scale=0.12]{i-space-time-elements.jpg}}}
\end{center}

\noindent
And this,
in an exclusive preview (All Rights Reserved),
would be the projected cover of \ELEMENTAL\
by \includegraphics[scale=0.05]{i-jackdaw.jpg}
to appear maybe in …

\noindent
\begin{center}
{\color{frame}\frame{\includegraphics[scale=0.03125]{i-elemental.jpg}}}
\end{center}

%\newpage

\subsection{Prague Fringe}

It was Friday 13 October,
the last summer day (in terms of temperature) of 2023 in Zürich,
around 3:15 pm,
with me walking back to where I had parked
after having been once more at Badi Tiefenbrunnen
for a very quick swim
and started to read in Gustav Meyrink’s \textsl{Das grüne Gesicht}
plus stumbling on Wikipedia over another book by Meyrink,
titled \textsl{Der Löwe Alois und andere Geschichten}.

On the way back,
actually quite a bit into the direction of the Astrodienst Villa,
I made first a rather stupid Gedankenexperiment,
not the first time of that kind
and usually not with a clearcut outcome.
%
I imagined two books falling down,
on the right \white{Space} \red{Time} \black{Elements}
in the form where I would quote and annotate all of the core text of my website,
and on the left an also finished version of \ELEMENTAL,
and that I would only have the option to grab and thus preserve one of the two books.

But then I imagined \ELEMENTAL\ in its current form:
hardly started to write,
just a few pages plus visual setup with cover and back cover, and so on.
%
My reaction astonished me or rather my conscious self.
%
I grabbed \ELEMENTAL,
without any hesitation or doubt,
like it was family or just something really, really loved.

That closes it.

\subsection{Tweets}

The next day,
via a very surreal and beautiful path
with clear references to ‘Prague’,
I got to a video where a bird was singing or talking
on the arm of a woman in the morning,
what I interpreted as ‘me to Prague’,
just a tiny bird that has fun saying this and that.
%
That bird was a starling,
in German ‘Star’.

Unlike many major discoveries in physics
or generally in exact sciences,
my findings are hard to prove with certainty
and in comparison easy to explain
even to someone who knows zero about math or science of philosophy, etc.
%
Hence,
writing a scientific paper seems relatively futile,
while writing a book for a all,
a rather small and short book,
seems to make much more sense.

Besides,
such a book would be cheaper to produce and buy,
easier to translate and transform to other media,
and so on.
%
And very likely also easier to advertise.
%
Note also that mainstream is usually movies
or books that tell a story,
not nonfiction books
nor ‘tweets’ of any kind in social media.

A website is usually not read sequentially,
hence not suitable to tell a story.
%
Visitors to my site usually land anywhere,
typically coming from a search engine,
and even if they start at the main page,
practically nobody starts reading more than maybe the first two sections
(always skipping ‘way’, by the way)
and then usually a few more random clicks,
and then quickly gone again.
%
Not absolutely everybody,
but that is the general picture.

Tweet, tweet, tweet…

\subsection{November}

Yesterday evening,
10 Nov 2023, around 21:55,
I realized that already the first sentence in {\small\ELEMENTAL},
“All experience in life is personal.”
is most likely not universally true.
%
Maybe that book is simply not for me to write.
%
Oracles into my discoveries document almost two years ago:
%
Question if as a dialog,
with two results
because the order of two coins was too close to call:
“\color{xphi}This is \textbf{\setlength{\fboxsep}{1.5pt}\colorbox{yellow!50}{\color{red}t}}he reason
why women are more adult than men,
why women do not play,
except with men resp.\ with their penises.
The female eleme\textbf{\setlength{\fboxsep}{1.5pt}\colorbox{yellow!50}{\color{red}n}}ts, earth and water,
are heavier than the male ones, fire and air.\color{black}”
with later in the document
“\color{xphi}The deep, dark secrets of the world
are guarded by the heavier elements,
i.e.\ collectively and unconsciously by the female side
in both women and men,
which also decides about life and death
(within the limits of physics).\color{black}”.
%
Question if as a monolog:
“\color{xphi}Venus is very hot and has a high pressure on its surface,
but on the other hand,
one could land on it without \textbf{\setlength{\fboxsep}{1.5pt}\colorbox{yellow!50}{\color{red}a}} parachute
if one was able to withstand the heat and the pressure.\color{black}”.
%
Could be interpreted as that I would not be able to do this as a monolog,
and need a woman to do it as a dialog.
%
In some sense,
all experience in life is personal,
\textsl{you} are experiencing it,
but there seem also to be things that are intrinsically shared,
even are just one thing,
which would contradict my take,
but my approach does in fact lead to the idea
that, because inside would be soft,
things would mix,
would be connected
or even \textsl{one} inside.
%
Maybe argue like that in the dialog in the book\,?
%
Stay tuned…

\newpage

\subsection{December}

On the morning of 6 December 2023 at about 6:50 AM,
with Neptune still slowly retrograde for a few hours,
I had the impression of having I finally unwound the core of this complex around \ELEMENTAL\ and more;
interesting also in relation to Neptune and illusions,
as it would then seem that he might also resolve them,
not only create them as maybe more commonly assumed.

I looked at the chart for my most recent Chiron return,
of late January 2018,
most likely the last one of my life,
unless I would get older than Henri Kissinger got.
%
And it mirrored \ELEMENTAL\ and a lot more around it,
also private to very private things,
for all that it appeared then very closely.

Just as I started to write this down,
initially very shortly after the finding,
I noticed one more detail,
which I deemed to again completely reverse my first impulse after the discovery,
which had been not to do \ELEMENTAL\
but instead do more in the way of exactphilosophy.net in the past 20 years.

But one thing after the other,
here is the chart for the return,
assuming I would then still live near Zürich, Switzerland,
which is very likely:

\begin{center}
\includegraphics[scale=0.23]{i-chiron.jpg}
\end{center}
\vspace{-3mm}

\noindent
Chiron has two strong major aspects,
a square to the moon in Gemini at the end of the second house,
hence at the cusp of the 3rd,
which might well suggest writing,
and a sextile to Mercury in Capricorn in the 10th house,
which indiciates that writings might be published.
%
Mercury is conjunct Pluto also in Capricorn in 10
(and Pluto at  weak sextile to Chiron itself, if you allow 6° orbs for sextiles),
with Pluto at the MC,
4° ahead of the MC in the middle of Capricorn.
%
There is also a weak square of Chiron to Saturn in early Capricorn in 9.

And there is a not so weak trine to Jupiter in 7 in Scorpio near my s.node,
and one to Mars in early Sagittarius already conjunct the cusp of the 8th house.
%
There are also octiles (semi-squares) to sun and Venus in Aquarius in 11,
and there is a {\color{xphi}close trine to retrograde Sedna}
near Algol in Taurus in the first house.

Venus is at the s.node,
opposite my natal sun that is thus right at the n.node of the return,
joined by retrograde Ceres at the return.

I do see a lot of \ELEMENTAL\ in that chart,
something that seems heavier, darker and less beautiful than before the return,
with Pluto at the MC,
prominent Capricorn
and first drafts of the book in early January 2019,
probably also less beautiful because Venus is at the south node,
resonating also with the idea of late spring 2021
with “Sleeping Beauty Dreaming” for this website.

Many things in my life since about the Chiron return
also seem to reflect quite well in this,
which were often also rather sad in comparison.

{\color{xphi}%
The ‘one more detail’ I noticed
is asteroid 278 with a particular (female) name
conjunct Chiron at its return,
just a few arc minutes behind it.
%
That is apparently why \ELEMENTAL\ has now apparently grown
into a dialogue on the very few pages it has so far,
and why the whole thing in that way might still be central to me,
and probably could have a chance
for a nontrivial Hollywood-style mainstream in the future.
%
And, as already mentioned,
Chiron (and the mentioned asteroid) were at a close trine to Sedna (and Algol).}

The mentioned Neptune station was at 2:22:07 PM CET,
with, near Zürich,
Asteroid 1969 Alain at the MC near the middle of Capricorn,
only a few degrees behind the position of the MC at the Chiron return.
%
So much about fate and
and when things can be discovered by mere humans,
I presume\,?

I forgot to say that Chiron at the return was in the 12th house,
ruled by Neptune in Pisces also in the same house,
and even Uranus in Aries “captured” there,
as well as in a way my natal moon and MC in Aries
and my natal Saturn and Lilith in Pisces.
%
Difficult to say what this might boil down to,
I think I am simply too small to see the full picture.

Especially with Venus at the return at the s.node,
better to have faith in women, fate, also money and beauty,
but not so much by thinking and practical considerations,
more by inituition and feeling,
and with Venus in Aquarius also with some hopes for the future.
%
In a way,
this also what remains to me at the time being,
“delusional hope”.

Hey, one more thing.
%
The asteroid 15992 Cynthia,
with the name of the female character of the dialogue with Jack in the book,
was at the Chiron return just a little less than 2° ahead of Mercury in Capricorn in 10,
and for the Neptune station very close to the moon in late Virgo in 6.
%
Both of that talks to me,
although the latter not consciously analytically so far.
%
And I guess you all know
where to mythologically place Cynthia,
right folks\,?

Tweet, tweet, tweet, tweet, tweet…

\vspace{2mm}
\noindent
{\footnotesize
* As it turned out in early December,
“Since relatively recently it has become quite calm at my website”
was likely just a bug in my filtering script that filters out webcrawlers,
so that it would sometimes hang at DNS and not send an email—%
visits seems to be essentially unchanged…\par

\vspace{1mm}
\noindent
I guess that solves an immediate mystery
from a practical perspective,
while it does not explain the mystery of \textsl{why}
having virtually been in Shangri-La in a way
for a few months this year.\par

\vspace{1mm}
\noindent
Here is something I quickly painted this evening,
7 Dec 2023,
accidentally scanned from the ‘wrong’ side.
%
In the end it all boils down to the last sign of the zodiac to me,
Pisces,
or faith,
just the minimal faith
(Saturn/Chiron in Pisces, in 9, last Chiron return in 12)
that what I discovered did not go unnoticed at all,
even most likely rather was part of a ‘bigger’ plan of some sort,
to say it in human words
for lack of better terms
at my limited disposal.\par}

\begin{center}
\includegraphics[scale=0.23]{i-penguin.jpg}
\end{center}

\noindent
{\footnotesize
Many of the other projects I envisaged,
or still envisage for one day, would be good,
but the more I try,
the more it turns out that none would be as good as \ELEMENTAL,
not as beautiful in print,
not as accessible,
not going as deep and further
in analysis and synthesis.

She wants \ELEMENTAL\ only already since quite some time.

Still also has to mirror me and my fate to some degree, just like x$\varphi$.

Well, x$\varphi$ has often been more of less dormant in the past,
and other projects as far as considered so far
would not be all that intellectually challenging,
hence I could still tackle them in 5 or 10 years,
so there is nothing immediately speaking against doing just \ELEMENTAL;
besides, a stable x$\varphi$ website is maybe also more approachable,
less of a moving target.
%
And \ELEMENTAL\ is definitely a challenge for me,
in several respects.

Let’s do it.
\par}

\vspace{2mm}
\noindent
{\footnotesize
Soon there will be a new moon,
in Sagittarius,
after the current one that had sun and moon
very close to my south node in Scorpio with Neptune and especially Artemis,
and in the 11th house,
while at birth the north node was in the 11th house (with Pythia),
so let’s see if the “Let’s do it.” above would survive into the next cycle;
remember the Sabian Symbol for 26° Pisces (my Chiron),
“A new moon that divides its influences. Finesse.”.
%
Then again,
Venus is now in Scorpio,
so women or even one in particular,
would want a decisions,
want some things ‘cut’,
removed from the picture,
to make room for new things to grow;
maybe not wrong to associate this with Scorpio-Taurus,
which also reminds of Artemis.
%
But how about just cutting ‘public ruminations’ around this
and just say “On verra.”\,?
%
Well, let’s do it.
\par}

\vspace{2mm}
\noindent
{\footnotesize
Note of early January 2024,
i.e.\ written after the following page:
As the image of the penguin above maybe shows,
in my feeling these are just “Sedna Times”,
especially as the core idea of this site
emerged shortly after the discovery of Sedna became public,
with Sedna near my north node.
%
This probably means that there will be no big leaps,
that the website would still gently evolve,
as well as other things,
probably including the mentioned books and maybe more,
and with Sedna in a few months in Gemini for a long time,
not unlikely more translations
and likely other shifts of some kind.
%
But at the core,
a theme of beauty and female apparent ‘passivity’ will probably remain.
%
Public attention might come at some time,
but probably ‘on its own’,
not as a consequence of ‘aggressive promotion’ or the like…
%
And very likely \ELEMENTAL\ would fit the Sedna theme better than other ones,
because it is in a way quite an ‘underdog’,
like Sedna in her myths…
%
And she was at the \textsl{origins} of this site, see oracle of next page…
\par}

\newpage

\subsection{origins}

Sorry,
but it did still not end exactly like that in December 2023.
%
On the evening of 15 December,
I asked my coin oracle
(c.f.\ previous articles
and Usenet posts about 20 years earlier to alt.astrology.tropical)
the following quite open question:

\begin{quote}
\begin{otherlanguage}{ngerman}
\textsl{\small\color{xphi}
Frage zu meiner Entdeckung mit den Elementen als Raum und Zeit,
v.a.\ dahingehend,
wohin sich die Idee entwickeln will
und was ich dazu tun könnte oder auch sein lassen sollte
(also auch ob eher viel und/oder rasch,
ob eher xphi oder elemental oder ste oder ganz was anderes,
alles so dinge auch implizit mit drin,
und doch offen die frage):}
\end{otherlanguage}
\end{quote}

\noindent
I started to shake the coins in my hands,
intending probably to do that for ten minutes
instead of the usual five.
%
But after close to three minutes the cat of the neighbors upstairs came by,
and so I “had” to toss the coins
already after about $\pi$ minutes,
which was at about 20:00:25 that evening,
at home in Adliswil.

The oracle pointed to the ‘g’ in the middle of the word ‘origins’
in the section about the Age of Aquarius,
and the moon was that evening in early Aquarius.
%
Anyways,
the word ‘origins’ resonated a lot with me,
even though I was not sure how exactly to analyze the meaning of the oracle consciously,
and still am not really able to do so.
%
What should be mentioned maybe is that Venus was then in Scorpio,
I guess already conjunct my s.node,
which is one way of looking at origins,
the other maybe being the IC,
which was not so far from the IC at my birth,
also in Aries,
and with the s.node near it.
And a third way to look at it would be etymologically, rise up, hence maybe AC.

Considering that my Neptune
and, more closely, my 105 Artemis
are near my s.node in 5 at birth,
my conclusion later that evening
was that it would have to be \ELEMENTAL\ that would grow,
and probably practically only that.

But the next morning was a surprise.
%
Between sometime probably not long after 9:30 and 10:53,
I wrote a welcome text for an additional translation to \easy\
for my website exactphilosophy.net,
and this time it was of a nature that seems to be able to carry on,
to grow,
even though \textsl{unfortunately}
not just a parallel translation in the same layout,
because the easy text is just too long for that.
%
The approach would be Aquarian,
in a way less timeless than what is or was here at the time,
for example,
using AI as an allegory in the welcome text,
but still of more than just good quality.

Maybe moot to discuss any further,
but I am much more optimistic for the future of x$\varphi$ now,
no matter which of these things might grow in time.
%
Actually,
I just got back mentally and emotionally and imaginatively and maybe realistically and “e5” to mainly \ELEMENTAL\,…

\noindent
\begin{center}
{\color{frame}\frame{\includegraphics[scale=0.035]{i-elemental.jpg}}}
\end{center}

\subsection{%
Jean Richer’s zodiac with Delphi at the center,
Ithaca/“Odysseus” west,
also AC of the station of Neptune/“Poseidon” at the DC}

A special Neptune (“Poseidon”) station occurred 2 July 2024
at 12:41:35 CEST at 29°55’55” Pisces.
%
It had never been that close to the end of the zodiac
in at least the past 2500 years,
while it occurs in the last degree of Pisces
about every few hundred years.
%
This may relate to my natal Saturn (“public status/work”)
as well as to the sun of the registration of this website’s domain,
which are both in the last degree of Pisces,
but not as close to the end of it,
hence possibly less universal than Neptune at that moment.

Besides “Poseidon” (with Priapus) at the DC (29°44’)
with also an emerging idea that reminded of the Odyssey,
moon and Uranus were
at the cusp of the 9th house
conjunct (in terms of longitude) with fixed star Algol,
the head of Medusa, 
which possibly also turned some of my perceptions to stone,
and took me several days during my summer holidays
to get my website again into a largely consistent state,
even though the core idea/way had only been grazed.

Was still deeper than it may appear now.
%
Ever since recently when Google effectively no longer values content on my website (general issue),
my website has become almost completely invisible to searches,
often completely invisible even to very specific searches,
which is unexcusable.
%
But more fundamentally,
it might simply be better to create individual documents (books, whatever)
for individual ideas,
which could then be promoted individually.
%
Only one of them would have to “land”
to get at least occasional attention to the others.

My ideas around early July had been
via the “Matt Helm Lounge” on bluray
to write “Reports for Future Academies”
in a nonchalant and cool way
that would not take itself all that serious,
inspired somewhat by those movies,
with an implicit nod to Kafka’s “Bericht für eine Akademie”.
%
It may simply be too much to expect from a lifetime
to find a “theory of everything”.
%
Probably better to promote individual pieces,
while still doing research on the side,
and maybe in time contributions by others
might even speed up the process,
maybe\,?

\begin{center}
\includegraphics[scale=0.09]{i-starrynight.jpg}
\end{center}

\noindent
In a way,
or maybe in many ways,
things are back to how they had been for many years. 
%
It is not sure if there is any major discovery at all.
%
Maybe do more research,
maybe promote some aspects,
or just let things grow,
just do whatever wants to grow at any moment
and don’t worry much,
just have faith\,?
%
Maybe simply fits my Saturn rx in late Pisces,
at a close trine to Mercury slowly rx in late 12 at the AC,
sort of both introverted thinking and, yet,
very universally mirroring life and the world\,?
%
Just relax…

If I somehow managed to recover
the beauty, simplicity and universality of the core idea,
including it being mirrored in modern as well as in ancient considerations
such that they would all be circling around the same core laws of life and nature,
then this might be an incentive to carry on,
otherwise things will probably rest quite a while, Sleeping Beauty Dreaming, of \ELEMENTAL\,?

\subsection{Mysteries}

I bought the 12” MacBook (10,1)
on which I am trying to write \ELEMENTAL\
Wed 15 Aug 2018 at 18:39 at Digitec in Zürich
(time of payment by debit card):

\begin{center}
\includegraphics[scale=0.167]{i-macbook-chart.jpg}
\end{center}

\noindent
Originally “space gray”,
I spray painted it a blend of slightly beige white in late 2022,
witch resulted also in a psychadelic effect around the screen.

\begin{center}
\includegraphics[scale=0.122]{i-macbook-open.jpg}\ \ \
\includegraphics[scale=0.122]{i-macbook-closed.jpg}
\end{center}

\begin{center}
\includegraphics[scale=0.22]{i-psychadelic.jpg}
\end{center}

\noindent
Several events around this tiny Macbook
seem to mirror the emergence of \ELEMENTAL\ out of x$\varphi$ so far;
why remains a mystery,
but the chart…

\subsection{What now\,?}

Well,
I guess \ELEMENTAL\ has the best chance
to eventually reach lots of people with the core new idea.
%
When{\small+}if that would happen,
and in which steps exactly in between,
I do not know.
%
Overall I think x$\varphi$ will continue to rest by changing,
wei chi, but if the little fox…
%
Likely there will also be a pair of books
‘\white{Space} \red{Time} \black{Elements}’/‘\white{Raum} \red{Zeit} \black{Elemente}’
at some point,
and/or also other things,
maybe mainly after I would have gone into pension,
or maybe not.

Reminds me of how the book Solaris ends.

\begin{center}
\includegraphics[scale=0.16]{i-indomitable-squirrel.jpg}
\end{center}

\end{document}
