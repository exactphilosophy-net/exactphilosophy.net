\documentclass[letterpaper]{article}
\pagestyle{empty}
\paperheight=3700mm
\textheight=3700mm
%\textwidth set depending on pdflatex or (experimentally)lualatex
\topmargin=-20mm
\oddsidemargin=25mm

\usepackage[utf8]{inputenc}
\usepackage[spanish,italian,french,ngerman,english]{babel} % last is main
\usepackage{graphicx}
\usepackage{multirow}
\usepackage{xcolor}
\usepackage{contour}
\usepackage{pict2e}
\usepackage{relsize}
\usepackage{amsmath}

\usepackage{iftex}
\ifpdftex
  % stronger fonts

% Find modes.mf, e.g. /usr/local/texlive/2025/texmf-dist/fonts/source/public/modes/modes.mf
%
% $ sudo cp modes.mf modes.mf.orig
%
% Add the following at the start of modes:
%
% mode_def xphi =
%   mode_param (pixels_per_inch, 1200);
%   mode_param (blacker, 1.9); % only difference to 'lexmarkr' (2 there)
%   mode_param (fillin, 0);
%   mode_param (o_correction, 1);
%   mode_common_setup_;
% enddef;
%
% Finally:
%
% $ sudo fmtutil-sys --byfmt mf

\pdfpkresolution=1200
\pdfpkmode={xphi}
\pdfmapfile{}

  \newcommand{\coretextwidth}{85.5mm}
\fi

% experimental, not used to produce the live website...
\ifluatex
  % about same heaviness in pdfs when rasterized in photoshop,
  % but since, unlike the metafont mechanisms I use, fake bold, "bleeds" in all directions,
  % seems heavier at least in core web page images
  \newcommand{\fontbleed}{0.8}
  % paragraphs wider and font looks larger, tried to fix, but then other things change a bit,
  % especially for section headings would have to change back, are now more narrow...
  \newcommand{\fontscale}{0.985}
  \newcommand{\coretextwidth}{85.2mm}
  \usepackage{fontspec}
  % microtype does maybe help and not help, maybe if would allow wider spaces...
  \usepackage{microtype}
  \setsansfont{Latin Modern Sans}[Scale=\fontscale, FakeBold=\fontbleed]
  \setmonofont{Latin Modern Mono}[Scale=\fontscale, FakeBold=\fontbleed]
  % this would be for "new computer modern" (but has many limitations so far)
  %\usepackage[default]{fontsetup}
  %\renewcommand{\familydefault}{\sfdefault}
\fi

\renewcommand{\familydefault}{\sfdefault}

\setcounter{secnumdepth}{-1}

\newcommand{\en}[1]{\iflanguage{english}{#1}{}}
\newcommand{\de}[1]{\iflanguage{ngerman}{#1}{}}
\newcommand{\fr}[1]{\iflanguage{french}{#1}{}}

% a bit less than ~255/256
\definecolor{almostwhite}{gray}{0.996}
\definecolor{xphi}{rgb}{0.0,0.5,0.5}
\definecolor{avant}{rgb}{1,0.5,0.5}

\definecolor{frame}{gray}{0.9}
\definecolor{lightgray}{gray}{0.8}
\definecolor{gray}{gray}{0.5}
\definecolor{darkgray}{gray}{0.3}

\definecolor{darkred}{rgb}{0.8,0.0,0.0}
\definecolor{darkyellow}{rgb}{0.7,0.7,0.0}
\definecolor{darkgreen}{rgb}{0.0,0.55,0.0}
\definecolor{darkviolet}{rgb}{0.5,0,0.5}

\definecolor{darkblue}{rgb}{0,0,0.7}
\definecolor{odyssey}{rgb}{0,0,0.8}
\definecolor{indigo}{rgb}{0.29,0,0.51}
\definecolor{indigoblue}{rgb}{0.1,0,0.6}

\definecolor{saffronback}{rgb}{1.000,0.878,0.627}
\definecolor{saffronfront}{rgb}{0.376,0.125,0.000}

\DeclareRobustCommand{\cometartemisscale}[1]{\includegraphics[scale=#1]{\sourcepath/i-comet.jpg}\hspace{-0.028453em} artemis}
\newcommand\cometartemis{\cometartemisscale{0.018}}
\newcommand\cometartemissection{\cometartemisscale{0.0225}}

\DeclareRobustCommand{\moebius}[1]{\includegraphics[scale=#1]{\sourcepath/i-moebius.jpg}}
\newcommand{\yinyang}{\includegraphics[scale=0.135]{\sourcepath/i-yinyang.jpg}}

\newcommand{\rarr}{\,$\rightarrow$\,}
\newcommand{\lrarr}{\,$\leftrightarrow$\,}

% greek elements
\newcommand{\elfire}{
\begin{picture}(9,6)
  \thicklines
  \put(1,-0.5){\line(1,0){7}}
  \put(1,-0.5){\line(1,1.732){3.5}}
  \put(8,-0.5){\line(-1,1.732){3.5}}
\end{picture}}
%
\newcommand{\elair}{
\begin{picture}(9,6)
  \thicklines
  \put(1,-0.5){\line(1,0){7}}
  \put(1,-0.5){\line(1,1.732){3.5}}
  \put(8,-0.5){\line(-1,1.732){3.5}}
  \put(2.75,1.9){\line(1,0){3.5}}
\end{picture}}
%
\newcommand{\elwater}{
\begin{picture}(9,6)
  \thicklines
  \put(1,5){\line(1,0){7}}
  \put(1,5){\line(1,-1.732){3.5}}
  \put(8,5){\line(-1,-1.732){3.5}}
\end{picture}}
%
\newcommand{\elearth}{
\begin{picture}(9,6)
  \thicklines
  \put(1,5){\line(1,0){7}}
  \put(1,5){\line(1,-1.732){3.5}}
  \put(8,5){\line(-1,-1.732){3.5}}
  \put(2.75,2.5){\line(1,0){3.5}}
\end{picture}}
%
\newcommand{\elhex}{
\begin{picture}(9,6)
  \thicklines
  \put(1,0.5){\line(1,0){7}}
  \put(1,0.5){\line(1,1.732){3.5}}
  \put(8,0.5){\line(-1,1.732){3.5}}
  \put(1,5){\line(1,0){7}}
  \put(1,5){\line(1,-1.732){3.5}}
  \put(8,5){\line(-1,-1.732){3.5}}
\end{picture}}

% i ching trigrams
\newcommand{\trigram}[3]{
\begin{picture}(9,6)
  \linethickness{0.36mm}
  \put(0,5){\line(1,0){#1}}
  \put(5.5,5){\line(1,0){3.5}}
  \put(0,2.5){\line(1,0){#2}}
  \put(5.5,2.5){\line(1,0){3.5}}
  \put(0,0){\line(1,0){#3}}
  \put(5.5,0){\line(1,0){3.5}}
\end{picture}}
\newcommand{\triheaven}{\trigram{5.5}{5.5}{5.5}}
\newcommand{\triearth}{\trigram{3.5}{3.5}{3.5}}
\newcommand{\trithunder}{\trigram{3.5}{3.5}{5.5}}
\newcommand{\triwater}{\trigram{3.5}{5.5}{3.5}}
\newcommand{\trimountain}{\trigram{5.5}{3.5}{3.5}}
\newcommand{\triwind}{\trigram{5.5}{5.5}{3.5}}
\newcommand{\trifire}{\trigram{5.5}{3.5}{5.5}}
\newcommand{\trilake}{\trigram{3.5}{5.5}{5.5}}

% i ching hexagrams
% 1+2 trigrams, 3 rest of line (see e.g. dreams.tex)
\DeclareRobustCommand{\hexagram}[3]{\raisebox{-3pt}{$\overset{\text{${#1}$}}{#2}$\,}#3\vspace{3pt}}

% white-red-black etc.
\DeclareRobustCommand{\outline}[1]{\contour{black}{{\color{white}#1}}}
\DeclareRobustCommand{\white}[1]{\outline{\textbf{#1}}}
\DeclareRobustCommand{\red}[1]{{\color{darkred}\textbf{#1}}}
\DeclareRobustCommand{\black}[1]{\textbf{#1}}
\DeclareRobustCommand{\yellow}[1]{{\color{darkyellow}\textbf{#1}}}
\DeclareRobustCommand{\green}[1]{{\color{darkgreen}\textbf{#1}}}
\DeclareRobustCommand{\violet}[1]{{\color{darkviolet}\textbf{#1}}}
\DeclareRobustCommand{\indigoblue}[1]{{\color{indigoblue}\textbf{#1}}}
\DeclareRobustCommand{\indigo}[1]{{\color{indigo}\textbf{#1}}}

% ELEMENTAL
\newcommand{\ELEMENTAL}{%
\colorlet{contour}{.}\textbf{\color{white}%
\raisebox{+0.001em}{\contour{contour}{E}}%
\raisebox{+0.015em}{\contour{contour}{L}}%
\raisebox{+0.016em}{\contour{contour}{E}}%
\raisebox{+0.023em}{\contour{contour}{M}}%
\raisebox{+0.023em}{\contour{contour}{E}}%
\raisebox{+0.017em}{\contour{contour}{N}}%
\raisebox{-0.020em}{\contour{contour}{T}}%
\raisebox{-0.002em}{\contour{contour}{A}}%
\raisebox{+0.006em}{\contour{contour}{L}}%
}}

% artemis pdf+web icons
\newcommand{\ipdfen}{\includegraphics[scale=0.5]{i-pdf-en.png}}
\newcommand{\ipdfde}{\includegraphics[scale=0.5]{i-pdf-de.png}}
\newcommand{\ipdffr}{\includegraphics[scale=0.5]{i-pdf-fr.png}}
\newcommand{\iweb}{\includegraphics[scale=0.055]{i-web.png}}
\newcommand{\ipdfblueen}{\includegraphics[scale=0.5]{i-pdf-blue-en.png}}
\newcommand{\ipdfbluede}{\includegraphics[scale=0.5]{i-pdf-blue-de.png}}
\newcommand{\ipdfbluefr}{\includegraphics[scale=0.5]{i-pdf-blue-fr.png}}
\newcommand{\iwebblue}{\includegraphics[scale=0.055]{i-web-blue.png}}


\textwidth=\coretextwidth



\usepackage{framed}

\renewenvironment{leftbar}{%
  \def\FrameCommand{{\color{xphi}\vrule width 3pt \hspace{10pt}}}%
  \MakeFramed {\advance\hsize-\width \FrameRestore}}%
{\endMakeFramed}

\begin{document}

\avantgarde

\section{\white{White}-\red{red}-black and the\,\green{“green”}\,goddess}

Some details around my article
\href{https://www.exactphilosophy.net/white-red-black-and-triple-moon-goddess.pdf}%
{\color{xphi}“White-red-black and triple moon goddess\,?”}
of April 2019,
mainly for people not so familiar with these things.

Contrary to the original article this one took longer to write,
is thus less art, less bohème, more bourgeoisie.
%
But, I hope, still a very interesting read.

In any case,
many of the details and arguments shown here
and in the original article
are not so easy to find
and some ideas are genuinely novel, original.

\begin{leftbar}
\noindent
\textbf{\white{White}-\red{red}-\black{black} and triple moon goddess\,?}
\end{leftbar}

\begin{leftbar}
\noindent
A dreamy little tour de force into a quote
by Robert Graves in \textsl{The White Goddess} (1948)
that relates \white{white}–\red{red}–\black{black} to a triple moon goddess:
\end{leftbar}

\noindent
Even though I start with a quote from Robert Graves,
the article is not about Robert Graves,
nor his book,
but about the goddess,
independently of what the article may in the end suggest about her existence.

Two weeks before writing the article,
just after leaving a cinema in Zürich in the early evening of a Saturday,
I had noticed a woman who was walking along,
smoking a cigarette,
which is somewhat rare these days.
%
She looked “hot” in a confident, yet wild, way,
and was wearing a coat that looked gray,
but maybe more likely seemed to be white with many black lines, possibly plaid.
%
Her leggings or stockings were black
and in her right hand she was holding a quite large lacquered red purse.
%
At the handle of the purse
I saw something in black and white,
affixed with some kind of ribbon:
a relatively large question mark “?” in black,
surrounded by a white frame in the same shape.

As I found out the next day,
that evening had been just after Shabbat Parah,
a Jewish holiday,
the Sabbath of the red heifer,
where a heifer is a cow that is able to have children,
but has never been pregnant, yet.
%
That holiday is three weeks before Passover (Pesach),
which is,
like Easter,
celebrated essentially on the first full moon after the beginning of spring
(spring equinox, the day at which day and night are of equal length).
%
In any case,
the moon phase was just a couple of days after waning half moon,
towards new moon.

When the moon is in the shape of a crescent,
the moon goddess Artemis
(or Diana, and her many more names)
goes hunting,
as the crescent of the moon is her hunting bow,
on the ecliptic,
the zodiac,
the circle of her animals.
%
This does,
of course,
not limit her power to only that part of the night sky,
nor anything else,
as I hope to expose a bit in the following.

\begin{leftbar}
\noindent
\textsl{\color{xphi}
I write of her as the White Goddess
because white is her principal colour,
the colour of the first member of her moon-trinity,
but when Suidas the Byzantine records
that Io was a cow that changed her colour
from white to rose and then to black
he means that the New Moon is the white goddess of birth and growth;
the Full Moon, the red goddess of love and battle;
the Old Moon, the black goddess of death and divination.
Suidas’s myth is supported by Hyginus’s fable
of a heifer-calf born to Minos and Pasiphae
which changed its colours thrice daily in the same way.
In response to a challenge from an oracle
one Polyidus son of Coeranus
correctly compared it to a mulberry—%
a fruit sacred to the Triple Goddess.}
(Chapter 4)
\end{leftbar}

\noindent
I verified that this paragraph was already
in the original edition of \textsl{The White Goddess} of 1948.
%
More precisely,
it was already in the first US edition by Creative Age Press of New York;
I could not find a copy of the first edition by Faber and Faber of London,
published 21 May 1948,
to verify that the paragraph was already in there,
but this seems very likely.

The 1948 edition has the same chapters as later editions,
except for two chapters added later on at the end of the book:
\textsl{The Return of the Goddess} and \textsl{Postscript 1960}.
%
It appears that Robert Graves later added some paragraphs in the chapters (and made corrections),
but essentially kept the flow of the original intact,
similarly to what I am doing here with my original text.

\begin{leftbar}
\noindent
I will trace back some sources from the quote above,
then expand a little into quite some directions
and finally come to some conclusions,
which maybe carve out the goddess a bit more clearly than ever before,
or maybe just add to the mystery.
\end{leftbar}

\noindent
It is difficult to enumerate
what exactly would have carved her out more precisely,
but relating her to 3+1 classical elements and their transformations
is probably new in our times,
and there are clearly ancient Egyptian roots,
albeit not with much visible triplicity or matriarchy.
%
This is different in, say, Ireland or Czechoslovakia, until practically today.
%
So,
the goddess might in some sense not be all that universal
in the sense of being “officially” recognized,
now and in the past.
%
More night and shadow than in plain daylight, like the moon\,?

\begin{leftbar}
\noindent
\footnotesize{%
This text is immediately based
on some discussions at the Astrodienst (astro.com) forum in early spring 2019,
which involved quite a few posts by \textsl{*Momo*},
as well as two by \textsl{Sonnenkind} and one by \textsl{Quadrix}.
%
Further back,
this was also related to an earlier and very voluminous thread
colloquially called \textsl{Mondfaden} (moon thread, December 2014 - June 2016)
where further people contributed:
\textsl{anna.} and \textsl{*Momo*} throughout,
and initially also \textsl{Novalis}, \textsl{fünftes Element}, \textsl{Moonman}, and quite a few more\,(!).
%
Going even further back,
this relates to and originates effectively from personal encounters with a woman—who would have guessed\,?
%
Of course,
reading in Robert Graves’ \textsl{The Greek Myths}
and later \textsl{The White Goddess} helped, too.}\end{leftbar}

\noindent
\footnotesize{%
The \textsl{Mondfaden} was called \textsl{Der Mond und die Astrologie (und die Welt)};
thread number with link:
\href{https://forum.astro.com/cgi/forum.cgi?lang=g&num=1419585854}{\color{xphi}1419585854}.
%
I started it 26 December 2014 at 10:24 in the morning (local time Zürich).
%
See the link for a complete list of who contributed what.
%
Similarly for the thread titled \textsl{Weiss-Rot-Schwarz und dreifaltige Mondgöttin?} of spring 2019,
\href{https://forum.astro.com/cgi/forum.cgi?lang=g&num=1554240294}{\color{xphi}1554240294}.
%
Note that there were other threads in the same timespans that also touched related themes at times.
%
Many thanks to all who contributed,
be it publicly with posts or silently by reading, feeling, considering\,!

Maybe I will write down something related to my personal encounters
with the mentioned woman one day,
or maybe not.
%
A tiny little bit will follow later here.

I never read any of the two books by Robert Graves in their entirety,
at least so far.
%
I was clearly coming from Greek mythology
and would often read relatively randomly in \textsl{The Greek Myths}
in summer at the lake, and still do.
%
A first look into \textsl{The White Goddess} followed only in 2015,
but, of course,
indirectly some of its content had already reached me before via various channels,
as his two books had influenced quite a few people.}

\normalsize
\begin{leftbar}
\noindent
Just keep reading and get immersed without knowing.
\end{leftbar}

\noindent
I wrote this mainly because I felt what was coming,
that I would be able to write something beautiful and coherent,
based on recent and past findings,
but not really under my conscious control
in which ways exactly things would arrange,
a bit like (or exactly like) the \textsl{écriture automatique} (automatic writing) of the surrealists,
which has much older roots,
of course.

So,
in a sense,
I myself also got immersed without knowing,
just a bit earlier.

But, again,
this is not about me,
nor Robert Graves,
nor probably about most readers of the text…

Enough bourgeois preamble,
time for more bourgeois explanations around the colors of the goddess.
%
But still fun,
after all to you this might be new.

\normalsize
\begin{leftbar}
\noindent
“Suidas the Byzantine” refers to what is now usually known as the first encyclopedia ever,
written in Byzantium (Istanbul, Constantinople) in the 10th century CE.
%
There is no mention of a cow in that encyclopedia under \textsl{Io},
but the following under \textsl{Isis} (an ancient Egyptian goddess):

\vspace{2mm}
\noindent
\textsl{\color{xphi}
She is [sc.\ also] called Io.
%
She was snatched by Zeus from Argos [Myth, Place]
and he,
fearing Hera,
changed her first into a white cow,
then into a black one,
and then into one that was violet-coloured.
%
After wandering around with her,
he came into Egypt.
%
The Egyptians, then, honour Isis,
and for this reason they carve the horns of a cow on the head of her statue,
alluding to the change from maiden to cow.}
%
(stoa.org, transl.\ Jennifer Benedict)
\end{leftbar}

\noindent
The Wikipedia entry for \href{https://en.wikipedia.org/wiki/Suda}{\textsl{\color{xphi}Suda}} states:

\begin{quote}
{\footnotesize
\textsl{\color{xphi}
The Suda or Souda […$\!$] is a large 10th-century Byzantine encyclopedia
of the ancient Mediterranean world,
formerly attributed to an author called Soudas […$\!$] or Souidas […$\!$].
%
It is an encyclopedic lexicon,
written in Greek,
with 30,000 entries,
many drawing from ancient sources that have since been lost,
and often derived from medieval Christian compilers.
%
The derivation is probably from the Byzantine Greek word souda,
meaning “fortress” or “stronghold”,
with the alternate name,
Suidas,
stemming from an error made by Eustathius,
who mistook the title for the author’s name.}
(retrieved April 2019)}
\end{quote}

\noindent
Be it as it may,
and even considering the rather late date of the encyclopedia compared to antiquity,
the entry is primarily about Isis,
even though Io is closely linked to Isis there,
which certainly makes sense,
considering also other sources.
%
So,
this leads to Egypt,
as Isis is an ancient Egyptian goddess.

Now,
at the time the great pyramids were built,
the constellation on the zodiac rising at the beginning of spring was Taurus,
the bull.
%
This links Egypt
and other cultures that emerged before around 1600 BCE
to bulls,
or also cows and calfs.
%
After that followed Aries,
the ram,
until about 0 CE,
since then Pisces,
the fish.
%
This is due to the precession of the Earth’s axis,
like the slow “wobbling” of a spinning top
besides the fast motion around its axis;
Earth rotates around its axis once every 24 hours,
the “wobbling” is about 26’500 years,
so roughly 2150 years per star sign,
but constellations have various sizes.

Between roughly 1550 and 1500 BCE,
around the time the first star of the constellation of Aries
passed the spring equinox,
symbolically starting the age of Aries,
there was a pretty huge volcanic eruption
on the island now known as Santorini,
which probably caused also damage on the larger island of Crete,
then the center of the Minoan culture,
with a prominent bull cult.
%
That volcano eruption probably
at least contributed to the subsequent downfall
of Minoan culture and influence in the region.
%
Note that “santorini” sounds superficially like “holy bull”,
even though at least at first sight,
the etymology seems to indicate different origins.
%
Crete will come up again later on.

Now to colors.
%
This appears still to be a subject of much active research,
so let me just say that black, white and some kind of red,
which included probably other colors not called red today,
were apparently often the first colors mentioned in history.
%
In a way this would make sense,
as these would be quite fundamental and strong signal colors:
white as very bright like the day,
black as very dark like the night
and red as the strongest signal color,
also used as a warning color by many poisonous plants and animals.

What word exactly would have been used
to describe “red” is often not so easy to find out.
%
Since written color words were usually \textsl{not} placed next to an actual color sample,
linking words to colors often requires to study many text passages,
with maybe some referring to something in nature that has the same color.
%
Or sometimes maybe the text is next to a painting,
then,
with chemical analysis,
the original color can possibly be determined.

If you are interested in colors in history,
see (besides the Internet)
also Earl A.\ Anderson’s book \textsl{Folk-Taxonomies in Early English}.

Robert Graves translated the “red” color as “rose”,
while the above translation by Jennifer Benedict used “violet-colored”.
%
Here is the original Greek, which you can find at
\href{http://www.stoa.org/sol/}{\color{xphi}stoa.org/sol}
if you search for ‘isis’:

\vspace{3mm}\noindent\hspace{0mm}%
\includegraphics[scale=0.38]{i-suda-isis.png}

% Ἴσις: αὕτη λέγεται Ἰώ:
% ἣν ἥρπασεν ὁ Ζεὺς ἐξ Ἄργους καὶ τὴν Ἥραν φοβούμενος μετέβαλεν αὐτὴν ποτὲ μὲν εἰς λευκὴν βοῦν,
% ποτὲ δὲ εἰς μέλαιναν,
% ποτὲ δὲ ἰάζουσαν:
% μεθ' ἧς πλανώμενος ἦλθεν εἰς Αἴγυπτον.
%
% τιμῶσιν οὖν Αἰγύπτιοι τὴν Ἴσιν:
% διὸ ἐπὶ τῆς κεφαλῆς τοῦ ἀγάλματος αὐτῆς κέρατα βοὸς γλύφουσι,
% δηλοῦντες τὴν ἐπὶ βοῦν τῆς κόρης μεταβολήν.

\vspace{3mm}\noindent
The first color would be “leukós”, white,
the second color “melania”, black.
%
I could not find the third color in online dictionaries so far,
but at least it was very likely neither black nor white.
%
Pity, but let me move on.

\begin{leftbar}
\noindent
Well, at first sight no mention of the moon at all,
and a different order of colors:
\white{white}–\black{black}–\red{‘red’} instead of \white{white}–\red{red}–\black{black}.
\end{leftbar}

\noindent
The order of colors is definitely different than what Robert Graves gives.
%
Unless there are different versions
or an error in the online Greek text,
Robert Graves would have misquoted,
except maybe if the source he used had already gotten the order wrong.

And,
yes,
immediately no mention of the moon at all,
instead the goddess Isis transformed into a cow
in sequentially changing colors white-black-“red”.

\begin{leftbar}
\noindent
To be sure,
I checked the words in the original Greek text (also at stoa.org).
%
It really speaks of a cow%
—as opposed to a bull—%
that was first white (color of light),
then black,
then “violet”.
%
Note that relating color words to actual colors is usually difficult.
%
A clear attribution is only possible
when text passages are known in which the described object has a well-defined color,
like e.g.\ a ripe blackberry,
or when the text is next to a painting
for which the original color could be reconstructed from chemical analysis,
or something like that…
\end{leftbar}

\noindent
Looks like I got somewhat ahead of myself
in this more detailed version.
%
Just one more thing to note:
It appears that the word used for “cow”
would also denote a bull or an ox,
but since a female goddess was transformed into it…

\begin{leftbar}
\noindent
Let me simply look at the other reference that Graves makes,
namely “Hyginus’s fable”.
%
It is not clear who exactly Hyginus was
and when exactly he lived,
but most probably the Latin text in its surviving form
is from roughly the 2nd century CE.
\end{leftbar}

\noindent
R.\ Scott Smith writes in \textsl{Apollodorus’ Library and Hyginus’ Fabulae} (2007):

\begin{quote}
{\footnotesize
\textsl{\color{xphi}%
[…$\!$] we know little to nothing about the collection of myths
that goes under the title Fabulae,
about the person who wrote it,
and about its date of composition.
%
For this Latin collection there are further difficulties:
over the course of its existence the original work
has been modified, reorganized, abridged,
and again expanded,
all while suffering mutilation and corruption along the way.
[…$\!$] we are better off speaking
about authors rather than a single author,
and dates instead of a single date.
%
Simply put,
the collection of myths we possess under the name Fabulae
is likely so far removed from the author’s original that,
we suspect,
he would have scarcely recognized it as his own.}}
\end{quote}

\noindent
Makes it,
of course,
difficult to be certain that no important elements
around the colors were added, modified or removed.
%
Note that the original text would apparently have been in Greek
and then translated to Latin.

I am not sure if
“the Latin text in its surviving form is from roughly the 2nd century CE”
is strictly true,
but it might come close,
also since that time (first 2-3 centuries CE)
is the source for so many texts
that preserved earlier texts from early to late antiquity,
while it is often difficult to distinguish
what was added or interpreted then from what is genuinely older.

\begin{leftbar}
\noindent
But even after that things did probably not settle,
as the following excerpt from “Hygini Fabuluae” (transl. Mauricius Schmidt, 1872) shows:

\vspace{2mm}
\noindent
\textsl{\color{xphi}
[…$\!$] cui dixerunt natum esse vitulum,
qui ter in die colorem mutaret per quaternas horas,
primum album secundo rubeum deinde nigrum.}
(Page 115)

\vspace{2mm}
\noindent
It speaks of a calf that had been born,
which changed its color three times in a day,
every four hours,
first white, then red, then black.
%
But, as a footnote in the book shows,
“every four hours” (“per quaternas horas”)
had apparently been added by “Tollius”,
which probably refers to the 17th century Dutch classicist Jakob Tollius.
%
This time it’s Graves’s order,
\white{white}–\red{red}–\black{black}.
\end{leftbar}

\noindent
Note that I do not know
for what reasons Tollius might have made the modification.
%
Maybe just interpretation;
but if so,
based on what reasoning or sources\,?
%
Or did he maybe even have access to a different source of the Fabulae\,?

Mauricius Schmidt’s book is available online,
like many older books that are no longer protected by copyright these days.

\begin{leftbar}
\noindent
Let me look into things so far and where a%
—possibly premature—%
conclusion would lead
\end{leftbar}

\noindent
I am aware that the period is missing in the sentence above.
%
I noticed that already sometime while writing,
but then considered it fitting
that a premature conclusion would not end with a period…

\begin{leftbar}
\noindent
In ancient Egypt,
the sun had three manifestations during a day.
%
The sun in the morning was Kephri,
the scarab,
the sun at noon was Ra,
the primary sun god
and mythologically also the first pharaoh Egypt ever had,
the sun towards noon was Atum.
%
So,
something that changes three times a day,
like the color of the calf,
maybe even in equal timeframes,
which is close to four hours closer to the equator,
as in Egypt,
where the length of a day varies less than in Europe.

In ancient Egypt,
there was also the Apis bull,
already since the first Dynasty,
a bull usually painted with black and white skin,
plus red genitals
and maybe a red blanket over his back.
%
So,
an ancient solar bull cult instead of a female moon goddess\,?
%
Also a red sun at noon might fit,
as several Egyptian gods and goddesses
are shown carrying a red sun disk over their heads.

As it will turn out,
this interpretation is most likely correct,
but probably only confirms the power of the goddess…
\end{leftbar}

\noindent
I do not know the exact primary source(s)
of the attribution of three Egyptian gods for the sun during a day;
I simply repeated what I found in several places.
%
According to Wikipedia,
\href{https://en.wikipedia.org/wiki/Khepri}{\color{xphi}Khepri}
would mean \textsl{“develop”, “come into being”, or “create”},
while \href{https://en.wikipedia.org/wiki/Atum}{\color{xphi}Atum}
would mean \textsl{to complete or finish}
(retrieved April 2019),
which would apparently largely confirm this view.

Independently of the exact duration of a day,
attributing three colors to different “phases” of the sun
during the day seems much more immediate
in a context that seems at first sight Egyptian,
with a calf and the same colors as around Isis.
%
After all,
the sun god Ra is often shown with a \red{red} sun disk over his head,
as well as other ancient Egyptian gods.
%
But then again,
why red for noon
considering that the sun is rather red at sunrise or sunset\,?
%
But some more considerations around that later…

The Apis bull is certainly very interesting.
%
The Ancient History Encyclopedia at
\href{https://www.ancient.eu/Apis/}{\color{xphi}ancient.eu/Apis}
writes the following
(among more interesting facts):

\begin{quote}
{\footnotesize
\textsl{\color{xphi}
There are no myths related to the origin of Apis,
but he is attested to through engravings from the Predynastic Period (c.\ 6000-3150 BCE).
%
[…$\!$]
%
In the Early Dynastic Period,
the ritual known as The Running of Apis was performed to fertilize the earth.
%
[…$\!$]
%
[The bull] had to be black with a white triangular marking on its forehead,
another white marking on its back in the shape of a hawk’s or vulture’s wings,
a white crescent on its side,
a separation of the hairs at the end of its tail,
(known as the “double hairs”)
and a lump under its tongue
in the shape of a scarab.}
(retrieved April 2019)}
\end{quote}

\noindent
A painted coffin footboard (8th to 4th century BCE)
at the same site
shows the Apis bull in black and white,
except red for genitals, the inside of the ears and the snout,
plus a red blanket over his back.
%
The only other color used is yellow
for more stuff attached to the bull.

I am not sure since when
these colors were used for the Apis bull,
but knowing this would certainly be interesting.

One thing caught my attention,
though,
but I just see that I completely forgot to give you a summary
of the fable around that calf.
%
The fable is numbered 136 and called \textsl{Polyidus}.
%
Glaucus,
the son of Minos and Pasiphaë,
king and queen of ancient Crete
(who would have guessed?),
fell into a large pot of honey,
while playing with a ball or maybe a mouse,
at least Robert Graves suggests the latter as a possibility.
%
The parents,
as it was maybe custom in those times,
ask Apollon for an oracle
that would help them to find their missing son.
%
The oracle is quite imprecise,
simply speaking of an omen
that would have been born
and that whoever could correctly interpret it,
would bring back the lost son.
%
That person was Polyidus,
who linked the color change
to the change of color of a ripening mulberry:
white to red to black.

Polyidus does actually find the boy,
but already drowned in the honey.
%
King Minos encloses Polyidus
in the cellar where he had found the boy,
demanding that he would bring him back to life,
as apparently promised by the oracle.
%
Polyidus sees a snake that approaches the boy and kills it.
%
Then a second snake comes
and restores the first snake to life with some herbs.
%
Polyidus does the same with the boy and both are saved.
%
See R.\ Scott Smith’s book for a 1:1 translation of the Latin original; 
it is quite a short fable,
just a few sentences.

Now I can briefly mention what Apis remembered me of:
That the latin name for biological genus of bees is \textsl{Apis}.
%
Although I did not bring this up in the original text,
you may want to relate the golden honey that bees produce to the goddess.
%
Note also that
when Odysseus awoke in the middle of the Odyssey
at his home island Ithaca,
but was not aware he was home,
there were honey pots in the cave he woke up in.
%
But I will not really delve into this here.
%
Let me just say
that there are quite some indications
that Odysseus’ faithful wife Penelope
was probably originally a goddess,
if not rather \textsl{the} goddess.

\begin{leftbar}
\noindent
The mulberry is mentioned just a bit further below in the fable:

\vspace{2mm}
\noindent
\textsl{\color{xphi}
arbori moro similem esse;
nam primum album est,
deinde rubrum, 
cum permatur[a]uit nigrum}
(Page 115)

\vspace{2mm}
\noindent
The “arbori moro” would be the mulberry tree,
most likely the black mulberry tree,
but I guess the white mulberry tree cannot be excluded for sure
(the red mulberry tree can,
because it is originally from North America).
%
Independently of the exact species,
mulberries ripen from white via red to black,
so exactly the order Graves gives,
\white{white}–\red{red}–\black{black}.
\end{leftbar}

\noindent
The Latin colors are quite specific,
album-white,
rubrum-red
and nigrum-black,
except,
of course,
that at some point someone had to translate from Greek to Latin
and could in principle have been imprecise.
%
But,
in any case,
the order is exactly as listed by Robert Graves
and as proposed as the “canonical” order.

R.\ Scott Smith translated “arbori moro” as “blackberry bush”,
which has fruits that are very similar to mulberries,
but I would lean strongly towards a mulberry tree,
also because mulberries grow on trees but blackberries on bushes.
%
But some crude botanical details about mulberry trees first,
based on some surfing on the Internet,
please verify as needed.
%
There are roughly three kinds:
white, red and black mulberry trees,
but the names are somewhat misleading,
as in all cases ripe berries are usually black.
%
The red mulberry tree can be excluded,
as it is native to North America
and the fable certainly dates back
to a long time before Columbus sailed to America.
%
The white mulberry tree is best known
for its use in China to feed silk worms with its leaves,
in order to produce silk,
but the white mulberry was apparently already known
in the Near East sometime in antiquity.
%
The black mulberry tree has the best tasting berries
and was also known in the region at the time,
although it appears to me
that the ancient Egyptians did not know
the white or black mulberry tree in the old kingdom (earliest dynasties),
only sometime later,
but probably already at the time of the fable.

A closer botanical look would certainly be fruitful,
but let me note that there are many kinds of fruit
that ripen from an initial light color via red to black,
although arguably not many of them start out
with a clear white color;
a light green seems more common.
In particular,
the fig-mulberry or sycamore fig
appears to have been know in ancient Egypt already very early on.

\begin{leftbar}
\noindent
Now,
and this is crucial,
the time it takes a mulberry to ripen
from white via red to black
is something like a moon cycle,
possibly more precisely
if you look at the whole tree
instead of a single berry,
possibly not.
%
This links two cycles,
the one of the sun during the day
and one for the phases of the moon.
%
Whichever god or goddess is related to that
in the end has the power
over \textsl{both} of these things,
over both sun and moon,
and possibly more.
\end{leftbar}

\noindent
Let me repeat and emphasize that:

\vspace{3mm}\noindent
\textsl{\color{darkgreen}
Whichever god or goddess is related to that
in the end has the power
over \textbf{both} of these things,
over both sun and moon,
and possibly more.
%
More might include everything else on the night sky,
as well as everything down on earth,
including the sea and the underworld.}

\vspace{3mm}\noindent
That is the basic idea here,
and I guess in the end
also largely of Robert Graves:
A universal deity that is more than “just” the moon.
%
In Greek mythology,
the goddess that stood maybe most directly
for the moon was Selene
(and Helios for the sun),
while the “moon” goddess Artemis was closer to that universal goddess,
or her twin brother,
the “sun” god Apollon to that universal god,
or maybe both, like Yin and Yang.
%
But maybe I am getting a bit too poetic,
yet.

\begin{leftbar}
\noindent
Going back to some facts,
note that Minos,
the king of Crete,
was in mythology one of the three sons of Europe,
who had ridden to Europe on Zeus disguised as a bull.
%
Another son of hers was Midas,
who turned everything to gold he touched.
%
Gold is usually associated rather with the sun than the moon,
since at least around roughly 0 CE,
while the moon would rather be silver.
%
Also,
jewelry and money would immediately rather be Venus,
who traditionally rules the astrological sign of Taurus,
the bull.
%
Then again,
the moon is considered exalted (a good guest) in Taurus,
at 3$^\circ$ Taurus%
—a hidden hint at a triplicity of a moon goddess,
maybe\,?
\end{leftbar}

\noindent
It appears not to be true that Midas
would be mentioned as a son of Europe
(but Minos was, and Europe was a daughter of Io),
so let me not insist here,
even though looking at the mythology,
there seem to be some links to Egypt.

About sun and gold,
resp.\ moon and silver,
I am not sure since when that association is commonly made,
even though I suspect that this is quite old,
at least for the sun and gold,
due to the color of gold.
%
Attributing silver with the moon
might then arguably have been a rather natural next step.

The exaltation of the moon at 3$^\circ$ Taurus is from Ptolemy,
but I am not sure how exactly he meant 3$^\circ$.
%
Did he mean the 3rd degree,
i.e.\ 2$^\circ$00’00” to 2$^\circ$59’59”
or maybe plus/minus half a degree around 3$^\circ$,
or maybe something else\,?
%
A look into primary sources could clarify that.
%
But more importantly,
I have no clue why exactly 3$^\circ$ Taurus and not another degree,
also not for the other exaltations.
%
One explanation could be the beginning of spring in the late Age of Taurus,
if the bull was really mainly associated with the moon then.

\begin{leftbar}
\noindent
In the fable,
the oracle had been cast after the son of Minos and Pasiphaë
had disappeared
and predicted that whoever could interpret it correctly
would save the child.
%
Polyidus actually finds the child,
but drowned in a large jar of honey,
while chasing a ball
(or maybe a mouse, as Graves writes).
%
Polyidus finds the child below ground,
I guess in the very labyrinth
in which later the Minotaur was kept.
%
Minos traps Polyidus down there,
as the oracle is not yet fulfilled.
%
Polyidus sees a snake,
kills it,
then a second snake comes
and reanimates the first one with some herbs,
which Polyidus uses,
in turn,
to reanimate the child.
\end{leftbar}

\noindent
About the reference to the Minotaur:
Is it known what color the thread of Ariadne had\,?
%
Germans today speak of “roter Faden” (red thread)
when they mean the figurative thread in a story,
the recurring theme or evolving story
that clearly defines where it leads.
%
Hmn.
%
Maybe I should return more to the main themes,
or is this exactly the theme\,?
%
Well,
let me just refer to Barbara G.\ Walker
in her book \textsl{The Woman’s Encyclopedia of Myths and Secrets}
under \textsl{Gunas},
the Indian (as in India) strands of Fate in the three expected colors.
%
Or that during the Ramadan
you are apparently allowed to eat in the evening
as soon as you cannot distinguish
a white thread from a black thread in your hand.

Yes,
in principle it is a “bad idea” in a text
aiming to clarify some sources of a partially improvised other text
to introduce new partially improvised associations.
%
But this is me,
and also in some sense how my mind works.
%
Then again,
I am a physicist (Ph.D.),
so I have also a side that is very grounded on facts and logic,
but still flexible while exploring
and which kinds of logic to apply.
%
Associative “female logic” is not nonsense,
not something that should be arrogantly smiled at.
%
It is simply how many things work in the world,
in language and also to some degree in nature overall.

If you have doubts,
read on,
if not,
anyways,
too. ;)

\begin{leftbar}
\noindent
This reminds again of ancient Egypt,
where Ra,
usually with the help of Seth,
defeats the Apophis snake each night below ground
and rises again every morning,
victorious as the morning sun.
%
It also reminds of the seer Tiresias in Greek mythology,
who observed two snakes at sex.
%
In ancient Egypt seeing such a thing
was believed to lead to the “female disease”,
(male) homosexuality,
which makes it clear what part of the male bodies
the snakes often stand for.
%
Apollon killed the snake Python at Delphi,
and Delphi means womb.
\end{leftbar}

\noindent
I cannot find a source for observing snakes and the “female disease” at the moment,
but remember clearly to have read that quite recently
(relative to April 2019).
%
In any case,
the association of snakes with that part of a man makes sense.
%
Note also that in Delphi the Pythia
(from the snake Python)
would sit on a tripod (3)
over an opening of the earth,
the womb of nature,
and that at Delphi there probably used to be
a temple of the earth goddess Gaia.

\begin{leftbar}
\noindent
Well,
again rather an ancient Egyptian sun god
and some later derived myths of the Greeks,
but no moon goddess in sight\,?
\end{leftbar}

\noindent
Well,
yes and no:
Considering Delphi maybe in the past
also at least some earth besides the sun\,?

\begin{leftbar}
\noindent
Let me present a different story first,
and then come back to Egypt later.
%
The Slavic fairy tale of the beautiful Vasilisa
features an old “witch” called Baba Yaga.
%
Vasilisa already lost her mother,
and her step mother and two step sisters are just as mean
with her as in the case of Cinderella.
%
They send her to Baba Yaga to fetch something.
%
While Vasilisa walks to Baba Yaga’s house in the woods,
like Hänsel and Gretel,
she encounters three horsemen.
%
The first horseman is all \white{white},
horse and clothes,
and passes her by just at dawn,
before sunrise,
when the sky starts to get a silver color.
%
The second horseman,
all \red{red},
rides by just a little bit later,
at sunrise.
%
Then she walks all day
and just when the night falls,
the third horseman,
all \black{black}, rides by,
and disappears just at Baba Yaga’s house
into the ground.
\end{leftbar}

\noindent
Baba Yaga gives an explanation herself later on.

\begin{leftbar}
\noindent
Later on,
Baba Yaga answers Vasilisa three questions,
one about each horseman.
%
According to her answers,
they would all three be her servants,
the white one would be day,
the red one the sun,
the black one night.
%
Vasilisa is wise enough not to ask Baba Yaga
about three pairs of hands that help in the house.

Now, what kind of goddess could possibly have six hands\,?

Yes, a triple goddess.
%
The Greek Hekate,
both a goddess of death and a midwife,
like also the moon goddess Artemis,
is often shown as three women standing back to back
(or sometimes just with three heads or faces).
%
But let me ask how you would complete
a sequence of supposedly four things that starts with:
day, sun, night, …?
%
Yes, obviously with “moon”.

As soon as the black horseman disappeared,
the eyes of the skulls on Baba Yaga’s
fence of bones started to glow
so much that her house was lighted as brightly as at day,
which reminds immediately of a full moon.
%
She later kills the step mother and the two sisters by creating a fire.
%
This relates her also to fire,
also in its ability to kill by burning.
\end{leftbar}

\noindent
Note that red for the sun in the middle
might even explain while the sun disk in ancient Egypt would have been painted red,
even though the sun is hardly ever red during midday.
%
Maybe,
just like in astrology,
the sun would be defined by its color at its birth in the morning,
hence often red\,?
%
I don’t know,
but maybe.

See the Internet for statues of Hekate
that are back to back at 120$^\circ$ angles,
and note that at least the ones I have seen so far
would be rather from late antiquity.
%
In any case,
goddesses seem to have appeared in triads
already in ancient Greek mythology,
and the brothers Zeus, Poseidon and Hades also form a triad,
ruling over heaven, sea and underworld.

But back to Baga Yaga
and maybe a crucial new element in this kind of reasoning,
based on my many explorations around the classical elements,
the main topic of my website at
\href{http://www.exactphilosophy.net/}{\color{xphi}exactphilosophy.net}.
%
This is where the actual “carving of the goddess” might really start…

\begin{leftbar}
\noindent
The four elements in antiquity were water, air, earth and fire.
%
Fire was a bit special in that group,
as it appeared to exist both on the earth and in the sky,
as sun, moon, planets and stars,
i.e.\ as the lights (fires) in the sky.
%
Aristotle introduced a fifth element
that would essentially only exist in space,
but not down on earth and move in circles.

Now,
life can only exist in water, air and earth,
but not in fire
(except in mythology the salamander).
%
So could Baba Yaga maybe stand for fire
and the three horsemen for the other three elements\,?
%
Seems a bit far fetched, right\,?
%
But wait:
Antiochus of Athens,
who lived roughly in the 2nd century CE,
attributed colors to the four elements,
as follows:
\white{water-white},
\red{air-red},
\black{earth-black}
and \yellow{fire-yellow}.
%
That would be exactly the colors of the three horsemen,
plus yellow for fire and Baba Yaga!

The colors of the four horsemen of the apocalypse are similar.
%
The first three are \white{white}–\red{red}–\black{black},
in that order.
%
The color of the fourth horseman,
“death”,
is usually translated as “pale”.
%
In the original Greek,
it was \textsl{khlōros},
which stands for roughly for a pale or yellowish green,
not unlikely referring here to the color of a dead person.
%
It is also the root of \textsl{chlorophyll},
which makes leaves green and allows them to do photosynthesis.
%
But back to that soon.
\end{leftbar}

\noindent
So,
a \green{“green”} goddess,
where green would stand for different colors
from pale greenish almost white to golden yellow
or the color of leaves on a tree\,?
%
Yes,
in a way,
that would be the general idea,
if only she would not resist in principle
to have a well-defined color pinned on her.
%
In that sense,
of course \white{white}-\red{red}-black are her colors,
too,
as well as any other color.

But let me reintroduce the moon here,
in a maybe astonishing way!

\begin{leftbar}
\noindent
Remember the idea of a goddess or god
that would govern the cycles of sun and moon\,?
%
That would be Baba Yaga here:
She is the boss over all fires in the sky
and also over all fires down on earth.
%
And also the other three elements (her horsemen) serve her.
%
So she would be all fires
and in command of all that moves,
via the fire (energy) that makes them move.

And she would also be the moon, for the following simple reason:
The sequence of day, sunrise and night
is always stereotypically the same.
%
Nobody has ever seen the sun rise
before the sky started to light up
or after the sun went down.
%
Sometimes the sun is not visible at all,
when hidden behind clouds,
but I guess in places like ancient Egypt
this was quite rare.
%
But the \yellow{moon} can rise at any time,
before or after \white{day}, \red{sun} or \black{night}.
%
Thus the moon is the boss of these three things.
%
That the moon is in the end stronger than the sun
also shows during a total solar eclipse,
where it is the moon that darkens the sun,
not the sun that outshines the moon.
\end{leftbar}

\noindent
Not much to add here.
%
Of course,
the moon can “swallow” all lights in the sky
(astronomically, of course, because the moon is the closest object to earth).
%
This would relate Baba Yaga also to Cronos in mythology,
the god who swallowed all of his children
(except Zeus!)
and who dismembered and killed his father Ouranos
with an instrument in the shape of a crescent,
following instructions by his mother Gaia (earth).

Note,
however,
that once more the women appear to operate rather in the background,
while men “officially” are in the positions of leadership
and do some dirty work.
%
Possibly really myths were rewritten,
but at some point a few concrete examples
of clearly matriarchal myths from very long ago
would be desirable.
%
Fairy tales such as Baba Yaga
might very well still reflect such times,
but could a priori also reflect
later developments or cultural differences.

\begin{leftbar}
\noindent
Baba Yaga’s house is often described as standing on chicken legs.
%
So it is mobile,
it can move,
figuratively in the sky,
like the moon.
%
An ancient historian describes fire sacrifices
of animals to Artemis at Ephesus (now in Turkey).
%
The animals which Artemis hunts, rules and protects,
are the zodiac,
and other constellations in the sky.
%
She,
in turn,
is reportedly,
just like her twin brother Apollon,
from Hyperborea,
a mythical country rather north
“beyond the North Wind”,
so maybe also hinting at the north pole in the sky
around which all else rotates.
%
The two constellations near the north pole are bears,
which fits with quite a few things in the mythology around Artemis.
\end{leftbar}

\noindent
I do not feel inclined to look for sources around Artemis
in detail at the moment,
but this is at least mostly something that is quite widespread.

Now to another core insight,
well,
actually even the one that had inspired me
to start the thread at astro.com in spring 2019.

\begin{leftbar}
\noindent
But let me present the basic idea I have
of the three aspects of the moon goddess and her three colors.
%
They would \textsl{not} simply be the colors of the moon at night,
but rather the colors that make the moon change its color,
as the \textsl{energy (fire)}
that is driving all changes.

The first phase would be the \white{white} goddess making a new moon bright again,
towards a full moon.
%
The second phase would be the \red{red} goddess around full moon,
somewhat before and after full moon,
and what makes the moon pregnant.
%
The third phase would be the \black{black} goddess from sometime after full moon,
making the moon dark again,
towards new moon.

The reason the middle phase would be red
would be the cycle of menstruation.
%
At full moon the seed for a new child would have grown inside her womb,
ripened, like a baby in the full, round belly of pregnant woman.
%
If not getting pregnant,
the seed for the baby would come out as menstrual blood
(and the placenta)
at new moon.

The first phase would also be
a child or a maiden before menstruation,
the second phase a mature woman who can have children,
and the third phase an old woman,
who cannot have children any more.
%
Remember that the encyclopedia entry above saw Isis
as at the transition from maiden to cow\,?
\end{leftbar}

\noindent
Just to make sure the idea got through:

\begin{itemize}
\item The \white{white} aspect of the goddess would be the \textsl{force}
that makes the moon \white{white} (bright) again after new moon.
\item The \red{red} aspect of the goddess would be the \textsl{force}
that makes the moon pregnant (\red{red} blood inside) around full moon.
\item The \black{black} aspect of the goddess would be \textsl{force}
that makes the moon \black{black} (dark) again towards old moon.
\end{itemize}

\noindent
I hope that by now some of the deep logic in these things is becoming clearer,
even though,
I guess,
you should never take anything as an absolute truth around these themes;
there are always different views overlapping to some degree,
for various reasons that can usually not be made 100\% compatible with each other.
%
This is,
by the way,
also why to a woman anything a man can say
can only be a better or worse approximation of the truth,
while coming close is usually rewarded,
but I am maybe getting too poetic again.

A triad of maiden, woman and crone is certainly present in Greek myths,
as well as in other places.
%
The woman who cannot have children,
yet,
the one who can have children,
and the one who cannot have children any more.
%
Three women are,
of course,
also related to the Fates
or also the Graves,
the “gray” old women who share one eye.
%
In ancient Egypt the eye of the falcon God Ra
was one way to see the sun,
which produces the light necessary to see,
or maybe even for the moon to shine.
%
As you can see,
there are arguments pro universal goddess or god,
but maybe still more pro universal goddess\,?
%
But I am getting ahead of the story.

\begin{leftbar}
\noindent
So three phases:
Growth, ripening (or fruit) and withering.
%
The mulberry turning \white{white}–\red{red}–\black{black},
the elements water (the sea),
air (heaven)
and earth (also the underworld),
over which a trinity would have ruled,
in ancient times according to Graves a triple goddess with ever changing member goddesses,
and later the male gods Poseidon, Zeus and Hades, for example.

But where is the \yellow{fire},
the light green color\,?
%
Well,
the mulberry first forms \textsl{catkins},
which are apparently often also slightly bent,
like a reborn moon after having become invisible for a few days.
%
So that color is both death and rebirth,
which is why Artemis and Hecate were midwifes.
%
They have the power to create new life, or not, if they desire.
%
And cats.
\end{leftbar}

\noindent
Since in astrology,
birth is so decisive,
defining the arrangement of the planets in the sky,
which are her children or servants,
it is not astonishing she is placed there,
or rather the other way round,
right at the secret of life.
%
And,
independently of whether men were in the past fully aware of their role in this,
the first phase before a newborn can proceed from child to adult to old person,
is growing hidden inside the body of a woman.
%
The secret is mainly between catkin and fruit,
not so much any more once the fruit is ripening.

I mentioned cats not as able to create life,
although in retrospect I also do not dare to contradict that,
but because cats with their eyes that resemble crescents
and the activity at night, etc.,
are often related to the moon,
and already were in ancient Greece.
%
At least there is one myth
in which Artemis turns into a cat,
but maybe that was borrowed,
like so many things in ancient Greek myths.
%
In Egypt there was Bastet,
while Hathor was maybe closer to the goddess,
or Heqet,
the frog goddess,
which might be the origin of Hecate.

\begin{leftbar}
\noindent
Aristotle put the four elements into a circle,
which they follow when changing from one element to the other:
fire-air-water-earth-fire-…\
%
Yes,
this is not the order of the mulberry.
%
But at least his proposed cyclic nature of the fifth element would confirm this picture:
Fire in the sky,
the fifth element,
would be related to the cycle of four elements down on earth.

After all,
if you have a cycle of four elements that starts with fire,
the fifth element in the cycle is fire again!
%
This is maybe also why Dionysos was first born from fire,
when Zeus had to reveal himself to his mother in his true form,
as lightning and burnt the poor mother Semele to ashes,
who had been tricked by jealous Hera into this.
%
After being woven into Zeus’ thigh,
he was born a second time,
this time not from fire,
but from earth.
%
Then he was cut up into pieces
(similar to Osiris in Egypt)
and cooked in water,
the third element.
%
Finally he was also stricken with madness,
so,
I guess,
his mind reborn from air.
%
That he often wore a lion’s skin,
might relate him to the fifth star sign in the zodiac,
Leo,
the lion,
a fire sign,
just like the first one,
Aries,
the ram.
\end{leftbar}

\noindent
Well,
yes,
once more a man in the center with Dionysos,
although also with clearly female seeming themes.
%
Astrologically,
Dionysos is often related to Pisces,
since at least Liz Greene,
where there is often a mother-son thing going on.
%
The constellation of Pisces is actually two fish,
in Greek mythology the love goddess Aphrodite (Venus) and her son Eros (Cupid, Amor).

Anyways,
I hope the idea that Aristotle sort of split the fire in the world into two fires got through:
Fire down on earth as the regular element Fire,
and fire in the sky that moves in circles
like the planets and the stars
and is round like sun and moon as the fifth element Aether,
while in the circle they become the same again,
except that the fifth element has more experience.

\begin{leftbar}
\noindent
In the scientific article
\textsl{Flowering and fruiting of cv.\ Pakistan mulberry under saline soil conditions in Egypt}
by Ahmed A. El Obeidy (Fruits, vol.\ 60 (6), 2005),
experimental introduction of a special breed of black mulberry on saline soils is described.
%
The fruit of the black mulberry are apparently the best,
but that is not what caught my attention:
“Fruit ripening began in the second week of March
and extended to the third week of April”.

In the Czech tale around Libuše, 
the mythological foundress of Prague,
the queen Niva and her husband Krok had three daughters,
all with magical abilities,
while the prophetic Libuše was the best of them all.
%
Niva is the snow,
winter,
Krok the crocus flower,
the first flower to start growing near the beginning of spring.
%
The life of Niva was tied to an oak tree,
which was guarded by Krok,
so he was her servant.
%
According to the Celtic tree circles found on the Internet,
the oak tree would be the first day of spring
(spring equinox around 21 March).
%
Niva died when lightning (fire) hit the oak tree,
and her youngest daughter,
Libuše,
became queen.
%
That it apparently was the youngest daughter who would follow in reign,
as opposed to the oldest son in patriarchal traditions,
would show that she had absolute control
about how many children would be born,
making the youngest the most gifted.

Libuše used to give council and settle disputes sitting under a linden tree,
which would apparently in the Celtic tree circle
be responsible for two periods of time,
including ten days before the beginning of spring,
which is close to when the mulberries started to ripen in the paper cited above.
\end{leftbar}

\noindent
As it turned out,
the “Celtic tree circle” I was referring to,
is a relatively recent “invention”
of French journalist and director Paula Delsol in the 1970s
for the French women’s magazine “Marie Claire”
(see e.g.\ German Wikipedia).

Yet,
placing the oak at the beginning of spring seems not arbitrary,
nor is the mythological placement of the linden (lime) tree just besides an oak.
%
In the myth of Philemon and Baucis,
at the end of their lives,
she was turned into a lime tree and he into an oak,
standing close-by,
because they had sheltered Zeus and Hermes in disguise in the past,
while all other inhabitants of their village had refused that.
%
This certainly also connects to Prague
with the Gehry building colloquially called “Ginger and Fred”:
two houses that dance with each other.
%
The connection of the oak tree to the beginning of spring
is a bit more complex to make,
but maybe I will here,
and I am quite confident that Paula Delsol
considered quite a few things for the zodiac,
considering that trees are historically important in France and,
last but first,
she is a woman.

So,
I consider the Egyptian scientific paper
rather an omen that points into the right direction,
the beginning of spring,
in the past associated also with oaks in some places.
%
One reason,
by the way,
might be that oaks are said to attract lightning more than other threes.
%
Now,
in the very distant past,
before people knew how to make fire,
practically the only source of fire was when lightning hit.
%
So the trees that attracted fire were probably sacred
(and dedicated to “Baba Yaga”)
and preserving fire probably, too.
%
Spring brings fire again (the sun).

Let me come back to some of this a bit later.

\begin{leftbar}
\noindent
At some point in time,
people no longer wanted a female rulership
and asked Libuše to find a husband
who would then be king and she his supportive wife.
%
She said that they should look for a young farmer
with just one shoe in a certain region.
%
And,
yes,
they found such a farmer,
be it because she actually was prophetic
or because she thought that a poor, young farmer would make a good lover
and probably not be too smart
(else he would have had a second shoe and not be a farmer),
so that she could easily direct him as king.

The lost shoe refers also to the last star sign in the Zodiac,
Pisces,
the fish,
associated with the feet in the human body.
%
It also refers to the sandal that Perseus lost when he helped Hera,
disguised as an old woman,
to cross a river by carrying her on his back.
%
In ancient Egypt,
the dead were buried on the western side of the Nile,
where the sun sets.
%
So,
crossing the river would also be both death and rebirth
by grace of the great goddess.
\end{leftbar}

\noindent
Let me just add
that this is all retold from memory,
like in old times;
feel free to research the original tales,
and so on.
%
All in all,
in Czech culture there is still quite a bit of the goddess,
in my experience.

\begin{leftbar}
\noindent
As Robert Graves also essentially writes,
not long after the initial quote on top of this text,
the single best reference to the “white goddess” is
Apuleius’ \textsl{The Golden Ass},
a Latin text from roughly the 2nd century CE.
%
Before going into some of the content of the book,
let me simply quote how Isis describes herself to Lucius
the night before the beginning of spring
when she appears to him at full moon at the beach,
rising out of the sea,
like Venus in greek mythology
or arguably her “hill” as the first earth
out of the primeval sea Nun in ancient Egyptian creation myths.

A shining disk hovers above her head,
which Lucius interprets as the moon
and thus that she is a goddess of the moon.
%
Her clothes are described in great detail:
\white{white},
\yellow[crocus-yellow]
and \red{red},
plus a \black{black} mantle
on which there are the stars and a full moon,
and flowers and fruit on the hem.

She says this to Lucius:

\vspace{2mm}
\noindent
\textsl{\color{xphi}
[…$\!$]
rerum naturae parens,
elementorum omnium domina,
saeculorum progenies initialis,
summa numinum,
regina manium,
prima caelitum,
deorum dearumque facies uniformis,
quae caeli luminosa culmina,
maris salubria flamina,
inferum deplorata silentia nutibus meis dispenso:
cuius numen unicum multiformi specie,
ritu vario,
nomine multiiugo totus veneratus orbis.}

\vspace{2mm}
\noindent
Impressive,right\,?
%
Oh, you don’t speak Latin\,?
%
Neither do I,
but I read several translations,
including the one by Robert Graves,
and ended up with this translation to English:

\vspace{2mm}
\noindent
\textsl{\color{xphi}
[…$\!$]
mother of nature,
all encompassing mistress of the elements,
first progeny of the times,
highest power/deity/queen,
first/best (sky) deity,
uniform face of gods and goddesses,
who dispenses over
heavenly, shining summits,
salty sea breezes
[and]
the dead down below in earth,
which are silently weeped.
A single/unique goddess in multiple shapes,
with changing rites,
many names,
worshipped all over the world.}

\vspace{2mm}
\noindent
This brings it all together,
the colors and rulership over air (heaven), water (sea) and earth (underworld),
and the moon, as well as the beginning of spring.
%
Why exactly spring,
here and around Libuše\,?
%
Well,
simply because that is again where the goddess lives,
at the point where the cycle both starts and ends,
beginning of the year, new moon and catkins.
\end{leftbar}

\noindent
Yes,
impressive,
quite up to the point,
including timing at the beginning of spring,
and full moon,
which would relate this to Easter or Pesach
or other festivities at that time,
likely at least back to times when Taurus
was the sign rising with the sun at the beginning of spring.

And,
once more,
the approach in term of the classical elements seems also to carry quite far.

Now,
female readers,
please be prepared to be shocked at first
by the crude theatrical pun below.
%
I will explain it in quite some detail subsequently,
but maybe only after you “passed out”
and regained consciousness.
%
Be warned.

\begin{leftbar}
\noindent
Before describing the story of the golden ass a bit in more detail
and relating it to its times%
—in maybe quite surprising ways—%
let me dive a bit into lesser know waters,
quite speculatively,
or so it might appear at first…

Baba apparently simply means an old woman or grandmother,
but there was an ancient Egyptian god Babi or Baba
(the ancient Egyptians only wrote consonants, but not vowels),
a baboon god especially known for his large, red genitals,
which even served as a mast on the ship in the underworld.
%
Like the Apis bull,
Babi was already present in the first Dynasty (before 3000 BCE).
%
There is apparently an image that shows the pharaoh with a white crown
running first in front of a baboon,
then behind or besides the Apis bull.

Sure,
maybe just a coincidence.
%
But then I remembered the Greek Baubo
who showed her genitals to Demeter,
who was weeping about her lost daughter Persephone,
who Hades had abducted into the underworld.
%
This cheered Demeter up,
probably reminded her of her unique power to create life.
%
So,
again a reference to genitals and a similar name.
%
Almost the same story appears also with Hathor
showing her genitals to her father at some point,
where he was angry about the process between Seth and Horus
about who should be pharaoh that took 80 years to settle.
%
This is why daughters even today
still show their genitals to their fathers
when fathers are feeling blue…– just making sure you are still reading attentively,
this is not a serious suggestion from my side.
\end{leftbar}

\noindent
Now,
about that comment about daughters and fathers today.
There is a lot more attached to it than it might appear at first,
in the end maybe even the question
if you are really ready for a less patriarchal society or not.

My assumption was that Baubo was an adult woman.
%
So she is in principle
(within the bounds of contemporary laws at the time)
free to do whatever she pleases,
she does not have to ask anyone for permission.
%
She decided to reveal herself;
she was for all that it appears not forced at all by anyone to do so.
%
Plus,
what she did apparently had a benign effect on the whole situation,
so one has to admit that she possibly had more insight into the situation
than the men involved.
%
Also,
her father did not see this as an invitation for sex,
for all that it appears.
%
(I have not see the original sources,
so I cannot be completely sure about that.)
%
Now,
had Baubo been a teenager or a child,
the situation would have already been different.
%
Then she cannot decide freely,
then her parents are automatically involved,
should decide if that is appropriate.
%
Considering that there is at least in current times still often
abuse by fathers of underage daughters,
the association is of course not far here.
%
And there is also sort of the opposite case,
where daughters have a hard time to mentally separate from their fathers
once they are grown up,
because they had fallen in love.

But let me look at the whole affair from a different angle.
%
As far as I know,
at least the catholic Christian view on menstruation would have been
that women bleed every month as a punishment for Eve having tasted the apple
(resp.\ whatever fruit that was,
more likely a pomegranate or a mulberry, etc.)
and thus Eve and Adam were evicted from paradise.
%
This is the catholic \textsl{original sin},
i.e.\ that being a woman and especially when doing anything sexual makes women guilty.
%
(Then again,
not having everyone running around naked all the time
is also something I personally welcome.)

Now,
if you are really a feminist
and not bound by the \textsl{original sin},
there is nothing “wrong” in what Baubo did,
assuming she was an adult
and doing this on her own initiative.
And,
yes,
the matter does even then remain somewhat ambivalent;
being a woman will,
I guess,
forever remain complicated.

So,
I am not in a position to suggest anything,
just providing a few insights
that may help in some way to find better ways
in the future to deal with Fate.

\begin{leftbar}
\noindent
Now,
Hathor is often a cow goddess,
I guess even was before Isis got that role,
too,
but it is often very hard to tell things apart with certainty
regarding ancient Egyptian gods and goddesses,
maybe because the culture evolved across several millennia,
or also because it is said that they all could transform into each other or,
of course,
into all kinds of animals.

As far as I remember,
there was the notion in ancient Egypt at some time
that the whole sky was a cow,
which would also explain,
where the milky way came from.
%
So,
be it the path of the sun during the day
or the one of the moon at any time,
it would always cross the universal sky goddess as the whole sky.
\end{leftbar}

\noindent
I am not sure where I read that about the whole sky as a goddess,
but maybe it was somewhere in Garry J. Shaw’s \textsl{The Egyptian Myths}.
%
In any case,
the sky as a cow would explain the Milky Way in a natural way,
as being the milk produced by the cow as the sky.
%
See also images with the nude Egyptian goddess Nut as the sky,
in profile (what else?),
happily smiling on all four.

\begin{leftbar}
\noindent
But is it certain
that this was always or originally a cow and not a bull\,?
%
Apis is a bull,
and the sun god Re is male,
too.
%
For example in Theban tomb TT359,
Re is shown as some kind of cat
(but curiously also resembling a rabbit)
in \yellow{yellow},
slaying the Apophis snake in \white{white}-\red{red}-\black{black} with a \red{red} knife.
\end{leftbar}

\noindent
This reminds me of the cover of the music album \textsl{Haus der Lüge} (House of Lie)
by the band \textsl{Einstürzende Neubauten}
({\footnotesize*}1 April 1980, Moon club, Berlin),
which would be in a way a male “bull” variant of the image with Nut as the sky.

That “rabbit” might actually really have made it to become the Easter bunny.
%
See the bunny beheading a man on manuscript
“Royal 10 E IV f.\ 61v” in the British Library,
which is of French origin.
%
Possibly someone in the past misread a depiction of Ra as a cat as a rabbit,
I would suspect.

\begin{leftbar}
\noindent
Well,
maybe the “Gretchenfrage” is a bit different.
%
Women are more cyclic than men
due to their menstrual cycles,
and they are physically rounder,
with their breasts,
and overall more curvy than men.
%
This would relate them more directly than men to things that go in circles,
like lunar phases, seasons, planets in the sky, and so on,
and also more to round objects in the sky like sun and moon.
%
So,
yes,
this would apparently mirror women much more than men.

Now,
this does not automatically mean that women would rule this.
%
In a patriarchal society,
men rule over women,
so why not also about anything female,
like sun and moon,
and all that motion in the sky\,?

Well,
this is maybe also not the question,
rather:
Should men do that,
can they do that\,?
%
I would say rather not,
but,
of course,
they can and should provide input to consider, ideas,
do stupid but loveable things.
%
But,
all in all,
I guess this remains a matter that also causes a lot of pain,
besides also often a lot of fun,
sometimes closely interlocked.
%
But actually I do not feel like I am a good person
to make judgements like these.
\end{leftbar}

\noindent
I hope I got maybe quite close to a truth here.
%
I would assume that women in the past
were maybe not all that more active than today,
also given the different anatomy,
where men are more designed for hunting.
%
But,
and I think this may be true,
women might well have been more respected as equals,
resp.\ superior in some aspects in the past,
and even if not,
this is how it should be now.

This is not a conclusion,
just a personal opinion,
or only a wish.

\begin{leftbar}
\noindent
But back to \textsl{The Golden Ass}.
%
It was written in about the 2nd century CE,
as already mentioned.
%
Now,
since about 103 BCE,
the star that rises with the sun at the beginning of spring
is in Pisces,
when presumably the astrological age of Aries
was replaced by an age of Pisces.
%
The first star of Pisces is Alrisha,
the knot,
and probably also symbolizes birth,
as it is part of the (umbilical) cord
that connects the two fish that constitute the constellation of Pisces.
%
These two fish are usually seen as mother and son,
with old roots.
%
In ancient Greece,
they were Aphrodite (Venus) and her son Eros (Amor, Cupid).

It is also Aphrodite that gives the pregnant Psyche four tasks,
in the fairy tale in the middle of the book,
told by an old “Baba” at night in a cave.
%
The four tasks are clearly related to the four elements
and all have to be solved during a day.
%
See my book \textsl{Elementary Star Signs} for an%
—arguably often quite bourgeois—%
exploration of these tasks
and a model that would describe the 12 star signs of the zodiac
as transitions between the four elements.
\end{leftbar}

\noindent
Yes,
just see my book.
%
And,
if you ever find a hint that relates the four tasks to older Egyptian roots,
I would be very happy to hear about it.
%
So far my only lead is that Seth once held Isis (and Nephthys) hostage
in the spinning house at Sais
and would give her tasks every month.
%
Note that this would again suggest an “identity” of days and months,
as in Apuleius version each task of Psyche has to be completed in a day,
between sunrise and sunset.

\begin{leftbar}
\noindent
Psyche is,
like later Snow White,
the most beautiful woman on earth.
%
Hey,
remember that in Grimms’ version of Snow White,
she is described as having snow \white{white} skin,
pitch \black{black} hair
and \red{red} lips,
related to her mother hurting her finger while sewing
and her blood dropping into the snow,
her window enclosed by a black wooden frame\,?
%
Frau Holle (Mother Hulda) was apparently
(as *Momo* noticed)
also described as travelling across the sky
in a chariot pulled by lady beetles,
which are,
yes,
colored \white{white}-\red{red}-\black{black}.

But,
hey,
why don’t you read \textsl{The Golden Ass} yourself\,?
Graves’ translation is very easy to read
and there is so much more in it than I could ever summarize,
from the theater prank they play on Lucius,
via his transformation to an ass by a woman with magical skills,
his dinner at the place of Artemis/Diana,
and so on,
and so on.
%
Let me just mention where he mainly lived: Egypt.

In any case,
the time frame also screams “beginning of spring”.
%
So,
is all of that maybe just self-confirming\,?
%
That was also a time in which Christian religion
arguably wanted to sort of unify the various pagan cults
into more or less a single deity,
in that case a male one.
%
This was also the time in which the Mithras cult
that included slaying of a bull
was having quite a few followers.
%
So,
is the “white goddess” Isis in \textsl{The Golden Ass},
who actually rather wears black plus other colors,
just a child of her time,
nothing that was before or even much after that\,?
%
Apuleius just one of several priests messiases back then\,?
\end{leftbar}

\noindent
Just to make sure you don’t miss this:
Was Isis in Apuleius’ book maybe simply the female version of monotheism,
possibly rather an invention of that time,
synthetic Pisces,
than something that was once a reality almost everywhere in the world\,?
%
Could be,
but is probably not the whole answer,
see above.

And just below,
of course.

\begin{leftbar}
\noindent
Well,
as always around the great goddess,
or around women,
in general,
the answer is a bit of both,
but quite specifically.
%
She does embody all that is female and which had been there before,
in all kinds of cults,
but possibly none of them combined all of that so synthetically into all stereotypes at once\,?
%
Yeah, sure.
%
But it is true that Pisces are generally seen a sign that is good at synthesis,
as opposed to Virgo,
who is rather analysis,
seen as taking things apart in order to see how exactly they related to each other.
%
Both are considered female signs in astrology,
and there are four more female signs.
%
So,
all in all,
only the carpet of associations is what creates the complete image.

Let me just leave it at that,
just follow all the leads I showed
and copy my approaches to these themes,
which,
of course,
also resemble the ones of Robert Graves somewhat:
Partially careful and precise research,
partially poetic, synthetic intuition.
%
I would add,
that in the end it should all not be taken all too seriously,
because ruling over it all in the end often is.
\end{leftbar}

\noindent
My conclusion is largely all the associations I made,
then hints for more I provided,
the few original new ideas I brought in,
and the few tentative conclusions I suggested,
nothing more,
I guess.
%
I hope this will help in many ways,
and do no real harm.
%
This is not the end of this text,
yet,
because had I written some additional notes and sources,
plus a postscript,
in the original text.

Let me just note that in the end,
as I will just show,
the similarity with \textsl{The White Goddess} was partially only a temporary phenomenon;
I will likely not continue to collect fact after fact about these things,
even more so since modern technology plus many people researching can do a better job than me.

In that sense,
my main contribution is rather related to \textsl{elementary} things.
%
Besides,
I am doing this in my spare time,
nobody is paying anything for this.

\begin{leftbar}
\noindent
Adliswil, 14 April 2019,
Alain Stalder
\end{leftbar}

\noindent
I am writing the first version of this text a bit more than a week later,
with Uranus still in the 3rd degree of Taurus,
but read on first.

\begin{leftbar}
\noindent
\textbf{Notes and Sources}

\vspace{2mm}\noindent
This text was written on the late afternoon and evening of 13 April
and the relatively early morning of 14 April 2019.
%
Similarly,
Graves wrote \textsl{The White Goddess} in a very short time,
starting sometime in early spring.
%
In 2019,
Mercury went retrograde around 5 March at the end of Pisces,
and is only in a few days going to return to that spot,
so this gives me some time to collect a few more facts and sources here. 
\end{leftbar}

\noindent
I can only say that writing the original text
and also very often the online discussions up to that
were a \white{lot} \red{of} \black{fun},
really satisfying,
I would hope I could write more like this.
%
In contrast,
this secondary text was already less fun,
partially still felt very good,
but often already smelled more like work.

\begin{leftbar}
\noindent
Robert Graves’ northern lunar node was in the middle of Pisces,
which may explain his approach.
%
This is also where Neptune is now,
or my natal Lilith
or maybe the sun of someone implicitly mentioned before.
%
My Saturn is in the last degree of Pisces,
where Mercury had gone retrograde,
as already mentioned.

Did you know that Robert Graves was briefly a professor in Cairo (Egypt)
between January and about June 1926\,?
%
Along with his wife at the time,
Nancy,
and their children,
the young poet Laura Riding came along.
%
In 1929 he separated from his wife,
wrote \textsl{Goodbye to all that} in a few weeks
and then the whole story with Laura Riding took really off,
which arguably was a main source for \textsl{The White Goddess},
and so on,
not to forget Beryl
(or Rosaleen).
\end{leftbar}

\noindent
Robert Graves’ life and especially also the life and work of Laura Riding
could certainly reveal more,
as well as,
of course,
other women, etc.\
in his life around \textsl{The White Goddess}.
%
I guess the avantgarde impulse that lead to the book
was besides his Irish roots not least due to Laura Riding.
%
Or more.
%
Then again Beryl reminds of “berry”,
and again so many women’s names are related to plants,
like Rosaleen,
or Laura to laurel:
according to Robert Graves in the past
chewed for its effect
and then became the crown of victors
in contests and of kings and emperors.
%
Laurel has also berries that are black in the end.
%
And a crown also reminds of the corona during a total solar eclipse,
when the moon totally swallows the sun,
except maybe the crown for the next king.

Francesco Petrarca,
who is often seen as one of the first voices of the Renaissance,
had a similar birth chart as Robert Graves (or me)
and some day he met a woman called Laura
who became his lifelong inspiration,
even though they never had a relationship.
%
I guess you cannot have a relationship with the goddess anyways.
%
(And no, in my case her name was not Laura at all.)

Let me just copy the rest of the notes and sources.

\begin{leftbar}
\noindent
Well,
for sources,
just ask Google.
%
Some information is,
of course,
from Wikipedia.
%
Did you know that Jimmy Wales,
one of its founders,
was born the same day
(in the same year)
as me\,?
%
The notion of offering all this for free and without ads
is something that I can relate well to.
%
But still,
of course,
it would also be nice to mention those sources explicitly.
%
Conversely,
I had more than once the impression
that some information in Wikipedia
had been obtained from original sources
that I had possibly mentioned first on the Internet and others had googled
and then worked into Wikipedia.

Hey,
this has nine pages,
an Ennead\,?
%
Uranus is in Taurus,
which might explain all the bull/cow symbolism,
\white{white}–\red{red}–\black{black}\,?

Thanks a lot for reading!
%
And see my website
(\href{https://www.exactphilosophy.net/}{\color{xphi}exactphilosophy.net})
and the Astrodienst
(\href{https://forum.astro.com/cgi/forum.cgi?lang=g}{\color{xphi}astro.com})
forum for more,
if you want.

Frozen except possibly tiniest formal fixes
Sun 14 April 2019 near noon.

\textsl{Egypt in Transition:
Social and Religious Development of Egypt in the First Millennium BCE},
eds.\ L.~Bareš, F.~Coppens and K.~Smoláriková, Prague, 2010.

\end{leftbar}

\noindent
I am pretty sure,
the text will not be modified any more,
also since I wrote this,
so that if someone wants to criticize the original text (or this one),
I can reply maybe here,
but hopefully will not have to touch the original document.

The Czech book mentioned at the bottom
possibly contains quite a few hidden jewels
that would not be easy to find otherwise,
which is why I listed it.

\begin{leftbar}
\noindent
\textbf{Postscript 18 April 2019}

\vspace{2mm}\noindent
A few days later (17 April),
I researched when Robert Graves started to write \textsl{The White Goddess},
mainly in Richard Perceval Graves’ \textsl{Robert Graves and the White Goddess},
near the end of chapter 4.
%
According to that,
he would have started sometime after Easter 1944,
which was 9 April,
and after “making serious headway”
with his maps for \textsl{The Golden Fleece},
but “could not get out of his head some lines from Alun Lewis’s last letter”
(died 5 March).

What would be a better moment to start a book about the white goddess,
if not a new moon,
the beginning of a new cycle,
with the first part her own\,?
%
New moon was Saturday 22 April at 22:43:23 in Galmpton (Devon).
%
The new moon was at 2$^\circ$\,34’\,41” Taurus.
%
Now,
the 3rd degree of Taurus is where,
according to Ptolemy,
the moon is exalted.
%
Earlier that day (around 2 AM),
Mercury had gone retrograde for almost three weeks;
Robert Graves writes that he wrote the first version in about three weeks.
%
An archetypal female lunar cycle would be three weeks of activity,
followed by one week of reorientation.

That is why the three colors of the cow came up,
and,
Hermes (Mercury) stole 50 cattle from Apollon in the \textsl{Homeric Hymns}
by walking backwards,
just like Robert Graves in the book.
%
There would be more details,
but…wow!

\vspace{2mm}
\includegraphics[scale=0.2]{i-new-moon-the-white-goddess.jpg}

\noindent
Uranus is in the 3rd degree of Taurus
since last Saturday 13 April at 19:45.
\end{leftbar}

\noindent
Thanks a lot for reading!
%
And hopefully this will be helpful in some ways.

Yes,
it would be possible to write on and on about this,
but I prefer to do this not all to myself,
except punctually.
%
My focus as a physicist is mainly to provide fundamental new ideas,
also related to such topics,
and I hope I did.

Otesánek,
Valley of the Bees (Údolí včel),
Torchbearer (Světlonoš),
Morgiana,
Marketa Lazarová,
Adelheid,
Lunacy (Šílení),
and more Czech movies…

Ah, trees.
%
The Czech word for the month March means “birch”,
a tree that is black and white and bleeds red resin.
%
The Danish word for the same month is related to Frau Holle
(Holunder is the elder tree, again black berries),
see also Jacob Grimm’s book about German mythology.
%
Or clover as a symbol for the trinity;
male as in Christianity,
or with the heart shaped leaves
as some round parts of the bodies of the triple goddess,
a bit like the “mulberry”
around the chest of Artemis at Ephesos.
%
The Sabian Symbol for 3$^\circ$ Taurus (2$^\circ$01’00” - 3$^\circ$00’59”)
is “steps up to a lawn blooming with clover”;
not bad,
Elsie Wheeler and Marc Edmund Jones,
for a single day in Balboa park (trees!) in San Diego.
%
Are Balboa, Baucis and Baubo related to the German word for tree, “Baum”\,?
%
Maybe.
%
Keep going,
if you like.
%
Good luck in any case\,!

The end was kind of fun again. :)

And,
maybe unavoidably,
largely written in just one day.

I guess,
I would simply write another text,
if ever necessary.

\vspace{3mm}\noindent
Alain Stalder (*1966)

\newpage

\subsection{Later notes}

\small
\begin{list}{$\bullet$}{\setlength{\leftmargin}{10pt}}

\item
Ptolemy only writes that the moon is exalted in Taurus,
but does not indicate a degree.
%
Vettius Valens,
his somewhat younger contemporary in the 2nd century CE,
does,
in \textsl{Anthology}, book 3, chapter 4:

\vspace{1mm}\hspace{3mm}\includegraphics[scale=0.22]{i-valens.png}

\vspace{1mm}
Note that the number of the degree is described as \textsl{tritēn},
which is,
I guess,
\textsl{tritos},
third,
as opposed to \textsl{tría},
three.
%
So he might actually say ‘2-3$^\circ$\,Taurus’,
as opposed to ‘around 3$^\circ$ Taurus’.
%
The origin of exalted signs and degrees is unclear.

\item
The following passage in Empedocles’ poem \textsl{On Nature}
is often considered the first mention of four elements in history:

\vspace{1mm}\hspace{3mm}\includegraphics[scale=0.16]{i-empedocles.jpg}

\vspace{1mm}
Would this maybe list the three aspects the goddess in her colors,
followed by the goddess herself,
on her own line:
Zeus as white gleaming (\white{white}, day),
Hera as pregnant and life bringing (\red{red}, sun),
Hades (\black{black}, night),
and Nestis (moon)\,?

Note that the \textsl{The White Goddess} is subtitled
\textsl{A Historical Grammar of Poetic Myth}
and starts with
“Since the age of fifteen poetry has been my ruling passion
and I have never intentionally undertaken any task or formed any relationship
that seemed inconsistent with poetic principles […$\!$]”.
%
The book is also quite a bit about trees (and their alphabets),
which would relate well to Empedocles
using the word \textsl{root} for the four things he lists.
%
So,
apparently a similar context.

Could the poem also help
to clarify the relation between the ancient Egyptian divine siblings
Osiris (black?), Seth (red), Isis (white?) and Nephthys (color?), etc.\,?

\item
The Greek names of the colors of the Four Horsemen
of the Apocalypse in the bible (Book of Revelation) are
\textsl{leukós}, \textsl{pyrrós}, \textsl{mélas} and \textsl{chlōrós}.

\textsl{Leukós} and \textsl{mélas} would apparently have come very close
to what is today called white and black,
respectively,
but \textsl{leukós} also includes the notion of a shining color,
of light, of brightness,
and similarly \textsl{mélas} includes the notion of darkness.
%
\textsl{Pyrrós} is literally the color of fire
and apparently could mean red as well as yellow,
and,
I presume then possibly also colors in between,
like orange.
%
This would apparently come quite close to Baba Yaga’s description
of the three riders as day (bright, white), sun (red/orange/yellow) and night (dark).
%
As mentioned,
\textsl{chlōrós} would apparently have been something
roughly been light green and light yellow,
with also the notion of shining,
which would apparently fit the moon quite well.

This seems to imply that in ancient times
the sun would not have been recognized as what makes it day,
but instead as simply rising shortly after the day began.
%
The sun would have been just one of the many bodies
that move across the sky,
as the moon can also be seen at daytime,
as well as some planets around sunrise and sunset.
%
This view is also not entirely wrong,
as night and day are also due to the rotation of earth.

Since the moon can temporarily “swallow” all other planets,
even the sun during a total solar eclipse,
it might not be far fetched to link
the deity that is at the origin of all motion in the sky with the moon,
also since the phases of the moon
might seem to imply full control over light and dark,
thus also over day and night.

\item
The three colors in ancient Egypt may have essentially been as follows:

\vspace{1mm}
\mbox{\ \ }\white{Isis white} – The water of the Nile, milk\newline
\mbox{\ \ }\red{Seth red} – The desert west and east of the Nile valley\newline
\mbox{\ \ }\black{Osiris black} – The fertile Nile valley
\vspace{1mm}

In the prehistoric past,
the earth of the Nile valley (Osiris)
would have been fertile after the annual Nile flood (Isis),
but would have dried up afterwards.
%
The desert (Seth) would have killed the vegetation (Osiris)
and also broken up the ground into a mosaic when drying up,
as in mythology Seth killed and dismembered Osiris.
%
The return of the Nile flood (Isis) brought back life,
as water and new black sediments,
just as Isis brought Osiris back to life with her “magic”.

I guess,
as soon as people learned how to irrigate the land,
drying up became less of an issue,
but the new sediments every year were still required as fertilizer.

This is certainly a somewhat incomplete picture,
for example without the “sun/falcon” as Ra or Horus,
and so on.
%
All in all,
there appear to have been many variations and transformations
between deities in ancient Egypt over the millennia.

\item
See this absolutely stunning article
by the Ethiopian “Shakespeare”, Tsegaye Gabre-Medhin:
\textsl{The Origin of the Trinity in Art \& Religion: Ethiopian Roots in the Egypto-Greek \& Hebrew},
on page 99-120 of
\textsl{African Origins of the Major World Religions},
ed.\ Amon Saba Saakana, Karnak House, 1988.

The trinity as KaBaRa
(in Ethiopia also HaBaSha and KaBaSa),
with Osiris-Ka and Isis-Ba
and Egypt as Kamit (black land),
as well as the trinity
“like a single sacred tree (like the Adbara or Baobab)”,
and as roots of the Kabbalah (KaBaRa)
or Osiris as Moses and others in similar myths%
—to just mention a few jewels in the article.

Reminded me also of the song “Shakara” by Fela,
with Sha maybe related to the god Shu (day?).
%
At a concert in the 1980s in Zürich,
Fela spoke first at least 10 minutes about colonial influences related to Greenwich,
which he called also “green witch” then.
%
He had come with his band at the time,
Egypt 80,
plus his about 50 women,
who acted also as singers and dancers,
usually in unison,
more like a choir.
%
I even found a baby basket in white-red-black called “Shakara” online today,
advertised with mention of Moses,
which leads all to the themes of “Artemis/Hecate”,
including theater, midwife, and so on.

All in all,
I guess Robert Graves and Laura Riding picked up that “beat”
during their relatively short stay in Egypt.

Let me call this implicitly part three of this “ode to the goddess”;
anything more would maybe follow in a new section here,
if absolutely necessary.
%
Today is 30 April 2019,
the last day of this period with Uranus in the third degree of Taurus.

Let this suffice,
respect the goddess,
I would say,
close to new moon…

\end{list}

\newpage

\normalsize

\subsection{Meta review}

Do you known the story of Zhuangzi and a friend
contemplating fishes in a river
and him concluding they are happy
because he is while watching them\,?

Writing this document%
—and the one it cites line-by-line—%
last year with Uranus in the third degree of Taurus
was probably the most fun and satisfaction
I ever had writing something for the public,
and even privately only surpassed by a series of messages
about in a way the same themes,
about eight years earlier with Uranus in early Aries.
%
I have no certainty whatsoever to what degree the things I wrote
would be “true” in any particular sense.
%
Cynthia Eller wrote a book called \textsl{The Myth of Matriarchal Prehistory},
plus another book about men in history attached to the myth.
%
A myth is something that people continue to tell each other over generations
because it feels important and deeply needed to do so.
%
People did not keep doing so
because they consciously knew it was “true” for this or that reason.
%
Men used to tell this myth,
then feminists,
and now in a way even me as a physicist
with likely one of the broadest and deepest overviews of these themes
in the time I am living in.
%
That must amount to something,
in some way,
independently of whether I or you may be able to grasp “why” in any analytical way.
%
Not long after writing the previous pages of this document,
I assembled the ‘definitive version’ of my writings at exactphilosophy.net
in the third book about it,
\textsl{exactphilosophy.net 2019}.
%
After that I felt a lot of drive gone that had sort of ‘pushed me’ in public areas during my life.

Today 25 January 2020,
Uranus is again back in the third degree of Taurus,
actually since Friday 13 December 2019,
which is also when my mother started to die and did so about three weeks later.
%
In retrospect things probably evolved maybe as smoothly and harmoniously
as one could ever hope for in such situations,
at least I feel thankful despite the loss.
%
That is not to say
that all this is directly related to my mother (also not astrologically),
even though it must have been important also to her,
also since my MC is in Aries with the moon there.
%
In psychological astrology
the MC is most often the wish of the mother,
and the moon normally stands for the mother.
%
Then again,
moon stuff is usually also a collective thing,
and the one I mentioned many times has the moon in Aries,
too,
feels to me most likely close to my moon
and then likely closer to my MC than my own moon.
%
Plus there were and are more women.
%
Thank you all.

By the way,
the name Cynthia,
as in Cynthia Eller,
is a surname of Artemis,
related to the mountain on the island Delos
where she and her twin brother Apollon were born by Leto.
%
And in Chinese Eller is,
I suppose,
written the same as Alain,
related also to the Artemisia plant in Chinese.
%
But who knows…

\begin{center}
\includegraphics[scale=0.13]{i-squirrel}
\end{center}

\section{\violet{Epilogue}}

\renewenvironment{leftbar}{%
  \def\FrameCommand{{\color{darkviolet}\vrule width 3pt \hspace{10pt}}}%
  \MakeFramed {\advance\hsize-\width \FrameRestore}}%
{\endMakeFramed}

The Suda entry for Io
(quoted early on)
listed the changing colors of the cow
in the order \white{“white”}-\black{“black”}-\violet{“violet”}:

\begin{leftbar}
\noindent
The first color would be “leukós”, white,
the second color “melania”, black.
%
I could not find the third color in online dictionaries so far,
but at least it was very likely neither black nor white.
%
Pity, but let me move on.
\end{leftbar}

\noindent
Yesterday,
I found a hint at that color “violet”.
%
The Greek word in the Suda was \includegraphics[trim=0mm 10mm 0mm 10mm, scale=0.06]{i-iazousan.png}
and I found dictionary entries for \includegraphics[trim=0mm 10mm 0mm 10mm, scale=0.06]{i-iazo.png}
in \textsl{Handwörterbuch der griechischen Sprache} by Wilhelm Pape, 1914 (online at zeno.org).

This gives two meanings,
“to speak Ionian, to act like an Ionian”
and “be the color of a violet, to shimmer in dark blue”.
%
Ionian is,
of course,
not far-fetched for a dictionary entry around Io.
%
Note also that the word \includegraphics[trim=0mm 10mm 0mm 10mm, scale=0.06]{i-iazo.png}
reminds me of \includegraphics[trim=0mm 10mm 0mm 10mm, scale=0.06]{i-iason.png} (Jason)
and thus of the map of the Argonauts’ journey that Robert Graves was drawing
shortly before starting to write \textsl{The White Goddess}:

\begin{leftbar}
\noindent
According to that,
he would have started sometime after Easter 1944,
which was 9 April,
and after “making serious headway”
with his maps for \textsl{The Golden Fleece},
but “could not get out of his head some lines from Alun Lewis’s last letter”
(died 5 March).
\end{leftbar}

\noindent
I have not investigated much further,
but feel free to do so.
%
In any case,
the Suda is a relatively recent source,
so things \textsl{might} have been mixed up.

The color \violet{violet} may also be related to the image of mulberry,
as its juice colors skin violet,
a color human skin also takes on after a person died
(except where the body touches the ground).
%
Violet would thus be related to death,
of humans and figuratively of mulberries.
%
This may even relate to the purple robes of ancient rulers
who might thus have been marked by the goddess,
if thinking in the line of Robert Graves in \textsl{The White Goddess}, etc.

Ancient cults around trees
might have involved basic alcoholic beverages made from berries.
%
In ancient Greece (red) wine diluted in water was believed to
make less drunk if drunken from a cup made of \violet{amethyst}.
%
This relates also to Dionysos,
the god of wine,
who was cut to pieces,
similarly to Osiris.

To wrap things up,
let me just note that there were way enough coincidences
around writing this document
to make this a habit
(c.\,f.\ the respective speech Robert Graves gave in 1957 in New York).
%
So,
all in all,
there must be something to all of this.
%
What exactly is not entirely clear to a conscious reasoning mind,
but all in all most likely “the moon” would still fit symbols best.

And don’t miss my work on
elementary star signs,
collective unconscious beings
and the 4+1 elements in terms of space (in/out) and (rest/move),
nor Jack Daw’s upcoming book \textsl{Elemental} that is bound to wrap it all up.

\vspace{2mm}\noindent
Adliswil, 12 February 2020

\vspace{3mm}\noindent
{\footnotesize
A week later sadly (to me) my father freely joined my mother on their last journey.

This relates maybe,
due to the birth chart of my father,
quite a bit to \violet{purple} as the color of Zeus and rulers in general.
%
You might also want to look at the corrected Combin\,/\,Davidson chart for my parents
(Suzy 25 April 1939 00:30 Couvet CH, Romeo 1 May 1936 23:35 Bern);
note e.\,g.\ the AC in the middle of Leo and Saturn at the end of Pisces in the 9th house.
%
And with slightly different birth times than in their BCs,
as maybe not unlikely in the case of my mother,
their combined MC might even have been in the third degree of Taurus…}

\newpage

\section{Closing Circles}

The Chandogya Upanishad,
one of the oldest Indian Upanishads (around 700 BCE),
lists three colors “of fire” at 6.4.1,
in the order red-white-black,
and attributes them apparently to fire, water, earth,
in that order.
%
Now, the word used for red,
\red{rohitam},
also denotes a female red deer,
as well as Rohini the red star Aldebaran,
one of the eyes of the bull in the constellation Taurus.

In ancient Greece,
deer were sacred to the moon goddess Artemis,
and the first version of Robert Graves’ \textsl{The White Goddess},
the one he wrote after that new moon in the third degree of Taurus in 1944,
was titled \textsl{The Roebuck in the Thicket}.
%
Sounds also quite Egyptian,
with Orion and Taurus and Pleiades close in the sky,
and related to spring in the Age of Taurus,
when the pyramids where built
and the Pharaoh used to run with the \white{white}-\red{red}-\black{black} Apis bull in spring.
%
In a triad Isis-Seth-Osiris
the only woman would be Isis and she would stand for white,
so the final title of Robert Graves’ book would appear accurate.

Still no proof, nor anything like complete clarity here,
but a lot points towards Taurus, moon and very ancient rather female roots…

I am aware that this is in some sense an incomplete and maybe unsatisfactory ending to this text.
%
It makes no sense to try to outrun all of the world’s historians here and now%
—time can do that much better than me.
%
Gives me more time for fundamental research,
which may please the goddess even a bit more…

Hail \white{Ar}\red{te}\black{mis}\,! {\scriptsize $^+$}\includegraphics[scale=0.02]{../i-comet.jpg}

\newpage

\section{Bonus: Simple Narrative}

Fire made a big difference to our human ancestors in the very distant past.
%
It brought light and warmth to the night,
as well as grilled and cooked food.
%
It has even been suggested that grilled meat allowed early humans to develop larger brains,
hence fire would have brought more intelligence.

The colors of a fire are the black of coal, the red-orange-yellow of flames and the white of ashes.
%
Presumably fire was first obtained where lightning struck and caused a tree to burn, or something similar.
%
Trees that seemed to attract lightning more than others may have been sacred.

Maybe even more if they carried berries that ripened in the colors of fire,
white-red-black,
like mulberries,
or other berries.

Sun and Moon, as well as planets and stars were likely associated with fire from early on,
simply because they shine.
%
And their boss may have been the sun,
because it shines most strongly,
or the moon,
because it can shine during day and night,
and even darken the sun during a total solar eclipse.

Possibly more likely the moon,
in the end,
even though it appears that in ancient times people where initially not consciously aware
that the moon causes solar eclipses,
as the moon is invisible then,
around new moon.

Why the bull and Taurus, and more, is harder to say.

(This simple narrative may be wrong,
of course,
but might still be useful as a starting point for looking further.)

\newpage

\section{Bonus: Relax}

The world is complex, or at least seems complex.

Simple things like colors and other first experiences like mother and moon are hard to ever analyze.

Simply because doing so would require more abstract concepts that naturally come after the basic ones.

\vspace{3mm}\noindent
{\footnotesize
If I had been a girl, my parents would have named me \indigoblue{Indigo};
(Alain 7 August 1966 04:12 Zürich), with the idea of probably more ultramarine than violet;
I guess for some things you never find out the reason(s).
%
They both had the sun in Taurus.
%
Looking back at Chinese ink paintings I made in early 2019,
in retrospect one could already see that their end was coming.
%
I was just too afraid to see it,
I guess;
at least I wrote this text then.
%
They were both so dear caring parents.
%
Hmn, dear and deer…\
%
So sad.\par}

{\footnotesize
Almost nobody ever read this document as I am adding this note,
but that will come,
and many people will love it,
I hope.
%
Just to have made things worth doing in a way.\par}

{\footnotesize
But relax.
%
And create a few beautiful things. Any color ;)\par}

{\footnotesize
(19 March 2020:
I guess “Artemis” was the big theme in the 2010s,
which is also why this seems to have come to a conclusion in that form
when the 2010s ended and the 2020s began,
while, of course, in a way,
the theme is omnipresent in every time.
Take care.)\par}

{\footnotesize
(Overall the theme of \white{white}–\red{red}–\black{black} seems to be more female than male,
but maybe that was, partially is, and maybe will be by definition mostly \green{unconscious};
note the pun on \green{green} (khlōros),
so prominent in nature and, yet, so late in languages.
%
So maybe also this article should have included triple moon goddess in its title?
%
Moonly does.)\par}

\vspace{-0.1mm}
\begin{center}
\includegraphics[scale=0.017]{i-tetrahedron.jpg}
\end{center}
\vspace{-0.24mm}

{\footnotesize
(Late April 2020,
some time after a new moon at 3$^\circ$24’ Taurus.
%
New findings,
every cycle brings new ideas and insights or experiences or expressions.
%
No color(s) without light,
hence initially white (bright) and black (dark)
plus the color(s) of light resp.\ of a flame (red-orange-yellow),
water-white and earth-black because water is transparent and earth intransparent,
which would all,
by the way,
be interesting insights into how archaic minds worked.
%
Menstrual blood is often already a bit oxidized when seen,
which would match the somewhat darker red
that was apparently meant with ‘red’,
and, of course,
relate to the moon.
%
See the main content of my web site for more,
in your time,
Jessica Hemming’s articles mentioned there,
which provided part of the inspiration for the above new findings.
%
But first still someone has to read this%
—likely for quite some time to come,
maybe many moons, nobody has…)\par}

{\footnotesize
(Deer would be sacred to Artemis because their antlers are like fire.)\par}

\vspace{2mm}
\noindent
{\scriptsize
\textbf{Postscript}
In April 2021
I heard of a woman in Germany
who had lost all sense of smell/taste already at Xmas 2019,
slept practically three days,
no fever,
no other symptoms;
and there appear to have been at least a few more such cases.
%
So my mother possibly got something similar to {\tiny COVID-19}
already 13 Dec 2019,
since at least later waves propagated essentially from South to North.
%
She hardly ate or drank anything,
had no visible symptoms,
no fever.
%
She just said she was not hungry at the moment and would eat later.
%
In retrospect this could have been related to a loss of smell and taste as with Corona.
%
Diagnosis had been a bacterial infection of the brain stem,
but I am not sure if ‘bacterial’ was specifically confirmed and ‘viral’ excluded.
%
It is known that {\tiny SARS/MERS} including Corona
can infect the brain stem,
possibly from the nose via nerves into the brain,
and such effects are also possibly one reason for Long-{\tiny COVID}.
%
But practically nobody knew that back then,
hardly even heard of Corona,
at first known as a lung disease far away in China.
%
My mother had asthma,
used the same spray
that a recent study indicates would reduce the chance of severe affection of the lungs.
%
My father and me had absolutely no symptoms,
nor did anybody we knew have anything like that at the time.
%
My mother had always clearly stated
that she did not want artificial feeding,
but if we had assumed a temporary infection
with a good chance of recovery…\ 
%
Even though my parents had lead an independent, active life right until then,
there were many concrete health issues
indicating that this would not continue much longer;
I would guess in any case not more than one year.
%
My parents had promised each other already when young
that they would only go together,
and in the end they almost did;
if legally possible
(they were at {\tiny EXIT},
but my mother was already not mentally fit enough any more),
they might have.
%
With my parents it will probably always remain some sort of \textsl{Rashomon} situation;
reality, fate, choice,
and maybe on several levels Gibbs’s rule \#39 (“no coincidences”),
or rather along Robert Graves’ anecdote (see page 26) it was enough for a habit%
—moon,
great goddess,
and,
related maybe also to my articles at the time,
hopefully rather an act of grace than disagreement.
%
RIP Romeo+Suzy\, \includegraphics[scale=0.018]{i-smiley.png}\par}

\end{document}
