\section{\en{psyche}\de{psyche}}

\en{Given the immediate experience of life would be essentially
along the circle of elements, everything the psyche does and
experiences, like thinking and feeling, would also essentially
be along that circle. In other words, life as personal experience
(psyche) would essentially happen along that circle.}%
\de{Da die unmittelbare Erfahrung des Lebens im Wesentlichen
entlang des Kreises der Elemente verlaufen würde, würde
auch alles, was die Psyche tut und erlebt, wie Denken und
Fühlen, im Wesentlichen entlang dieses Kreises verlaufen.
Mit anderen Worten: Das Leben als persönliche Erfahrung
(Psyche) würde sich im Wesentlichen entlang dieses Kreises
abspielen.}

\en{In the model of elemental transformations in the zodiac
from the \textsl{star signs} section, all star signs transform from
outer to inner elements (except for the desired element).
Inside is where one might suspect the psyche to be.}%
\de{Im Modell der elementaren Wandlungen im Tierkreis
der Sektion \textsl{sternzeichen} wandeln sich alle Sternzeichen
von den äusseren zu den inneren Elementen (ausser dem
erwünschten Zielelement). Das Innere ist der Ort, an dem
man die Psyche vermuten könnte.}

\vspace{-1mm}
\begin{center}
\en{\includegraphics[scale=0.088]{i-circle.jpg}}%
\de{\includegraphics[scale=0.088]{i-circle-de.jpg}}
\end{center}
\vspace{-1mm}

\en{Could the argument be reversed, would an assumption
that the psyche is inside imply the transitions of the zodiac\,?
At least they are general in the sense that for each element
they select the transition from the outer adjacent element
via the element itself to the inner adjacent element.}%
\de{Könnte das Argument umgekehrt werden, würde die
Annahme, dass die Psyche innen ist, die Übergänge des
Tierkreises implizieren\,? Zumindest sind sie allgemein in
dem Sinne, dass sie für jedes Element den Übergang vom
äusseren Nachbarelement über das Element selbst zum
inneren Nachbarelement wählen.}

\en{So all transformations in life would be about learning in
the broadest sense, end up inside, but with hopes also for
outside, maybe even often as offspring, new life.}%
\de{Alle Transformationen im Leben wären also im weitesten
Sinne ein Lernen, enden im Inneren, aber mit
Hoffnungen auch für das Aussen, vielleicht sogar oft als
Nachkommenschaft, neues Leben.}

\en{And the psyche would be closely related to e5.}%
\de{Und die Psyche wäre eng verbunden mit e5.}

\en{Was the prehistoric psyche of people maybe not much
able, yet, to separate the two active elements from each
other, what the individual can move outside (emo) from the
thoughts related to that (eri?), and labelled both as a single
experience of everyday life as fire/red, leaving black/dark
for the largely given state of things outside (ero) and white/ bright
for the largely given flow of things inside (emi)\,?}%
\de{War die prähistorische Psyche der Menschen vielleicht
noch nicht so sehr in der Lage gewesen, die beiden aktiven
Elemente voneinander zu trennen, das, was das Individuum
aussen bewegen kann (emo) von den darauf bezogenen
Gedanken (eri?), und bezeichnete beides als eine einzige
Erfahrung des Alltags als Feuer/Rot, was Schwarz/dunkel
für den weitgehend gegebenen Zustand der Dinge aussen
(ero) und Weiss/hell für den weitgehend gegebenen Fluss
der Dinge im Inneren (emi) übrig liess\,?}
\vspace{-1mm}

% page
\begin{center}
\en{\includegraphics[scale=0.09]{i-circle-archaic.jpg}}%
\de{\includegraphics[scale=0.09]{i-circle-archaic-de.jpg}}
\end{center}
\vspace{-1mm}

\en{Thus first just three proto-elements, of which ‘red’ fire
later split into yellow fire plus red air\,? And before learning
to preserve and later to create fire, a more passive psyche
of mostly just ‘black and white’, with fire/light essentially
as a given phenomenon outside and inside\,? And maybe
black for outside, because it is dark at night without the
fire of the sun or also in a cave without a campfire, while
inside the mind it can be bright anytime\,? And before that
black and white mixed into one in perpetual change, as in
yin-yang \yinyang\ or in some of Heraclitus’ fragments\,?}%
\de{Also zunächst nur drei Proto-Elemente, von denen sich
das ‘rote’ Feuer später in gelbes Feuer plus rote Luft
aufspaltete\,? Und bevor die Menschen lernten, Feuer zu
bewahren und später zu erschaffen, eine passivere Psyche von
meist nur ‘schwarz und weiss’, mit Feuer/Licht im Wesentlichen
als einem gegebenen Phänomen aussen und innen\,?
Und vielleicht Schwarz für aussen, weil es nachts ohne das
Feuer der Sonne dunkel ist oder auch in einer Höhle ohne
Lagerfeuer, während es im Inneren des Geistes jederzeit
hell sein kann\,? Und davor Schwarz und Weiss in ständiger
Veränderung in eins vermischt, wie im Yin-Yang \yinyang\ oder in
einigen Fragmenten von Heraklit\,?}

\subsection{\en{leads}\de{fährten}}

\small
\begin{list}{$\bullet$}{\setlength{\leftmargin}{10pt}}

\item
\en{The four tasks of Psyche in Apuleius’ \textsl{The Golden Ass} are
about elemental transformations of nominally the \textsl{psyche}.}%
\de{Bei den vier Aufgaben der Psyche in Apuleius’ \textsl{Der goldene
Esel} geht es nominell um elementare Wandlungen der Psyche.}

\en{The four tasks are in the middle of the book, nested threefold
into the outer story of Lucius as an ass, the fairy tale of Cupid
and Psyche, and Psyche’s visits to different deities for help,
until she ends up at Venus who poses the four tasks to her.}%
\de{Die vier Aufgaben befinden sich in der Mitte des Buches,
dreifach verschachtelt in die äussere Geschichte von Lucius
als Esel, das Märchen von Amor und Psyche und die Besuche
von Psyche bei verschiedenen Gottheiten um Hilfe, bis sie bei
Venus landet, die ihr die vier Aufgaben stellt.}

\en{While the two outmost stories are based in part on well-known
older myths and folk tales, and the ancient gods reflect their
well-known natures, this appears not to be the case for Psyche’s
tasks. Instead it is more likely that Apuleius devised
them himself or at least that they emerged around his time,
as a way to convey certain new ideas.}%
\de{Während die beiden äussersten Geschichten zum Teil auf
bekannten älteren Mythen und Volksmärchen beruhen und die
antiken Götter ihr bekanntes Wesen widerspiegeln, scheint
dies bei den Aufgaben der Psyche nicht der Fall zu sein.
Vielmehr ist es wahrscheinlich, dass Apuleius sie sich selbst
ausgedacht hat oder dass sie zumindest zu seiner Zeit entstanden
sind, um bestimmte neue Ideen zu vermitteln.}

\en{Only few of Apuleius’ works have survived. One is \textsl{On Plato and his Doctrine},
a short version of Plato’s philosophy,
another is \textsl{On the God of Socrates},
with thoughts on ‘daemons’
as beings that live in the air, and he translated Plato’s \textsl{Phaedo}
to Latin, where Socrates argues for the immortality of the
soul on the evening before his death by hemlock.}%
\de{Von Apuleius’ Werken sind nur wenige erhalten geblieben.
Eines davon ist \textsl{Platon und seine Lehre}, eine Kurzfassung von
Platons Philosophie, ein anderes ist \textsl{Über den Gott des Sokrates},
mit Gedanken zu ‘Dämonen’ als Wesen, die in der
Luft leben, und er übersetzte Platons \textsl{Phaidon} ins Lateinische,
wo Sokrates am Abend vor seinem Tod durch Schierling für
die Unsterblichkeit der Seele argumentiert.}

\en{The word that Plato used for soul is \textsl{psychê}, literally ancient
Greek for a \textsl{butterfly}, that mystical short-lived creature.}%
\de{Das Wort, das Platon für Seele verwendete, ist \textsl{psychê}, was
wörtlich altgriechisch für \textsl{Schmetterling} ist, dieses mystische
kurzlebige Wesen.}

\vspace{1.2mm}\hspace{0mm}
\noindent
\includegraphics[scale=0.185]{i-psyche.jpg}
\vspace{1mm}

\en{A butterfly is often seen as either resting on a flower or else
fluttering on to the next one, which reminds of the psyche,
which often dwells a while on a topic, then “flutters” on to
the next, often also in a rather random looking way.}%
\de{Ein Schmetterling wird oft erblickt, wenn er sich entweder
auf einer Blume ausruht oder zur nächsten weiterflattert, was
an die Psyche erinnert, die oft eine Weile bei einem Thema
verweilt und dann zum nächsten “flattert”, oft auch auf eine
eher zufällig wirkende Weise.}

% page
\en{Apuleius lived in a very fruitful time in which many symbolic
systems found a form to stay in for many centuries by melting
Greek and Egyptian/African views into something new: Star
signs got their attributed elements; in Stoicism the highest,
lightest form of pneuma was called psychê; in alchemy the
transition towards the philosopher’s stone black-white-yellow-red
is the same order of elements as apparently in Psyche’s
tasks; a mummy reminds of the chrysalis into which a caterpillar
weaves itself and later emerges as a butterfly, a cocoon
as sort of a vessel towards a higher life; leading back in time
to silkworms, the changing colors of a mulberry and the great
goddess, or forward to then upcoming religions like Christianity
that feature the idea of an immortal soul, and so on.}%
\de{Apuleius lebte in einer sehr fruchtbaren Zeit, in der viele
symbolische Systeme eine Form fanden, in der sie dann viele
Jahrhunderte lang blieben, indem sie griechische und
ägyptische/ afrikanische Ansichten zu etwas Neuem verschmolzen:
Sternzeichen bekamen ihre zugeordneten Elemente; im Stoizismus
wurde die höchste, leichteste Form des Pneuma psychê
genannt; in der Alchemie ist der Übergang zum Stein der Weisen
schwarz-weiss-gelb-rot die gleiche Reihenfolge der Elemente
wie offenbar in den Aufgaben der Psyche; eine Mumie
erinnert an die Puppe, in die sich eine Raupe einspinnt und später
als Schmetterling schlüpft, ein Kokon als eine Art Gefäss auf
dem Weg zu einem höheren Leben; der Weg führt in der
Zeit zurück zu Seidenraupen, den wechselnden Farben einer
Maulbeere und der grossen Göttin, oder vorwärts zu damals
aufkommenden Religionen wie dem Christentum, die die Idee
einer unsterblichen Seele beinhalten, und so weiter.}

\en{The original title of the book was \textsl{Metamorphoseon Libri XI},
which is likely why Apuleius might have devised the tasks of
Psyche as elemental transformations of the soul and placed
them at the very center of his masterpiece.}%
\de{Der Originaltitel des Buches war \textsl{Metamorphoseon Libri XI},
was wahrscheinlich der Grund ist, wieso Apuleius die Aufgaben
der Psyche als elementare Wandlungen der Seele konzipierte
und sie in den Mittelpunkt seines Meisterwerks stellte.}

\item
\en{Myths may have carved out the cycle of elements more closely
and in a more streamlined way than most other stories, as
in myths originally only what felt ultimately important was
worth the effort of remembering it by heart during life and
transmitting it orally from one generation to the next. Myths
around star signs, in particular, might even more specifically
reflect only certain segments of the cycle of elements.}%
\de{Mythen haben den Zyklus der Elemente vielleicht genauer
und stromlinienförmiger ausgearbeitet als die meisten
anderen Geschichten, denn in den Mythen war ursprünglich nur
das wichtig, was die Mühe wert war, es während des Lebens
auswendig zu lernen und mündlich von einer Generation an
die nächste weiterzugeben. Insbesondere die Mythen um die
Sternzeichen herum könnten sogar noch spezifischer nur
bestimmte Segmente des Zyklus der Elemente widerspiegeln.}

\item
\en{Observing something that happens outside (emo) can lead to
insights into the workings of the world (eri), so the psyche
would have operated along the circle of elements, emo\rarr eri.
Natural sciences would be a lot about this part of the cycle,
relating essentially experiment (emo) and theory (eri).}%
\de{Etwas beobachten, das aussen geschieht (emo), kann zu
Einsichten in die Funktionsweise der Welt (eri) führen, so dass
die Psyche entlang des Kreislaufs der Elemente, emo\rarr eri,
operiert hätte. In den Naturwissenschaften ginge es viel um
diesen Teil des Kreislaufs, der im Wesentlichen Experiment
(emo) und Theorie (eri) verbindet.}

\en{You could, for example, not learn much of what a cube is,
unless it moves (emo) or if you move yourself and look at it
from various angles or turn it in your hand (eri\rarr emo). Just
looking at a cube from a single perspective (ero) would not
allow to learn much about a cube as a physical object with
specific properties and symmetries, but could still change your
mood (emi?). Such a mood might still allow to learn something
in the sense of later being able to recognize a cube
if you encounter another one from a similar angle, but not
much in a consciously analytical way, and arguably recognition
might rather come indirectly from a transition emi\rarr eri,
from learning inside from different moods.}%
\de{Man könnte zum Beispiel nicht viel darüber lernen, was ein
Würfel ist, wenn er sich nicht bewegt (emo) oder man sich
selbst bewegt und ihn aus verschiedenen Winkeln betrachtet
oder ihn in der Hand dreht (eri\rarr emo). Das Betrachten
eines Würfels aus einer einzigen Perspektive (ero) würde es
nicht erlauben, viel über einen Würfel als physisches Objekt
mit spezifischen Eigenschaften und Symmetrien zu lernen,
könnte aber dennoch die Stimmung (emi?) verändern. Eine
solche Stimmung könnte es immer noch ermöglichen, etwas
zu lernen, in dem Sinne, dass man später in der Lage wäre,
einen Würfel wiederzuerkennen, wenn man einem anderen aus
einem ähnlichen Blickwinkel begegnet, aber nicht gross in
einer bewusst analytischen Art und Weise, und das Erkennen
könnte wohl eher indirekt aus einem Übergang emi\rarr eri, aus
dem Lernen aus verschiedenen Stimmungen heraus entstehen.}

\en{Even though in the model of the star signs, transitions would
in the end tend to go inside, in practice things would often
involve both ways, for example, when looking at a cube from
different sides, both moving it, eri\rarr emo, and learning from
its movements, emo\rarr eri, in a close feedback loop.}%
\de{Auch wenn im Modell der Sternzeichen die Übergänge letztlich
eher nach innen gehen würden, würde es in der Praxis
oft in beide Richtungen gehen, z.B.\ wenn man einen
Würfel von verschiedenen Seiten betrachtet, ihn sowohl
bewegt, eri\rarr emo, als auch aus seinen Bewegungen lernt,
emo $\rightarrow$\,eri, in einer engen Rückkopplungsschleife.}

\en{At emi much more may already be going on unconsciously
than is obvious, there may already be a lot of comparing
of different experiences (ero) happening in the background,
which then eri could analyze by observing emi inside similarly
to emo outside. And what eri would postulate, might again
create an emotional reaction, and so on\,?}%
\de{Bei emi mag bereits viel mehr unbewusst ablaufen, als
offensichtlich ist, es mögen vielleicht bereits eine Menge Vergleiche
verschiedener Erfahrungen (ero) im Hintergrund ablaufen, die
eri dann analysieren könnte, indem es emi innen ähnlich wie
emo aussen beobachtet. Und das, was eri postulieren würde,
könnte wiederum eine emotionale Reaktion hervorrufen, und
so weiter\,?}

% page
\item
\en{How would the maybe more subtle view of dual female and
male elements in the I Ching fit here\,? And generally into the
astrological model of transformations in the Western zodiac\,?
What about the Chinese zodiac, which probably emerged
roughly around the year zero like similar systems in the West\,?
Does it also mirror elemental transitions of the psyche\,? Or
maybe something else\,? What about other zodiacs\,?}%
\de{Wie würde die vielleicht etwas subtilere Sichtweise der dualen
weiblichen und männlichen Elemente des I Ging hierher
passen\,? Und ganz allgemein in das astrologische Modell der
Wandlungen im westlichen Tierkreis\,? Wie verhält es sich mit
dem chinesischen Tierkreis, der wahrscheinlich etwa um das
Jahr Null herum entstanden ist, wie ähnliche Systeme im
Westen\,? Spiegelt er auch elementare Übergänge der Psyche
wider\,? Oder vielleicht etwas anderes\,? Wie steht es mit anderen
Tierkreisen\,?}

\item
\en{Is it true that the psyche is inside, that all would travel inside
during life, or is that more of a Western view, not ultimately
true\,? But maybe part of the truth if adding similar transitions
“the other way” to balance it, or in some other ways\,?}%
\de{Ist es wahr, dass die Psyche im Inneren ist, dass alles im Laufe
des Lebens nach innen wandert, oder ist das eher eine
westliche Sichtweise, die letztlich nicht wahr ist\,? Aber vielleicht
ein Teil der Wahrheit, wenn man ähnliche Übergänge “in die
andere Richtung” hinzufügen würde, um das auszugleichen,
oder auf andere Weisen\,?}

\item
\en{I switched the fire colors in the illustration of the archaic
circle so that the darker color (red) is closer to black and the
lighter color (yellow) closer to white, actually as in alchemy.}%
\de{Ich habe die Feuerfarben in der Illustration des archaischen
Kreises vertauscht, so dass die dunklere Farbe (Rot) näher
an Schwarz und die hellere Farbe (Gelb) näher an Weiss ist,
genau wie in der Alchemie.}

\item
\en{Before agriculture, people essentially had to follow nature.
Where to stay, where to find something to eat, was beyond
human control. Similarly, the flow of feelings, dreams, visions
was not something people could approach analytically at first.
That probably came in time by telling stories, with mythology
and other stories. Abstract concepts like love were first
personified as deities like Aphrodite/Venus, only later things
became more abstract, as in Greek philosophy.}%
\de{Bevor es Landwirtschaft gab, mussten die Menschen im
Wesentlichen der Natur folgen. Wo sich aufhalten, wo etwas zu
essen finden, lag jenseits der menschlichen Kontrolle. Auch
der Fluss von Gefühlen, Träumen und Visionen war anfangs
nichts, was die Menschen analytisch angehen konnten. Das
kam wahrscheinlich mit der Zeit durch das Erzählen von
Geschichten, mit Mythologie und anderen Geschichten. Abstrakte
Konzepte wie die Liebe wurden zunächst als Gottheiten wie
Aphrodite/Venus personifiziert, erst später wurden die Dinge
noch abstrakter, wie in der griechischen Philosophie.}

\item
\en{Reading a text silently was apparently not usually done in
antiquity. Texts were rather recited aloud, hence also texts
often in rhymes to give them rhythm and melody. Thus in
and out of what the psyche actively did were still somewhat
mixed up into one: no thinking without speaking or acting.}%
\de{Das stille Lesen eines Textes war in der Antike offenbar nicht
üblich. Texte wurden vielmehr laut rezitiert, daher auch Texte
oft in Reimen, um ihnen Rhythmus und Melodie zu geben.
So waren innen und aussen von dem, was die Psyche aktiv
tat, noch zum Teil ineinander verwoben: kein Denken ohne
Sprechen oder Handeln.}

\item
\en{\textsl{New Testament Recovery Version}, 1 Th 5:23, footnote 5c:}%
\de{\textsl{Neues Testament Wiedererlangungs-Übersetzung}, 1 Th 5:23, Fussnote 5c:}

\en{“The spirit as our inmost part is the inner organ, possessing
God-consciousness, that we may contact God (John 4:24;
Rom.\ 1:9). The soul is our very self (cf.\ Matt.\ 16:26; Luke
9:25), a medium between our spirit and our body, possessing
self-consciousness, that we may have our personality. The
body as our external part is the outer organ, possessing
world-consciousness, that we may contact the material world.”}%
\de{“Der Geist als unser innerster Teil ist das innere Organ, das
das Gottesbewusstsein besitzt, damit wir mit Gott in Kontakt
treten können (Johannes 4,24; Röm.\ 1,9). Die Seele ist unser
eigenes Ich (vgl.\ Mt 16,26; Lk 9,25), ein Medium zwischen
unserem Geist und unserem Körper, das ein Selbstbewusstsein
besitzt, damit wir unsere Persönlichkeit haben können. Der
Körper als unser äusserer Teil ist das äussere Organ, das ein
Weltbewusstsein besitzt, damit wir mit der materiellen Welt
in Kontakt treten können.”}

\en{Hence from inside to outside this could be seen as the colors
of the ripening mulberry, spirit-water-white, soul-air{\footnotesize$\vert$}fire-red,
body-earth-black; part of the cycle emi-eri{\footnotesize$\vert$}emo-ero; inside
and outside the passive elements, in-between the active ones.}%
\de{Von innen nach aussen könnte man dies also als die Farben
der reifenden Maulbeere sehen, Geist-Wasser-Weiss,
Seele-Luft{\footnotesize$\vert$}Feuer-Rot, Körper-Erde-Schwarz; Teil des Zyklus
emi-eri{\footnotesize$\vert$}emo-ero; innen und aussen die passiven Elemente,
dazwischen die aktiven.}

\en{Note, however, that the footnote might be more of a modern
interpretation/insight than explicit or implicit ancient views;
spirit-soul-body were pneuma-psyche-soma in Greek.}%
\de{Es ist jedoch zu beachten, dass es sich bei der Fussnote
eher um eine moderne Interpretation/Einsicht handeln könnte
als um explizite oder implizite antike Ansichten; Geist-Seele-Körper
waren im Griechischen pneuma-psyche-soma.}

% page
\item
\en{Where does the identification air-thinking and water-feeling
come from in contemporary psychological astrology\,? Maybe
from Jung’s “Psychological Types”, but is it like that there\,?
And what about the ancient theories of humors and temperaments,
do they reflect this view\,?}%
\de{Woher kommt in der heutigen psychologischen Astrologie
die Identifikation von Luft-Denken und Wasser-Fühlen\,?
Vielleicht aus Jungs “Psychologischen Typen”, aber ist das dort
auch so\,? Und wie sieht es mit den antiken Theorien der
Humore und Temperamente aus, spiegeln sie diese Sichtweise\,?}

\en{Temperaments seem to have been almost exclusively describing
outer appearance and behavior, but not internal motivation.
For example, phlegmatic types (‘water’) were considered
rather slow and weak. And temperaments seem to have
been hardly mentioned in the context of astrology.}%
\de{Temperamente scheinen fast nur das äussere Erscheinungsbild
und das Verhalten beschrieben zu haben, nicht aber die
innere Motivation. Zum Beispiel wurden Phlegmatiker (‘Wasser’)
als eher langsam und schwach betrachtet. Und die
Temperamente scheinen im Zusammenhang mit der Astrologie
kaum erwähnt worden zu sein.}

\en{It seems that a shift towards psychology would have only
come roughly with the 20th century, along also with developments
around Freud and others. In Alan Leo’s descriptions
of the elements in \textsl{The Art of Synthesis} (1912) there seem
to be already some tendencies into that direction.}%
\de{Eine Hinwendung zur Psychologie scheint erst im 20.\ Jahrhundert
stattgefunden zu haben, auch im Zug der Entwicklungen
um Freud und andere. In Alan Leos Beschreibungen
der Elemente in \textsl{The Art of Synthesis} (1912) scheint es
bereits einige Tendenzen in diese Richtung zu geben.}

\en{But it seems that Jung would have been first to define four
purely psychological functions to derive 8 psychological types
via intra- and extraversion. In \textsl{Psychologische Typen} he has
an extensive chapter with definitions of terms, where he also
defines ‘thinking’, ‘feeling’, ‘sensation’ and ‘intuition’.}%
\de{Aber es scheint, dass Jung als erster vier rein psychologische
Funktionen definiert hat, um daraus über Intro- und Extraversion
8 psychologische Typen abzuleiten. In  \textsl{Psychologische
Typen} hat es ein umfangreiches Kapitel mit Begriffsdefinitionen,
in dem er auch ‘Denken’, ‘Fühlen’, ‘Empfinden’ und
‘Intuition’ definiert.}

\en{From my perspective, sensation goes beyond just registering
things, it also covers what I would call ‘measurement’,
the translation of sensory input to more abstract concepts
like, for example, ‘dog’, which also includes some categorization.
The conscious mind then operates mostly on those
categories, logically or also in other conscious ways, including
judging things. A feeling is sort of an unconscious categorization
in response to other things going on in the mind,
sort of feeding data into a neural network and the network
then maybe detecting something that is suspicions, resulting
in the corresponding feeling, but this probably already simplifies
too much. Finally, intuition in Jung’s sense would in my
view include unconscious creations that become conscious for
further judgement and analysis, sort of the result of a generative
neural network that produces various ideas of which some
after some basic unconscious testing are allowed to become
conscious. That would in a nutshell be how I would describe
things, and Jung arguably was already close to that in 1921.}%
\de{Aus meiner Sicht geht Empfindung über das blosse Registrieren
von Dingen hinaus, sie umfasst auch das, was ich
als ‘Messung’ bezeichnen würde, die Übersetzung von
Sinneseindrücken in abstraktere Konzepte wie z.B.\ ‘Hund’, was
auch eine gewisse Kategorisierung beinhaltet. Der bewusste
Verstand arbeitet dann hauptsächlich mit diesen Kategorien,
logisch oder auch auf andere bewusste Weise, einschliesslich
der Beurteilung von Dingen. Ein Gefühl ist eine Art unbewusste
Kategorisierung als Reaktion auf andere Dinge, die
im Gehirn vor sich gehen, eine Art Einspeisung von Daten
in ein neuronales Netzwerk, und das Netzwerk erkennt dann
vielleicht etwas, das verdächtig ist, was zu dem entsprechenden
Gefühl führt, aber das vereinfacht wahrscheinlich schon
zu sehr. Schliesslich würde Intuition im Sinne von Jung meiner
Ansicht nach unbewusste Schöpfungen umfassen, die zur
weiteren Beurteilung und Analyse bewusst werden, sozusagen
das Ergebnis eines generativen neuronalen Netzwerks, das
verschiedene Ideen produziert, von denen einige nach einer
elementaren unbewussten Prüfung bewusst werden dürfen.
So würde ich die Dinge kurz gefasst beschreiben, und Jung
war wohl schon 1921 nahe daran.}

\en{But I am not sure if these categories are all that natural, even
though they certainly mirror many things in useful ways.}%
\de{Aber ich bin mir nicht sicher, ob diese Kategorien wirklich
so natürlich sind, auch wenn sie sicherlich viele Dinge auf
nützliche Weise widerspiegeln.}

\en{In my personal feelings, Jung’s book would be biased in ways
that later lead to troubles for him including the red book.}%
\de{Meinem persönlichen Gefühl nach, wäre Jungs Buch einseitig
auf Weisen, die später für ihn zu mühseligen Wirrungen
führten, inklusive dem roten Buch.}

\end{list}
