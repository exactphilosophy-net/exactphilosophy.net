\section{\en{dreams}\de{träume}}

\en{I dream of the way I just presented to catch on and to grow
by itself, without me having to do anything for it.}%
\de{Ich träume, dass der Weg den ich soeben präsentiert habe,
ansteckend wäre und von selbst wachsen würde, ohne dass
ich irgendwas dafür tun müsste.}

\subsection{\ \ \ \ \includegraphics[scale=0.092]{i-elemental}}

\en{Rumor has it that \includegraphics[scale=0.048]{i-jackdaw} (yes, the legendary Jack Daw)
would be writing a new book that would include some key
ideas from this website in his inimitable style.}%
\de{Man munkelt, dass \includegraphics[scale=0.048]{i-jackdaw}
(ja, der legendäre Jack Daw)
ein neues Buch schreiben würde, das einige der wichtigsten
Ideen dieser Website in seinem unnachahmlichen Stil
aufgreifen würde.}

\subsection{\en{leads}\de{fährten}}

\small
\begin{list}{$\circ$}{\setlength{\leftmargin}{10pt}}

\item
\en{I guess I would not be made for publicity anyways.}%
\de{Ich schätze, ich wäre sowieso nicht für die Öffentlichkeit
geschaffen.}

\color{odyssey}

\item
\en{Since 2025 I dream of close to completely recovering the way
so that I could also convey my other findings going that way,
more naturally and with less resistance in people and me.}%
\de{Seit 2025 träume ich davon, den Weg fast vollständig wieder
zu finden, so dass ich auch meine anderen Entdeckungen über
diesen Weg gehend vermitteln könnte, natürlicher und mit
weniger Widerstand in den Menschen und mir.}

\end{list}

\color{odyssey}

\subsection{\en{dreams}\de{träume}}

\small
\begin{list}{$\bullet$}{\setlength{\leftmargin}{10pt}}

\item
\en{What moves inside (emi), even conscious thinking, is mostly
an automatic process and thus passive from the point of view
of an observing self, with just tiny fully conscious glimpses
where things are static (eri), like the flowers for a butterfly or
the islands Odysseus visited in the Odyssey.}%
\de{Was sich im Inneren bewegt (emi), sogar bewusstes Denken,
ist grösstenteils ein automatischer Prozess und somit passiv
aus der Perspektive eines beobachtenden Selbst, mit nur
winzigen vollständig bewussten Anblicken wo die Dinge statisch
sind (eri), wie die Blumen für einen Schmetterling oder die
Inseln, die Odysseus in der Odyssee besuchte.}

\item
\en{A cycle of elements in nature can be seen as fire rising up, air
resting above, water flowing down and earth resting below.
In that cycle rest/move and active/passive would be a good
fit with emo-fire, eri-air, emi-water and ero-earth.}%
\de{Ein Kreislauf der Elemente in der Natur kann als aufsteigendes
Feuer, oben ruhende Luft, herunterfliessendes Wasser
und unten ruhende Erde gesehen werden. In diesem Kreislauf
würden ruhen/bewegen und aktiv/passiv gut passen zu
emo-Feuer, eri-Luft, emi-Wasser und ero-Erde.}

\item
\en{In psychological astrology, realism (earth) and vision (fire)
could be considered more outside than feelings (water) and
thoughts (air), but feelings-moving and thinking-resting seem
not to fit immediately, even though it would seem to fit the
central air and water signs, Libra and Scorpio, in a way.}%
\de{In der psychologischen Astrologie könnten Realismus (Erde)
und Vision (Feuer) als eher aussen als Gefühle (Wasser) und
Gedanken (Luft) erachtet werden, aber Gefühle-bewegen und
Denken-ruhen scheint nicht unmittelbar zu passen, obschon
es zu den zentralen Luft- und Wasserzeichen, der Waage und
dem Skorpion, zu passen scheinen würde, in gewisser Weise.}

\en{The transformations of the zodiac signs, as described in the
section ‘star signs’, seem to go in the above direction of the
cycle for fire and air signs, but for water and earth seem to
go the opposite way. For water it would be first melting and
then evaporating, reminding of Werner Held’s article “Flusswissen”,
where he imagines a river that flows upwards; his
sun is in Scorpio, the central water sign.}%
\de{Die Wandlungen der Tierkreiszeichen, wie beschrieben in der
Sektion ‘sternzeichen’, scheinen für Feuer- und Luftzeichen
in die oben beschriebene Richtung im Kreislauf zu verlaufen,
aber für Wasser und Erde scheinen sie umgekehrt zu verlaufen.
Bei Wasser wäre erst Schmelzen und dann Verdampfen,
was an Werner Helds Artikel “Flusswissen” erinnert, wo er
sich einen Fluss vorstellt, der aufwärts fliesst; seine Sonne ist
im Skorpion, dem zentralen Wasserzeichen.}

\en{Thus water signs and maybe the watery part of the psyche
would not be foremost about feelings, but more about a quest
against the flow towards more consciousness and light, therefore
popping up from the unconscious again and again (emi
to eri), or something along those lines\,?}%
\de{Hätten daher Wasserzeichen und vielleicht der wässerige Teil
der Psyche nicht primär mit Gefühlen zu tun, sondern eher
mit einer Suche gegen den Strom zu mehr Bewusstsein und
Licht hin, wodurch immerzu aus dem Unbewussten wieder
etwas auftauchen würde (emi zu eri), oder etwas in der Art\,?}

\item
\en{There is just one experience of living, thus also ero and emo
are part of what you might call the psyche.}%
\de{Es gibt nur ein Erlebnis des Lebens, daher sind auch ero und
emo Teil dessen, was man als Psyche bezeichnen könnte.}

\item
\en{Related to a lead in the \textsl{origins} section about the Red Sparrow
in Bukowski’s \textsl{Pulp}: At the spring point astrologically Pisces
and Aries meet, with Pisces as a water sign mainly made of
(triune?)\ air for the human condition and Aries as a fire sign
mainly made of (triune?)\ earth for the physical end of it\,?
And rebirth in some way at the same point\,?}%
\de{In Bezug auf die Fährte in der Sektion \textsl{ursprünge} über den
roten Spatz in Bukowskis \textsl{Pulp}: Am Frühlingspunkt treffen
sich astrologisch Fische und Widder, mit Fischen als
Wasserzeichen vorwiegend aus (dreieiniger?) Luft und Widder als
Feuerzeichen vorwiegend aus (dreieiniger?) Erde\,? Und
Wiedergeburt auf eine Weise am gleichen Punkt\,?}

\end{list}
