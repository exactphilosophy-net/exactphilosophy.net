\section{\en{mixed feelings}\de{gemischte gefühle}}

\en{The inner elements eri and emi are softer than the outer
ones, which suggests that they would \textsl{mix} more easily.}%
\de{Die inneren Elemente eri und emi sind weicher als die
äu\-sse\-ren, was suggeriert, dass sie sich leichter \textsl{mischen}
würden.}

\en{The idea is now that what appears outside as individual
and separate beings is unconsciously connected inside…}%
\de{Die Idee ist nun, dass was aussen als individuelle,
separate Wesen erscheint, innen unbewusst verbunden wäre…}

\vspace{1.1mm}
\hspace{8mm}
\en{\includegraphics[scale=0.05]{i-feelings.jpg}}%
\de{\includegraphics[scale=0.05]{i-feelings-de.jpg}}
\vspace{2.5mm}

\en{…and that these connections result in feelings that
change for often not obvious reasons (emi), while naming
inner concepts allows to attain some abstract calm (eri).}%
\de{…und dass diese Verbindungen zu Gefühlen führen, die
sich aus oft nicht offensichtlichen Gründen wandeln (emi),
während das Benennen innerer Konzepte es ermöglicht,
eine abstrakte Ruhe zu erlangen (eri).}

\en{Astrology links water (emi) to feelings, love, music, art,
religion, the collective and/or individual unconscious, and
more. Now, the idea of a \textsl{collective} unconscious goes back
to Jung, while it may in the end still be so that such unconscious
collective connections are created by more Freudian
individual unconsciouses, via subliminal channels in normal
day-to-day external interactions between beings.}%
\de{Die Astrologie verbindet Wasser (emi) mit Gefühlen,
Liebe, Musik, Kunst, Religion, dem kollektiven und/oder
individuellen Unbewussten und vielem mehr. Nun, die Idee
eines \textsl{kollektiven} Unbewussten geht auf Jung zurück, auch
wenn es letztlich immer noch so sein mag, dass solche
unbewussten kollektiven Verbindungen von eher freudianischen
individuellen Unbewussten über unterschwellige Kanäle in
den normalen alltäglichen äusseren Interaktionen zwischen
Wesen geschaffen würden.}

\en{But let me explore things in the Jungian picture first,
as a Gedankenexperiment, because it is initially easier, and
because it mirrors the initial assumption more directly.}%
\de{Aber lasst mich die Dinge zunächst im Jungschen Bild
erkunden, als Gedankenexperiment, weil das zunächst
einfacher ist und die Ausgangsannahme direkter widerspiegelt.}

\en{How one feels at any moment would be a mixture of
individual and collective influences. Not that what other
people think would be directly accessible, just indirectly
with regard to how one feels in a particular situation, or
how one feels regarding individual possible next steps.}%
\de{Wie man sich in einem bestimmten Moment fühlt, ist
eine Mischung aus individuellen und kollektiven Einflüssen.
Was andere Menschen denken, wäre nicht direkt zugänglich,
sondern nur indirekt in Bezug darauf, wie man sich in einer
bestimmten Situation fühlt oder wie man sich hinsichtlich
individueller möglicher nächster Schritte fühlt.}

\en{Individuals that are emotionally and physically close
would likely have the strongest influence on a person,
but also large groups of people, like same village, country,
religion, etc., could together have a strong influence.}%
\de{Individuen, die einem emotional und physisch nahe stehen,
hätten wahrscheinlich den stärksten Einfluss auf eine
Person, aber auch grosse Gruppen von Menschen, wie
dasselbe Dorf, Land, die gleiche Religion usw., könnten
zusammen einen starken Einfluss haben.}

\en{Influences from a collective unconscious could go well
beyond the sum of what is in individual conscious minds.
Since the collective unconscious would effectively be a very
large brain, consisting of many more brain cells than any
individual being, it might have a much more complex and
sophisticated mind than any conscious individual and it could
know all kinds of details about everybody.}%
\de{Einflüsse eines kollektiven Unbewussten könnten weit
über die Summe dessen hinausgehen, was in den einzelnen
bewussten Köpfen vorhanden ist. Da das kollektive
Unbewusste praktisch ein sehr großes Gehirn wäre, das aus
viel mehr Gehirnzellen besteht als jedes einzelne Wesen,
könnte es einen viel komplexeren und ausgefeilteren
Verstand haben als jedes bewusste Individuum und es könnte
alle möglichen Details über jeden Menschen wissen.}

% page
\en{Such a view of a collective unconscious would resemble
the concept of god or gods in many religions, and it would
likely be fragmented into smaller units at several scales,
like families, countries, religions, etc., each with its own
collective feelings, plans, and so on.}%
\de{Eine solche Sichtweise eines kollektiven Unbewussten
würde dem Konzept von Gott oder Göttern in vielen
Religionen ähneln, und es wäre wahrscheinlich in kleinere
Einheiten auf verschiedenen Ebenen unterteilt, wie Familien,
Länder, Religionen usw., jede mit ihren eigenen kollektiven
Gefühlen, Plänen und so weiter.}

\en{Jung noticed that in dreams and in cultural creations
some archetypal patterns repeat. These archetypes might
simply be part of the thoughts, experiences and knowledge
of the collective unconscious.}%
\de{Jung bemerkte, dass sich in Träumen und in kulturellen
Schöpfungen einige archetypische Muster wiederholen.
Diese Archetypen könnten einfach Teil von Gedanken,
Erfahrungen und Wissen des kollektiven Unbewussten sein.}

\en{Precognition in dreams or art might simply be picking
up collective intentions that are only later realized and can
be felt and dreamed about already while the collective
unconscious is only planning or considering them.}%
\de{Präkognition in Träumen oder in der Kunst könnte
einfach kollektive Absichten aufgreifen, die erst später
verwirklicht werden aber die bereits gefühlt und erträumt werden
können, während das kollektive Unbewusste sie nur plant
oder in Erwägung zieht.}

\en{How would the collective unconscious effectively direct
the individuals it consists of\,? Telling each and every one
what to do at each moment would likely not be possible,
just like the conscious individual mind would not be able
to tell each of its nerve cells when to fire.}%
\de{Wie würde das kollektive Unbewusste die Individuen,
aus denen es besteht, effektiv lenken\,? Jedem Einzelnen zu
sagen, was er in jedem Moment zu tun hat, wäre
wahrscheinlich nicht möglich, sowenig wie der bewusste
individuelle Verstand nicht in der Lage wäre, jeder seiner
Nervenzellen zu sagen, wann sie feuern sollen.}

\en{But maybe with a general concept like astrology, which
creates a balanced and relatively complete set of individuals,
each with its own approach to new problems\,? Faced
with a particular problem, a Leo, for example, would feel
more like solving it in a “Leo way”, due to collective feedback,
so that in any situation different approaches would
be tried by different individuals and a good solution would
usually emerge. Since astrology tries to reflect all possible
approaches in a structured way, the search space for solutions
would usually be quite complete.}%
\de{Aber vielleicht mit einem allgemeinen Konzept wie der
Astrologie, die eine ausgewogene und relativ vollständige
Gruppe von Individuen schafft, von denen jedes einen
eigenen Ansatz für neue Probleme hat\,? Ein Löwe zum Beispiel
würde ein bestimmtes Problem aufgrund des kollektiven
Feedbacks eher auf “Löwe-Art” lösen, so dass in jeder
Situation verschiedene Ansätze von verschiedenen Individuen
ausprobiert würden und sich üblicherweise eine gute Lösung
fände. Da die Astrologie versucht, alle möglichen Ansätze
in strukturierter Weise widerzuspiegeln, wäre der Suchraum
für Problemlösungen in der Regel recht vollständig.}

\en{In other words, a culture with a system like astrology
would have an evolutionary advantage in the sense of Darwin.
Astrology would then not necessarily need to have
anything to do with planets and stars in the sky, more so
with relatively ancient beliefs about them.}%
\de{Mit anderen Worten: Eine Kultur mit einem System wie
der Astrologie hätte einen evolutionären Vorteil im Sinne
Darwins. Astrologie müsste dann nicht unbedingt etwas mit
Planeten und Sternen am Himmel zu tun haben, sondern
eher mit relativ alten Glaubensvorstellungen darüber.}

\en{Assuming the collective unconscious would extend to
matter considered inanimate, oracles like the I Ching or
Tarot could really reveal some intentions of the collective
unconscious, maybe paired with emotional feedback which
parts of the response to focus on or how to interpret it.
If so, also astrology might a priori still have natural causes,
direct influences from planets and stars, collective feedback
from the universe itself.}%
\de{Angenommen, das kollektive Unbewusste würde sich
auch auf unbelebte Materie erstrecken, dann könnten Orakel
wie das I Ging oder das Tarot tatsächlich einige Absichten
des kollektiven Unbewussten offenbaren, vielleicht
gepaart mit einer emotionalen Rückmeldung, auf welche
Teile der Reaktion man sich konzentrieren oder wie man sie
interpretieren sollte. Wenn dem so wäre, könnte auch die
Astrologie a priori noch natürliche Ursachen haben, direkte
Einflüsse von Planeten und Sternen, kollektiven Feedback
aus dem Universum selbst.}

\en{However, there are some arguments that speak against
astrology having dominantly natural causes from the sky.
There are different astrologies in different cultures, each of
which comes in different flavours and has different schools
of thought. Besides many small examples for a detachment
from actual constellations in the sky, the most prominent
one is Pluto in Western astrology.}%
\de{Es gibt jedoch einige Argumente, die dagegen sprechen,
dass die Astrologie überwiegend natürliche Ursachen vom
Himmel hat. Es gibt verschiedene Astrologien in verschiedenen
Kulturen, die jeweils unterschiedliche Ausprägungen
haben und verschiedene Denkschulen vertreten. Neben
vielen kleinen Beispielen für eine Loslösung von den tatsäch\-li\-chen
Konstellationen am Himmel ist das prominenteste
Beispiel Pluto in der westlichen Astrologie.}

% page
\en{Pluto was at its discovery in 1930 thought to be a planet
that is about as big as planet Earth. Over the following
decades it first emerged that Pluto is much smaller, consists
mainly of ice and finally in the early 21st century that
Pluto is rather part of a belt of objects in similar orbits
and with similar sizes. In 20th century astrology, however,
Pluto was attributed a major role, both in mundane events
and personal fates. In my perception, part of that view \textsl{did}
reflect reality, so that it seems most plausible to me that astrology
is \textsl{largely} a cultural creation of mankind that works
by collective feedback.}%
\de{Bei seiner Entdeckung im Jahr 1930 wurde Pluto für
einen Planeten gehalten, der etwa so gross wie die Erde ist.
In den folgenden Jahrzehnten stellte sich zunächst heraus,
dass Pluto viel kleiner ist, hauptsächlich aus Eis besteht und
schliesslich Anfang des 21.\ Jahrhunderts, dass Pluto eher
Teil eines Gürtels von Objekten in ähnlichen Umlaufbahnen
mit einander ähnlichen Grössen ist. In der Astrologie
des 20.\ Jahrhunderts wurde Pluto jedoch eine grosse Rolle
zugeschrieben, sowohl bei weltlichen Ereignissen als auch
bei persönlichen Schicksalen. Meiner Ansicht nach spiegelt
ein Teil dieser Ansicht \textsl{tatsächlich} die Realität wider, so
dass es mir am plausibelsten erscheint, dass die Astrologie
eine kulturelle Schöpfung der Menschheit wäre, die durch
kollektive Rückkopplung funktionieren würde.}

\en{Now let me come back to the initial question or to how
something with the properties of a collective unconscious
could come about in view of contemporary physics.}%
\de{Lasst mich nun auf die ursprüngliche Frage zurück\-kom\-men
oder darauf, wie etwas mit den Eigenschaften eines
kollektiven Unbewussten aus Sicht der heutigen Physik
zustandekommen könnte.}

\en{The most immediate explanation would be that there
are direct connection between brains, mediated by some
kind of “waves”. But this can largely be excluded today,
except maybe at close range, in the sense that any
explanation of that sort would require new physics.}%
\de{Die naheliegendste Erklärung wäre, dass es direkte
Verbindungen zwischen Gehirnen gäbe, vermittelt durch eine
Art “Wellen”. Aber das kann man heute weitgehend
ausschliessen, ausser vielleicht in relativ grosser Nähe, also jede
Erklärung dieser Art würde eine neue Physik erfordern.}

\en{So let me focus on known physics and try to look for
the most simple and obvious explanation. What I propose
is that people simply \textsl{mirror} who and what they encounter
in their lives inside their brains.}%
\de{Ich möchte mich also auf die bekannte Physik konzentrieren
und versuchen, die einfachste und naheliegendste
Erklärung zu finden. Ich schlage vor, dass die Menschen
einfach in ihren Gehirnen \textsl{spiegeln} würden, wen und was
ihnen in ihrem Leben begegnet.}

\en{People’s brains would thus contain “copies” of everyone
they know, most prominently and precisely of their loved
ones. What exactly the neural networks would mirror would
not be consciously available to individuals nor would it likely
be easy to analyze scientifically even if the full structure was
known. But it could in principle allow people to make fairly
accurate predictions about what their loved ones would do
and when. For example, one person could possibly think of
the other one almost exactly the moment that other person
would have picked up the phone to call.}%
\de{Die Gehirne der Menschen würden also “Kopien” von
allen enthalten, die sie kennen, am stärksten und genauesten
von denen, die sie lieben. Was genau die neuronalen
Netze widerspiegeln würden, wäre dem Einzelnen nicht
bewusst und liesse sich wahrscheinlich auch nicht leicht
wissenschaftlich analysieren, selbst wenn die vollständige
Struktur bekannt wäre. Aber im Prinzip könnte es den
Menschen ermöglichen, ziemlich genaue Vorhersagen darüber
zu machen, was ihre Angehörigen wann tun würden. Zum
Beispiel könnte eine Person möglicherweise fast genau in
dem Moment an die andere denken, in dem diese ihr Telefon
um anzurufen in die Hand nimmt.}

\en{In terms of network terminology, this would be a \textsl{store and forward}
network instead of one where information is
propagated immediately.}%
\de{In der Netzwerkterminologie wäre dies ein \textsl{Store \mbox{and\ }
Forward}-Netzwerk (Speichern und Weiterleiten), statt
einem, in dem Informationen sofort weitergegeben werden.}

\vspace{-0.2mm} % keep
% page
\subsection{\en{leads}\de{fährten}}

\small
\begin{list}{$\bullet$}{\setlength{\leftmargin}{10pt}}

\item
\en{Mirroring the outside world is such a central part of the
human psyche that it would seem likely that nature would try
to make use of any physical effect it could.}%
\de{Die Spiegelung der Aussenwelt ist ein so zentraler Bestandteil
der menschlichen Psyche, dass es nahe läge, dass die Natur
versuchen würde, sich jeden physikalischen Effekt zunutze zu
machen, den sie könnte.}

\item
\en{Experimentally distinguishing different effects that could
explain such phenomena seems to be very difficult.}%
\de{Experimentell verschiedene Effekte, die solche Phänomene
erklären könnten, unterscheiden scheint sehr schwierig zu sein.}

\item
\en{Candidates would include entangled quantum states, as in
the EPR paradox, and self-similarity as in fractals.}%
\de{Kandidaten wären verschränkte Quantenzustände, wie beim
EPR-Paradoxon, und Selbstähnlichkeit wie bei Fraktalen.}

\en{There would be neither senders nor receivers in these views;
sharing would be fundamentally symmetric. Would maybe
different people simply look at the \textsl{same} things inside\,?}%
\de{Bei diesen Thesen gäbe es weder Sender noch Empfänger; der
Austausch wäre fundamental symmetrisch. Würden vielleicht
verschiedene Menschen einfach die \textsl{gleichen} Dinge innen
betrachten\,?}

\en{If there was just one inner world, seen from different perspectives
by different people, similarly to what is usually assumed
about the outer world, would maybe the inner world be as
important or “real” as the outer one, or even more, unlike
nowadays usually assumed in science and technology\,?}%
\de{Wenn es nur eine innere Welt gäbe, die von verschiedenen
Menschen aus verschiedenen Perspektiven betrachtet wird,
ähnlich wie das gewöhnlich für die äussere Welt angenommen
wird, wäre dann vielleicht die innere Welt genauso wichtig
oder “real” wie die äussere, oder sogar noch mehr, anders
als heutzutage gewöhnlich in Wissenschaft und Technik
angenommen wird\,?}

\item
\en{See the article “How astrology might really work\,?” in the
section \cometartemis, about the same themes, but with a different
starting point and focus, also available in German.}%
\de{Siehe den Artikel “Wie Astrologie wirklich funktionieren könn\-te\,?”
in der Sektion \cometartemis, der sich mit denselben Themen
befasst, aber einen anderen Ausgangspunkt und Schwerpunkt
hat; auch auf Englisch vorhanden.}

\item
\en{My personal best guess is still that there would be long-range
emotional connections between people, with spin 1 symmetry
between their heads, see “A few new discoveries in physics”
of 2002 in “Zeitzeugnisse” under \cometartemis. But in the end
it maybe boils down to a question of faith:}%
\de{Meine persönlich favorisierte Vermutung ist immer noch, dass
es emotionale Verbindungen zwischen Menschen über grosse
Distanzen gäbe, mit Spin-1-Symmetrie zwischen ihren Köpfen,
siehe  “A few new discoveries in physics” von 2002 in
“Zeitzeugnisse” under \cometartemis. Aber am Ende läuft es vielleicht
auf eine Glaubensfrage hinaus:}

\en{Would you prefer to believe in a world in which love is a real
immediate connection shared between people, or one in which
it were essentially an illusion created inside of you\,?}%
\de{Würdest du lieber an eine Welt glauben, in der die Liebe eine
echte, unmittelbare Verbindung zwischen Menschen ist, oder
an eine, in der sie im Wesentlichen eine Illusion wäre, die
jede(r) sich selbst erschaffen hätte\,?}

\en{It appears that I am clearly happier since $\pi$ is living in Zürich
again, even though she has a younger boyfriend (with sun and
moon and nodes almost like me). It even appears that when
I sit down somewhere that I would often choose a direction
to look in into which she would most likely be.}%
\de{Es scheint, dass ich eindeutig glücklicher bin, seit $\pi$ wieder in
Zürich lebt, obwohl sie einen jüngeren Freund hat (mit Sonne
und Mond und Mondknoten fast genau wie ich). Es scheint
sogar so zu sein, dass ich, wenn ich mich irgendwo hinsetze,
oft (unbewusst) eine Blickrichtung wählen würde, in der sie
sich am ehesten aufhalten würde.}

\en{“Say, A and P are standing opposite each other, facing each
other, their heads upright. Then the feeling is maximal. Now
A starts to turn around slowly, the head still upright and in
the direction of the body. The feeling will get weaker, be
weakest at 90°, then get stronger again and be practically
maximal again at 180°, then get weaker again, weakest at
270°, and get stronger again towards 0° again. Now A tilts
the head to the right, the feeling will get weaker, be again
minimal at 90°, then A stands on the head, 180°, where the
feeling will be practically maximal again, and then back to
minimal at 270° and back to maximal at 0°.”}%
\de{“Nehmen wir an, A und P stehen sich gegenüber, schauen
sich an, den Kopf aufgerichtet. Dann ist das Gefühl maximal.
Nun beginnt A, sich langsam umzudrehen, den Kopf immer
noch aufrecht und in Richtung des Körpers. Das Gefühl wird
schwächer, am schwächsten bei 90°, wird dann wieder stärker
und ist bei 180° praktisch wieder maximal, wird dann wieder
schwächer, am schwächsten bei 270°, und wird wieder stärker
gegen 0°. Nun neigt A den Kopf nach rechts, das Gefühl wird
schwächer, ist wieder minimal bei 90°, dann steht A auf dem
Kopf, 180°, wo das Gefühl wieder praktisch maximal ist, und
dann wieder minimal bei 270° und wieder maximal bei 0°.”}

% page
\item
\en{Contemporary limitations of ‘Artificial Intelligence’ (AI) via
classical computers plus general considerations make it very
likely to me that consciousness would have to do with quantum
mechanics. But even that would not exclude that collective
beings could come about via everyday person-to-person
interactions, possibly resulting in entangled states that could
be preserved across large distances.}%
\de{Einschränkungen der zeitgenössischen ‘künstlichen Intelligenz’
(KI) via klassische Computer und allgemeine Überlegungen
machen es für mich sehr wahrscheinlich, dass Bewusstsein mit
Quantenmechanik zu tun hätte. Aber selbst das würde
nicht ausschliessen, dass kollektive Wesen durch alltägliche
Interaktionen von Mensch zu Mensch entstehen könnten, und
möglicherweise zu verschränkten Zuständen führen, die über
grosse Entfernungen hinweg erhalten bleiben könnten.}

\item
\en{Deities as collective beings would live much longer than
individual humans, which is one reason why gods and goddesses
are considered immortal, the other being that at last some of
them also stand for abstract concepts, like Venus for love.}%
\de{Götter als kollektive Wesen würden viel länger leben als
einzelne Menschen, was ein Grund dafür ist, dass Götter und
Göttinnen als unsterblich gelten, der andere wäre, dass einige
von ihnen auch für abstrakte Konzepte stehen, wie Venus für
die Liebe.}

\en{In Plato’s \textsl{Menon}, Socrates argues for the immortality of
the soul by trying to prove that knowledge is universally available
to all if only made conscious, as exemplified when Socrates
helps a slave solve a geometrical problem by just asking him
some simple questions, apparently not aware at the time that
asking questions can convey information.}%
\de{In Platons \textsl{Menon} argumentiert Sokrates für die Unsterblichkeit
der Seele, indem er zu beweisen versucht, dass Wissen
universell für alle verfügbar ist, wenn es nur bewusst gemacht wird,
wie gezeigt indem Sokrates einem Sklaven hilft, ein
geometrisches Problem zu lösen, indem er ihm nur ein paar
einfache Fragen stellt, wobei ihm offenbar damals nicht bewusst
war, dass Fragen stellen Informationen übermitteln kann.}

\en{How about a \textsl{Socrates Test} as a variation of the Turing Test,
where an AI would help a human solve all kinds of problems
by just asking questions, and/or vice-versa\,?}%
\de{Wie wäre es mit einem \textsl{Sokrates-Test} als Variation des Turing-Tests,
bei dem eine KI einem Menschen bei der Lösung aller
möglichen Probleme helfen würde, indem sie einfach Fragen
stellen würde, und/oder mit vertauschten Rollen\,?}

\en{Maybe not unexpectedly, the very early “AI” program {\footnotesize{ELIZA}}
did often just ask back very simple questions. It did however
fundamentally already do the same as contemporary AIs do,
it essentially mirrored human input.}%
\de{Vielleicht nicht unerwartet hat das sehr frühe “KI”-Programm
{\footnotesize{ELIZA}} oft nur sehr einfache Fragen zurückgestellt. Fundamental
tat es jedoch bereits dasselbe wie heutige KIs, es spiegelte
im Wesentlichen menschlichen Input.}

\item
\en{In Plato’s \textsl{Critias}, Critias explains how deities guide mortals:}%
\de{In Platons \textsl{Kritias} erklärt Kritias, wie die Götter die Sterblichen
leiten:}

\en{“[…$\!$] they tended us, their nurselings and possessions, as
shepherds tend their flocks, excepting only that they did not
use blows or bodily force, as shepherds do, but governed us
like pilots from the stern of the vessel, which is an easy way
of guiding animals, holding our souls by the rudder of persuasion
according to their own pleasure;—thus did they guide all
mortal creatures.” (translated by B.~Jowett)}%
\de{“[…$\!$] sie hüteten uns, ihre Kinder und ihren Besitz, wie die
Hirten ihre Herden hüten, nur dass sie nicht Schläge oder
körperliche Gewalt anwendeten, wie es die Hirten tun,
sondern sie lenkten uns wie Lotsen vom Heck des Schiffes aus,
was eine einfache Art ist, Tiere zu lenken, und hielten unsere
Seelen durch das Ruder der Überredung nach ihrem eigenen
Wohlgefallen;—so lenkten sie alle sterblichen Geschöpfe.”\newline
(übersetzt auf engl.\ von B.~Jowett)}

\item
\en{Big data and deep learning could be used to find and analyze
such collective structures, including astrological ones.}%
\de{Big Data und Deep Learning könnten genutzt werden, um
solche kollektiven Strukturen zu finden und zu analysieren,
auch astrologische.}

\item
\en{Science is based on some implicit, but fundamentally unprovable
assumptions, like that nature is more stupid than people
and repeats stoically given the same questions. Since numbers
only come to be after a measurement, it is difficult to
compare a mathematical model of the situation before measurement
with reality. So, the “Veil of Isis” may not be easy
to lift, if at all, also related to e5, etc.}%
\de{Die Wissenschaft beruht auf einigen impliziten, aber grund\-sätz\-lich
unbeweisbaren Annahmen, wie der, dass die Natur
dümmer wäre als der Mensch und sich bei gleichen Fragen
stoisch wiederholen würde. Da Zahlen erst nach einer
Messung zustande kommen, ist es schwierig, ein mathematisches
Modell der Situation vor der Messung mit der Realität zu
vergleichen. Der “Schleier der Isis” ist also nicht leicht zu lüften,
wenn überhaupt, auch in Bezug auf e5 usw.}

\end{list}
