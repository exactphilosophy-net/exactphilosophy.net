
\section{\en{greek philosophy}\de{griechische philosophie}}

\en{Aristotle defines elements to be composed of properties
that can be felt by touching. He uses two pairs of opposites,
hot-cold and wet-dry, to define four elements, which he
names \textsl{fire}, \textsl{earth}, \textsl{water }and \textsl{air}. And he identifies wet-dry
with soft-hard, viscous-brittle and smooth-rough. Unlike
later commonly the case, he does not consistently identify
hot-cold with active-passive and light-heavy. If you do, you
get a one-to-one correspondence to my previous definition
of the elements in terms of in/out and rest/move:}%
\de{Aristoteles definiert Elemente als zusammengesetzt aus
Eigenschaften, die man durch Berührung fühlen kann. Er
benutzt zwei Paare von Gegensätzen, warm-kalt und feucht-trocken,
um vier Elemente zu definieren, welche er \textsl{Feuer},
\textsl{Erde}, \textsl{Wasser} und \textsl{Luft} nennt. Und er identifiziert
feucht-trocken mit weich-hart, viskos-spröde und glatt-rau. Anders
als später üblicherweise der Fall, identifiziert er nicht konsistent
warm-kalt mit aktiv-passiv und leicht-schwer. Wenn
man das tut, erhält man eine Eins-zu-Eins-Entsprechung zu
meiner vorherigen Definition der Elemente als Kombinationen
von innen/aussen und ruht/bewegt:}

{\small\begin{center}
\begin{tabular}{|l|l|l|l|l|l|}\hline
\en{\elfire & \textbf{fire} & hot (active) & dry (hard) & \textbf{emo} \\ \hline
\elearth & \textbf{earth} & cold (passive) & dry (hard) & \textbf{ero} \\ \hline
\elwater & \textbf{water} & cold (passive) & wet (soft) & \textbf{emi} \\ \hline
\elair & \textbf{air} & hot (active) & wet (soft) & \textbf{eri} \\ \hline}%
\de{\elfire & \textbf{feuer} & warm (aktiv) & trocken (hart) & \textbf{emo} \\ \hline
\elearth & \textbf{erde} & kalt (passiv) & trocken (hart) & \textbf{ero} \\ \hline
\elwater & \textbf{wasser} & kalt (passiv) & feucht (weich) & \textbf{emi} \\ \hline
\elair & \textbf{luft} & warm (aktiv) & feucht (weich) & \textbf{eri} \\ \hline}
\end{tabular}
\end{center}}

\en{Aristotle defines a fifth element as immutable, moving
only in circles and existing only in space, while the other
four elements move linearly. And he also arranges the four
elements essentially in a circle in which they transform into
each other by flipping one of hot\lrarr cold or wet\lrarr dry at
each transition, while not completely excluding transitions
that flip both at the same time, but considering them more
difficult and slower. The shared theme of a circle links the
transformation of elements to the fifth element.}%
\de{Aristoteles definiert ein fünftes Element als unver\-än\-der\-lich,
sich nur im Kreis bewegend und nur im Weltraum
existierend, während sich die anderen vier Elemente linear
bewegen. Und er ordnet die vier Elemente auch im Wesentlichen
in einem Kreis an, in dem sie sich ineinander wandeln,
indem bei jedem Übergang eines von warm-kalt oder
feucht-trocken ins Gegenteil kippt, wobei er Übergänge, bei
denen beide gleichzeitig kippen, nicht völlig ausschliesst, sie
aber für schwieriger und langsamer erachtet. Das gemeinsame
Thema des Kreises verbindet die Umwandlung der
Elemente mit dem fünften Element.}

\vspace{1mm}\hspace{13mm}
\en{\includegraphics[scale=0.102]{i-circle.jpg}}%
\de{\includegraphics[scale=0.102]{i-circle-de.jpg}}
\vspace{-0.3mm}

\en{In other words, the same circle as tentatively derived
earlier on from my definition of the elements, and a similar
meaning related to e5, as also derived earlier.}%
\de{Mit anderen Worten, derselbe Kreis wie zuvor tentativ
aus meiner Definition der Elemente abgeleitet, und eine
ähnliche Bedeutung bezüglich e5, wie ebenfalls vorher
abgeleitet.}

% page
\en{Passive is inertial in a sense: Outside ero resists more
to get into motion than emo resists to get to rest; inside
emi resists more to get to rest than eri resists to get into
motion. In rough equivalence to inertial and gravitational
mass in physics, inert (passive) would be heavy and dense,
swift (active) would be light and thin.}%
\de{Passiv ist in gewisser Weise träge: Aussen widersteht
ero stärker, in Bewegung zu kommen als emo zur Ruhe zu
kommen; innen widersteht emi mehr, zur Ruhe zu kommen
als eri in Bewegung zu kommen. In grober Entsprechung
zu träger und schwerer Masse in der Physik, wäre träge
(passiv) schwer und dicht, agil (aktiv) leicht und dünn.}

\vspace{-0.2mm} % keep
\subsection{\en{leads}\de{fährten}}

\small
\begin{list}{$\bullet$}{\setlength{\leftmargin}{10pt}}

\item
\en{Aristotle. \textsl{On Generation and Corruption}. Around 350 BCE.}%
\de{Aristoteles. \textsl{Über Werden und Vergehen}. Um 350 v.\,Chr.}

\item
\en{“Since, then, we are looking for ‘originative sources’ of
perceptible body; and since ‘perceptible’ is equivalent to ‘tangible’,
and ‘tangible’ is that of which the perception is touch;
it is clear that not all the contrarieties constitute ‘forms’ and
‘originative sources’ of body, but only those which correspond
to touch.” (Book II, translated by H.~Joachim)}%
\de{“Da wir also nach ‘Ursprungsquellen’ des wahrnehmbaren
Körpers suchen, und da ‘wahrnehmbar’ gleichbedeutend mit
‘greifbar’ ist, und ‘greifbar’ das ist, was durch Berührung
wahrgenommen wird, ist es klar, dass nicht alle Gegensätze
‘Formen’ und ‘Ursprungsquellen’ des Körpers darstellen,
sondern nur diejenigen, die der Berührung entsprechen.”
(Buch II, übersetzt auf engl.\ von H.~Joachim)}

\item
\en{“From moist and dry are derived (iii) the fine and coarse,
viscous and brittle, hard and soft, and the remaining tangible
differences. For (a) since the moist has no determinate shape,
but is readily adaptable and follows the outline of that which
is in contact with it, it is characteristic of it to be ‘such as
to fill up’. Now ‘the fine’ is ‘such as to fill up’. For ‘the fine’
consists of subtle particles; but that which consists of small
particles is ‘such as to fill up’, inasmuch as it is in contact
whole with whole–and ‘the fine’ exhibits this character in a
superlative degree. Hence it is evident that the fine derives
from the moist, while the coarse derives from the dry. Again
(b) ‘the viscous’ derives from the moist: for ‘the viscous’
(e.g.\ oil) is a ‘moist’ modified in a certain way. ‘The brittle’,
on the other hand, derives from the dry: for ‘brittle’ is that
which is completely dry–so completely, that its solidification
has actually been due to failure of moisture. Further (c) ‘the
soft’ derives from the moist. For ‘soft’ is that which yields
to pressure by retiring into itself, though it does not yield
by total displacement as the moist does–which explains why
the moist is not ‘soft’, although ‘the soft’ derives from the
moist. ‘The hard’, on the other hand, derives from the dry: for
‘hard’ is that which is solidified, and the solidified is dry.”}%
\de{“Aus feucht und trocken werden (iii) das Feine und das Grobe
abgeleitet, zähflüssig und spröde, hart und weich, und die
übrigen spürbaren Unterschiede. Denn (a) da das Feuchte
keine bestimmte Form hat, sondern leicht anpassungsfähig
ist und den Umrissen dessen folgt, was dem, was mit ihm
in Berührung kommt, ist es charakteristisch für es, ‘so dass
es ausfüllt’ zu sein. Nun ‘das Feine’ ist ‘so dass es ausfüllt’.
Denn ‘das Feine’ besteht aus kleinen Teilchen; aber das, was
aus kleinen Teilchen besteht, ist ‘ausfüllend’, insofern es in
Kontakt ganz mit dem Ganzen in Kontakt steht–und ‘das
Feine’ hat diesen Charakter in höchstem Grad. Daher ist
es offensichtlich, dass das Feine aus dem Feuchten stammt,
während das Grobe aus dem Trockenen stammt. Wiederum
(b) ‘das Zähflüssige’ leitet sich vom Feuchten ab: denn
‘das Zähflüssige’ (z.B.\ Öl) ist ein auf eine bestimmte Weise
modifiziertes ‘Feuchtes’. Das ‘Spröde’, leitet sich dagegen
vom Trockenen ab: denn ‘spröde’ ist das das vollkommen
trocken ist – so vollkommen, dass seine Verfestigung
tatsächlich auf den Ausfall von Feuchtigkeit zurückzuführen
ist. Ferner (c) ‘das Weiche’ leitet sich vom Feuchten ab. Denn
‘weich’ ist, was dem Druck nachgibt dem Druck nachgibt,
indem es sich in sich selbst zurückzieht, obwohl es nicht durch
totale Verdrängung nachgibt, wie es das Feuchte tut–was
erklärt, warum das Feuchte nicht ‘weich’ ist, obwohl ‘das Weiche’
vom Feuchten abstammt feucht. ‘Das Harte’ hingegen
leitet sich vom Trockenen ab: denn ‘hart’ ist das, was erstarrt
ist, und das Erstarrte ist trocken.”}

\item
\en{“The elementary qualities are four […$\!$]. Hence it is evident
that the ‘couplings’ of the elementary qualities will be four:
hot with dry and moist with hot, and again cold with dry and
cold with moist. […$\!$]\, Fire is hot and dry, whereas Air is hot
and moist (Air being a sort of aqueous vapour); and Water
is cold and moist, while Earth is cold and dry.”}%
\de{“Es gibt vier elementare Qualitäten […$\!$]. Daher ist es
offensichtlich, dass die ‘Kopplungen’ der elementaren Qualitäten
vier sind: warm mit trocken und feucht mit warm, und
wiederum kalt mit trocken und kalt mit feucht. […$\!$]\, Feuer ist
warm und trocken, während Luft warm und feucht ist (Luft
ist eine Art wässriger Dampf); und Wasser ist kalt und feucht,
während Erde kalt und trocken ist.”}

% page
\item
\en{Aristotle arranges the elements in a cycle fire-air-water-earth:}%
\de{Aristoteles ordnet die Elemente in einem Kreislauf
Feuer-Luft-Wasser-Erde an:}

\en{“Thus (i) the process of conversion will be quick between
those which have interchangeable ‘complementary factors’,
but slow between those which have none. The reason is that
it is easier for a single thing to change than for many. Air, e.g.\
will result from Fire if a single quality changes: for Fire, as we
saw, is hot and dry while Air is hot and moist, so that there
will be Air if the dry be overcome by the moist. Again, Water
will result from Air if the hot be overcome by the cold: for Air,
as we saw, is hot and moist while Water is cold and moist,
so that, if the hot changes, there will be Water. So too, in
the same manner, Earth will result from Water and Fire from
Earth, since the two ‘elements’ in both these couples have
interchangeable ‘complementary factors’. For Water is moist
and cold while Earth is cold and dry–so that, if the moist
be overcome, there will be Earth: and again, since Fire is
dry and hot while Earth is cold and dry, Fire will result from
Earth if the cold pass-away. […$\!$] (ii) the transformation of
Fire into Water and of Air into Earth, and again of Water and
Earth into Fire and Air respectively, though possible, is more
difficult because it involves the change of more qualities.”}%
\de{“So wird (i) der Prozess der Umwandlung zwischen denjenigen,
die austauschbare ‘komplementäre Faktoren’ haben,
schnell, zwischen denjenigen, die keine haben, aber langsam
verlaufen. Der Grund dafür ist, dass es für eine einzelne Sache
leichter ist, sich zu verändern als für viele. Luft z.B.\ wird aus
Feuer entstehen, wenn sich eine einzelne Eigenschaft ändert:
denn Feuer ist, wie wir gesehen haben, warm und trocken,
während Luft warm und feucht ist, so dass es Luft geben wird,
wenn das Trockene durch das Feuchte überwunden wird.
Ebenso wird aus der Luft Wasser entstehen, wenn das Warme
durch das Kalte überwunden wird; denn die Luft ist, wie wir
sahen, warm und feucht, während das Wasser kalt und feucht
ist, so dass, wenn das Warme sich ändert, Wasser entsteht.
Auf dieselbe Weise wird Erde aus Wasser und Feuer aus Erde
entstehen, da die beiden ‘Elemente’ in diesen beiden Paaren
austauschbare ‘komplementäre Faktoren’ haben. Denn
Wasser ist feucht und kalt, während Erde kalt und trocken
ist–wenn also das Feuchte überwunden wird, entsteht Erde;
und da Feuer trocken und warm ist, während Erde kalt und
trocken ist, wird Feuer aus Erde entstehen, wenn die Kälte
vergeht. […$\!$] (ii) Die Verwandlung von Feuer in Wasser
und von Luft in Erde, und wiederum von Wasser und Erde
in Feuer bzw.\ Luft ist zwar möglich, aber schwieriger, weil sie
die Veränderung von mehr Qualitäten beinhaltet.”}

\item
\en{In \textsl{On Generation and Corruption}, Aristotle considers
light-heavy not to be an attribute of any specific elements:}%
\de{In \textsl{Über Werden und Vergehen} erachtet Aristoteles
leicht-schwer nicht als Eigenschaft von spezifischen Elementen:}

\en{“(i) heavy and light are neither active nor susceptible. Things
are not called ‘heavy’ and ‘light’ because they act upon, or
suffer action from, other things. But the ‘elements’ must be
reciprocally active and susceptible, since they ‘combine’ and
are transformed into one another. On the other hand (ii) hot
and cold, and dry and moist, are terms, of which the first pair
implies power to act and the second pair susceptibility.”}%
\de{“(i) schwer und leicht sind weder aktiv noch empfänglich.
Die Dinge werden nicht ‘schwer’ und ‘leicht’ genannt, weil
sie auf andere Dinge einwirken oder von ihnen beeinflusst
werden. Aber die ‘Elemente’ müssen wechselseitig aktiv und
empfänglich sein, da sie sich ‘verbinden’ und ineinander
umgewandelt werden. Andererseits sind (ii) ‘warm’ und ‘kalt’
sowie ‘trocken’ und ‘feucht’ Begriffe, von denen das erste Paar
Wirkkraft und das zweite Paar Empfänglichkeit impliziert.”}

% page
\en{But in \textsl{On the Heavens}, he considers air and fire as light and
water and earth as heavy, in the order earth-water-air-fire, and
postulates the existence of an immutable fifth element that
dominates in the sky, is neither light nor heavy and moves in
circles, while the first four elements move linearly:}%
\de{Aber in \textsl{Über den Himmel} betrachtet er Luft und Feuer als
leicht und Wasser und Erde als schwer, in der Reihenfolge
Erde-Wasser-Luft-Feuer, und postuliert die Existenz eines
unveränderlichen fünften Elements, das im Himmel dominiert,
weder leicht noch schwer ist und sich in Kreisen bewegt,
während sich die ersten vier Elemente linear bewegen:}

\en{“[…$\!$] all locomotion, as we term it, is either straight or
circular or a combination of these two, which are the only simple
movements. […$\!$]\, Now revolution about the centre is circular
motion, while the upward and downward movements are in a
straight line, ‘upward’ meaning motion away from the centre,
and ‘downward’ motion towards it. […$\!$]\, For if the natural
motion is upward, it will be fire or air, and if downward, water
or earth. […$\!$] circular motion is necessarily primary. For the
perfect is naturally prior to the imperfect, and the circle is a
perfect thing. […$\!$]\, These premises clearly give the conclusion
that there is in nature some bodily substance other than
the formations we know, prior to them all and more divine
than they. […$\!$] there is something beyond the bodies that
are about us on this earth, different and separate from them;
and that the superior glory of its nature is proportionate to its
distance from this world of ours. […$\!$] things are heavy and
light relatively to one another; air, for instance, is light relatively
to water, and water light relatively to earth. The body,
then, which moves in a circle cannot possibly possess either
heaviness or lightness. For neither naturally nor unnaturally
can it move either towards or away from the centre. […$\!$] this
body will be ungenerated and indestructible and exempt from
increase and alteration […$\!$] earth is enclosed by water, water
by air, air by fire, and these similarly by the upper bodies.”
(Book I, translated by J.~Stocks)}%
\de{“[…$\!$] alle Fortbewegung, wie wir sie nennen, ist entweder
gerade oder Kreisbewegung oder eine Kombination dieser
beiden, die die einzigen einfachen Bewegungen sind. […$\!$]\, Nun
ist die Drehung um den Mittelpunkt eine Kreisbewegung,
während die Aufwärts- und Abwärtsbewegungen in einer
geraden Linie sind, wobei ‘aufwärts’ die Bewegung vom
Zentrum weg bedeutet, und ‘abwärts’ die Bewegung zu ihm hin.
[…$\!$]\, Denn wenn die natürliche Bewegung nach oben ist, ist es
Feuer oder Luft, und wenn sie nach unten ist, Wasser oder Erde.
[…$\!$] kreisförmige Bewegung ist notwendigerweise primär.
Denn das Vollkommene geht natürlich dem Unvollkommenen
voraus, und der Kreis ist ein vollkommenes Ding. […$\!$]\, Aus
diesen Voraussetzungen ergibt sich klar die Schlussfolgerung,
dass es in der Natur eine andere körperliche Substanz gibt
als die uns bekannten Gebilde, die ihnen allen vorausgeht und
göttlicher ist als sie. […$\!$] es gibt etwas jenseits der Körper,
die das uns auf dieser Erde umgibt, anders und getrennt von
ihnen; und dass die höhere Herrlichkeit seiner Natur proportional
zu seiner Entfernung von dieser unserer Welt ist. […$\!$]
Dinge sind schwer und leicht im Vergleich zueinander; Luft
zum Beispiel ist leicht im Vergleich zu Wasser, und Wasser
ist leicht im Vergleich zu Erde. Der Körper, der sich im Kreis
bewegt, kann also unmöglich entweder Schwere oder Leichtigkeit
besitzen. Denn weder natürlich noch unnatürlich kann
er sich auf den Mittelpunkt zu oder von ihm weg bewegen.
[…$\!$] dieser Körper wird unerschaffen und unzerstörbar sein
und frei vom Zuwachs und Veränderung. […$\!$] Erde wird von
Wasser umschlossen, Wasser von Luft, Luft von Feuer, und
diese wiederum ähnlich von den oberen Körpern.” (Buch I,
übersetzt auf engl.\ von J.~Stocks)}

\item
\en{Aristotle appears to consistently consider the pair of opposites
hot/cold active and the pair wet/dry passive, see the quote
from \textsl{On Generation and Corruption} above, or the following
quote from \textsl{Meteorology}:}%
\de{Aristoteles scheint konsistent das Gegensatzpaar warm/kalt
als aktiv und das Paar feucht/trocken als passiv zu betrachten,
siehe das obige Zitat aus \textsl{Über Werden und Vergehen}
oder das folgende Zitat aus \textsl{Meteorologie}:}

\en{“All this makes it clear that bodies are formed by heat and
cold and that these agents operate by thickening and solidifying.
It is because these qualities fashion bodies that we
find heat in all of them, and in some cold in so far as heat is
absent. These qualities, then, are present as active, and the
moist and the dry as passive, and consequently all four are
found in mixed bodies.” (Book IV, translated by E.~Webster)}%
\de{“All dies macht deutlich, dass Körper durch Wärme und Kälte
geformt werden und dass diese Mittel durch Verdickung und
Verfestigung wirken. Weil diese Eigenschaften die Körper
formen, finden wir in allen Körpern Wärme und in manchen
Kälte, wenn die Wärme fehlt. Diese Qualitäten sind also
aktiv, die feuchten und trockenen passiv, und folglich sind alle
vier in gemischten Körpern zu finden.” (Buch IV, übersetzt
auf engl.\ von E.~Webster)}

\item
\en{In the outside world, the elements water and air (essentially
liquids and gases or gas-like phenomena like clouds or smoke)
appear softer and more fluidly in motion than the element
earth (solid matter). The element fire (flames, lightning),
however, does not appear to be visibly hard, while, like earth,
quite closely related to dryness.}%
\de{In der Aussenwelt erscheinen die Elemente Wasser und Luft
(im Wesentlichen Flüssigkeiten und Gase oder gasähnliche
Erscheinungen wie Wolken oder Rauch) weicher und flüssiger in
der Bewegung als das Element Erde (feste Materie). Das
Element Feuer (Flammen, Blitze) hingegen scheint nicht sichtbar
hart zu sein, während es wie die Erde recht eng mit Trockenheit
verbunden ist.}

% page
\item
\en{While many works of Aristotle and Plato have been preserved
in their entirety, works of earlier philosophers, as well of many
later ones, like the Stoics, have usually only survived as
fragmentary quotes by later philosophers, typically around early
CE or even later. Since this was also the time in which the
“canonical view” on the elements emerged for centuries to
follow in astrology, alchemy, medicine, etc., it is difficult to
reconstruct other views with certainty. Moreover, it seems
that some schools of philosophy might have had oaths which
would bind their members not to speak about certain
fundamental views, or only in carefully veiled form.}%
\de{Während viele Werke von Aristoteles und Platon vollständig
erhalten sind, sind die Werke früherer Philosophen sowie
vieler späterer Philosophen, wie der Stoiker, in der Regel nur als
fragmentarische Zitate späterer Philosophen überliefert,
typischerweise um die Zeit früh n.\,Chr.\ oder noch später. Da dies
auch die Zeit war, in der sich die “kanonische Sicht” auf die
Elemente herausbildete, die über Jahrhunderte hinweg in der
Astrologie, Alchemie, Medizin usw.\ zum Tragen kam, ist es
schwierig, andere Ansichten mit Sicherheit zu rekonstruieren.
Darüber hinaus scheint es in einigen Philosophenschulen Eide
gegeben zu haben, die ihre Mitglieder verpflichteten, über
bestimmte grundlegende Ansichten nicht oder nur in sorgfältig
verschleierter Form zu sprechen.}

\en{In a nutshell, the earliest source I know of that attributes fire
and air to active, and water and earth to passive is Cicero
in \textsl{Academica} (45 BCE), possibly influenced by the Stoics.
The first attribution of the same elements to male-female in
astrology is Vettius Valens in \textsl{Anthologia} (2nd century CE).
Aristotle names Empedocles at least twice as the first to have
considered four elements. Plato introduces a fifth element in
the\,\textsl{Timaeus}, most likely predating Aristotle.}%
\de{Kurz gesagt, die früheste mir bekannte Quelle, die Feuer und
Luft als aktiv und Wasser und Erde als passiv einstuft, ist
Cicero in \textsl{Academica} (45 v.\,Chr.), möglicherweise beeinflusst
von den Stoikern. Die erste Zuordnung der gleichen Elemente
zu männlich-weiblich in der Astrologie ist Vettius Valens in
\textsl{Anthologia} (2.\ Jahrhundert n.\,Chr.). Aristoteles nennt
Empedokles mindestens zweimal als den ersten, der vier Elemente
in Betracht gezogen hat. Platon führt ein fünftes Element im
\textsl{Timaios} ein, höchstwahrscheinlich vor Aristoteles.}

\en{A fragmentary closer look below and in following sections.}%
\de{Eine fragmentarische nähere Betrachtung folgt unten und in
späteren Sektionen.}

\item
\en{David Sedley writes in chapter 11 of \textsl{The Cambridge History
of Hellenistic Philosophy} (2000) that the Stoic’s identification
of fire and air with active emerged from medical tradition,
from \textsl{pneuma}, breath, which was seen as a mixture of fire and
air, and mentions also that this identification was originally
not exclusively the only view of the Stoics in their time.}%
\de{David Sedley schreibt in Kapitel 11 von \textsl{The Cambridge
History of Hellenistic Philosophy} (2000), dass die stoische
Identifikation von Feuer und Luft mit aktiv aus der medizinischen
Tradition stammt, von \textsl{pneuma}, dem Atem, der als eine
Mischung aus Feuer und Luft angesehen wurde, und erwähnt
auch, dass diese Identifikation ursprünglich nicht die einzige
Ansicht der Stoiker zu ihrer Zeit war.}

\item
\en{In \textsl{Academica} (45 BCE), Cicero lets Antiochus of Ascalon say
the following, influenced by Aristotle and maybe the Stoics:}%
\de{In \textsl{Academica} (45 v.\,Chr.) lässt Cicero Antiochus von Askalon
Folgendes sagen, beeinflusst von Aristoteles und vielleicht den
Stoikern:}

\en{“Accordingly air […$\!$] and fire and water and earth are primary;
while their derivatives are the species of living creatures
and of the things that grow out of the earth. Therefore those
things are termed […$\!$] elements; and among them air and
fire have motive and efficient force, and the remaining divisions
[…$\!$] water and earth, receptive and ‘passive’ capacity.
Aristotle deemed that there existed a certain fifth sort of
element, in a class by itself and unlike the four that I have
mentioned above, which was the source of the stars and of
thinking minds.” (Book I 26, translated by H.\ Rackham)}%
\de{“Demnach sind Luft […$\!$] und Feuer und Wasser und Erde
primär; während deren Derivate Arten von Lebewesen und
die Dinge sind, die aus der Erde wachsen. Daher werden diese
Dinge […$\!$] Elemente genannt; und unter ihnen haben Luft
und Feuer treibende und wirksame Kraft, und die übrigen
Abteilungen […$\!$] Wasser und Erde, rezeptive und ‘passive’
Fähigkeit. Aristoteles war der Meinung, dass es eine bestimmte
fünfte Art von Element gibt, das zu einer eigenen Klasse
gehört und sich von den vier, die ich oben erwähnt habe,
unterscheidet, und das die Quelle der Sterne und der denkenden
Geister ist.” (Buch I 26, übersetzt auf engl.\ von H.\ Rackham)}

\item
\en{A bit later astrological views emerged that see fire and air as
male, and water and earth as female. See Vettius Valens’s
\textsl{Anthologia} in the 2nd century CE and hints in earlier texts by
Dorotheus of Sidon and Marcus Manilius. These views have
essentially prevailed, including in medieval alchemy and up to
contemporary astrology.}%
\de{Etwas später kamen astrologische Ansichten auf, die Feuer
und Luft als männlich und Wasser und Erde als weiblich
ansehen. Siehe Vettius Valens’ \textsl{Anthologia} im 2.\ Jahrhundert
n.\,Chr. und Hinweise in früheren Texten von Dorotheus von
Sidon und Marcus Manilius. Diese Ansichten haben sich im
Wesentlichen durchgesetzt, auch in der mittelalterlichen
Alchemie und bis hin zur zeitgenössischen Astrologie.}

\item
\en{In contemporary astrology, the element fire is associated with
(visual) imagination and impulse, air with (abstract) thinking
and communication, water with feelings and faith, earth with
pragmatic realism—to give just a rough summary.}%
\de{In der zeitgenössischen Astrologie wird das Element Feuer mit
(visueller) Vorstellungskraft und Impuls, Luft mit (abstraktem)
Denken und Kommunikation, Wasser mit Gefühlen und
Glauben, Erde mit pragmatischem Realismus in Verbindung
gebracht—um nur eine grobe Zusammenfassung zu geben.}

% page
\item
\en{Most things in the sky beyond clouds are round or cyclic: sun
and moon are round; planets, as well as stars during night
and seasons, move periodically in predictable cycles.}%
\de{Die meisten Dinge am Himmel jenseits der Wolken sind rund
oder zyklisch: Sonne und Mond sind rund; die Planeten sowie
die Sterne während der Nacht und den Jahreszeiten bewegen
sich periodisch in vorhersagbaren Zyklen.}

\item
\en{The fifth element is also called ether or aether and quin\-tes\-sence.
Many different views of the fifth element and closely
related concepts have emerged over time.}%
\de{Das fünfte Element wird auch Äther und Quintessenz
genannt. Im Laufe der Zeit haben sich viele verschiedene
Ansichten über das fünfte Element und eng verwandte Konzepte
herausgebildet.}

\en{Plato used the word aether to describe the purest form of air
in the \textsl{Timaeus}. But there is also a strong association of the
sky with fire, because stars and planets appear to emit light
and the sun provides heat, and also because fire was often
considered the lightest of the four elements.}%
\de{Plato verwendete das Wort Äther, um die reinste Form der
Luft im Timaios zu beschreiben. Es gibt aber auch eine starke
Assoziation des Himmels mit dem Feuer, weil Sterne und
Planeten Licht auszustrahlen scheinen und die Sonne Wärme
spendet, und auch weil das Feuer oft als das leichteste
der vier Elemente angesehen wurde.}

\en{The fifth element is generally considered “divine” because
gods were often believed to live in heaven. And it is often
also seen as special in other ways, like able to create life, or
immortal like the soul or maybe pneuma, or able to create
matter and to hold it together, or maybe identified by some
alchemists with the philosopher’s stone, which was believed
to be able to transform matter, like lead to gold, etc.\,?}%
\de{Das 5.\ Element wird allgemein als “göttlich” gesehen, weil
man oft glaubte, dass Götter im Himmel leben. Und es wird
oft auch in anderer Hinsicht als etwas Besonderes gesehen,
wie fähig, Leben zu erschaffen, oder unsterblich wie die Seele
oder vielleicht Pneuma, oder fähig, Materie zu erschaffen und
zusammenzuhalten, oder vielleicht von einigen Alchemisten
mit dem Stein der Weisen identifiziert, von dem man glaubte,
er könne Materie verwandeln, z.B.\ Blei in Gold usw.\,?}

\item
\en{Do such associations (historically founded or not) fit well with
the definition of e5 simply because they all keep going in
circles around the same questions\,?}%
\de{Passen solche Assoziationen (historisch begründet oder nicht)
gut zur Definition von e5 einfach weil sie sich immer wieder
in Kreisen um die gleichen Fragen drehen\,?}

\item
\en{Apuleius in $\!$\textsl{The Doctrines of Plato} in the 2nd century CE:}%
\de{Apuleius in \textsl{Platon und seine Lehre} im 2.\ Jahrhundert n.\,Chr.:}

\en{“In the first place, the twin pupils of the eyes are very clear, 
and, shining with a certain light of vision, they possess the 
office of knowing light; while hearing, by partaking of the 
nature of air, has a perception of sounds, through messengers 
in the air; whereas the taste, being a sense more relaxed, is 
on that account suited to things rather moist and watery;
but  the touch, as being of the earth and corporeal, perceives 
things, that are rather solid, and which can be handled and 
struck against. Of those things likewise, which are changed, 
when corrupted there is a separate perception. For in the 
middle of the region of the face Nature has placed the nostrils, 
by the double door-way of which there passes an odour
together with the breath; and that conversions and changes 
furnish the causes of smelling; and that they are perceived 
from substances, when corrupted or burnt, or in a mucous 
or moistened state; […$\!$].” (Book I, 14, translated by G.\ Burges)}%
\de{“An erster Stelle die Zwillingspupillen der Augen; sie sind
sehr klar, und da sie mit einem bestimmten Licht des Sehens
leuchten, haben sie die Aufgabe, Licht zu erkennen; während
das Gehör, da es an der Natur der Luft teilhat, eine
Wahrnehmung von Tönen durch Boten in der Luft hat; während
der Geschmack, da er ein entspannterer Sinn ist, aus diesem
Grund für Dinge geeignet ist, die eher feucht und wässrig
sind; aber der Tastsinn, da er von der Erde und körperlich
ist, nimmt Dinge wahr, die eher fest sind und die angefasst
und angeschlagen werden können. Auch von den Dingen, die
sich verändern, wenn sie verdorben sind, gibt es eine eigene
Wahrnehmung. Denn in der Mitte der Gesichtsregion hat
die Natur die Nasenlöcher angebracht, durch deren doppelten
Eingang ein Geruch zusammen mit dem Atem strömt;
und dass Umwandlungen und Veränderungen die Ursachen
des Geruchs sind, und dass sie von Substanzen wahrgenommen
werden, wenn sie verdorben oder verbrannt oder in einem
schleimigen oder befeuchteten Zustand sind; […$\!$].” (Buch I,
14, übersetzt auf englisch von G.\ Burges)}

\en{Even though most of what he writes is from Platos’ \textsl{Timaeus},
it seems that a view of see-fire, hear-air, taste-water, touch-earth
is something that Apuleius implicitly added. Even more
so with the implicit association of transformations of the
elements with smell and the 5th element, which would maybe
already mirror the definition of \textsl{e5} in my model…}%
\de{Obwohl das meiste, was er schreibt, aus Platons $\!$\textsl{Timaios}
stammt, scheint es, dass eine Sicht von Sehen-Feuer, Hören-Luft,
Schmecken-Wasser, Tasten-Erde etwas ist, das Apuleius
implizit hinzugefügt hat. Dies gilt umso mehr für die implizite
Assoziation von Transformationen der Elemente mit Riechen
und dem 5.\ Element, was vielleicht schon die Definition von
\textsl{e5} in meinem Modell widerspiegeln würde…}

% page
\item
\en{According to Diogenes Laërtius in the third century CE, the
Stoics would have identified fire with hot, earth with dry,
water with wet, and air with cold (and dry):}%
\de{Gemäss Diogenes Laërtius im 3.\ Jahrhundert n.\,Chr.\ hätten
die Stoiker Feuer mit warm, Erde mit trocken, Wasser mit feucht
und Luft mit kalt (und trocken) gleichgesetzt:}

\en{“[…$\!$] the four elements are all equally an essence without any
distinctive quality, namely, matter; but fire is the hot, water
the moist, air the cold, and earth the dry—though this last
quality is also common to the air. The fire is the highest,
and that is called aether, in which first of all the sphere was
generated in which the fixed stars are set, then that in which
the planets revolve; after that the air, then the water; and
the sediment as it were of all is the earth, which is placed in
the centre of the rest.” (7.\,LXIX, translated by C.\ Yonge)}%
\de{“[…$\!$] die vier Elemente sind alle gleichermassen eine Essenz
ohne irgendeine unterscheidende Qualität, nämlich die Materie
aber Feuer ist das Warme, Wasser das Feuchte, Luft das
Kalte und Erde das Trockene—obwohl diese letzte Qualität
auch der Luft gemeinsam ist. Das Feuer ist das Höchste, und
das heisst Äther, in dem zuerst die Sphäre entstanden ist, in
der die Fixsterne stehen, dann die, in der die Planeten kreisen;
danach die Luft, dann das Wasser; und der Bodensatz
von allen ist gleichsam die Erde, die in der Mitte der übrigen
steht.” (7.\ LXIX, übersetzt auf engl.\ von C.\ Yonge)}

\en{The papyrus \textsl{Anonymus Londinensis} from about the first
century CE says essentially the same about Philistion (apparently
Philistion of Locri, a contemporary of Plato):}%
\de{Das Papyrus \textsl{Anonymus Londinensis} aus dem ersten
Jahrhundert n.\,Chr.\ sagt im Wesentlichen dasselbe über Philistion
(offenbar Philistion von Locri, ein Zeitgenosse Platons):}

\en{“Philiston thinks that we are composed of four ‘forms’, that
is, of four elements—fire, air, water, earth. Each of these too
has its own power; of fire the power is the hot, of air it is
the cold, of water the moist, and of earth the dry.” (XX 24,
translated by W.\ Jones)}%
\de{“Philiston glaubt, dass wir aus vier ‘Formen’ bestehen, d.h.\
aus vier Elementen—Feuer, Luft, Wasser, Erde. Jedes dieser
Elemente hat seine eigene Kraft; die Kraft des Feuers ist das
Warme, die der Luft das Kalte, die des Wassers das Feuchte
und das der Erde das Trockene.” (XX 24, übersetzt auf engl.\
von W.\ Jones)}

\en{According to David Hahm in \mbox{\textsl{The Origins of Stoic Cosmology}}
(1977), this view might have already been quite common
among physicians in classical times. Artistotle’s texts about
biology seem to implicitly reflect that view, like that air is
inhaled cold and exhaled hot (pneuma). Although there
appear to be no contemporary sources that would directly prove
such an identification, Hahm’s detailed argumentation that
the Stoics aimed for a unified view of the elements (unlike
apparently Aristotle) across all fields seems plausible.}%
\de{Laut David Hahm in $\!$\textsl{The Origins of Stoic Cosmology} (1977)
könnte diese Ansicht bereits in der Antike unter Ärzten weit
verbreitet gewesen sein. Artistoteles’ Texte über die Biologie
scheinen diese Ansicht implizit widerzuspiegeln, z.B.\ dass die
Luft kalt eingeatmet und warm ausgeatmet wird (Pneuma).
Obwohl es keine zeitgenössischen Quellen zu geben scheint,
die eine solche Identifizierung direkt belegen würden, erscheint
die detaillierte Argumentation von Hahm, dass die Stoiker
(anders als offenbar Aristoteles) eine einheitliche Sicht der
Elemente über alle Gebiete hinweg anstrebten, plausibel.}

\en{In Stoic belief, the cosmos emerged from fire via air to water
to earth, and back (see Hahm for details), essentially along
Aristotle’s circle of the elements or light to heavy and back.}%
\de{In stoischer Vorstellung entwickelte sich der Kosmos von Feuer
über Luft zu Wasser zu Erde und zurück (siehe Hahm für
Details), im Wesentlichen entlang Aristoteles’ Kreis der
Elemente oder vom Leichten zum Schweren und zurück.}

\item
\en{In ancient Greek philosophy there was also the idea of matter
consisting of indivisible physical units (atoms). In Plato’s
\textsl{Timaeus}, a model is presented that combines both views by
associating the elements with the five Platonic solids:
fire-tetrahedron, air-octahedron, water-icosahedron, earth-cube
and the “roundest” one, the dodecahedron, for the whole
world/universe (pan). Kepler’s drawings (1619):}%
\de{In der antiken griechischen Philosophie gab es auch die
Vorstellung, dass die Materie aus unteilbaren physikalischen
Einheiten (Atomen) besteht. In Platons $\!$\textsl{Timaios} wird ein Modell
vorgestellt, das beide Ansichten miteinander verbindet,
indem die Elemente mit den fünf platonischen Körpern in
Verbindung gebracht werden: Feuer-Tetraeder, Luft-Oktaeder,
Wasser-Ikosaeder, Erde-Würfel und der “rundeste”, der
Dodekaeder, für die ganze Welt/das Universum (pan). Keplers
Zeichnungen (1619):}

\vspace{2mm}
\hspace{0.8mm}\includegraphics[scale=0.13]{i-platonic.jpg}
\vspace{1mm}

\en{Today they are usually paired cube-octahedron, dodecahedron-icosahedron
and tetrahedron-itself, because the centers of the
surfaces yield the corners of the dual body.}%
\de{Heute werden die platonischen Körper in der Regel in Paaren
als Würfel-Oktaeder, Dodekaeder-Ikosaeder und Tetraeder-(sich selbst)
gesehen, weil die Mittelpunkte der Flächen die
Ecken des dualen Körpers ergeben.}

\en{In 4 dimensions there are 6 generalized Platonic solids, in 5
and more dimensions always only 3, namely generalizations
of tetrahedron, cube and octahedron.}%
\de{In 4 Dimensionen gibt es 6 verallgemeinerte platonische Körper,
in 5 und mehr Dimensionen immer nur 3, nämlich Verallgemeinerungen
von Tetraeder, Würfel und Oktaeder.}

\end{list}
