% note the trailing "I" for correct height when cutting out
\section{\en{measurement}\de{messung} {\color{almostwhite}I}}

\en{Using a camera, emo and ero could be defined as the
difference between two images taken shortly after each other.
Differing pixels would be emo, same pixels ero. For example,
a ball that rolls down a slope would itself not be emo
as a physical object, but emo would be the area the ball
spawns between the two images (excluding the middle if
the ball is uniformly colored).}%
\de{Wenn man eine Kamera benutzt, könnte man emo und ero
als den Unterschied zwischen zwei kurz nacheinander
aufgenommenen Bildern definieren. Unterschiedliche Pixel wären
emo, gleiche Pixel ero. Ein Ball, der einen Abhang hinunterrollt,
wäre beispielsweise als physisches Objekt selbst nicht
emo, aber emo wäre der Bereich, den der Ball zwischen den
beiden Bildern erzeugt (mit Ausnahme der Mitte, wenn der
Ball gleichmässig gefärbt ist).}

\vspace{0.5mm}\hspace{16mm}
\noindent
\includegraphics[scale=0.05]{i-labrys.jpg}
\vspace{1mm}

\en{A camera can only register ero and emo, and thus only
transitions ero\lrarr emo, while transitions that would cross
between in and out would not be part of the picture.}%
\de{Eine Kamera kann nur ero und emo aufzeichnen, und
somit nur Übergänge ero\lrarr emo, währenddem Übergänge
zwischen innen und aussen nicht Teil des Bildes wären.}

\en{Measurements inside might be performed indirectly by
measuring brain activity outside, or maybe by focussing on
what is recurring inside and thus in a way stable inside, on
maybe often abstract insights (eri) of a logical nature.}%
\de{Messungen innen könnten indirekt durch Messung der
Hirnaktivität draussen erfolgen, oder vielleicht durch Fokussieren
auf das, was innen wiederkehrt und somit in gewisser
Weise stabil wäre, auf vielleicht oft abstrakte Einsichten
(eri) einer logischen Natur.}

\en{The most basic form of eri might be pairs of opposites,
which could maybe be assembled to form more complex
concepts, possibly inspired or guided by zodiacs and similar
cultural concepts.}%
\de{Die grundlegendste Form von eri könnte Gegensatzpaare
sein, die vielleicht zu komplexeren Konzepten zusammengesetzt
werden könnten, eventuell inspiriert oder geleitet
von Tierkreisen und ähnlichen kulturellen Konzepten.}

\subsection{\en{leads}\de{fährten}}

\small
\begin{list}{$\bullet$}{\setlength{\leftmargin}{10pt}}

\item
\en{Even if a formal model of the elements defined in terms of
in/out and rest/move and their transformations grew into a
‘scientific way of doing metaphysics’, as aimed at in Kant’s
\textsl{Prolegomena to Any Future Metaphysics That Will Be Able
to Present Itself as a Science}, it would essentially be air,
something that rests inside the mind (eri). It would not be
complete without also including the other three elements in
some form, say, in performance art, or whatever.}%
\de{Selbst wenn ein formales Modell der Elemente, das durch
innen/aussen und ruht/bewegt und ihren Transformationen
definiert wäre, zu einer ‘wissenschaftlichen Art und Weise,
Metaphysik zu betreiben’ heranwachsen würde, wie es in Kants
\textsl{Prolegomena zu einer jeden künftigen Metaphysik, die als
Wissenschaft wird auftreten können} angestrebt wird, wäre
das im Wesentlichen Luft, etwas, das im Geist ruht (eri). Es
wäre nicht vollständig, solange es nicht auch die anderen drei
Elemente in irgendeiner Form einschliessen würde, zum
Beispiel via Performance-Kunst oder wie auch immer.}

\item
\en{Moreover, it would likely not be possible to deduce the whole
world from the definition of elements alone, at least doing
so would likely be as hard as finding a theory of everything
in modern science. Some additional, a priori unprovable
assumptions might be necessary to synthesize the world.}%
\de{Darüber hinaus wäre es vermutlich nicht möglich, die gesamte
Welt allein aus der Definition der Elemente abzuleiten,
zumindest wäre dies wahrscheinlich so schwierig wie die Suche
nach einer Theorie für alles in der modernen Wissenschaft.
Einige zusätzliche, a priori unbeweisbare Annahmen könnten
notwendig sein, um die Welt zu synthetisieren.}

% page
\item
\en{The concept of a “ball” is a priori much more complex than
comparing two images, which becomes evident once you try
to program computers to recognize (3-dimensional) items on
2-dimensional images. How a ball comes to be in the mind
appears to require a lot of interaction with the environment
(often quite early as a child), and in the end it is philosophically
not so clear whether a “ball” is rather a natural thing,
something that objectively exists, or instead rather a purely
abstract cultural creation useful for interaction with others.
See also Kant or Plato’s Allegory of the Cave.}%
\de{Das Konzept eines “Balls” ist a priori viel komplexer als der
Vergleich zweier Bilder, was deutlich wird, wenn man
versucht, Computer so zu programmieren, dass sie (dreidimensionale)
Gegenstände auf zweidimensionalen Bildern erkennen.
Wie ein Ball im Kopf entsteht, scheint eine Menge
Interaktion mit der Umwelt zu erfordern (oft schon im
Kindesalter), und letztlich ist es philosophisch nicht so klar, ob
ein “Ball” eher ein natürliches Ding ist, etwas, das objektiv
existiert, oder eher eine rein abstrakte kulturelle Schöpfung,
die für die Interaktion mit anderen nützlich ist. Siehe auch
Kant oder Platons Höhlengleichnis.}

\en{The above definition of emo\lrarr ero appears thus fundamental,
but is possibly already different from immediate experience of
the world in which a rolling ball is never seen as two crescents.
It reminds also of the shadows in Plato’s Cave, which even
remind of the souls of the dead that dwell in Hades as shadows,
as depicted in Homer’s Odyssey. In other words, the
above definition might already project reality onto something
in which crucial information might already be lost, or not.}%
\de{Die obige Definition von emo\lrarr ero scheint also fundamental
zu sein, unterscheidet sich aber möglicherweise bereits von
der unmittelbaren Erfahrung der Welt, in der eine rollende
Kugel niemals als zwei Halbmonde gesehen wird. Sie erinnert
auch an die Schatten in Platons Höhle, die sogar an die Seelen
der Toten erinnern, die als Schatten im Hades wohnen, wie in
Homers Odyssee dargestellt. Mit anderen Worten: Die obige
Definition projiziert die Realität möglicherweise bereits auf
etwas, in dem entscheidende Informationen bereits verloren
gegangen sind, oder auch nicht.}

\item
\en{Could maybe only activity cross between in and out, but not
elements\,? Would activity travelling from in to out transform
both eri to emi and ero to emo\,? That would at least be
consistent with a camera only recording ero and emo.}%
\de{Könnte vielleicht nur Aktivität zwischen innen und aussen
hin- und herwandern, aber keine Elemente\,? Würde Aktivität,
die sich von innen nach aussen bewegt, sowohl eri in emi als
auch ero in emo verwandeln\,? Das wäre zumindest konsistent
damit, dass eine Kamera nur ero und emo registriert.}

\item
\en{In a harmonic oscillator, two kinds of energies are transformed
into each other. For example, for a mass on a spring, the
energy in the spring transforms into the kinetic energy of the
moving mass and vice-versa. This gives the motion of the
oscillator four special states, when either of the energies is
extremal. And the motion between these states is periodic,
thus overall reminding of the circle of elements.}%
\de{In einem harmonischen Oszillator werden zwei Arten von
Energie ineinander umgewandelt. Bei einer Masse an einer Feder
zum Beispiel wandelt sich die Energie der Feder in die
kinetische Energie der bewegten Masse um und umgekehrt.
Dadurch erhält die Bewegung des Oszillators vier spezielle
Zustände, wenn jeweils eine der Energien extrem ist. Und die
Bewegung zwischen diesen Zuständen ist periodisch, so dass
sie insgesamt an den Kreis der Elemente erinnert.}

\en{However, the natural pairing of extremal states of a harmonic
oscillator is opposite states in the cycle, which naturally fits
rest/move in the elemental circle, but makes it hard to relate
two pairs of \textsl{adjacent} states to opposites like active/passive
or in/out in a natural way.}%
\de{Die natürliche Paarung der Extremzustände eines harmonischen
Oszillators sind jedoch entgegengesetzte Zustände im
Zyklus, was natürlich zu ruht/bewegt im Kreis der Elemente
passt, es aber schwierig macht, zwei Paare von \textsl{benachbarten}
Zuständen auf natürliche Weise mit Gegensätzen wie
aktiv/passiv oder innen/aussen zu verbinden.}

\item
\en{The four elements can be grouped into 3 different pairs with
opposing attributes, including maybe these:}%
\de{Die vier Elemente können in drei verschiedene Paare mit
gegensätzlichen Attributen gruppiert werden, inklusive vielleicht
diesen:}

\vspace{-1mm}
\begin{center}
{\small\begin{tabular}{ccc}
\en{\textbf{rest/move} & \textbf{in/out} & \textbf{passive/active} \\
bind/release & wet/dry & cold/hot \\
 & soft/hard & heavy/light \\
 & malleable/brittle & inert/swift \\
 & mixed/isolated & dense/thin \\
 & collective/individual & dark/light \\
 & & female/male \\
 & & moon/sun \\
 & & night/day \\
 & & un-/conscious \\}%
\de{\textbf{ruht/bewegt} & \textbf{innen/aussen} & \textbf{passiv/aktiv} \\
binden/lösen & feucht/trocken & kalt/warm \\
 & weich/hart & schwer/leicht \\
 & viskos/spröde & träge/agil \\
 & gemischt/isoliert & dicht/dünn \\
 & kollektiv/individuell & dunkel/hell \\
 & & weiblich/männlich \\
 & & mond/sonne \\
 & & nacht/tag \\
 & & un-/bewusst \\} 
\end{tabular}}
\end{center}
\vspace{-1mm}

\en{Some pairs on the right have a historically patriarchal touch,
which however still partially reflects nature.}%
\de{Einige Paare rechts haben einen historisch-patriarchalischen
Touch, welcher dennoch zum Teil die Natur spiegelt.}

\end{list}
